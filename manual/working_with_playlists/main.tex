% $Id$ %
\section{\label{ref:working_with_playlists}Working with Playlists}

\subsection{Playlist terminology}
Some common terms that are used in Rockbox when referring to
playlists:

\begin{description}
\item[Directory.] A playlist! One of the keys to getting the most out of
  Rockbox is understanding that Rockbox \emph{always} considers the song that
  it is playing to be part of a playlist, and in some situations, Rockbox will
  create a playlist automatically. For example, if you are playing the
  contents of a directory, Rockbox will automatically create a playlist
  containing all songs in it. This means that just about anything
  that is described in this chapter with respect to playlists also applies to
  directories.

\item[Dynamic playlist.]  A dynamic playlist is a playlist that is created
  ``On the fly.'' Any time you insert or queue tracks using the
  \setting{Playing Next Submenu} (see \reference{ref:playingnext_submenu}), you are
  creating (or adding to) a dynamic playlist.

\item[Play/Add.] In Rockbox, to \setting{Play} or \setting{Add} an item means
  putting it into a playlist and leaving it there, even after it is
  played.

\item[Queue.] To \setting{Queue} a song means to put it into a playlist, but then
  to remove the song from the playlist once it has been played.
\end{description}

\subsection{Creating playlists}

Rockbox can create playlists in four different ways.

\subsubsection{By selecting (``playing'') a song from the File Browser}
Whenever a song is selected from the \setting{File Browser} with
\ActionTreeEnter, Rockbox will automatically create a playlist containing
all of the songs in that directory and start playback with the selected
song.

\note{If you already have created a dynamic playlist, playing a new
  song will \emph{erase} the current dynamic playlist and create a new one.
  If you want to add a song to the current playlist
  rather than erasing the current
  playlist, see the section below on how to add music to a playlist.}

\subsubsection{By using the Play or Play Shuffled functions}
The \setting{Play} function as described in \ref{ref:playingnext_submenu}
will replace the dynamic playlist with the selected tracks. The \setting{Play
Shuffled} function is similar, except the selected tracks will be added to the
playlist in a random order.

\subsubsection{\label{ref:addtoplaylist_submenu}By using the Add to Playlist submenu}
The \setting{Add to Playlist submenu} makes it possible to modify and create
playlists that are not currently playing. To do this select \setting{Add to Playlist...}
in the \setting{Context Menu}. There you will have two choices,
\setting{Add to Existing Playlist} adds the selected track or directory to an existing
playlist and \setting{Add to New Playlist} creates a new playlist containing
the selected track or directory.

\note{All playlists in the \setting{Playlist catalogue} are stored by default
  in the \fname{/Playlists} directory in the root of your \daps{} disk and
  playlists stored in other locations are not included in the catalogue. It is
  however possible to move existing playlists there (see
  \reference{ref:Contextmenu}).}

\subsubsection{By using the Main Menu}
To create a playlist containing some or all of the music on your \dap{}, you can use the
\setting{Create Playlist} command in the \setting{Playlist Catalogue Context Menu}
(see \reference{ref:playlistcatalogue_contextmenu}).

\subsection{Adding music to playlists}

\subsubsection{\label{ref:playingnext_submenu}Adding music to a dynamic playlist}
\screenshot{rockbox_interface/images/ss-playlist-menu}{The Playing Next Submenu}{}
The \setting{Playing Next Submenu} is a submenu in the \setting{Context Menu} (see
\reference{ref:Contextmenu}), it allows you to put tracks into a
``dynamic playlist''. The place in which the newly
selected tracks are added to the playlist is determined by these
options:

\begin{description}
\item [Play Next.] Play track(s) immediately after the currently playing track.

\item [Add.] Add track(s) after the most recently added tracks or, if tracks
have not been added yet, immediately after the currently playing track.

\item [Play Last.] Add track(s) to the end of the playlist.

\item [Add Shuffled.] Add track(s) to the playlist at random positions.

\item [Play Last Shuffled.] Add tracks in a random order to the end of the playlist.
\end{description}

If you'd like to replace the current playlist with the new selection, the
following two options will achieve that effect.

\begin{description}
\item [Play.] Replace all entries in the dynamic playlist with the selected
  tracks. If \setting{Keep Current Track When Replacing Playlist} is set to
  \setting{Yes}, the new tracks will play after the current track finishes
  playing; if no track is playing or the setting is \setting{No}, the new
  tracks will begin playing immediately.

\item [Play Shuffled.] Similar, except the tracks will be added to the new
  playlist in random order.
\end{description}

Another possibility is to add tracks \emph{temporarily} to the dynamic playlist.
In Rockbox’s parlance, this is called queuing. Queued tracks are automatically
removed from the playlist after they have been played. They are also not saved
to the playlist file (see \reference{ref:playlistoptions}).

\begin{description}
\item [Queue Next.] Corresponds to \setting{Play Next}.

\item [Queue.] Corresponds to \setting{Add}.

\item [Queue Last.] Corresponds to \setting{Play Last}.

\item [Queue Shuffled.] Corresponds to \setting{Add Shuffled}.

\item [Queue Last Shuffled.] Corresponds to \setting{Play Last Shuffled}.
\end{description}

\note{Visibility of options to add shuffled tracks or to queue tracks can be toggled by going to
\setting{Settings} $\rightarrow$ \setting{General Settings} $\rightarrow$ \setting{Playlists}
$\rightarrow$ \setting{Current Playlist}. Select either \setting{Show Shuffled Adding Options}
or \setting{Show Queue Options} to customize the displayed set of options.}

The \setting{Playing Next Submenu}  can be used to add either single tracks or
entire directories to a playlist. If the \setting{Playing Next Submenu} is
invoked on a single track, it will put only that track into the playlist.
On the other hand, if the \setting{Playing Next Submenu} is invoked on a
directory, Rockbox adds all of the tracks in that directory to the
playlist.

\note{You can control whether or not Rockbox includes the contents of
  subdirectories when adding an entire directory to a playlist. Set the
  \setting{Settings $\rightarrow$ General Settings $\rightarrow$ Playlist
  $\rightarrow$ Recursively Insert Directories} setting to \setting{Yes} if
  you would like Rockbox to include tracks in subdirectories as well as tracks
  in the currently-selected directory.}

Dynamic playlists are saved so resume will restore them exactly as they
were before shutdown.

\note{To view, save, reshuffle, or display the play time of the current
  dynamic playlist use the
  \setting{Playlist} sub menu in the WPS context menu or in the
  \setting{Main Menu}.}

\subsection{Modifying playlists}
\subsubsection{Reshuffling}
Reshuffling the current playlist is easily done from the \setting{Current Playlist}
sub menu in the WPS, just select \setting{Reshuffle}.

\subsubsection{Moving and removing tracks}
To move or remove a track from the current playlist enter the
\setting{Playlist Viewer} by selecting \setting{View Current Playlist} in the
\setting{Current Playlist} submenu in the WPS context menu or the \setting{Main Menu}.
Once in the \setting{Playlist Viewer} open the context menu on the track you
want to move or remove. If you want to move the track, select \setting{Move} in
the context menu and then move the blinking cursor to the place where you want
the track to be moved and confirm with \ActionStdOk. To remove a track, simply
select \setting{Remove} in the context menu.

\subsection{Saving playlists}
To save the current playlist either enter the \setting{Current Playlist} submenu
in the \setting{WPS Context Menu} (see \reference{sec:contextmenu}) and
select \setting{Save Current Playlist} or enter the context menu for the
\setting{Playlist catalogue} in the \setting{Main Menu} and select
\setting{Save Current Playlist}.
Either method will bring you to the \setting{Virtual Keyboard} (see
\reference{sec:virtual_keyboard}), enter a filename for your playlist and
accept it and you are done.

\subsection{Loading saved playlists}
\subsubsection{Through the \setting{File Browser}}
Playlist files, like regular music tracks, can be selected through the
\setting{File Browser}. When loading a playlist from disk it will replace
the current dynamic playlist. If you want to look at a playlist's
content without starting playback immediately, access the \setting{Context Menu} (see
\reference{ref:Contextmenu}) with \ActionStdContext{} and choose \setting{View}.

\subsubsection{Through the \setting{Playlist catalogue}}
The \setting{Playlist catalogue} offers a shortcut to all playlists in your
\daps{} specified playlist directory.
It can be used like the \setting{File Browser} but will display
the content of a playlist when one is selected.

