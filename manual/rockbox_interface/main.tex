% $Id$ %
\chapter{\label{ref:rockbox_interface}Quick Start}
\section{Basic Overview}
\subsection{The \daps{} controls}

% include the front image. Using \specimg makes this fairly easy,
% but requires to use the exact value of \specimg in the filename!
% The extension is selected in the preamble, so no further \Ifpdfoutput
% is necessary.
%
% The check looks for a png file -- we use png for the HTML manual, so that
% format needs to be present. It can also be used for the pdf manual, but
% usually we provide a pdf version of the file for that. Picking the correct
% one is done by LaTeX automatically, but for checking the filename we need to
% specify the extension.
\begin{center}
\IfFileExists{rockbox_interface/images/\specimg-front.png}
   {\Ifpdfoutput{\includegraphics[height=8cm,width=10cm,keepaspectratio=true]%
    {rockbox_interface/images/\specimg-front}}
   {\includegraphics{rockbox_interface/images/\specimg-front}}
   }
   {\color{red}{\textbf{WARNING!} Image not found}%
    \typeout{Warning: missing front image}
   }
\end{center}
\opt{HAVEREMOTEKEYMAP}{
  % spacing between the two pictures, could possibly be improved
  \begin{center}
  \IfFileExists{rockbox_interface/images/\specimg-remote.png}
    {\Ifpdfoutput{\includegraphics[height=5.6cm,width=10cm,keepaspectratio=true]{rockbox_interface/images/\specimg-remote}}
    {\includegraphics{rockbox_interface/images/\specimg-remote}}
    }
    {\color{red}{\textbf{WARNING!} Image not found}%
     \typeout{Warning: missing remote image}
    }
  \end{center}
}

Throughout this manual, the buttons on the \dap{} are labelled according to the
picture above.
\opt{touchscreen}{
The areas of the touchscreen in the 3$\times$3 grid mode are in turn referred as follows:
\begin{table}
    \centering
    \begin{tabular}{|c|c|c|}
	\hline
        \TouchTopLeft & \TouchTopMiddle & \TouchTopRight \\ [5ex]
	\hline
	\TouchMidLeft & \TouchCenter & \TouchMidRight \\ [5ex]
	\hline
	\TouchBottomLeft & \TouchBottomMiddle & \TouchBottomRight \\ [5ex]
	\hline
    \end{tabular}
\end{table}
}%
Whenever a button name is prefixed by ``Long'', a long press of approximately
one second should be performed on that button. The buttons are described in
detail in the following paragraph.
\blind{%
  Additional information for blind users is available on the Rockbox website at
  \wikilink{BlindFAQ}.

  %
  \opt{iriverh100}{
  Hold or lay the \dap{} so that the side with the joystick and LCD is facing
  towards you, and the curved side is at the top. The joystick functions as
  the \ButtonUp{}, \ButtonRight{}, \ButtonLeft{}, and \ButtonDown{} buttons when
  pressed in the appropriate direction. Pressing the joystick down functions as
  \ButtonSelect{}.
  On the right side of the \dap{} are the \ButtonOn{}, \ButtonOff{},
  \ButtonMode{} buttons, and the \ButtonHold{} switch. When this switch is
  switched towards the bottom of the \dap{}, hold is on, and none of the other
  buttons have any effect.

  On the left side is the \ButtonRec{} button. Above that is the internal microphone.

  On the top panel of the \dap{}, from left to right, you can find the
  following: headphone mini jack plug, remote port, Optical line-in, Optical line-out.

  On the bottom panel of the \dap{}, from left to right, you can find the
  following: power jack, reset switch, and USB port. In the event that your
  \dap{} hard locks, you can reset it by inserting a paper clip into the hole
  where the reset switch is.}
  %
  \opt{iriverh300}{
  Hold or lay the \dap{} so that the side with the button pad and
  LCD is facing towards you.  The buttons on the button pad are as follows:  top
  left corner: \ButtonOn{}, bottom left corner: \ButtonOff{}, top right corner:
  \ButtonRec, bottom right corner: \ButtonMode{}.  In the center of the button pad
  is a button labelled \ButtonSelect{}.  Surrounding the \ButtonSelect{} button are
  the \ButtonUp{}, \ButtonDown{}, \ButtonLeft{}, and \ButtonRight{} buttons.

  On the top panel of the \dap{}, from left to right, you can find the
  following: headphone mini jack plug, remote port, line-in, line-out.

  On the left hand side of the \dap{} is the internal microphone. Just underneath
  this is a small hole, the reset switch. In the event that your \dap{} hard locks,
  you can reset it by inserting a paper clip into the hole where the reset switch
  is.

  On the right hand side of the \dap{} is the \ButtonHold{} switch. When this is
  switched towards the bottom of the \dap{}, hold is on, and none of the other
  buttons have any effect.

  On the bottom panel of the \dap{}, from left to right, you can find the
  following:  power jack and two USB ports.  The USB port on the right is used
  to connect your \dap{} to your computer.  The USB port on the left is not
  used in Rockbox.
  }
  %
  \opt{mpiohd200}{
  Hold or lay the \dap{} so that the side with the LCD is facing towards you.
  On the right hand side there is a rocker switch at the top which serves as
  \ButtonRew{} and \ButtonFF{} when rocked up or down, respectively.
  Pressing the rocker in functions as the \ButtonFunc{} button. Below the rocker
  there are the \ButtonRec{} and \ButtonPlay{} buttons. At the bottom of the
  right panel there is the \ButtonHold{} switch. When this is switched towards the
  bottom of the \dap{}. hold is on, and none of the other buttons have any effect.

  On the top panel of the \dap{} there is another rocker which serves as the
  \ButtonVolDown{} and \ButtonVolUp{} buttons when pressed to the left or right,
  respectively.

  On the left hand side of the \dap{} there is a headphone mini jack plug at the top
  and a small hole at the bottom, the reset switch. In the event that your \dap{}
  hard locks, you can reset it by inserting a paper clip into the hole where the
  reset switch is.

  On the bottom panel of the \dap{}, from left to right, you can find the
  following: power jack, line-in jack and USB port (under rubber cover).
  }
  %
  \opt{ipod4g,ipodcolor,ipodvideo,ipodmini}{
  The main controls on the \dap{} are a slightly indented scroll wheel
  with a flat round button in the center. Hold the \dap{} with these controls
  facing you.

  The top of the player will have the following, from left to
  right:
  \opt{ipod4g,ipodcolor}{remote connector, headphone socket, \ButtonHold{}
    switch.}
  \opt{ipodvideo}{\ButtonHold{} switch, headphone socket.}
  \opt{ipodmini}{\ButtonHold{} switch, remote connector, headphone socket.}

  The dock connector that is used to connect your \dap{} to your computer is on
  the bottom panel of the \dap{}.

  The button in the middle of the wheel is called \ButtonSelect{}. You can
  operate the wheel by pressing the top, bottom, left or right sections,
  or by sliding your finger around it.  The top is \ButtonMenu{}, the bottom is
  \ButtonPlay{}, the left is \ButtonLeft{}, and the right is \ButtonRight{}.
  When the manual says to \ButtonScrollFwd{}, it means to slide your finger
  clockwise around the wheel. \ButtonScrollBack{} means to slide your finger
  counterclockwise. Note that the wheel is sensitive, so you will need to move
  slowly at first and get a feel for how it works.

  Note that when the \ButtonHold{} switch is pushed toward the center of the \dap{},
  hold is on, and none of the other controls do anything.  Be sure
  \ButtonHold{} is off before trying to use your player.
  }
  %
  \opt{ipod3g}{
  The main controls on the \dap{} are a slightly indented touch wheel
  with a flat round button in the center, and four buttons in a row above the
  touch wheel. Hold the \dap{} with these controls
  facing you.

  The top of the player will have the following, from left to
  right: remote connector, headphone socket, \ButtonHold{} switch.

  The dock connector that is used to connect your \dap{} to your computer is on
  the bottom panel of the \dap{}.

  The button in the middle of the wheel is called \ButtonSelect{}. You can
  operate the wheel by sliding your finger around it.  The row of
  buttons consists of, from left to right, the \ButtonLeft{},
  \ButtonMenu{}, \ButtonPlay{}, and \ButtonRight{} buttons.
  When the manual says to \ButtonScrollFwd{}, it means to slide your finger
  clockwise around the wheel. \ButtonScrollBack{} means to slide your finger
  counterclockwise. Note that the wheel is sensitive, so you will need to move
  slowly at first and get a feel for how it works.

  Note that when the \ButtonHold{} switch is pushed toward the center of the \dap{},
  hold is on, and none of the other controls do anything.  Be sure
  \ButtonHold{} is off before trying to use your player.
  }
  %
  \opt{ipod1g2g}{
  The main controls on the \dap{} are a slightly indented wheel
  with a flat round button in the center, and four buttons surrounding
  it. On the 1st generation iPod, this wheel physically turns. On the
  2nd generation iPod, this wheel is touch-sensitive. Hold the \dap{} with these controls
  facing you.

  The top of the player will have the following, from left to
  right: FireWire port, headphone socket, \ButtonHold{} switch.

  The FireWire port is used to connect your \dap{} to the computer and
  to charge its battery via a wall charger.

  The button in the middle of the wheel is called \ButtonSelect{}. You can
  operate the wheel by turning it, or sliding your finger around
  it. The top is \ButtonMenu{}, the bottom is \ButtonPlay{}, the left
  is \ButtonLeft{}, and the right is \ButtonRight{}.
  When the manual says to \ButtonScrollFwd{}, it means to slide your finger
  clockwise around the wheel. \ButtonScrollBack{} means to slide your finger
  counterclockwise. Note that the wheel is sensitive, so you will need to move
  slowly at first and get a feel for how it works.

  Note that when the \ButtonHold{} switch is pushed toward the center of the \dap{},
  hold is on, and none of the other controls do anything.  Be sure
  \ButtonHold{} is off before trying to use your player.
  }
  %
  \opt{ipodnano,ipodnano2g}{
  The main controls on the \dap{} are a slightly indented wheel with a
  flat round button in the center. Hold the \dap{} with these controls on the
  top surface. There is a \ButtonHold{} switch at one end, and
  headphone and dock connector at the other; be sure the end with the
  switch is facing away from you.

  The button in the middle of the wheel is called \ButtonSelect{}. You can
  operate the wheel by pressing the top, bottom, left or right sections,
  or by sliding your finger around it.  The top is \ButtonMenu{}, the bottom is
  \ButtonPlay{}, the left is \ButtonLeft{}, and the right is \ButtonRight{}.
  When the manual says to \ButtonScrollFwd{}, it means to slide your finger
  clockwise around the wheel. \ButtonScrollBack{} means to slide your finger
  counterclockwise. Note that the wheel is sensitive, so you will need to move
  slowly at first and get a feel for how it works.

  Note that when the \ButtonHold{} switch is pushed toward the center of the \dap{},
  hold is on, and none of the other controls do anything; be sure \ButtonHold{} is
  off before trying to use your player.
  }
  %
  \opt{iriverh10,iriverh10_5gb}{
  Hold or lay the \dap{} so that the side with the scroll pad and
  LCD is facing towards you. In the centre below the lcd is the scroll pad. It
  is oriented vertically. Touching the top and bottom half of it acts as the
  \ButtonScrollUp{}  and \ButtonScrollDown{} buttons respectively. On the left
  of the scroll pad is the \ButtonLeft{} button and on the right is the
  \ButtonRight{} button.

  There are three buttons on the right hand side of the \dap{}. From top to
  bottom, they are: \ButtonRew{}, \ButtonPlay{} and \ButtonFF{}. On the left
  hand side is the \ButtonPower{} button.

  On the top panel of the \dap{}, from left to right, you can find the
  following: \ButtonHold{} switch, \opt{iriverh10}{reset pin hole, }remote port
  and headphone mini jack plug.

  On the bottom panel of the \dap{} is the data cable port.}
  %
  \opt{gigabeatf}{
  \note{The following description is for the Gigabeat F, but can also apply for the
  Gigabeat X. The Gigabeat F is slightly larger and more rectangular shaped, while the
  Gigabeat X is smaller and has a slightly tapered back.}

  Hold the \dap{} with the screen on top and the controls on the right hand side.
  Below the screen is a cross-shaped touch sensitive pad which contains the
  \ButtonUp{}, \ButtonDown{}, \ButtonLeft{} and \ButtonRight{} controls.  On the
  Gigabeat X, this pad will feel slightly raised up, while it will feel slightly
  sunken in on the Gigabeat F. On the top of the unit, from left to right, are the
  power socket, the \ButtonHold{} switch, and the headphone socket.  The
  \ButtonHold{} switch puts the \dap{} into hold mode when it is switched to the
  right of the unit. The buttons will have no effect when this is the case.

  Starting from the left hand side on the bottom of the unit, nearer to the front
  than the back, is a recessed switch which
  controls whether the battery is on or off.  When this switch is to the left,
  the battery is disconnected.  This can be used for a hard reset of the unit,
  or if the \dap{} is being placed in storage.  Next to that is a connector for
  the docking station and finally on the right hand side of the bottom of the
  unit is a mini USB socket for connecting directly to USB.

  Finally on the right hand side of the unit are some control buttons.  Going from
  the bottom of the unit to the top there is a small round \ButtonA{} buttton then a
  rocker volume switch with of the \ButtonVolDown{} button below the \ButtonVolUp{}
  button.  Above that is are two more small round buttons, the \ButtonMenu{}
  button and nearest to the top of the unit the \ButtonPower{} button, which is held
  down to turn the \dap{} on or off. If you have a Gigabeat X, these buttons are small
  metallic buttons that are place further up on the right hand side, and closer
  together. The layout is still the same, however.}
  %
  \opt{gigabeats}{
  Hold the \dap{} with the screen on top and the controls on the right hand side.
  Directly below the bottom edge of the screen are two buttons, \ButtonBack{}
  on the left and \ButtonMenu{} on the right. Below them is a cross-shaped pad
  which contains the \ButtonUp{}, \ButtonDown{}, \ButtonLeft{}, \ButtonRight{}
  and \ButtonSelect{} controls.
  On the top of the unit from left to right are the headphone socket and the
  \ButtonHold{} switch.  The \ButtonHold{} switch puts the \dap{} into
  hold mode when it is switched to the right of the unit.
  The buttons will have no effect when this is the case.

  Starting from the left hand side on the bottom of the unit, nearer to the back
  than the front, is a recessed switch which controls whether the battery is on
  or off.  When this switch is to the left, the battery is disconnected.
  This can be used for a hard reset of the unit, or if the \dap{} is being placed
  in storage.  Next to that is a mini USB socket for connecting directly to USB,
  and finally a custom connector, presumably for planned accessories which were
  never released.

  Finally on the right hand side of the unit are some control buttons and the power
  connector.  Going from the bottom of the unit to the top, there is the power
  connector socket, followed by three small round buttons, the
  \ButtonNext{} buttton, \ButtonPlay{} button, and \ButtonPrev{} button (from bottom
  to top) then a rocker volume switch with of the \ButtonVolDown{} button below the
  \ButtonVolUp{} button.  Above that is one more small round button, the \ButtonPower{}
  button, which is held down to turn the \dap{} on or off.}
  %
  \opt{mrobe100}{
  Hold the \dap{} with the black front facing you such that the m:robe writing
  is readable. Below the writing is the touch sensitive pad with the
  \ButtonMenu{}, \ButtonPlay{}, \ButtonLeft{}, \ButtonRight{} and \ButtonDisplay
  controls indicated by their symbols. The dotted center strip is devided in
  three parts: \ButtonUp{}, \ButtonSelect{} and \ButtonDown. On the top of the
  unit, on the right, is the \ButtonPower{} switch, which is held down to turn
  the \dap{} on or off.

  The \ButtonHold{} switch is located on the left of the \dap{}, below the
  headphone socket. It puts the \dap{} into hold mode when it is switched to the
  top of the unit. The buttons will have no effect when this is the case. On the
  bottom of the unit, there is a connector for the docking station or the
  proprietary USB connector for connecting directly to USB.}
  %
  \opt{iaudiom5,iaudiox5}{
  The \dap{} is curved so that the end with the screen on it is thicker than the
  other end.  Hold the \dap{} wih the thick end towards the top and the screen
  facing towards you.  Half way up the front of the unit on the right hand side
  is a four way joystick which is the \ButtonUp{}, \ButtonDown{},
  \ButtonLeft{}, and \ButtonRight{} buttons. When pressed it serves as \ButtonSelect{}.

  On the right hand side of the \dap{} from top to bottom, first there is a two
  way switch.  the \ButtonPower{} button is activated by pushing this switch up,
  and pushing this switch down until it clicks slightly will activate the
  \ButtonHold{} button.  When the switch is in this position, none of the other
  keys will have an effect.

  Below the switch is a lozenge shaped button which is the \ButtonRec{}
  button, and below that the final button on this side of the unit, the
  \ButtonPlay{} button.  Just below this is a small hole which is difficult to
  locate by touch which is the internal microphone.  At the very bottom of
  this side of the unit is the reset hole, which can be used to perform a hard
  reset by inserting a paper clip.

  On the bottom of the unit is the connector for the
  \playerman{} subpack or dock.  On the top of the unit is a charge
  indicator light, which may feel a bit like a button, but is not.

  From the top of the \dap{} on the left hand side is the headphone socket, then the
  remote connector.  Below this is a cover which protects the \opt{iaudiox5}{USB
  host connector.}\opt{iaudiom5}{USB and charging connector}.}
  %
  \opt{e200,e200v2}{
  Hold the \dap{} with the turning wheel at the front and bottom.  On the bottom left
  of the front of the \dap{} is a raised round button, the \ButtonPower{} button.
  Above and to the left of this, on the outside of the turning wheel are four
  buttons.  These are the \ButtonUp{}, \ButtonDown{}, \ButtonLeft{} and
  \ButtonRight{} buttons.  Inside the wheel is the \ButtonSelect{} button.  Turning
  the wheel to the right activates the \ButtonScrollFwd{} function, and to the
  left, the \ButtonScrollBack{} function.

  On the right of the unit is a slot for inserting flash cards.  On the bottom is
  the connector for the USB cable.  On the left is the \ButtonRec{} button, and
  on the top, there is the headphone socket to the right, and the \ButtonHold{}
  switch.  Moving this switch to the right activates hold mode in which none of the
  other buttons have any effect.  Just to the left of the \ButtonHold{} switch is a
  small hole which contains the internal microphone.}
  %
  \opt{c200,c200v2}{
  Hold the \dap{} with the buttons on the right and the screen on the left. On
  the right side of the unit, there is a series of four connected buttons that
  form a square. The four sides of the square are the \ButtonUp{},
  \ButtonDown{}, \ButtonLeft{} and \ButtonRight{} buttons, respectively. Inside
  the square formed by these four buttons is the \ButtonSelect{} button. At the
  bottom right corner of the square is a small separate button, the
  \ButtonPower{} button.

  Moving clockwise around the outside of the unit, on the top are the \ButtonVolUp{}
  and \ButtonVolDown{} buttons, which control the volume of playback. The buttons can
  be distinguished by a sunken triangle on the \ButtonVolDown{} button, and a
  raised triangle on the \ButtonVolUp{} button. To the right of
  the volume buttons on the top of the unit is the slot for inserting flash
  memory cards. On the right side of the unit is the connector for the USB
  cable. At center of the bottom of the \dap{} is the \ButtonRec{} button. To
  the left of the \ButtonRec{} button is the \ButtonHold{} switch. Moving this
  switch to the right activates hold mode, in which none of the other buttons
  have any effect. On the lower left side of the unit is the headphone socket.
  Immediately above the headphone socket is a lanyard loop and the microphone.
  }
  %
  \opt{fuze,fuzev2}{
  Hold the \dap{} with the controls on the bottom and the screen on the top. The main
  controls are a scroll wheel with four clickable points and a button in the centre; pressing
  this centre button functions as \ButtonSelect{}. Going clockwise from the top, the clickable
  points on the wheel are the \ButtonUp{}, \ButtonRight{}, \ButtonDown{}, and \ButtonLeft{}
  buttons. Turning the wheel clockwise is \ButtonScrollFwd{}, and turning it counter-clockwise
  is \ButtonScrollBack{}. Immediately above and to the right of the wheel is the \ButtonHome{}
  button.

  On the lower left of the unit is a slot for inserting microSD cards. Immediately below that is
  the opening for the microphone.

  On the bottom of the unit is the connector for connecting a USB cable and the headphone socket.
  On the lower right hand side of the unit is a two-way switch. Pressing this switch up acts as
  \ButtonPower{}, and clicking it down until it locks acts as the \ButtonHold{} switch. When the
  \ButtonHold{} switch is on, none of the other buttons have any effect.
  }
  %
  \opt{clipplus,clipv1,clipv2,clipzip}{
  Hold the \dap{} with the controls on the bottom and the screen on the top. The main
  controls are a four-way pad with a button in the centre; pressing this centre button
  functions as \ButtonSelect{}. Going clockwise from the top, the four-way pad contains
  the \ButtonUp{}, \ButtonRight{}, \ButtonDown{}, and \ButtonLeft{} buttons.
  Immediately above and to the \nopt{clipzip}{right}\opt{clipzip}{left} of the four-way
  pad is the \ButtonHome{} button.
  }
  %
  \opt{clipplus,clipzip}{
  The \ButtonPower{} button is on the top of the \dap{}\opt{clipplus}{, towards the right side.}

  At the bottom of the right side of the \dap{} is a slot for microSD cards.
  Above this slot on the right side is the headphone socket.

  On the left hand panel is a two-way button that acts as \ButtonVolDown{} when
  pressed on the bottom, and \ButtonVolUp{} when pressed on the top. Immediately
  above the switch is a mini-USB port to connect the \dap{} to a computer.

  }
  %
  \opt{clipv1,clipv2}{
  On the left hand panel is a two way switch. Pressing this switch up acts as
  \ButtonPower{}, and clicking it down until it locks acts as the \ButtonHold{}
  switch. When the \ButtonHold{} switch is on, none of the other buttons have any
  effect. Immediately above the switch is a mini-USB port to connect the \dap{} to
  a computer.

  On the right hand panel is a two-way button that acts as \ButtonVolDown{} when
  pressed on the bottom, and \ButtonVolUp{} when pressed on the top. Immediately
  above this button is the headphone socket.
  }
  %
  \opt{vibe500}{
  Hold or lay the \dap{} so that the side with the controls and
  LCD is facing towards you. Below the LCD is the touch sensitive pad with the \ButtonMenu{},
  \ButtonPlay{}, \ButtonLeft{}, \ButtonRight{} controls and the scroll pad in the centre. The
  scroll pad is oriented vertically between the \ButtonOK{} and \ButtonCancel{} buttons.
  Sliding a finger up or down the scroll pad acts as \ButtonUp{} and \ButtonDown{} respectively.
  Note that the scroll pad is sensitive, so you will need to move
  slowly at first and get a feel for how it works.

  There are two buttons on the right hand side of the \dap{}: \ButtonPower{} on the top and
  \ButtonRec{} underneath. Under these buttons, from top to bottom you can find: USB connector,
  power connector and the reset hole if you need to perform a hardware reset.

  The \ButtonHold{} switch is located on the left hand side of the \dap{}. Note that when the
  \ButtonHold{} switch is moved towards the top of the \dap{}, hold is turned on and all the
  other controls are disabled. Be sure \ButtonHold{} is off before trying to use your player.

  On the top on the \dap{} is the internal microphone on the left and the line-in socket on the
  right, near the headphone socket.}
  %
  \opt{samsungyh820}{
  Hold or lay the \dap{} so that the side with the controls and
  LCD is facing towards you. Directly below the bottom edge of the screen are three buttons:
  \ButtonRew{} on the left, \ButtonPlay{} in the middle and \ButtonFF{} on the right. Below them
  is a four-way pad which contains the \ButtonDown{}, \ButtonUp{}, \ButtonLeft{} and
  \ButtonRight{} controls.

  At the top of the right hand side of the \dap{} is the \ButtonRec{} button.

  On the top panel of the \dap{}, from left to right, you can find the following: headphone
  socket, line-in socket, internal microphone, and the \ButtonHold{} switch. Note that when the
  \ButtonHold{} switch is moved towards the center of the \dap{}, hold is turned on and all the
  other controls are disabled. Be sure \ButtonHold{} is off before trying to use your player.

  At the top of the back side of the player, just under the \ButtonHold{} button is the reset
  hole, if you need to perform a hardware reset.

  The USB/dock connector that is used to connect your \dap{} to your computer is on
  the bottom panel of the \dap{}.
  }
  %
  \opt{samsungyh920,samsungyh925}{
  Hold or lay the \dap{} so that the side with the controls and
  LCD is facing towards you. Below the LCD is a four-way pad with the \ButtonDown{},
  \ButtonUp{}, \ButtonLeft{} and \ButtonRight{} buttons.

  There are three buttons at the top of the right hand side of the \dap{}: \ButtonFF{} on the top,
  \ButtonPlay{} in the middle and \ButtonRew{} underneath. Below these buttons is the \ButtonRec{}
  switch. Rockbox doesn't take note of the actual \emph{position} of the switch, but reacts to a
  \emph{switching movement} like pressing a regular button.

  On the top panel of the \dap{}, from left to right, you can find the following: headphone/remote
  socket, line-in socket, internal microphone, and the \ButtonHold{} switch. Note that when the
  \ButtonHold{} switch is moved towards the center of the \dap{}, hold is turned on and all the
  other controls are disabled. Be sure \ButtonHold{} is off before trying to use your player.

  At the top of the back side of the player, just under the \ButtonHold{} button is the reset hole,
  if you need to perform a hardware reset.

  The USB/dock connector that is used to connect your \dap{} to your computer is on
  the bottom panel of the \dap{}.
  }
  %
}

\subsection{Turning the \dap{} on and off}
\opt{cowond2}{Rockbox has a dual-boot feature with the original firmware being
  the default.\\}
To turn on and off your Rockbox enabled \dap{} use the following keys:
    \begin{btnmap}
      \opt{IRIVER_H100_PAD,IRIVER_H300_PAD}{\ButtonOn}%
      \opt{MPIO_HD200_PAD,MPIO_HD300_PAD,SAMSUNG_YH92X_PAD,SAMSUNG_YH820_PAD}%
          {Long \ButtonPlay}%
      \opt{IPOD_4G_PAD}{\ButtonMenu{} / \ButtonSelect}%
      \opt{IPOD_3G_PAD}{\ButtonMenu{} / \ButtonPlay}%
      \opt{IAUDIO_X5_PAD,IRIVER_H10_PAD,SANSA_E200_PAD,SANSA_C200_PAD,ONDA_VX777_PAD%
          ,GIGABEAT_PAD,MROBE100_PAD,GIGABEAT_S_PAD,sansaAMS,PBELL_VIBE500_PAD%
          ,SANSA_FUZEPLUS_PAD,XDUOO_X3_PAD,AIGO_EROSQ_PAD%
          }{\ButtonPower}%
      \opt{COWON_D2_PAD} {\ButtonPower{}, then \ButtonHold}%
      \opt{ONDA_VX777_PAD} {\ButtonPower{}}%
      \opt{AGPTEK_ROCKER_PAD}{\ButtonPower{}}%
          &
      \opt{HAVEREMOTEKEYMAP}{
          \opt{IRIVER_RC_H100_PAD}{\ButtonRCOn}%
          \opt{IAUDIO_RC_PAD}{\ButtonRCPlay}
          &}

      Start Rockbox
          \\

      \opt{IRIVER_H100_PAD,IRIVER_H300_PAD}{Long \ButtonOff}%
      \opt{MPIO_HD200_PAD,MPIO_HD300_PAD,SAMSUNG_YH92X_PAD,SAMSUNG_YH820_PAD}%
          {Long \ButtonPlay}%
      \opt{IPOD_4G_PAD,IPOD_3G_PAD}{Long \ButtonPlay}%
      \opt{IAUDIO_X5_PAD,IRIVER_H10_PAD,SANSA_E200_PAD,SANSA_C200_PAD%
          ,GIGABEAT_PAD,MROBE100_PAD,GIGABEAT_S_PAD,sansaAMS,COWON_D2_PAD%
          ,PBELL_VIBE500_PAD,ONDA_VX777_PAD,SANSA_FUZEPLUS_PAD,XDUOO_X3_PAD,AIGO_EROSQ_PAD%
          }{Long \ButtonPower}%
      \opt{AGPTEK_ROCKER_PAD}{Long \ButtonPower{}}%
          &
      \opt{HAVEREMOTEKEYMAP}{
          \opt{IRIVER_RC_H100_PAD}{Long \ButtonRCStop}%
          \opt{IAUDIO_RC_PAD}{Long \ButtonRCPlay}
          &}

      Shutdown Rockbox
          \\
    \end{btnmap}

\label{ref:Safeshutdown}On shutdown, Rockbox automatically saves its settings.

\opt{IRIVER_H100_PAD,IRIVER_H300_PAD,IAUDIO_X5_PAD,SANSA_E200_PAD%
  ,SANSA_C200_PAD,IRIVER_H10_PAD,IPOD_4G_PAD,GIGABEAT_PAD}{%
  If you have problems with your settings, such as accidentally having
  set the colours to black on black, they can be reset at boot time.  See
  the Reset Settings in \reference{ref:manage_settings_menu} for details.
}%

\opt{GIGABEAT_PAD,IPOD_4G_PAD,SANSA_E200_PAD%
,SANSA_C200_PAD,IAUDIO_X5_PAD,IAUDIO_M5_PAD,IPOD_3G_PAD}{%
  In the unlikely event of a software failure, hardware poweroff or reset can be
  performed by holding down
  \opt{GIGABEAT_PAD}{the battery switch}\opt{IPOD_4G_PAD}
  {\ButtonMenu{} and \ButtonSelect{} simultaneously}%
  \opt{IPOD_3G_PAD}{\ButtonMenu{} and \ButtonPlay{} simultaneously}%
  \opt{SANSA_E200_PAD,SANSA_C200_PAD,IAUDIO_X5_PAD,IAUDIO_M5_PAD}
  {\ButtonPower} until the \dap{} shuts off or reboots.
}%
\opt{IRIVER_H100_PAD,IRIVER_H300_PAD,IAUDIO_M3_PAD,IRIVER_H10_PAD,MROBE100_PAD
    ,PBELL_VIBE500_PAD,MPIO_HD200_PAD,MPIO_HD300_PAD,SAMSUNG_YH92X_PAD%
    ,SAMSUNG_YH820_PAD,XDUOO_X3_PAD}{%
  In the unlikely event of a software failure, a hardware reset can be
  performed by inserting a paperclip gently into the Reset hole.
}%

\nopt{gigabeatf,iaudiom3,iaudiom5,iaudiox5}
  {
  \subsection{Starting the original firmware}
  \label{ref:Dualboot}
  \opt{ipod4g,ipodcolor,ipodvideo,ipodnano,ipodnano2g,ipodmini}
    {
    Rockbox has a dual-boot feature. To boot into the original firmware, shut
    down the device as described above. Turn on the \ButtonHold{} switch
    immediately after turning the player on. The Apple logo will
    display for a few seconds as Rockbox loads the original firmware.

    You can also load the original firmware by shutting down the device,
    then clicking the \ButtonHold{} switch on and connecting the iPod
    to your computer.

    Regardless of which method you use to boot to the original firmware, you can
    return to Rockbox by pressing and holding \ButtonMenu{} and \ButtonSelect{}
    simultaneously until the player hard resets.
    }

  \opt{ipod1g2g,ipod3g}
    {
    Rockbox has a dual-boot feature. To boot into the original firmware, shut
    down the device as described above. Turn on the \ButtonHold{} switch
    immediately after turning the player on. The Apple logo will
    display for a few seconds as Rockbox loads the original firmware.

    You can also load the original firmware by shutting down the device,
    then clicking the \ButtonHold{} switch on and connecting the iPod
    to your computer.

    Regardless of which method you use to boot to the original firmware, you can
    return to Rockbox by pressing and holding \ButtonMenu{} and \ButtonPlay{}
    simultaneously until the player hard resets.
    }

  \opt{iriverh100,iriverh300}
    {
    Rockbox has a dual-boot feature. To boot into the original firmware,
    when the \dap{} is turned off, press and hold the \ButtonRec{} button,
    and then press the \ButtonOn{} button.
    }
  \opt{fuzeplus}
    {
    Rockbox has a dual-boot feature. To boot into the original firmware,
    when the \dap{} is turned off, press and hold the \ButtonVolDown{} button,
    and then press and hold the \ButtonPower{} button while keeping the
    \ButtonVolDown{} button pressed. After 5 to 10 seconds the original
    firmware should boot.

    It is also possible to connect your \dap{} to your computer using the
    original firmware. To do so you may press and hold the \ButtonVolDown{}
    button and connect your device to the computer while keeping the
    \ButtonVolDown{} button pressed. After 5 to 10 seconds the original
    firmware should boot into USB mode.
    }
  \opt{mpiohd200,mpiohd300}
    {
    Rockbox has a dual-boot feature. To boot into the original firmware,
    when the \dap{} is turned off, press and hold the \ButtonRec{} button,
    and then press the \ButtonPlay{} button. This will bring you to the
    short menu where you can choose among: Boot Rockbox, Boot MPIO firmware
    and Shutdown. Select the option you need with \ButtonRew{} and \ButtonFF{}
    and confirm with long \ButtonPlay{}.
    }
  \opt{iriverh10,iriverh10_5gb}
    {
    Rockbox has a dual-boot feature. It loads the original firmware from
    the file \fname{/System/OF.mi4}. To boot into the original firmware,
    press and hold the \ButtonLeft{} button while turning on the player.
    \note{The iriver firmware does not shut down properly when you turn it off,
    it only goes to sleep. To get back into Rockbox when exiting from the
    iriver firmware, you will need to reset the player by \opt{iriverh10}{%
    inserting a pin in the reset hole}\opt{iriverh10_5gb}{removing and
    reinserting the battery}.}
    }

  \opt{sansa,sansaAMS}
    {
    Rockbox has a dual-boot feature. To boot into the original firmware,
    press and hold the \ButtonLeft{} button while turning on the player.
    }

  \opt{clipv2,fuzev2,clipplus}
    {
        \note{Rockbox does not boot into the original firmware when powered by
        a USB connection. Older versions of Rockbox do not provide USB support.
        If you have such a version installed you need to manually boot into the
        original firmware for data transfer via USB.}
    }

  \opt{mrobe100}
    {
    Rockbox has a dual-boot feature. It loads the original firmware from
    the file \fname{/System/OF.mi4}. To boot into the original firmware,
    when the \dap{} is turned off, press the \ButtonPower{} button once and then
    a second time when the m:robe bootlogo (the headphone) appears. Hold the
    \ButtonPower{} button until you see the ``Loading original firmware...''
    message on the screen.
    }

  \opt{gigabeats}
    {
    Rockbox has a dual-boot feature. To boot into the original firmware,
    turn the \ButtonHold{} switch on just after turning on the \dap{}.
    To return to Rockbox, shutdown the \dap{}, then turn the battery switch
    on the bottom off then on again. Rockbox should now start.
    }

  \opt{cowond2}
    {
    Use \ButtonPower{} to boot the original \playerman{} firmware.
    }

  \opt{vibe500}
    {
    Rockbox has a dual-boot feature where it is possible to load the original firmware from
    the file \fname{/System/OF.mi4}. To boot into the original firmware press and release
    \ButtonPower{} and then immediately after the backlight turns on, press the \ButtonOK{}
    button and keep it pressed until the original firmware starts.
    }

  \opt{samsungyh}
    {
    Rockbox has a dual-boot feature. It loads the original firmware from
    the file \fname{/System/OF.mi4}. To boot into the original firmware, press and hold
    for awhile the \ButtonPlay{} button and then immediately after the Samsung logo appears,
    press the \ButtonLeft{} button and keep it pressed until the original firmware starts.
    }

  \opt{ondavx777}
    {
    Rockbox has a dual-boot feature where it is possible to load the original firmware from
    the file \fname{/SD/ccpmp.bin}. To boot into the original firmware press and release
    \ButtonPower{} immediately after the Rockbox Logo appear on the screen.
    }

  \opt{xduoox3}
    {
    Rockbox has a dual-boot feature. To boot into the original firmware,
    when the \dap{} is turned off, set the \ButtonLock{} switch to locked,
    and then press the \ButtonPower{} button.
    }

  \opt{fiiom3k,shanlingq1,erosqnative}
    {
    Rockbox has a dual-boot feature. To boot into the original firmware,
    hold \ActionBootOFPlayer{} when powering on the \dap{}.

    \nopt{erosqnative}{
      You can trigger a normal \playerman{} firmware update by holding
      \ActionBootOFRecovery{} when powering on the \dap{}.
      \warn{Updating the original firmware will \textbf{erase} the Rockbox
      bootloader.}
    }

    \subsection{Entering the recovery menu}
    You can access the Rockbox bootloader's ``recovery menu'' by holding
    \ActionBootRecoveryMenu{}. This menu can be used to connect your \dap{}
    over USB to transfer files, update the Rockbox bootloader, or revert to a
    bootloader you've previously backed up.
    }

  }
\subsection{Putting music on your \dap{}}

\opt{usb_hid}{
\note{Due to a bug in some OS X versions, the \dap{} can not be mounted, unless
    the USB HID feature is disabled. See \reference{ref:USB_HID} for more
    information.\newline
}
}

With the \dap{} connected to the computer as an MSC/UMS device (like a
USB Drive), music files can be put on the player via any standard file
transfer method that you would use to copy files between drives (e.g. Drag-and-Drop).
Files may be placed wherever you like on the \dap{}, but it is strongly
suggested \emph{NOT} to put them in the \fname{/.rockbox} folder and instead
put them in any other folder, e.g. \fname{/}, \fname{/music} or \fname{/audio}.
The default directory structure that is assumed by some parts of Rockbox
\opt{albumart}{%
    (album art searching, and missing-tag fallback in some WPSes) uses the
    parent directory of a song as the Album name, and the parent directory of
    that folder as the Artist name. WPSes may display information incorrectly if
    your files are not properly tagged, and you have your music organized in a
    way different than they assume when attempting to guess the Artist and Album
    names from your filetree. See \reference{ref:album_art} for the requirements
    for Album Art to work properly.
}%
\nopt{albumart}{%
    (missing-tag fallback in some WPSes) uses the parent directory of a song
    as the Album name, and the parent directory of that folder as the Artist
    name. WPSes may display
    information incorrectly if your files are not properly tagged, and you have
    your music organized in a way different than they assume when attempting to
    guess the Artist and Album names from your filetree.
}%
    See \reference{ref:Supportedaudioformats} for a list of supported audio
    formats.

\subsection{The first contact}

After you have first started the \dap{}, you'll be presented by the
\setting{Main Menu}. From this menu you can reach every function of Rockbox,
for more information (see \reference{ref:main_menu}). To browse the files
on your \dap{}, select \setting{Files} (see \reference{ref:file_browser}), and to
browse in a view that is based on the meta-data\footnote{ID3 Tags, Vorbis
comments, etc.} of your audio files, select \setting{Database} (see
\reference{ref:database}).

\subsection{Basic controls}
When browsing files and moving through menus you usually get a list view
presented. The navigation in these lists are usually the same and should be
pretty intuitive.
In the tree view use \ActionStdNext{} and \ActionStdPrev{} to move around
the selection. Use \ActionStdOk{} to select an item. \opt{wheel_acceleration}{
Note that the scroll speed is accelerating the faster you rotate the wheel.}
When browsing the file system selecting an audio file plays it. The view
switches to the ``While playing screen'', usually abbreviated as ``WPS'' (see
\reference{ref:WPS}. The dynamic playlist gets replaced with the contents of
the current directory. This way you can easily treat directories as playlists.
The created dynamic playlist can be extended or modified while playing. This is
also known as ``on-the-fly playlist''.
To go back to the \setting{File Browser} stop the playback with the
\ActionWpsStop{} button or return to the file browser while keeping playback
running using \ActionWpsBrowse{}.
In list views you can go back one step with \ActionTreeParentDirectory.

\subsection{Basic concepts}
\subsubsection{Playlists}
Rockbox is playlist oriented. This means that every time you play an audio file,
a so-called ``dynamic playlist'' is generated, unless you play a saved
playlist. You can modify the dynamic playlist while playing and also save
it to a file. If you do not want to use playlists you can simply play your
files directory based.
Playlists are covered in detail in \reference{ref:working_with_playlists}.

\subsubsection{Menu}
From the menu you can customise Rockbox. Rockbox itself is very customisable.
Also there are some special menus for quick access to frequently used
functions.

\subsubsection{Context Menu}
Some views, especially the file browser and the WPS have a context menu.
From the file browser this can be accessed with \ActionStdContext{}.
The contents of the context menu vary, depending on the situation it gets
called. The context menu itself presents you with some operations you can
perform with the currently highlighted file. In the file browser this is
the file (or directory) that is highlighted by the cursor. From the WPS this is
the currently playing file. Also there are some actions that do not apply
to the current file but refer to the screen from which the context menu
gets called. One example is the playback menu, which can be called using
the context menu from within the WPS.

\section{Customising Rockbox}
Rockbox' User Interface can be customised using ``Themes''. Themes usually
only affect the visual appearance, but an advanced user can create a theme
that also changes various other settings like file view, LCD settings and
all other settings that can be modified using \fname{.cfg} files. This topic
is discussed in more detail in \reference{ref:manage_settings}.
The Rockbox distribution comes with some themes that should look nice on
your \dap{}.

\note{Some of the themes shipped with Rockbox need additional
fonts from the fonts package, so make sure you installed them.
Also, if you downloaded additional themes from the Internet make sure you
have the needed fonts installed as otherwise the theme may not display
properly.}

  \opt{usb_power}{
    \section{USB Charging}
    Your \dap{} will automatically charge when connected to USB. By default
    Rockbox will connect in mass storage mode to transfer files, but you can
    prevent this by holding down any button while plugging in the USB cable,
    or by changing the \setting{USB Mode} setting to \setting{Charge Only}.
    \nopt{fuzeplus}{
    \note{Be aware that holding a button may still perform its normal function,
    so it is recommended to use a button without harmful side effects, such as
    \ActionStdUsbCharge{}.}
    }
  }

% $Id$ %
\chapter{Browsing and playing}
\section{\label{ref:file_browser}File Browser}
\screenshot{rockbox_interface/images/ss-file-browser}{The file browser}{}
Rockbox lets you browse your music in either of two ways. The 
\setting{File Browser} lets you navigate through the files and directories on 
your \dap, entering directories and executing the default action on each file.
To help differentiate files, each file format is displayed with an icon. 

The \setting{Database Browser}, on the other hand, allows you to navigate 
through the music on your player using categories like album, artist, genre,
etc.

You can select whether to browse using the \setting{File Browser} or the
\setting{Database Browser} by selecting either \setting{Files} or
\setting{Database} in the \setting{Main Menu}.
If you choose the \setting{File Browser}, the \setting{Show Files} setting
lets you select what types of files you wish to view. See
\reference{ref:ShowFiles} for more information on the \setting{Show Files}
setting.

\note{The \setting{File Browser} allows you to manipulate your files in ways
that are not available within the \setting{Database Browser}. Read more about
\setting{Database} in \reference{ref:database}. The remainder of this section
deals with the \setting{File Browser}.}

\opt{iriverh10,iriverh10_5gb}{\note{
If your \dap{} is a MTP model, the Music directory where all your music is stored
may be hidden in the \setting{File Browser}. This may be fixed by either
changing its properties (on a computer) to not hidden, or by changing
the \setting{Show Files} setting to all.
}}

\subsection{\label{ref:controls}File Browser Controls}
\begin{btnmap}
      \ActionStdPrev{}/\ActionStdNext{}
      \opt{HAVEREMOTEKEYMAP}{& \ActionRCStdPrev{}/\ActionRCStdNext{}}
         & Go to previous/next item in list. If you are on the first/last 
           entry, the cursor will wrap to the last/first entry.\\
      %
      \opt{IRIVER_H100_PAD,IRIVER_H300_PAD}
        {
          \ButtonOn+\ButtonUp{}/ \ButtonDown
          \opt{HAVEREMOTEKEYMAP}{&
            \opt{IRIVER_RC_H100_PAD}{\ButtonRCSource{}/ \ButtonRCBitrate}
          }
          & Move one page up/down in the list.\\
        }
      \opt{IRIVER_H10_PAD}
        {
          \ButtonRew{}/ \ButtonFF
          & Move one page up/down in the list.\\
        }
      %
      \ActionTreeParentDirectory
      \opt{HAVEREMOTEKEYMAP}{& \ActionRCTreeParentDirectory}
      & Go to the parent directory.\\
      %
      \ActionTreeEnter
      \opt{HAVEREMOTEKEYMAP}{& \ActionRCTreeEnter}
      & Execute the default action on the selected file or enter a
        directory.\\
      %
      \ActionTreeWps 
      \opt{HAVEREMOTEKEYMAP}{& \ActionRCTreeWps}
         & If there is an audio file playing, return to the
         \setting{While Playing Screen} (WPS) without stopping playback.\\
      %
      \nopt{player,SANSA_C200_PAD}%
        {%
          \ActionTreeStop 
          \opt{HAVEREMOTEKEYMAP}{& \ActionRCTreeStop}
          & Stop audio playback.\\%
        }%
      %
      \ActionStdContext{}
      \opt{HAVEREMOTEKEYMAP}{& \ActionRCStdContext}
      & Enter the \setting{Context Menu}.\\
      %
      \ActionStdMenu{}
      \opt{HAVEREMOTEKEYMAP}{& \ActionRCStdMenu}
      & Enter the \setting{Main Menu}.\\
      %
      \opt{quickscreen}{
        \ActionStdQuickScreen
        \opt{HAVEREMOTEKEYMAP}{& \ActionRCStdQuickScreen}
        & Switch to the \setting{Quick Screen}
        (see \reference{ref:QuickScreen}). \\
      }
      %
      \opt{SANSA_E200_PAD}{
        \ActionStdRec & Switch to the \setting{Recording Screen}.\\
      %
      }
      \nopt{touchscreen}{\opt{hotkey}{
        \ActionTreeHotkey
            &
        \opt{HAVEREMOTEKEYMAP}{
            &}
        Activate the \setting{Hotkey} function
        (see \reference{ref:Hotkeys}).
            \\
      }}
\end{btnmap}

\subsection{\label{ref:Contextmenu}\label{ref:PartIISectionFM}Context Menu}
\screenshot{rockbox_interface/images/ss-context-menu}{The Context Menu}{}

The \setting{Context Menu} allows you to perform certain operations on files or 
directories.  To access the \setting{Context Menu}, position the selector over a file 
or directory and access the context menu with \ActionStdContext{}.\\

\note{The \setting{Context Menu} is a context sensitive menu.  If the 
\setting{Context Menu} is invoked on a file, it will display options available 
for files.  If the \setting{Context Menu} is invoked on a directory, 
it will display options for directories.\\}

The \setting{Context Menu} contains the following options (unless otherwise noted, 
each option pertains both to files and directories):

\begin{description}
\item [Current Playlist.]
  Enters the \setting{Current Playlist Submenu} (see \reference{ref:currentplaylist_submenu}).
\item [Playlist Catalogue.]
  Enters the \setting{Playlist Catalogue Submenu} (see 
  \reference{ref:playlist_catalogue}).
\item [Rename.]
  This function lets the user modify the name of a file or directory.
\item [Cut.]
  Copies the name of the currently selected file or directory to the clipboard
  and marks it to be `cut'.
\item [Copy.]
  Copies the name of the currently selected file or directory to the clipboard
  and marks it to be `copied'.
\item [Paste.]
  Only visible if a file or directory name is on the clipboard. When selected
  it will move or copy the clipboard to the current directory.
\item [Delete.]
  Deletes the currently selected file. This option applies only to files, and
  not to directories. Rockbox will ask for confirmation before deleting a file.
  Press \ActionYesNoAccept{}
  to confirm deletion or any other key to cancel.
\item [Delete Directory.]
  Deletes the currently selected directory and all of the files and subdirectories
  it may contain. Deleted directories cannot be recovered. Use this feature with
  caution!
\opt{lcd_non-mono}{
\item [Set As Backdrop.]
  Set the selected \fname{bmp} file as background image. The bitmaps need to meet the
  conditions explained in \reference{ref:LoadingBackdrops}.
}
\item [Open with.]
  Runs a viewer plugin on the file. Normally, when a file is selected in Rockbox,
  Rockbox automatically detects the file type and runs the appropriate plugin.
  The \setting{Open With} function can be used to override the default action and
  select a viewer by hand.
  For example, this function can be used to view a text file
  even if the file has a non-standard extension (i.e., the file has an extension
  of something other than \fname{.txt}). See \reference{ref:Viewersplugins}
  for more details on viewers.
\item [Create Directory.]
  Create a new directory in the current directory on the disk.
\item [Properties.]
  Shows properties such as size and the time and date of the last modification
  for the selected file. If used on a directory, the number of files and
  subdirectories will be shown, as well as the total size.
\opt{recording}{
  \item [Set As Recording Directory.]
    Save recordings in the selected directory.
}
\item [\label{ref:StartFileBrowserHere}Start File Browser Here.]
  This option allows users to set the currently selected directory as the default
  start directory for the file browser. This option is not available for files.
  \note{If you have \setting{Auto-Change Directory} and
  \setting{Constrain Auto-Change} enabled, the directories returned will
  be constrained to the directory you have chosen here and those below it.
  See \reference{ref:ConstrainAutoChange}}
\item [Add to Shortcuts.]
  Adds a link to the selected item in the \fname{shortcuts.link} file.
  If the file does not already exist it will be created in the root directory.
  Note that if you create a shortcut to a file, Rockbox will not open it upon
  selecting, but simply bring you to its location in the \setting{File Browser}.
\end{description}

\subsection{\label{sec:virtual_keyboard}Virtual Keyboard}
\screenshot{rockbox_interface/images/ss-virtual-keyboard}{The virtual keyboard}{}
This is the virtual keyboard that is used when entering text in Rockbox, for 
example when renaming a file or creating a new directory.
The virtual keyboard can be easily changed by making a text file
with the required layout. More information on how to achieve this can be found
on the Rockbox website at \wikilink{LoadableKeyboardLayouts}.

\opt{morse_input}{
  Also you can switch to Morse code input mode by changing the
  \setting{Use Morse Code Input} setting%
  \opt{IRIVER_H100_PAD,IRIVER_H300_PAD,IPOD_4G_PAD,IPOD_3G_PAD,IRIVER_H10_PAD%
      ,GIGABEAT_PAD,GIGABEAT_S_PAD,MROBE100_PAD,SANSA_E200_PAD,PBELL_VIBE500_PAD%
      ,SANSA_FUZEPLUS_PAD,SAMSUNG_YH92X_PAD,SAMSUNG_YH820_PAD}
    { or by pressing \ActionKbdMorseInput{} in the virtual keyboard}%
  .}

% no "Actions" yet in the Player's virtual keyboard

\note{When the cursor is on the input line, \ActionKbdSelect{} deletes the preceding character}

\begin{btnmap}
    \opt{IRIVER_H100_PAD,IRIVER_H300_PAD,GIGABEAT_PAD,GIGABEAT_S_PAD%
        ,MROBE100_PAD,SANSA_E200_PAD,SANSA_FUZE_PAD,SANSA_C200_PAD,SANSA_FUZEPLUS_PAD%
        ,SAMSUNG_YH820_PAD}{
        \ActionKbdCursorLeft{} / \ActionKbdCursorRight
            &
        \opt{HAVEREMOTEKEYMAP}{\ActionRCKbdCursorLeft{} / \ActionRCKbdCursorRight
            &}
        Move the line cursor within the text line.
            \\
        %
        \ActionKbdBackSpace
            &
        \opt{HAVEREMOTEKEYMAP}{
            &}
        Delete the character before the line cursor.
            \\
    }%
    \ActionKbdLeft{} / \ActionKbdRight
        &
    \opt{HAVEREMOTEKEYMAP}{\ActionRCKbdLeft{} / \ActionRCKbdRight
        &}
    Move the cursor on the virtual keyboard.
    If you move out of the picker area, you get the previous/next page of
    characters (if there is more than one).
        \\
    %
    \ActionKbdUp{} / \ActionKbdDown
        &
    \opt{HAVEREMOTEKEYMAP}{\ActionRCKbdUp{} / \ActionRCKbdDown
        &}
    Move the cursor on the virtual keyboard.
    If you move out of the picker area you get to the line edit mode.
        \\
    %
    \nopt{IPOD_3G_PAD,IPOD_4G_PAD,IRIVER_H10_PAD,PBELL_VIBE500_PAD%
         ,SANSA_FUZEPLUS_PAD,SAMSUNG_YH92X_PAD,SAMSUNG_YH820_PAD}{
        \ActionKbdPageFlip
            &
        \opt{HAVEREMOTEKEYMAP}{\ActionRCKbdPageFlip
            &}
        Flip to the next page of characters (if there is more than one).
            \\
    }
    %
    \ActionKbdSelect
        &
    \opt{HAVEREMOTEKEYMAP}{\ActionRCKbdSelect
        &}
    Insert the selected keyboard letter at the current line cursor position.
        \\
    %
    \ActionKbdDone
        &
    \opt{HAVEREMOTEKEYMAP}{\ActionRCKbdDone
        &}
    Exit the virtual keyboard and save any changes.
        \\
    %
    \ActionKbdAbort
        &
    \opt{HAVEREMOTEKEYMAP}{\ActionRCKbdAbort
        &}
    Exit the virtual keyboard without saving any changes.
        \\
% to be done - create a separate section for morse imput and update the info
      \opt{morse_input}{
        \opt{IRIVER_H100_PAD,IRIVER_H300_PAD,GIGABEAT_PAD,GIGABEAT_S_PAD,MROBE100_PADD%
            ,SANSA_E200_PA,IPOD_4G_PAD,IPOD_3G_PAD,IRIVER_H10_PAD,PBELL_VIBE500_PAD%
            ,SAMSUNG_YH92X_PAD,SAMSUNG_YH820_PAD}{
          \ActionKbdMorseInput
          \opt{HAVEREMOTEKEYMAP}{& \ActionRCKbdMorseInput}
          & Toggle keyboard input mode and Morse code input mode. \\}
        %
        \ActionKbdMorseSelect
        \opt{HAVEREMOTEKEYMAP}{& \ActionRCKbdMorseSelect}
        & Tap to select a character in Morse code input mode. \\
      } 
\end{btnmap}

% $Id$ %
\section{\label{ref:database}Database}

\subsection{Introduction}
This chapter describes the Rockbox music database system. Using the information
contained in the tags (ID3v1, ID3v2, Vorbis Comments, Apev2, etc.) in your
audio files, Rockbox builds and maintains a database of the music
files on your player and allows you to browse them by Artist, Album, Genre, 
Song Name, etc.  The criteria the database uses to sort the songs can be completely
 customised. More information on how to achieve this can be found on the Rockbox
 website at \wikilink{DataBase}. 

\subsection{Initializing the Database}
The first time you use the database, Rockbox will scan your disk for audio files.
This can take quite a while depending on the number of files on your \dap{}.
This scan happens in the background, so you can choose to return to the
Main Menu and continue to listen to music.
If you shut down your player, the scan will continue next time you turn it on.
After the scan is finished you may be prompted to restart your \dap{} before
you can use the database.

\subsubsection{Ignoring Directories During Database Initialization}

You may have directories on your \dap{} whose contents should not be added
to the database. Placing a file named \fname{database.ignore} in a directory
will exclude the files in that directory and all its subdirectories from
scanning their tags and adding them to the database. This will speed up the
database initialization.

If a subdirectory of an `ignored' directory should still be scanned, place a
file named \fname{database.unignore} in it. The files in that directory and
its subdirectories will be scanned and added to the database.

\subsubsection{Issues During Database Commit}

You may have files on your \dap{} whose contents might not be displayed
correctly or even crash the database.
Placing a file named \fname{/.rockbox/database_commit.ignore}
will prevent the device from committing the database automatically
you can manually commit the database using the db_commit plugin in APPS
with logging

\subsection{\label{ref:databasemenu}The Database Menu}

\begin{description}
  \opt{tc_ramcache}{
  \item[Load To RAM]
    The database can either be kept on \disk{} (to save memory), or
    loaded into RAM (for fast browsing). Setting this to \setting{Yes} loads
    the database to RAM, allowing faster browsing and searching. Setting this
    option to \setting{No} keeps the database on the \disk{}, meaning slower 
    browsing but it does not use extra RAM and saves some battery on boot up. 
    
    \opt{HAVE_DISK_STORAGE}{
    \note{If you browse your music frequently using the database, you should
      load to RAM, as this will reduce the overall battery consumption because
      the disk will not need to spin on each search.}
    }
  }
  
\item[Auto Update]
  If \setting{Auto update} is set to \setting{on}, each time the \dap{}
  boots, the database will automatically be updated.

\item[Initialize Now]
  You can force Rockbox to rescan your disk for tagged files by
  using the \setting{Initialize Now} function in the \setting{Database
    Menu}.
  \warn{\setting{Initialize Now} removes all database files (removing
    runtimedb data also) and rebuilds the database from scratch.}

\item[Update Now]
  \setting{Update now} causes the database to detect new and deleted files
    \note{Unlike the \setting{Auto Update} function, \setting{Update Now}
      will update the database regardless of whether the \setting{Directory Cache}
      is enabled. Thus, an update using \setting{Update now} may take a long
      time.
  }
  Unlike \setting{Initialize Now}, the \setting{Update Now} function
  does not remove runtime database information.
  
\item[Gather Runtime Data]
  When enabled, rockbox will record how often and how long a track is being played, 
  when it was last played and its rating. This information can be displayed in
  the WPS and is used in the database browser to, for example, show the most played, 
  unplayed and most recently played tracks.
  
\item[Export Modifications]
  This allows for the runtime data to be exported to the file \\
  \fname{/.rockbox/database\_changelog.txt}, which backs up the runtime data in
  ASCII format. This is needed when database structures change, because new
  code cannot read old database code. But, all modifications
  exported to ASCII format should be readable by all database versions.
  
\item[Import Modifications.]
  Allows the \fname{/.rockbox/database\_changelog.txt} backup to be 
  conveniently loaded into the database. If \setting{Auto Update} is
  enabled this is performed automatically when the database is initialized.
  
\end{description}

\subsection{Using the Database}
Once the database has been initialized, you can browse your music 
by Artist, Album, Genre, Song Name, etc.  To use the database, go to the
 \setting{Main Menu} and select \setting{Database}.\\

\note{You may need to increase the value of the \setting{Max Entries in File
Browser} setting (\setting{Settings $\rightarrow$ General Settings
$\rightarrow$ System $\rightarrow$ Limits}) in order to view long lists of
tracks in the ID3 database browser.\\

There is no option to turn off database completely. If you do not want
to use it just do not do the initial build of the database and do not load it
to RAM.}%

\begin{table}
  \begin{rbtabular}{.75\textwidth}{XXX}%
  {\textbf{Tag}   & \textbf{Type}  & \textbf{Origin}}{}{}
  filename              & string    & system \\ 
  album                 & string    & id tag \\
  albumartist           & string    & id tag \\
  artist                & string    & id tag \\
  comment               & string    & id tag \\
  composer              & string    & id tag \\
  genre                 & string    & id tag \\
  grouping              & string    & id tag \\
  title                 & string    & id tag \\
  bitrate               & numeric   & id tag \\
  discnum               & numeric   & id tag \\
  year                  & numeric   & id tag \\
  tracknum              & numeric   & id tag/filename \\
  autoscore             & numeric   & runtime db \\
  lastplayed            & numeric   & runtime db \\
  playcount             & numeric   & runtime db \\
  Pm (play time -- min)  & numeric   & runtime db \\
  Ps (play time -- sec)  & numeric   & runtime db \\
  rating                & numeric   & runtime db \\
  commitid              & numeric   & system \\
  entryage              & numeric   & system \\
  length                & numeric   & system \\
  Lm (track len -- min)  & numeric   & system \\
  Ls (track len -- sec)  & numeric   & system \\
  \end{rbtabular}
\end{table}

% $Id$ %
\section{\label{ref:WPS}While Playing Screen}
The While Playing Screen (WPS) displays various pieces of information about the
currently playing audio file.
%
The appearance of the WPS can be configured using WPS configuration files.
The items shown depend on your configuration -- all items can be turned on
or off independently. Refer to \reference{ref:wps_tags} for details on how
to change the display of the WPS.
\begin{itemize}
\item Status bar: The Status bar shows Battery level, charger status,
  volume, play mode, repeat mode, shuffle mode\opt{rtc}{ and clock}.
  In contrast to all other items, the status bar is always at the top of
  the screen.
\item (Scrolling) path and filename of the current song.
\item The ID3 track name.
\item The ID3 album name.
\item The ID3 artist name.
\item Bit rate. VBR files display average bitrate and ``(avg)''
\item Elapsed and total time.
\item A slidebar progress meter representing where in the song you are.
\item Peak meter.
\end{itemize}
%

See \reference{ref:ConfiguringtheWPS} for details of customising
your WPS (While Playing Screen).


\subsection{\label{ref:WPS_Key_Controls}WPS Key Controls}

  \begin{btnmap}
      \ActionWpsVolUp{} / \ActionWpsVolDown
      \opt{HAVEREMOTEKEYMAP}{& \ActionRCWpsVolUp{} / \ActionRCWpsVolDown}
      & Volume up/down.\\
      %
      \ActionWpsSkipPrev
       \opt{HAVEREMOTEKEYMAP}{& \ActionRCWpsSkipPrev}
      & Go to beginning of track, or if pressed while in the
        first seconds of a track, go to the previous track.\\
      %
      \ActionWpsSeekBack
      \opt{HAVEREMOTEKEYMAP}{& \ActionRCWpsSeekBack}
      & Rewind in track.\\
      %
      \ActionWpsSkipNext
      \opt{HAVEREMOTEKEYMAP}{& \ActionRCWpsSkipNext}
      & Go to the next track.\\
      %
      \ActionWpsSeekFwd
      \opt{HAVEREMOTEKEYMAP}{& \ActionRCWpsSeekFwd}
      & Fast forward in track.\\
      %
      \ActionWpsPlay
      \opt{HAVEREMOTEKEYMAP}{& \ActionRCWpsPlay}
      & Toggle play/pause.\\
      %
      \ActionWpsStop
      \opt{HAVEREMOTEKEYMAP}{& \ActionRCWpsStop}
      & Stop playback.\\
      %
      \ActionWpsBrowse
      \opt{HAVEREMOTEKEYMAP}{& \ActionRCWpsBrowse}
      & Return to the \setting{File Browser} / \setting{Database}.\\
      %
      \ActionWpsContext
      \opt{HAVEREMOTEKEYMAP}{& \ActionRCWpsContext}
      & Enter \setting{WPS Context Menu}.\\
      %
      \ActionWpsMenu
      \opt{HAVEREMOTEKEYMAP}{& \ActionRCWpsMenu}
      & Enter \setting{Main Menu}%
      .\\%
      %
      \opt{quickscreen}{%
        \ActionWpsQuickScreen
        \opt{HAVEREMOTEKEYMAP}{& \ActionRCWpsQuickScreen}
          & Switch to the \setting{Quick Screen}
          (see \reference{ref:QuickScreen}). \\}%
      %
      % software hold targets
      \nopt{hold_button}{%
          \opt{SANSA_CLIP_PAD}{\ButtonHome+\ButtonSelect}
          \opt{SANSA_FUZEPLUS_PAD}{\ButtonPower}
          & Key lock (software hold switch) on/off.\\
      }%
      % We explicitly list all the appropriate targets here and do no condition
      % on the 'pitchscreen' feature since some players have the feature but do
      % not have the button to go from the WPS to the pitch screen.
      \opt{IRIVER_H100_PAD,IRIVER_H300_PAD,IRIVER_H10_PAD,MROBE100_PAD%
          ,GIGABEAT_PAD,GIGABEAT_S_PAD,SANSA_E200_PAD,SANSA_C200_PAD,SANSA_FUZEPLUS_PAD}{%
        \ActionWpsPitchScreen
        \opt{HAVEREMOTEKEYMAP}{& \ActionRCWpsPitchScreen}
          & Show \setting{Pitch Screen} (see \reference{sec:pitchscreen}).\\%
      }%
      \opt{GIGABEAT_PAD,GIGABEAT_S_PAD,SANSA_CLIP_PAD,MROBE100_PAD,PBELL_VIBE500_PAD%
          ,SAMSUNG_YH92X_PAD,SAMSUNG_YH820_PAD,XDUOO_X3_PAD}{%
        \ActionWpsPlaylist
        \opt{HAVEREMOTEKEYMAP}{&}
          & Show current \setting{Playlist}.\\%
      }%
      \opt{IRIVER_H100_PAD,IRIVER_H300_PAD,IRIVER_H10_PAD%
          ,SANSA_E200_PAD,SANSA_C200_PAD,SANSA_FUZEPLUS_PAD}{%
        \ActionWpsIdThreeScreen
          \opt{HAVEREMOTEKEYMAP}{& \ActionRCWpsIdThreeScreen}
          & Enter \setting{ID3 Viewer}.\\%
      }%
      \opt{hotkey}{%
        \ActionWpsHotkey \opt{HAVEREMOTEKEYMAP}{& }
        & Activate the \setting{Hotkey} function (see \reference{ref:Hotkeys}).\\
      }
      \opt{ab_repeat_buttons}{%
         \ActionWpsAbSetBNextDir{} or }%
         % not all targets have the above action defined but the one below works on all
      Short \ActionWpsSkipNext{} + Long \ActionWpsSkipNext
      \opt{HAVEREMOTEKEYMAP}{
        &
          \opt{IRIVER_RC_H100_PAD}{\ActionRCWpsAbSetBNextDir{} or}
        Short \ActionRCWpsSkipNext{} + Long \ActionRCWpsSkipNext}
      & Skip to the next directory.\\
      %
      \opt{ab_repeat_buttons}{%
         \ActionWpsAbSetAPrevDir{} or }%
      Short \ActionWpsSkipPrev{} + Long \ActionWpsSkipPrev
      \opt{HAVEREMOTEKEYMAP}{
        &
          \opt{IRIVER_RC_H100_PAD}{\ActionRCWpsAbSetAPrevDir{} or}
        Short \ActionRCWpsSkipPrev{} + Long \ActionRCWpsSkipPrev}
      & Skip to the previous directory.\\
      %
      \opt{SANSA_E200_PAD,SANSA_C200_PAD,IRIVER_H100_PAD,IRIVER_H300_PAD}{
        \ActionStdRec
          \opt{HAVEREMOTEKEYMAP}{&}
          & Switch to the \setting{Recording Screen}.\\
      }%
  \end{btnmap}


\subsection{\label{ref:peak_meter}Peak Meter}
The peak meter can be displayed on the While Playing Screen and consists of
several indicators.
\opt{recording}{
  For a picture of the peak meter, please see the While
  Recording Screen in \reference{ref:while_recording_screen}.
}
\opt{ipodvideo}{
  \note{Especially the \playerman{} \playertype{}'s performance and battery runtime
   suffers when this feature is enabled. For this \dap{} it is highly recommended
   to not use peak meter.}
}

\begin{description}
\item [The bar:]
  This is the wide horizontal bar. It represents the current volume value.
\item [The peak indicator:]
  This is a little vertical line at the right end of the bar. It indicates
  the peak volume value that occurred recently.
\item [The clip indicator:]
  This is a little black block that is displayed at the very right of the
  scale when an overflow occurs. It usually does not show up during normal
  playback unless you play an audio file that is distorted heavily.
  \opt{recording}{
    If you encounter clipping while recording, your recording will sound distorted.
    You should lower the gain.
  }
  \note{Note that the clip detection is not very precise.
   Clipping might occur without being indicated.}
\item [The scale:]
  Between the indicators of the right and left channel there are little dots.
  These dots represent important volume values. In linear mode each dot is a
  10\% mark. In dBFS mode the dots represent the following values (from right
  to left): 0~dB, {}-3~dB, {}-6~dB, {}-9~dB, {}-12~dB, {}-18~dB, {}-24~dB, {}-30~dB,
  {}-40~dB, {}-50~dB, {}-60~dB.
\end{description}

\subsection{\label{sec:contextmenu}The WPS Context Menu}
Like the context menu for the \setting{File Browser}, the \setting{WPS Context Menu}
allows you quick access to some often used functions.

\subsubsection{Playlist}
The \setting{Playlist} submenu allows you to view, save, search, reshuffle,
and display the play time of the current playlist. These and other operations
are detailed in \reference{ref:working_with_playlists}. To change settings for
the \setting{Playlist Viewer} press \ActionStdContext{} while viewing the
current playlist to bring up the \setting{Playlist Viewer Menu}. In this
menu, you can find the \setting{Playlist Viewer Settings}.

\paragraph{Playlist Viewer Settings}
  \begin{description}
    \item[Show Icons.] This toggles display of the icon for the currently
    selected playlist entry and the icon for moving a playlist entry
    \item[Show Indices.] This toggles display of the line numbering for
       the playlist
    \item[Track Display.] This toggles between filename only and full path
       for playlist entries
  \end{description}


\subsubsection{Playlist catalogue}
  \begin{description}
    \item [Add to playlist.] Adds the currently playing file to a playlist.
    Select the playlist you want the file to be added to and it will get
    appended to that playlist.
    \item [Add to new playlist.] Similar to the previous entry this will
    add the currently playing track to a playlist. You need to enter a name
    for the new playlist first.
  \end{description}

\subsubsection{Sound Settings}
This is a shortcut to the \setting{Sound Settings Menu}, where you can configure volume,
bass, treble, and other settings affecting the sound of your music.
See \reference{ref:configure_rockbox_sound} for more information.

\subsubsection{Playback Settings}
This is a shortcut to the \setting{Playback Settings Menu}, where you can configure shuffle,
repeat, party mode, skip length and other settings affecting the playback of your music.

\subsubsection{Rating}
The menu entry is only shown if \setting{Gather Runtime Information} is
enabled. It allows the assignment of a personal rating value (0 -- 10)
to a track which can be displayed in the WPS and used in the Database
browser. The value wraps at 10.

\subsubsection{Bookmarks}
This allows you to create a bookmark in the currently-playing track.

\subsubsection{\label{ref:trackinfoviewer}Show Track Info}
\screenshot{rockbox_interface/images/ss-id3-viewer}{The track info viewer}{}
This screen is accessible from the WPS screen, and provides a detailed view of
all the identity information about the current track. This info is known as
meta data and is stored in audio file formats to keep information on artist,
album etc. To access this screen, %
\opt{IRIVER_H100_PAD,IRIVER_H300_PAD,IRIVER_H10_PAD,%
      SANSA_C200_PAD,SANSA_E200_PAD,SANSA_FUZE_PAD,SANSA_FUZEPLUS_PAD}{
  press \ActionWpsIdThreeScreen. }%
\opt{IPOD_4G_PAD,IPOD_3G_PAD,IAUDIO_X5_PAD,IAUDIO_M3_PAD,FIIO_M3K_PAD,%
      GIGABEAT_PAD,GIGABEAT_S_PAD,MROBE100_PAD,SANSA_CLIP_PAD,PBELL_VIBE500_PAD,%
      MPIO_HD200_PAD,MPIO_HD300_PAD,SAMSUNG_YH92X_PAD,SAMSUNG_YH820_PAD,XDUOO_X3_PAD}%
      {press \ActionWpsContext{} to access the
      \setting{WPS Context Menu} and select \setting{Show Track Info}. }

\subsubsection{Open With...}
This \setting{Open With} function is the same as the \setting{Open With}
function in the file browser's \setting{Context Menu}.

\subsubsection{Delete}
Delete the currently playing file. The file will be deleted but the playback
of the file will not stop immediately. Instead, the part of the file that
has already been buffered (i.e. read into the \daps\ memory) will be played.
This may even be the whole track.

\opt{pitchscreen}{
  \subsubsection{\label{sec:pitchscreen}Pitch}

  The \setting{Pitch Screen} allows you to change the rate of playback
  (i.e. the playback speed and at the same time the pitch) of your
  \dap.  The rate value can be adjusted
  between 50\% and 200\%. 50\% means half the normal playback speed and a
  pitch that is an octave lower than the normal pitch. 200\% means double
  playback speed and a pitch that is an octave higher than the normal pitch.

  The rate can be changed in two modes: procentual and semitone.
  Initially, procentual mode is active.

    If you've enabled the \setting{Timestretch} option in
    \setting{Sound Settings} and have since rebooted, you can also use
    timestretch mode. This allows you to change the playback speed
    without affecting the pitch, and vice versa.

    In timestretch mode there are separate displays for pitch and
    speed, and each can be altered independently.  Due to the
    limitations of the algorithm, speed is limited to be between 35\%
    and 250\% of the current pitch value.  Pitch must maintain the
    same ratio as well as remain between 50\% and 200\%.

  The value of the rate, pitch and speed
  is not persistent, i.e. after the \dap\ is turned on it will
  always be set to 100\%.  However, the rate, pitch and speed
  information will be stored in any bookmarks you may create
  (see \reference{ref:Bookmarkconfigactual}) and will be restored upon
  playing back those bookmarks.

  \begin{btnmap}
    \ActionPsToggleMode
    \opt{HAVEREMOTEKEYMAP}{& \ActionRCPsToggleMode}
    & Toggle pitch changing mode (cycle through all available modes).\\
    %
    \ActionPsIncSmall{} / \ActionPsDecSmall
    \opt{HAVEREMOTEKEYMAP}{& \ActionRCPsIncSmall{} / \ActionRCPsDecSmall}
    & Increase~/ Decrease pitch by 0.1\% (in procentual mode) or 0.1
      semitone (in semitone mode).\\
    %
    \nopt{PBELL_VIBE500_PAD}{ % there is no long scroll up or down because of slide
    \ActionPsIncBig{} / \ActionPsDecBig
    \opt{HAVEREMOTEKEYMAP}{& \ActionRCPsIncBig{} / \ActionRCPsDecBig}
    & Increase~/ Decrease pitch by 1\% (in procentual mode) or a semitone
      (in semitone mode).\\
    }
    %
    \ActionPsNudgeLeft{} / \ActionPsNudgeRight
    \opt{HAVEREMOTEKEYMAP}{& \ActionRCPsNudgeLeft{} / \ActionRCPsNudgeRight}
    & Temporarily change pitch by 2\% (beatmatch), or modify speed (in timestretch mode).\\
    %
    \ActionPsReset
    \opt{HAVEREMOTEKEYMAP}{& \ActionRCPsReset}
    & Reset pitch and speed to 100\%. \\
    %
    \ActionPsExit
    \opt{HAVEREMOTEKEYMAP}{& \ActionRCPsExit}
    & Leave the \setting{Pitch Screen}. \\
    %
  \end{btnmap}

}


%Include playlist section
% $Id$ %
\chapter{Plugins}\label{ref:plugins}
\opt{mpiohd200}{%
\fixme{The manual for MPIO HD200 is incomplete. Keymap definitions for plugins are missing.}\\

}
Plugins are programs that Rockbox can load and run. Only one plugin can
be loaded at a time. Plugins have exclusive control over the user interface.
This means you cannot switch back and forth between a plugin and Rockbox. When
a plugin is loaded, you need to exit it to return to the Rockbox interface.
Most plugins will not interfere with music playback but some of them will stop
playback while running. Plugins have the file extension \fname{.rock}. Most of
them can be started from \setting{Browse Plugins} in the \setting{Main Menu}.\\

Viewer plugins get started automatically by opening an associated file (i.e.
text files%
, chip8 games), or from the \setting{Open with} option on the \setting{Context Menu}.

\section{Games}
  See also the Chip{}-8 emulator in \reference{ref:Chip8emulator},
  Frotz in \reference{ref:Frotz},
  \opt{iriverh100,iaudiom5,lcd_color}
  {Rockboy in \reference{ref:Rockboy}}
  and ZXBox in \reference{ref:ZXBox}.

\subsection{2048}
%\screenshot{plugins/images/ss-2048}{2048}{fig:2048}

2048 is a simple, addictive puzzle game played by moving tiles in around on a 4x4 grid. Tiles slide as far as possible in the direction chosen by the player each turn until they are stopped by either another tile or the edge of the grid. If two tiles of the same number collide while moving, they merge into a tile with the total value of the two tiles that collided. The resulting tile cannot merge with another the same move. After each move, a tile with the value of 2 or 4 is created in an empty spot on the grid.

The game is won when a tile with a value of 2048 is created, and the player loses when there are no more possible moves.

\note{On players with a small screen tiles with a value greater than 1000 are shortened
to ``1k'', ``2k'' and so forth (\emph{k} is the abbreviation of \emph{kilo},
which -- in computer talk -- means a multiple of 1024).}

\begin{btnmap}
  \PluginUp, \PluginDown, \PluginLeft, \PluginRight
  \opt{HAVEREMOTEKEYMAP}{& \PluginRCUp, \PluginRCDown, \PluginRCLeft, \PluginRCRight}
  & Slide tiles\\

  \PluginCancel
  \opt{HAVEREMOTEKEYMAP}{& \PluginRCCancel}
  & Go to menu\\
\end{btnmap}


\subsection{Blackjack}
\screenshot{plugins/images/ss-blackjack}{Blackjack}{fig:blackjack}

Blackjack, a game played in casinos around the world, is now available
in the palm of your hand! The rules are simple: try to get as close to 21
without going over or simply beat out the dealer for the best hand.
Although this may not seem difficult, blackjack is a game renowned for the
strategy involved. This version includes the ability to split, buy insurance,
and double down. 

For the full set of rules to the game, and other fascinating information
visit\\
\url{http://www.blackjackinfo.com/blackjack-rules.php}

    \begin{btnmap}
    \opt{RECORDER_PAD,ONDIO_PAD,IRIVER_H100_PAD,IRIVER_H300_PAD,IAUDIO_X5_PAD%
        ,GIGABEAT_PAD,GIGABEAT_S_PAD,MROBE100_PAD,SANSA_C200_PAD,PBELL_VIBE500_PAD%
        ,SANSA_FUZEPLUS_PAD,SANSA_CLIP_PAD,SAMSUNG_YH92X_PAD,SAMSUNG_YH820_PAD}
      {\ButtonLeft{} / \ButtonRight{} / \ButtonUp{} / \ButtonDown}
    \opt{IPOD_4G_PAD,IPOD_3G_PAD,SANSA_E200_PAD,SANSA_FUZE_PAD}
      {\ButtonLeft{} / \ButtonRight{} / \ButtonScrollFwd{} / \ButtonScrollBack}
    \opt{IRIVER_H10_PAD}
      {\ButtonLeft{} / \ButtonRight{} / \ButtonScrollUp{} / \ButtonScrollDown}
    \opt{MPIO_HD300_PAD}
      {\ButtonRew{} / \ButtonFF{} / \ButtonScrollUp{} / \ButtonScrollDown}
    \opt{COWON_D2_PAD}{\TouchMidRight{} / \TouchMidLeft}
       \opt{HAVEREMOTEKEYMAP}{& }
      & Enter betting amount\\
    \opt{RECORDER_PAD,IRIVER_H10_PAD,GIGABEAT_S_PAD,SAMSUNG_YH92X_PAD%
        ,SAMSUNG_YH820_PAD}{\ButtonPlay}
    \opt{IRIVER_H100_PAD,IRIVER_H300_PAD}{\ButtonOn}
    \opt{IPOD_4G_PAD,IPOD_3G_PAD,IAUDIO_X5_PAD,GIGABEAT_PAD%
        ,SANSA_E200_PAD,SANSA_C200_PAD,SANSA_CLIP_PAD,MROBE100_PAD,SANSA_FUZE_PAD,SANSA_FUZEPLUS_PAD%
        }{\ButtonSelect}
    \opt{ONDIO_PAD}{\ButtonMenu}
    \opt{COWON_D2_PAD}{\TouchTopRight}
    \opt{PBELL_VIBE500_PAD}{\ButtonOK}
    \opt{MPIO_HD300_PAD}{\ButtonEnter}
       \opt{HAVEREMOTEKEYMAP}{& }
    & Hit (Draw new card) / Select\\
    \opt{RECORDER_PAD}{\ButtonFOne}
    \opt{IRIVER_H100_PAD,IRIVER_H300_PAD,IAUDIO_X5_PAD,SAMSUNG_YH92X_PAD%
        ,SAMSUNG_YH820_PAD}{\ButtonRec}
    \opt{IRIVER_H10_PAD}{\ButtonFF}
    \opt{ONDIO_PAD,IPOD_4G_PAD,IPOD_3G_PAD,SANSA_E200_PAD,SANSA_C200_PAD,SANSA_CLIP_PAD,SANSA_FUZE_PAD}{\ButtonRight}
    \opt{GIGABEAT_PAD,GIGABEAT_S_PAD}{\ButtonVolDown}
    \opt{MROBE100_PAD}{\ButtonDisplay}
    \opt{COWON_D2_PAD}{\TouchBottomLeft}
    \opt{SANSA_FUZEPLUS_PAD}{\ButtonBack}
    \opt{PBELL_VIBE500_PAD}{\ButtonCancel}
    \opt{MPIO_HD300_PAD}{\ButtonPlay}
       \opt{HAVEREMOTEKEYMAP}{& }
    & Stay (End hand)\\
    \opt{RECORDER_PAD}{\ButtonFTwo}
    \opt{IRIVER_H100_PAD,IRIVER_H300_PAD,GIGABEAT_S_PAD}{\ButtonSelect}
    \opt{IAUDIO_X5_PAD,SANSA_FUZEPLUS_PAD}{\ButtonPlay}
    \opt{IRIVER_H10_PAD}{\ButtonRew}
    \opt{GIGABEAT_PAD}{\ButtonA}
    \opt{ONDIO_PAD}{\ButtonUp}
    \opt{MROBE100_PAD}{\ButtonDown}
    \opt{IPOD_4G_PAD,IPOD_3G_PAD,SANSA_E200_PAD,SANSA_C200_PAD,SANSA_CLIP_PAD,SANSA_FUZE_PAD}{\ButtonLeft}
    \opt{COWON_D2_PAD}{\ButtonMinus}
    \opt{PBELL_VIBE500_PAD}{\ButtonMenu}
    \opt{MPIO_HD300_PAD}{\ButtonRec}
    \opt{SAMSUNG_YH92X_PAD,SAMSUNG_YH820_PAD}{\ButtonFF}
       \opt{HAVEREMOTEKEYMAP}{& }
    & Double down\\
    \opt{RECORDER_PAD,ONDIO_PAD,IRIVER_H100_PAD,IRIVER_H300_PAD}{\ButtonOff}
    \opt{IPOD_4G_PAD,IPOD_3G_PAD}{\ButtonMenu}
    \opt{IAUDIO_X5_PAD,IRIVER_H10_PAD,SANSA_E200_PAD,SANSA_C200_PAD,SANSA_CLIP_PAD%
        ,GIGABEAT_PAD,MROBE100_PAD,COWON_D2_PAD,SANSA_FUZEPLUS_PAD}{\ButtonPower}
    \opt{GIGABEAT_S_PAD}{\ButtonBack}
    \opt{SANSA_FUZE_PAD}{Long \ButtonHome}
    \opt{PBELL_VIBE500_PAD}{\ButtonRec}
    \opt{MPIO_HD300_PAD}{\ButtonMenu}
    \opt{SAMSUNG_YH92X_PAD,SAMSUNG_YH820_PAD}{\ButtonRew}
       \opt{HAVEREMOTEKEYMAP}{& }
        & Pause game and go to menu / Cancel\\
    \end{btnmap}


\opt{large_plugin_buffer}{\subsection{Boomshine}

This is a game coded in Lua that's a clone of \url{http://www.yvoschaap.com/chainrxn/}.
It is a rather basic game, but probably a good way to show off some of Lua's features
in Rockbox.
}

% $Id$
\subsection{BrickMania}
\screenshot{plugins/images/ss-brickmania}{BrickMania}%
{img:brickmania}
BrickMania is a clone of the classic game Breakout. The aim of the game is to
destroy all the bricks by hitting them with the ball once or more. Sometimes a
special item falls down when you destroy a brick. For a special item to take
effect, you must catch it with the paddle. Look out for the bad ones.\\

\subsubsection{Special items}
\begin{table}
    \begin{rbtabular}{.75\textwidth}{clX}%
        {\textbf{Displayed} & \textbf{Name} & \textbf{Description}}{}{}
    N & Normal & Returns paddle to normal.\\
    D & Die & Ball dies; lose a life.\\
    L & Life & Gain a life.\\
    F & Fire & Allows you to shoot bricks with paddle.\\
    G & Glue & Ball sticks to paddle each time it hits.\\
    B & Ball & Immediately fires another ball.\\
    FL & Flip & Flip left / right movement.\\
    \end{rbtabular}
\end{table}

    \begin{btnmap}
    \opt{IAUDIO_X5_PAD,IRIVER_H100_PAD,IRIVER_H300_PAD%
        ,SANSA_C200_PAD,SANSA_CLIP_PAD,GIGABEAT_PAD,GIGABEAT_S_PAD,MROBE100_PAD,IPOD_4G_PAD%
        ,IPOD_3G_PAD,SANSA_E200_PAD,IRIVER_H10_PAD,SANSA_FUZE_PAD,SANSA_FUZEPLUS_PAD%
        ,SAMSUNG_YH92X_PAD,SAMSUNG_YH820_PAD}
        {\ButtonLeft\ / \ButtonRight}
    \opt{SANSA_C200_PAD,SANSA_CLIP_PAD}{\\
        \ButtonVolDown\ / \ButtonVolUp}
    \opt{IPOD_4G_PAD,IPOD_3G_PAD,SANSA_E200_PAD,SANSA_FUZE_PAD}{
        \ButtonScrollBack\ / \ButtonScrollFwd}
    \opt{COWON_D2_PAD}{\ButtonMinus{} or \TouchBottomLeft{} / \ButtonPlus{} or \TouchBottomRight}
    \opt{PBELL_VIBE500_PAD}{\ButtonLeft{} or \ButtonMenu{} / \ButtonRight{} or \ButtonPlay}
    \opt{MPIO_HD300_PAD}{\ButtonRew{} / \ButtonFF{}}
       \opt{HAVEREMOTEKEYMAP}{& }
    & Moves the paddle\\
    \opt{IAUDIO_X5_PAD,SAMSUNG_YH92X_PAD,SAMSUNG_YH820_PAD}{\ButtonPlay\ / \ButtonUp}
    \opt{IRIVER_H100_PAD,IRIVER_H300_PAD,SANSA_C200_PAD,SANSA_CLIP_PAD,GIGABEAT_PAD%
      ,GIGABEAT_S_PAD,MROBE100_PAD}{\ButtonSelect\ / \ButtonUp}
    \opt{IPOD_4G_PAD,IPOD_3G_PAD,SANSA_E200_PAD,SANSA_FUZE_PAD,SANSA_FUZEPLUS_PAD%
      }{\ButtonSelect}
    \opt{IRIVER_H10_PAD}{\ButtonPlay\ / \ButtonScrollUp}
    \opt{COWON_D2_PAD}{\ButtonMenu{} or \TouchCenter}
    \opt{PBELL_VIBE500_PAD}{\ButtonOK{} or \ButtonUp}
    \opt{MPIO_HD300_PAD}{\ButtonEnter{}}
       \opt{HAVEREMOTEKEYMAP}{& }
    & Release the ball / Fire\\
    \opt{IRIVER_H100_PAD,IRIVER_H300_PAD}{\ButtonOff}
    \opt{IPOD_4G_PAD,IPOD_3G_PAD}{\ButtonMenu}
    \opt{IAUDIO_X5_PAD,IRIVER_H10_PAD,SANSA_E200_PAD,SANSA_C200_PAD,SANSA_CLIP_PAD%
        ,GIGABEAT_PAD,MROBE100_PAD,COWON_D2_PAD,SANSA_FUZEPLUS_PAD}{\ButtonPower}
    \opt{SANSA_FUZE_PAD}{Long \ButtonHome}
    \opt{GIGABEAT_S_PAD}{\ButtonBack}
    \opt{PBELL_VIBE500_PAD}{\ButtonRec}
    \opt{MPIO_HD300_PAD}{\ButtonMenu}
    \opt{SAMSUNG_YH92X_PAD,SAMSUNG_YH820_PAD}{\ButtonRew}
       \opt{HAVEREMOTEKEYMAP}{& 
          \opt{IRIVER_RC_H100_PAD}{\ButtonRCStop}
          }
    & Open menu / Quit\\
    \end{btnmap}


% $Id$ %
\subsection{Bubbles}
\screenshot{plugins/images/ss-bubbles}{Bubbles}{img:bubbles}
The goal of the game is to beat each level as quickly as possible by clearing
the board of all bubbles. Bubbles are removed from the board when a cluster of
three of more of the same type is formed. The game is over when any bubbles on
the board extend below the bottom line. To make things more difficult, the
entire board is shifted down every time a certain number of shots have been
fired. Points are awarded depending on how quickly the level was completed.

    \begin{btnmap}
    \opt{IRIVER_H10_PAD,SAMSUNG_YH92X_PAD,SAMSUNG_YH820_PAD}{\PluginSelect}
    \opt{IPOD_4G_PAD,IPOD_3G_PAD}{\ButtonPlay}
    \nopt{IRIVER_H10_PAD,SAMSUNG_YH92X_PAD,SAMSUNG_YH820_PAD,IPOD_4G_PAD,IPOD_3G_PAD}{\PluginUp}
       \opt{HAVEREMOTEKEYMAP}{& \PluginRCUp}
        & Pause game\\

    \nopt{scrollwheel}{\PluginLeft{} / \PluginRight}
    \opt{scrollwheel}{\PluginScrollFwd{} / \PluginScrollBack}
       \opt{HAVEREMOTEKEYMAP}{& \PluginRCLeft{} / \PluginRCRight}
        & Aim the bubble\\

    \opt{IRIVER_H10_PAD,SAMSUNG_YH92X_PAD,SAMSUNG_YH820_PAD}{\PluginUp}
    \nopt{IRIVER_H10_PAD,SAMSUNG_YH92X_PAD,SAMSUNG_YH820_PAD}{\PluginSelect}
       \opt{HAVEREMOTEKEYMAP}{& \PluginRCSelect}
        & Fire bubble\\

    \nopt{IRIVER_H10_PAD,SAMSUNG_YH92X_PAD,SAMSUNG_YH820_PAD,IPOD_4G_PAD,IPOD_3G_PAD}{\PluginCancel{} or \PluginExit}
    \opt{IPOD_4G_PAD,IPOD_3G_PAD}{\ButtonMenu}
       \opt{HAVEREMOTEKEYMAP}{& \PluginRCCancel}
        & Exit to menu\\
    \end{btnmap}


% $Id$ %
\subsection{Chessbox}
\screenshot{plugins/images/ss-chessbox}{Chessbox}{img:chessbox}
Chessbox is a one-person chess game with computer artificial intelligence. 
The chess engine is a port of GNU Chess 2 by John Stanback.

It also works as a PGN file viewer. Instead of executing the game from the
plugin menu, look for any file with \fname{.pgn} extension in the file browser
and execute it. Chessbox will show the list of matches included in the file
and allow you to select the one you want to watch. After that, you can scroll
back and forth through the moves of the game. If the menu is invoked while in
the viewer, the user is allowed to select a new match from the same file or
quit the game.

``Force play'' while the computer is thinking will cause it to make its move
immediately.  If done while it's your turn, the computer will move
for you and flip the board so that you are playing from the other side.  If you
want, you can force play an entire game and watch the artificial intelligence
 fight against itself.

When you quit the game the current state will be saved and restored when
you resume the game. The menu also allows the user to reload the last game
saved, save the current position and start a new game without having to quit
the game.

\opt{archosrecorder,archosfmrecorder,ondio}{
\note{This plugin will stop playback.}
}

\subsubsection{Keys}
    \begin{btnmap}
    \opt{IPOD_4G_PAD,IPOD_3G_PAD}{\ButtonMenu, \ButtonPlay, \ButtonLeft, \ButtonRight}
    \opt{MPIO_HD300_PAD}{\ButtonRew, \ButtonFF, \ButtonScrollUp, \ButtonScrollDown}
    \nopt{IPOD_4G_PAD,IPOD_3G_PAD,MPIO_HD300_PAD}{Direction keys}
       \opt{HAVEREMOTEKEYMAP}{& }
    & Move the cursor\\
    \opt{RECORDER_PAD,SAMSUNG_YH92X_PAD,SAMSUNG_YH820_PAD}{\ButtonPlay}
    \opt{ONDIO_PAD}{\ButtonMenu}
    \opt{IRIVER_H100_PAD,IRIVER_H300_PAD,IPOD_4G_PAD,IPOD_3G_PAD,IAUDIO_X5_PAD%
        ,SANSA_E200_PAD,SANSA_C200_PAD,SANSA_CLIP_PAD,GIGABEAT_PAD,GIGABEAT_S_PAD,MROBE100_PAD%
        ,SANSA_FUZE_PAD,SANSA_FUZEPLUS_PAD}
        {\ButtonSelect}
    \opt{IRIVER_H10_PAD}{\ButtonRew}
    \opt{COWON_D2_PAD}{\TouchCenter}
    \opt{PBELL_VIBE500_PAD}{\ButtonOK}
    \opt{MPIO_HD300_PAD}{\ButtonEnter}
       \opt{HAVEREMOTEKEYMAP}{& }
    & Pick up / Drop piece\\
    \opt{RECORDER_PAD}{\ButtonFOne}
    \opt{ONDIO_PAD}{\ButtonMenu+\ButtonOff}
    \opt{IRIVER_H100_PAD,IRIVER_H300_PAD}{\ButtonMode}
    \opt{IPOD_4G_PAD,IPOD_3G_PAD}{\ButtonSelect+\ButtonRight}
    \opt{IAUDIO_X5_PAD,SANSA_E200_PAD,SANSA_C200_PAD}{\ButtonRec}
    \opt{SANSA_CLIP_PAD}{\ButtonHome}
    \opt{SANSA_FUZE_PAD}{\ButtonSelect+\ButtonLeft}
    \opt{IRIVER_H10_PAD}{\ButtonFF}
    \opt{GIGABEAT_PAD,GIGABEAT_S_PAD}{\ButtonMenu}
    \opt{MROBE100_PAD}{\ButtonDisplay}
    \opt{COWON_D2_PAD}{\ButtonPlus}
    \opt{PBELL_VIBE500_PAD}{\ButtonCancel}
    \opt{MPIO_HD300_PAD}{\ButtonRec}
    \opt{SANSA_FUZEPLUS_PAD}{\ButtonBottomRight}
    \opt{SAMSUNG_YH92X_PAD,SAMSUNG_YH820_PAD}{\ButtonFF}
       \opt{HAVEREMOTEKEYMAP}{& }
    & Change level\\
    \opt{RECORDER_PAD,IRIVER_H100_PAD,IRIVER_H300_PAD}{\ButtonOn}
    \opt{ONDIO_PAD}{Long \ButtonMenu}
    \opt{IPOD_4G_PAD,IPOD_3G_PAD}{\ButtonSelect+\ButtonPlay}
    \opt{IAUDIO_X5_PAD,IRIVER_H10_PAD,GIGABEAT_S_PAD,MROBE100_PAD,PBELL_VIBE500_PAD%
        ,MPIO_HD300_PAD}{\ButtonPlay}
    \opt{SANSA_E200_PAD,SANSA_FUZE_PAD}{\ButtonSelect+\ButtonRight}
    \opt{SANSA_C200_PAD,SANSA_CLIP_PAD}{\ButtonVolUp}
    \opt{GIGABEAT_PAD}{\ButtonA}
    \opt{COWON_D2_PAD}{Long \TouchCenter}
    \opt{SANSA_FUZEPLUS_PAD}{Long \ButtonPlay}
    \opt{SAMSUNG_YH92X_PAD}{\ButtonRecOn{} or \ButtonRecOff}
    \opt{SAMSUNG_YH820_PAD}{\ButtonRec}
       \opt{HAVEREMOTEKEYMAP}{& }
    & Force play\\
    \opt{RECORDER_PAD,ONDIO_PAD,IRIVER_H100_PAD,IRIVER_H300_PAD}{\ButtonOff}
    \opt{IPOD_4G_PAD,IPOD_3G_PAD}{\ButtonSelect+\ButtonMenu}
    \opt{IAUDIO_X5_PAD,IRIVER_H10_PAD,SANSA_E200_PAD,SANSA_C200_PAD,SANSA_CLIP_PAD,GIGABEAT_PAD,GIGABEAT_S_PAD%
        ,MROBE100_PAD,SANSA_FUZEPLUS_PAD}{\ButtonPower}
    \opt{SANSA_FUZE_PAD}{Long \ButtonHome}
    \opt{COWON_D2_PAD,PBELL_VIBE500_PAD,MPIO_HD300_PAD}{\ButtonMenu}
    \opt{SAMSUNG_YH92X_PAD,SAMSUNG_YH820_PAD}{\ButtonRew}
       \opt{HAVEREMOTEKEYMAP}{& }
    & Show the menu\\
    \end{btnmap}


% $Id: chopper.tex 18466 2008-09-08 18:15:06Z rmenes $ %
\subsection{Chopper}
\screenshot{plugins/images/ss-chopper}{Chopper}{img:chopper}

  Navigate a cavernous maze without banging into walls, the
  ceiling, or the floor. How long can you fly your chopper?

    \begin{btnmap}
      \opt{RECORDER_PAD}{\ButtonPlay}%
      \opt{ONDIO_PAD}{\ButtonUp{} / \ButtonMenu}
      \opt{IRIVER_H10_PAD,SAMSUNG_YH92X_PAD}{\ButtonRight}
      \opt{IPOD_4G_PAD,IPOD_3G_PAD,SANSA_E200_PAD,SANSA_C200_PAD,SANSA_CLIP_PAD,MROBE100_PAD%
        ,GIGABEAT_PAD,SANSA_FUZE_PAD,SANSA_FUZEPLUS_PAD}
        {\ButtonSelect}
      \opt{IRIVER_H100_PAD,IRIVER_H300_PAD,IAUDIO_X5_PAD}{\ButtonSelect{} / \ButtonUp}
      \opt{GIGABEAT_S_PAD}{\ButtonSelect{} / \ButtonMenu}
      \opt{COWON_D2_PAD}{\ButtonPlus{} / \TouchBottomLeft}
      \opt{PBELL_VIBE500_PAD}{\ButtonPlay{} / \ButtonUp}
      \opt{MPIO_HD300_PAD}{\ButtonEnter}
       \opt{HAVEREMOTEKEYMAP}{& }
      & Make chopper fly\\

      \opt{RECORDER_PAD,ONDIO_PAD,IRIVER_H100_PAD,IRIVER_H300_PAD}{\ButtonOff}
      \opt{IAUDIO_X5_PAD,IRIVER_H10_PAD,MROBE100_PAD,SANSA_E200_PAD%
          ,SANSA_C200_PAD,SANSA_CLIP_PAD,COWON_D2_PAD,SANSA_FUZEPLUS_PAD}{\ButtonPower}
      \opt{SANSA_FUZE_PAD}{\ButtonHome}
      \opt{IPOD_4G_PAD,IPOD_3G_PAD,GIGABEAT_PAD,MPIO_HD300_PAD}{\ButtonMenu}
      \opt{GIGABEAT_S_PAD}{\ButtonBack}%
      \opt{PBELL_VIBE500_PAD}{\ButtonRec}
      \opt{SAMSUNG_YH92X_PAD}{\ButtonLeft}
       \opt{HAVEREMOTEKEYMAP}{& }
      & Enter menu\\
    \end{btnmap}


\opt{lcd_color}{\subsection{Clix}
\screenshot{plugins/images/ss-clix}{Clix}{img:clix}

The aim is to remove all blocks from the board. You can only
remove blocks, if at least two blocks with the same color have a direct connection.
The more blocks you remove per turn, the more points you get.

  \begin{btnmap}
    \nopt{touchscreen}{
    \opt{RECORDER_PAD,IRIVER_H10_PAD,ONDIO_PAD,IRIVER_H100_PAD,IRIVER_H300_PAD%
      ,IPOD_4G_PAD,IPOD_3G_PAD,SANSA_E200_PAD,SANSA_C200_PAD,SANSA_CLIP_PAD,GIGABEAT_PAD%
      ,MROBE100_PAD,IAUDIO_X5_PAD,GIGABEAT_S_PAD,SANSA_FUZE_PAD,PBELL_VIBE500_PAD%
      ,SANSA_FUZEPLUS_PAD,SAMSUNG_YH92X_PAD,SAMSUNG_YH820_PAD}
        {\ButtonLeft/\ButtonRight/}
    \opt{RECORDER_PAD,ONDIO_PAD,IRIVER_H100_PAD,IRIVER_H300_PAD,SANSA_E200_PAD%
      ,SANSA_C200_PAD,SANSA_CLIP_PAD,GIGABEAT_PAD,MROBE100_PAD,IAUDIO_X5_PAD,GIGABEAT_S_PAD%
      ,SANSA_FUZE_PAD,PBELL_VIBE500_PAD,SANSA_FUZEPLUS_PAD,SAMSUNG_YH92X_PAD%
      ,SAMSUNG_YH820_PAD}
      {\ButtonUp/\ButtonDown}
    \opt{IPOD_4G_PAD,IPOD_3G_PAD}{\ButtonMenu/\ButtonPlay}
    \opt{IRIVER_H10_PAD}{\ButtonScrollUp/\ButtonScrollDown}
       \opt{HAVEREMOTEKEYMAP}{& }
    & Move the cursor around the blocks \\
    \opt{RECORDER_PAD,IRIVER_H10_PAD,SAMSUNG_YH92X_PAD,SAMSUNG_YH820_PAD}{\ButtonPlay}
    \opt{ONDIO_PAD}{\ButtonMenu}
    \opt{IRIVER_H100_PAD,IRIVER_H300_PAD,IPOD_4G_PAD,IPOD_3G_PAD,SANSA_E200_PAD%
      ,SANSA_C200_PAD,SANSA_CLIP_PAD,GIGABEAT_PAD,MROBE100_PAD,IAUDIO_X5_PAD,GIGABEAT_S_PAD%
      ,SANSA_FUZE_PAD,SANSA_FUZEPLUS_PAD}
      {\ButtonSelect}
    \opt{PBELL_VIBE500_PAD}{\ButtonOK}
       \opt{HAVEREMOTEKEYMAP}{& }
    & Remove a block \\
    }
    \opt{RECORDER_PAD,ONDIO_PAD,IRIVER_H100_PAD,IRIVER_H300_PAD}
      {\ButtonOff}
    \opt{IRIVER_H10_PAD,SANSA_E200_PAD,SANSA_C200_PAD,SANSA_CLIP_PAD,GIGABEAT_PAD,MROBE100_PAD%
      ,IAUDIO_X5_PAD,COWON_D2_PAD,SANSA_FUZEPLUS_PAD}{\ButtonPower}
    \opt{SANSA_FUZE_PAD}{\ButtonHome}
    \opt{GIGABEAT_S_PAD}{\ButtonBack}
    \opt{PBELL_VIBE500_PAD}{\ButtonRec}
    \opt{SAMSUNG_YH92X_PAD,SAMSUNG_YH820_PAD}{\ButtonRew}
       \opt{HAVEREMOTEKEYMAP}{& }
     & Exit \\
  \end{btnmap}
}

\opt{lcd_color}{% $Id$
\subsection{Codebuster}
\screenshot{plugins/images/ss-codebuster}{Codebuster}{img:codebuster}

Codebuster is a clone of the classic mastermind game. The computer selects a
random combination of coloured pegs and the aim is to guess the correct combination
in the smallest number of moves. After each attempt to guess the combination the
results are displayed in the form of red and white pegs.  A red peg signifies
a correct peg in the correct position, and a white peg signifies a correct
peg in the wrong position.

  \begin{btnmap}
    \opt{IPOD_4G_PAD,IPOD_3G_PAD}{\ButtonMenu}
    \nopt{IPOD_4G_PAD,IPOD_3G_PAD}{\PluginCancel}
       \opt{HAVEREMOTEKEYMAP}{& \PluginRCCancel}
        & Show menu \\

    \PluginSelect
       \opt{HAVEREMOTEKEYMAP}{& \PluginRCSelect}
        & Check suggestion and move to next line \\

    \PluginLeft{} / \PluginRight
       \opt{HAVEREMOTEKEYMAP}{& \PluginRCLeft{} / \PluginRCRight}
        & Select a peg \\

    \nopt{scrollwheel}{\PluginUp{} / \PluginDown}
    \opt{scrollwheel}{\PluginScrollFwd{} / \PluginScrollBack}
       \opt{HAVEREMOTEKEYMAP}{& \PluginRCUp{} / \PluginRCDown}
        & Change current peg \\
  \end{btnmap}
}

\subsection{Dice}
Dice is a simple dice rolling simulator. Select number and type of dice to roll
in a menu and start by choosing ``Roll Dice''. The result is shown as individual
numbers as well as the total of the rolled dice.

\begin{btnmap}
    \PluginSelect
          \opt{HAVEREMOTEKEYMAP}{& \PluginRCSelect}
        & Roll dice again\\

    \nopt{IPOD_4G_PAD,IPOD_3G_PAD}{\PluginCancel}
    \opt{IPOD_4G_PAD,IPOD_3G_PAD}{\ButtonMenu}
       \opt{HAVEREMOTEKEYMAP}{& \PluginRCCancel}
        & Quit\\
\end{btnmap}


\nopt{xduoox3}{\nopt{lowmem}{% $Id$ %
\subsection{Doom}
\screenshot{plugins/images/ss-doom}{Doom}{}
This is the famous Doom game.

\subsubsection{Getting started}
For the game to run you need \fname{.wad} game files located in
\fname{/.rockbox/doom/} on your player. Create the directory and save the
following files there:
\begin{description}
\item[rockdoom.wad.] The Rockbox \fname{.wad}, based on \fname{prboom.wad}
from prboom-2.2.6
\item[Your wad files.] Copy all Doom wads you wish to play into that directory.
\end{description}
The needed files can be found at
\wikilink{PluginDoom}

To play addon wads create the \fname{addons} directory within the doom directory.
Place \fname{wad} files in this directory. Currently doom only supports 
a maximum number of 10 addons.

A free alternative for Doom 2 is FreeDoom (\url{http://freedoom.sourceforge.net}).
This can be used in place of \fname{doom2.wad}, or it may be used as an addon in 
Doom, by placing it in the \fname{addons} directory.

\subsubsection{Menus}
\begin{description}
  \item[Rockdoom Menu. ] The Rockdoom menu is shown when Doom is first launched.  
This is the only time it can be accessed (before starting the game).  To re-adjust 
Rockdoom options, you will need to quit your current game and restart the plugin.
  \item[Main Menu. ]
  The Doom plugin has a main menu, which is brought up before a game is started. It 
  has the following entries:
  
  \emph{Game. } Select which (official) wad to launch\\
  \emph{Addon. } Select which unofficial addon wad to launch (From 
  \fname{/.rockbox/doom/addons} directory)\\
  \emph{Demos. } Select which demo file to play on game start\\
  \emph{Options. } Configure low-level Doom options\\
  \emph{Play Game. } Launch the wad/addon/Demo chosen%\\
  
  \item[Options Menu. ]This menu has the following options:
  
  \emph{Sound. } Enable or Disable sound in Doom\\
  \emph{Set Keys. }  Change the game key configuration\\
  \emph{Time Demo. } Run a timed demo, to test game speed on a player (Only runs on Doom Shareware)\\
  \emph{Player Bobbing. } Enable or Disable player up/Down movement\\
  \emph{Translucency. } Enable or Disable sprite translucency (Fireballs, Plasma...)\\
  \emph{Fake Contrast.} Enable or Disable modified game lighting\\
  \emph{Always Run.} Make the player always run\\
  \emph{Headsup Display.} Show the player status when in fullscreen\\
  \emph{Statusbar Always Red.} Disable colour response statusbar%\\

  \item[InGame Main Menu. ]This menu can only be accessed from within a running game, and is displayed by
    \opt{IRIVER_H100_PAD,IRIVER_H300_PAD}{pressing \ButtonOff}%
    \opt{IPOD_3G_PAD,IPOD_4G_PAD}{flipping your \ButtonHold{} switch a couple of times}%
    \opt{IAUDIO_X5_PAD,IRIVER_H10_PAD,SANSA_E200_PAD,SANSA_C200_PAD,SANSA_CLIP_PAD%
         ,GIGABEAT_PAD,GIGABEAT_S_PAD,COWON_D2_PAD}{pressing \ButtonPower}%
    \opt{SANSA_FUZE_PAD}{pressing \ButtonHome}
    \opt{PBELL_VIBE500_PAD,SAMSUNG_YH92X_PAD,SAMSUNG_YH820_PAD}{pressing \ButtonRec}
    \opt{SANSA_FUZEPLUS_PAD}{pressing \ButtonBack}

  \emph{New Game. } Start a new game\\
  \emph{Options. } In game options\\
  \emph{Load Game. } Load a saved game\\
  \emph{Save Game. } Save the current game\\
  \emph{Quit. } Quit the game%\\

  \item[InGame Options Menu. ]This menu has the following options:
  
  \emph{End Game. } Ends the current game\\
  \emph{Messages. }  Enable or Disable in game messages\\
  \emph{Screen Size. } Shrink or Enlarge the displayed portion of the game\\
  \emph{Gamma. } Change the brightness (Gamma) of the game\\
  \emph{Sound Volume. } Change the sound, music and system volume%\\
  \note{In game music is not currently supported}

\end{description}

\subsubsection{Keys}
\begin{btnmap}
  \opt{IRIVER_H100_PAD,IRIVER_H300_PAD,IAUDIO_X5_PAD,SANSA_E200_PAD,SANSA_FUZE_PAD,SANSA_C200_PAD,SANSA_CLIP_PAD%
      ,GIGABEAT_PAD,GIGABEAT_S_PAD,MROBE100_PAD,SANSA_FUZEPLUS_PAD,SAMSUNG_YH92X_PAD%
      ,SAMSUNG_YH820_PAD}{\ButtonUp}
  \opt{IPOD_3G_PAD,IPOD_4G_PAD}{\ButtonMenu}
  \opt{IRIVER_H10_PAD,MPIO_HD300_PAD}{\ButtonScrollUp}
  \opt{COWON_D2_PAD}{\TouchTopMiddle}
  \opt{PBELL_VIBE500_PAD}{\ButtonOK}
       \opt{HAVEREMOTEKEYMAP}{& }
    & Move Forward \\
%
  \nopt{IPOD_3G_PAD,IPOD_4G_PAD,COWON_D2_PAD}{
    \opt{IRIVER_H100_PAD,IRIVER_H300_PAD,IAUDIO_X5_PAD,SANSA_E200_PAD,SANSA_FUZE_PAD,SANSA_C200_PAD,SANSA_CLIP_PAD%
        ,GIGABEAT_PAD,GIGABEAT_S_PAD,MROBE100_PAD,SANSA_FUZEPLUS_PAD,SAMSUNG_YH92X_PAD%
        ,SAMSUNG_YH820_PAD}{\ButtonDown}
    \opt{IRIVER_H10_PAD,MPIO_HD300_PAD}{\ButtonScrollDown}
    \opt{COWON_D2_PAD}{\TouchBottomMiddle}
    \opt{PBELL_VIBE500_PAD}{\ButtonCancel}
       \opt{HAVEREMOTEKEYMAP}{& }
      & Down\\
  }
%
  \opt{IRIVER_H100_PAD,IRIVER_H300_PAD,IAUDIO_X5_PAD,SANSA_E200_PAD,SANSA_CLIP_PAD%
      ,SANSA_FUZE_PAD,SANSA_C200_PAD,GIGABEAT_PAD,GIGABEAT_S_PAD,MROBE100_PAD%
      ,IPOD_3G_PAD,IPOD_4G_PAD,IRIVER_H10_PAD,PBELL_VIBE500_PAD,SANSA_FUZEPLUS_PAD%
      ,SAMSUNG_YH92X_PAD,SAMSUNG_YH820_PAD}{\ButtonLeft}
  \opt{MPIO_HD300_PAD}{\ButtonRew}
  \opt{COWON_D2_PAD}{\TouchMidLeft}
      \opt{HAVEREMOTEKEYMAP}{& }
    & Turn Left \\
%
  \opt{IRIVER_H100_PAD,IRIVER_H300_PAD,IAUDIO_X5_PAD,SANSA_E200_PAD,SANSA_CLIP_PAD%
      ,SANSA_FUZE_PAD,SANSA_C200_PAD,GIGABEAT_PAD,GIGABEAT_S_PAD,MROBE100_PAD%
      ,IPOD_3G_PAD,IPOD_4G_PAD,IRIVER_H10_PAD,PBELL_VIBE500_PAD,SANSA_FUZEPLUS_PAD%
      ,SAMSUNG_YH92X_PAD,SAMSUNG_YH820_PAD}{\ButtonRight}
  \opt{MPIO_HD300_PAD}{\ButtonFF}
  \opt{COWON_D2_PAD}{\TouchMidRight}
      \opt{HAVEREMOTEKEYMAP}{& }
    & Turn Right \\
%
  \opt{IRIVER_H100_PAD,IRIVER_H300_PAD}{\ButtonRec}
  \opt{IPOD_3G_PAD,IPOD_4G_PAD,GIGABEAT_S_PAD,SAMSUNG_YH92X_PAD,SAMSUNG_YH820_PAD}{\ButtonPlay}
  \opt{IAUDIO_X5_PAD,SANSA_E200_PAD,SANSA_FUZE_PAD,SANSA_C200_PAD,SANSA_CLIP_PAD,SANSA_FUZEPLUS_PAD}{\ButtonSelect}
  \opt{IRIVER_H10_PAD}{\ButtonRew}
  \opt{GIGABEAT_PAD}{\ButtonA}
  \opt{MROBE100_PAD}{\ButtonDisplay}
  \opt{COWON_D2_PAD}{\TouchBottomRight}
  \opt{PBELL_VIBE500_PAD}{\ButtonMenu}
  \opt{MPIO_HD300_PAD}{\ButtonEnter}
       \opt{HAVEREMOTEKEYMAP}{& }
    & Shoot \\
%
  \opt{IRIVER_H100_PAD,IRIVER_H300_PAD}{\ButtonMode}
  \opt{IPOD_3G_PAD,IPOD_4G_PAD}{\ButtonMenu}
  \opt{IAUDIO_X5_PAD,IRIVER_H10_PAD,SANSA_FUZEPLUS_PAD}{\ButtonPlay}
  \opt{SANSA_E200_PAD,SANSA_C200_PAD}{\ButtonRec}
  \opt{SANSA_FUZE_PAD}{\ButtonPower}
  \opt{SANSA_CLIP_PAD}{\ButtonHome}
  \opt{GIGABEAT_PAD,GIGABEAT_S_PAD,MROBE100_PAD,MPIO_HD300_PAD}{\ButtonMenu}
  \opt{PBELL_VIBE500_PAD}{\ButtonOK}
  \opt{SAMSUNG_YH92X_PAD,SAMSUNG_YH820_PAD}{\ButtonRew}
       \opt{HAVEREMOTEKEYMAP}{& }
    & Open \\
%
  \opt{IRIVER_H100_PAD,IRIVER_H300_PAD}{\ButtonOff}
  \opt{IPOD_3G_PAD,IPOD_4G_PAD}{\ButtonHold{} switch}
  \opt{IAUDIO_X5_PAD,IRIVER_H10_PAD,SANSA_E200_PAD,SANSA_C200_PAD,GIGABEAT_PAD%
      ,GIGABEAT_S_PAD,MROBE100_PAD,SANSA_CLIP_PAD}{\ButtonPower}
  \opt{SANSA_FUZE_PAD}{\ButtonHome}
  \opt{COWON_D2_PAD}{\TouchTopLeft}
  \opt{PBELL_VIBE500_PAD,MPIO_HD300_PAD,SAMSUNG_YH92X_PAD,SAMSUNG_YH820_PAD}{\ButtonRec}
  \opt{SANSA_FUZEPLUS_PAD}{\ButtonBack}
       \opt{HAVEREMOTEKEYMAP}{& }
    & InGame Menu \\
%
  \opt{IRIVER_H100_PAD,IRIVER_H300_PAD,IPOD_3G_PAD,IPOD_4G_PAD,IAUDIO_X5_PAD%
      ,GIGABEAT_PAD,GIGABEAT_S_PAD,SANSA_E200_PAD,SANSA_FUZE_PAD,SANSA_C200_PAD%
      ,MROBE100_PAD,SANSA_CLIP_PAD,SANSA_FUZEPLUS_PAD}{\ButtonSelect}
  \opt{IRIVER_H10_PAD}{\ButtonRew}
  \opt{COWON_D2_PAD}{\TouchCenter}
  \opt{PBELL_VIBE500_PAD}{\ButtonPower}
  \opt{MPIO_HD300_PAD}{Long \ButtonPlay}
  \opt{SAMSUNG_YH92X_PAD,SAMSUNG_YH820_PAD}{\ButtonFF}
       \opt{HAVEREMOTEKEYMAP}{& }
    & Enter \\
%
  \opt{IRIVER_H100_PAD,IRIVER_H300_PAD}{\ButtonOn}
  \opt{IPOD_3G_PAD,IPOD_4G_PAD}{\ButtonSelect}
  \opt{IAUDIO_X5_PAD}{\ButtonRec}
  \opt{IRIVER_H10_PAD,SAMSUNG_YH92X_PAD,SAMSUNG_YH820_PAD}{\ButtonFF}
  \opt{SANSA_E200_PAD,SANSA_FUZE_PAD}{\ButtonScrollFwd}
  \opt{SANSA_C200_PAD,SANSA_CLIP_PAD}{\ButtonVolUp}
  \opt{GIGABEAT_PAD,GIGABEAT_S_PAD}{\ButtonVolDown}
  \opt{COWON_D2_PAD}{\TouchBottomLeft}
  \opt{PBELL_VIBE500_PAD,MPIO_HD300_PAD}{\ButtonPlay}
  \opt{SANSA_FUZEPLUS_PAD}{\ButtonBottomLeft}
       \opt{HAVEREMOTEKEYMAP}{& }
    & Change Weapon \\
%
\end{btnmap}

\subsubsection{Playing the game}
After installation of the \fname{wad} files is complete you can start the
game.
\fixme{more description is needed}
}}

\opt{lcd_color}{\nopt{lowmem,iaudiox5,iriverh300}{\subsection{Duke3D}

\screenshot{plugins/images/ss-duke3d}{``Hollywood Holocaust'', the
  first level of Duke Nukem 3D}{fig:Duke3D}

This is a port of Duke Nukem 3D, derived from Fabien Sanglard's
Chocolate Duke.

\subsubsection{Installation}
The \fname{.GRP} and \fname{.CON} files from an original installation
of the game must be placed in the \fname{/.rockbox/duke3d/} on your
device. The shareware files work as well, and are available at
\wikilink{PluginDuke3D}.

\subsubsection{Music}
In-game music will not work by default. For it to work, you must
install a modified Timidity patchset in the
\fname{/.rockbox/timidity/} directory on your device. There should be
a \fname{/.rockbox/timidity/timidity.cfg} file located in this
directory, along with the instrument files. You must edit the
\fname{.cfg} file so that all the path names are absolute (i.e. in the
form \fname{/.rockbox/timidity/instruments/*.cfg}).

As above, there is a free patchset available from
\wikilink{PluginDuke3D}.

\subsubsection{Video}
Rotation of the video output is possible by choosing the correct video
option in the in-game menu. If your device's display is normally
320x240, for example, choosing the 240x320 option will rotate the
screen 90 degrees, and the keymap will be updated as well.

\subsubsection{Caveats}
Sound effects, enabled by default, could have a detrimental effect on
playability on some devices. If you notice excessive lag, try
disabling sound.

The default button mapping may not be optimal for gameplay. Set a
different mapping in the ``Keyboard'' section of the game setup. Note
that not all keys are mappable on all devices.

Some devices will low memory or large GRP files will prevent the game
from completely caching the GRP file in RAM, which could lead to disk
reads during gameplay. This might cause the game to lag slightly when
it happens, which is normal. The game should resume in a second or so.

On other devices, large GRP files will render the game completely
unplayable. If this occurs, try using the smaller shareware GRP. If
this still fails, then please see Appendix \ref{sec:feedback} for
instructions to report a bug.
}}

% $Id$ %
\subsection{Flipit}
\screenshot{plugins/images/ss-flipit}{Flipit}{img:flipit}
Flipping the colour of the token under the cursor also flips the tokens
above, below, left and right of the cursor.  The aim is to end up with
a screen containing tokens of only one colour.

\begin{btnmap}
\opt{PLAYER_PAD}{\ButtonOn{} / \ButtonMenu{} / \ButtonLeft{} / \ButtonRight}
\opt{RECORDER_PAD,ONDIO_PAD,IRIVER_H100_PAD,IRIVER_H300_PAD,IAUDIO_X5_PAD%
    ,SANSA_E200_PAD,SANSA_FUZE_PAD,SANSA_C200_PAD,SANSA_CLIP_PAD,GIGABEAT_PAD,GIGABEAT_S_PAD%
    ,MROBE100_PAD,PBELL_VIBE500_PAD,SANSA_FUZEPLUS_PAD,SAMSUNG_YH92X_PAD,SAMSUNG_YH820_PAD}
    {\ButtonUp{} / \ButtonDown{} / \ButtonLeft{} / \ButtonRight}
\opt{IPOD_4G_PAD,IPOD_3G_PAD}{\ButtonMenu{} / \ButtonPlay{} / \ButtonLeft{} / \ButtonRight}
\opt{IRIVER_H10_PAD}{\ButtonScrollUp{} / \ButtonScrollDown{} / \ButtonLeft{} / \ButtonRight}
\opt{MPIO_HD300_PAD}{\ButtonScrollUp / \ButtonScrollDown / \ButtonRew / \ButtonFF}
       \opt{HAVEREMOTEKEYMAP}{& }
\opt{COWON_D2_PAD}{\TouchTopMiddle{} / \TouchBottomMiddle{} / \TouchMidLeft{} / \TouchMidRight}
    & Move the cursor \\
\opt{PLAYER_PAD,RECORDER_PAD}{\ButtonPlay}
\opt{ONDIO_PAD}{\ButtonMenu}
\opt{IRIVER_H100_PAD,IRIVER_H300_PAD,IPOD_4G_PAD,IPOD_3G_PAD,IAUDIO_X5_PAD%
    ,SANSA_E200_PAD,SANSA_FUZE_PAD,SANSA_C200_PAD,GIGABEAT_PAD,GIGABEAT_S_PAD%
    ,MROBE100_PAD,SANSA_FUZEPLUS_PAD}
    {\ButtonSelect}
\opt{IRIVER_H10_PAD,SAMSUNG_YH92X_PAD,SAMSUNG_YH820_PAD}{\ButtonRew}
\opt{COWON_D2_PAD}{\TouchCenter}
\opt{PBELL_VIBE500_PAD}{\ButtonOK}
\opt{MPIO_HD300_PAD}{\ButtonEnter}
\opt{SANSA_CLIP_PAD}{\ButtonHome}
       \opt{HAVEREMOTEKEYMAP}{& }
    & Flip \\
\opt{PLAYER_PAD}{\ButtonOn+\ButtonLeft}
\opt{RECORDER_PAD}{\ButtonFOne}
\opt{ONDIO_PAD}{\ButtonMenu+\ButtonLeft}
\opt{IRIVER_H100_PAD,IRIVER_H300_PAD}{\ButtonMode}
\opt{IPOD_4G_PAD,IPOD_3G_PAD}{\ButtonSelect+\ButtonLeft}
\opt{IAUDIO_X5_PAD,IRIVER_H10_PAD}{\ButtonPlay+\ButtonLeft}
\opt{SANSA_E200_PAD,SANSA_C200_PAD}{\ButtonRec+\ButtonLeft}
\opt{SANSA_CLIP_PAD}{\ButtonHome+\ButtonLeft}
\opt{SANSA_FUZE_PAD}{\ButtonSelect+\ButtonLeft}
\opt{GIGABEAT_PAD,GIGABEAT_S_PAD,MROBE100_PAD}{\ButtonMenu}
\opt{COWON_D2_PAD}{\TouchTopRight}
\opt{PBELL_VIBE500_PAD,MPIO_HD300_PAD,SANSA_FUZEPLUS_PAD,SAMSUNG_YH92X_PAD,SAMSUNG_YH820_PAD}%
    {\ButtonPlay}
       \opt{HAVEREMOTEKEYMAP}{& }
    & Shuffle \\
\opt{PLAYER_PAD}{\ButtonOn+\ButtonRight}
\opt{RECORDER_PAD}{\ButtonFTwo}
\opt{ONDIO_PAD}{\ButtonMenu+\ButtonUp}
\opt{IRIVER_H100_PAD,IRIVER_H300_PAD}{\ButtonOn}
\opt{IPOD_4G_PAD,IPOD_3G_PAD}{\ButtonSelect+\ButtonPlay}
\opt{IAUDIO_X5_PAD,IRIVER_H10_PAD}{\ButtonPlay+\ButtonRight}
\opt{SANSA_E200_PAD,SANSA_C200_PAD}{\ButtonRec+\ButtonRight}
\opt{SANSA_CLIP_PAD}{\ButtonHome+\ButtonRight}
\opt{SANSA_FUZE_PAD}{\ButtonSelect+\ButtonDown}
\opt{GIGABEAT_PAD,GIGABEAT_S_PAD}{\ButtonVolUp}
\opt{MROBE100_PAD}{\ButtonPlay}
\opt{COWON_D2_PAD}{\TouchBottomLeft}
\opt{PBELL_VIBE500_PAD,MPIO_HD300_PAD}{\ButtonMenu}
\opt{SANSA_FUZEPLUS_PAD}{\ButtonBack}
\opt{SAMSUNG_YH92X_PAD,SAMSUNG_YH820_PAD}{\ButtonFF}
       \opt{HAVEREMOTEKEYMAP}{& }
    & Solve \\
\opt{PLAYER_PAD}{\ButtonOn+\ButtonPlay}
\opt{RECORDER_PAD}{\ButtonFThree}
\opt{ONDIO_PAD}{\ButtonMenu+\ButtonRight}
\opt{IRIVER_H100_PAD,IRIVER_H300_PAD}{\ButtonRec}
\opt{IPOD_4G_PAD,IPOD_3G_PAD}{\ButtonSelect+\ButtonRight}
\opt{IAUDIO_X5_PAD}{\ButtonPlay+\ButtonUp}
\opt{IRIVER_H10_PAD}{\ButtonPlay+\ButtonScrollUp}
\opt{SANSA_E200_PAD,SANSA_C200_PAD}{\ButtonRec+\ButtonSelect}
\opt{SANSA_CLIP_PAD}{\ButtonHome+\ButtonSelect}
\opt{SANSA_FUZE_PAD}{\ButtonSelect+\ButtonRight}
\opt{GIGABEAT_PAD,GIGABEAT_S_PAD}{\ButtonVolDown}
\opt{MROBE100_PAD}{\ButtonDisplay}
\opt{COWON_D2_PAD}{\TouchBottomRight}
\opt{PBELL_VIBE500_PAD}{\ButtonCancel}
\opt{MPIO_HD300_PAD}{\ButtonRec}
\opt{SANSA_FUZEPLUS_PAD}{\ButtonBottomLeft}
\opt{SAMSUNG_YH92X_PAD,SAMSUNG_YH820_PAD}{\ButtonFF+\ButtonUp}
       \opt{HAVEREMOTEKEYMAP}{& }
    & Solve step by step \\
\opt{PLAYER_PAD}{\ButtonStop}
\opt{RECORDER_PAD,ONDIO_PAD,IRIVER_H100_PAD,IRIVER_H300_PAD}{\ButtonOff}
\opt{IPOD_4G_PAD,IPOD_3G_PAD}{\ButtonSelect+\ButtonMenu}
\opt{IAUDIO_X5_PAD,IRIVER_H10_PAD,SANSA_E200_PAD,GIGABEAT_PAD,MROBE100_PAD%
    ,SANSA_C200_PAD,SANSA_CLIP_PAD,COWON_D2_PAD,SANSA_FUZEPLUS_PAD}{\ButtonPower}
\opt{SANSA_FUZE_PAD}{Long \ButtonHome}
\opt{GIGABEAT_S_PAD}{\ButtonBack}
\opt{PBELL_VIBE500_PAD,SAMSUNG_YH92X_PAD,SAMSUNG_YH820_PAD}{\ButtonRec}
\opt{MPIO_HD300_PAD}{Long \ButtonMenu}
       \opt{HAVEREMOTEKEYMAP}{& 
          \opt{IRIVER_RC_H100_PAD}{\ButtonRCStop}
        }
    & Quit the game \\
\end{btnmap}


\subsection{Goban}
\screenshot{plugins/images/ss-goban}{Goban}{The Rockbox Goban plugin}
Goban is a a plugin for playing, viewing and recording games of Go (also known
as Weiqi, Baduk, Igo and Goe).  It uses standard Smart Game Format (SGF) files
for saving and loading games.  You can find a short introduction to Go at
\url{http://senseis.xmp.net/?WhatIsGo} and more information about SGF files
can be read at \url{http://senseis.xmp.net/?SmartGameFormat} or the SGF
specification at \url{http://www.red-bean.com/sgf/}.\\

This plugin can load all modern SGF files (file format 3 or 4) with few problems.
It attempts to preserve SGF properties which it doesn't understand, and most common
SGF properties are handled fully.  It is possible to view (and edit if you like)
Kogo's Joseki Dictionary (\url{http://waterfire.us/joseki.htm}) with this plugin,
although the load and save times can be on the order of a minute or two on
particularly slow devices.  Large SGF files may stop audio playback for the duration
of the plugin's run in order to free up more memory and some very large SGF files will
not even load on devices with little available memory.\\

\note{The plugin does \emph{NOT} support SGF files with multiple games in
one file.  These are rare, but if you have one don't even try it (the file will most
likely be corrupted if you save over it). You have been warned.\\}

The file \fname {/sgf/gbn\_def.sgf} is used by the plugin to store any unsaved
changes in the most recently loaded game.  This means that if you forget to save your
changes, you should load \fname {/sgf/gbn\_def.sgf} immediately to offload the changes
to another file.  If you load another file first then your changes will be lost
permanently.  The \fname {/sgf/gbn\_def.sgf} file is also the file loaded if another
is not selected.\\

The information panel which displays the current move number may also contain
these markers: 

\begin{table}
    \begin{rbtabular}{\textwidth}{lX}%
      {\textbf{Mark} & \textbf{Meaning}}{}{}
      \emph{+ } & There are nodes after the current node in the SGF tree. \\
      \emph{* } & There are sibling variations which can be navigated to using the %
      \emph{Next Variation} menu option of the \emph{Context Menu}%
      \opt{SANSA_E200_PAD,SANSA_C200_PAD,SANSA_CLIP_PAD,%
         RECORDER_PAD,MROBE100_PAD,GIGABEAT_PAD,GIGABEAT_S_PAD,IRIVER_H100_PAD,%
         IRIVER_H300_PAD,PBELL_VIBE500_PAD,MPIO_HD200_PAD,SANSA_FUZEPLUS_PAD,%
         SAMSUNG_YH92X_PAD}{ or the %
         \opt{SANSA_FUZEPLUS_PAD}{\ButtonBottomRight}%
         \opt{SANSA_E200_PAD,SANSA_C200_PAD}{\ButtonRec}%
         \opt{SANSA_CLIP_PAD}{\ButtonHome}%
         \opt{RECORDER_PAD}{\ButtonOn}%
         \opt{MROBE100_PAD}{\ButtonPower}%
         \opt{GIGABEAT_PAD}{\ButtonA}%
         \opt{GIGABEAT_S_PAD}{\ButtonPlay}%
         \opt{PBELL_VIBE500_PAD}{\ButtonCancel}%
         \opt{MPIO_HD200_PAD}{Long \ButtonFunc}%
         \opt{MPIO_HD300_PAD}{\ButtonMenu}%
         \opt{SAMSUNG_YH92X_PAD}{\ButtonPlay+\ButtonDown}
         \opt{IRIVER_H100_PAD,IRIVER_H300_PAD}{\ButtonRec} button}. \\
      \emph{C } & There is a comment at the current node.  It can be viewed/edited using
                the \emph{Add/Edit Comment} menu option of the \emph{Context Menu}. \\
    \end{rbtabular}
\end{table}

\subsubsection{Controls}
    \begin{btnmap}
        \nopt{IPOD_1G2G_PAD,IPOD_3G_PAD,IPOD_4G_PAD,IRIVER_H10_PAD,%
            MPIO_HD200_PAD,MPIO_HD300_PAD,touchscreen}{\ButtonUp}%
        \opt{IPOD_1G2G_PAD,IPOD_3G_PAD,IPOD_4G_PAD}{\ButtonMenu}%
        \opt{IRIVER_H10_PAD,MPIO_HD300_PAD}{\ButtonScrollUp}
        \opt{MPIO_HD200_PAD}{\ButtonRew}
        \opt{touchscreen}{\TouchTopMiddle}
            &
        \opt{HAVEREMOTEKEYMAP}{
            &}
        Move cursor up
        \\
        
        \nopt{IPOD_1G2G_PAD,IPOD_3G_PAD,IPOD_4G_PAD,IRIVER_H10_PAD,%
            MPIO_HD200_PAD,MPIO_HD300_PAD,touchscreen}{\ButtonDown}%
        \opt{IPOD_1G2G_PAD,IPOD_3G_PAD,IPOD_4G_PAD}{\ButtonPlay}%
        \opt{IRIVER_H10_PAD,MPIO_HD300_PAD}{\ButtonScrollDown}
        \opt{MPIO_HD200_PAD}{\ButtonFF}
        \opt{touchscreen}{\TouchBottomMiddle}
            &
        \opt{HAVEREMOTEKEYMAP}{
            &}
        Move cursor down
        \\
            
        \nopt{MPIO_HD200_PAD,MPIO_HD300_PAD,touchscreen}{\ButtonLeft}
        \opt{MPIO_HD200_PAD}{\ButtonVolDown}
        \opt{MPIO_HD300_PAD}{\ButtonRew}
        \opt{touchscreen}{\TouchMidLeft}
            &
        \opt{HAVEREMOTEKEYMAP}{
            &}
        Move cursor left
        \opt{ONDIO_PAD}{if in \emph{board} navigation mode, or retreat one
            node in the game tree if in \emph{tree} navigation mode}
        \\
             
        \nopt{MPIO_HD200_PAD,MPIO_HD300_PAD,touchscreen}{\ButtonRight}
        \opt{MPIO_HD200_PAD}{\ButtonVolUp}
        \opt{MPIO_HD300_PAD}{\ButtonFF}
        \opt{touchscreen}{\TouchMidRight}
            &
        \opt{HAVEREMOTEKEYMAP}{
            &}
        Move cursor right
        \opt{ONDIO_PAD}{if in \emph{board} navigation mode, or advance one
            node in the game tree if in \emph{tree} navigation mode}
        \\
            
        \opt{ONDIO_PAD}{
            \ButtonOff
                &
            Toggle between \emph{board} and \emph{tree} navigation modes
            \\
        }
        
        \nopt{IRIVER_H10_PAD,ONDIO_PAD,RECORDER_PAD,IAUDIO_M3_PAD,PBELL_VIBE500_PAD%
            ,MPIO_HD200_PAD,MPIO_HD300_PAD,SAMSUNG_YH92X_PAD,touchscreen}{%
            \ButtonSelect}%
        \opt{IRIVER_H10_PAD,RECORDER_PAD,PBELL_VIBE500_PAD,SAMSUNG_YH92X_PAD}{\ButtonPlay}%
        \opt{ONDIO_PAD}{\ButtonMenu}
        \opt{MPIO_HD200_PAD}{\ButtonFunc}
        \opt{MPIO_HD300_PAD}{\ButtonEnter}
        \opt{touchscreen}{\TouchCenter}
            &
        \opt{HAVEREMOTEKEYMAP}{
            &}
        Play a move (or use a tool if play-mode has been changed).
        \\
            
        \nopt{ONDIO_PAD}{
            \opt{SANSA_E200_PAD,SANSA_FUZE_PAD,IPOD_1G2G_PAD,IPOD_3G_PAD%
                ,IPOD_4G_PAD}{\ButtonScrollBack}%
            \opt{SANSA_CLIP_PAD,SANSA_M200_PAD,SANSA_C200_PAD,GIGABEAT_PAD%
                ,GIGABEAT_S_PAD}{\ButtonVolDown}%
            \opt{IRIVER_H10_PAD}{\ButtonFF}%
            \opt{IRIVER_H100_PAD,IRIVER_H300_PAD}{\ButtonOff}%
            \opt{MROBE100_PAD}{\ButtonMenu}%
            \opt{SANSA_FUZEPLUS_PAD}{\ButtonBack}%
            \opt{IAUDIO_X5_PAD}{\ButtonPlay}%
            \opt{RECORDER_PAD}{\ButtonFOne}
            \opt{touchscreen}{\TouchBottomLeft}
            \opt{PBELL_VIBE500_PAD}{\ButtonOK{} + \ButtonLeft}
            \opt{MPIO_HD200_PAD,MPIO_HD300_PAD}{\ButtonRec + \ButtonRew}
            \opt{SAMSUNG_YH92X_PAD}{\ButtonRew}
                &
            \opt{HAVEREMOTEKEYMAP}{
                &}
            Retreat one node in the game tree
            \\
            
            \opt{scrollwheel}{\ButtonScrollFwd}%
            \opt{SANSA_CLIP_PAD,SANSA_M200_PAD,SANSA_C200_PAD,GIGABEAT_PAD%
                ,GIGABEAT_S_PAD}{\ButtonVolUp}%
            \opt{IRIVER_H10_PAD}{\ButtonRew}%
            \opt{IRIVER_H100_PAD,IRIVER_H300_PAD}{\ButtonOn}%
            \opt{MROBE100_PAD,SANSA_FUZEPLUS_PAD}{\ButtonPlay}%
            \opt{IAUDIO_X5_PAD}{\ButtonRec}%
            \opt{RECORDER_PAD}{\ButtonFThree}
            \opt{touchscreen}{\TouchBottomRight}
            \opt{PBELL_VIBE500_PAD}{\ButtonOK{} + \ButtonRight}
            \opt{MPIO_HD200_PAD,MPIO_HD300_PAD}{\ButtonRec + \ButtonFF}
            \opt{SAMSUNG_YH92X_PAD}{\ButtonFF}
                &
            \opt{HAVEREMOTEKEYMAP}{
                &}
            Advance one node in the game tree
            \\
        }
        
        \opt{SANSA_E200_PAD,SANSA_FUZE_PAD,SANSA_CLIP_PAD,SANSA_M200_PAD%
            ,SANSA_C200_PAD,IRIVER_H10_PAD,IAUDIO_X5_PAD,SANSA_FUZEPLUS_PAD%
            }{\ButtonPower}%
        \opt{MROBE100_PAD}{\ButtonDisplay}%
        \opt{IPOD_1G2G_PAD,IPOD_3G_PAD,IPOD_4G_PAD}{Long \ButtonSelect}%
        \opt{GIGABEAT_PAD,GIGABEAT_S_PAD,PBELL_VIBE500_PAD,MPIO_HD300_PAD}%
            {\ButtonMenu}%
        \opt{IRIVER_H100_PAD,IRIVER_H300_PAD}{\ButtonMode}%
        \opt{RECORDER_PAD}{\ButtonFTwo}%
        \opt{ONDIO_PAD}{Long \ButtonMenu}
        \opt{MPIO_HD200_PAD}{Long \ButtonPlay}
        \opt{SAMSUNG_YH92X_PAD}{\ButtonPlay+\ButtonLeft}
        \opt{touchscreen}{\TouchTopLeft}
            &
        \opt{HAVEREMOTEKEYMAP}{
            &}
        Main Menu
        \\

%        \nopt{IPOD_1G2G_PAD,IPOD_3G_PAD,IPOD_4G_PAD,ONDIO_PAD,RECORDER_PAD}{%
        \opt{SANSA_E200_PAD,SANSA_FUZE_PAD,SANSA_C200_PAD,GIGABEAT_PAD,GIGABEAT_S_PAD%
            ,IRIVER_H100_PAD,IRIVER_H300_PAD,MROBE100_PAD,IAUDIO_X5_PAD,IRIVER_H10_PAD%
            ,MPIO_HD200_PAD,PBELL_VIBE500_PAD,touchscreen,SANSA_FUZEPLUS_PAD%
            ,SAMSUNG_YH92X_PAD}{% 
            \nopt{IRIVER_H10_PAD,touchscreen,PBELL_VIBE500_PAD,SAMSUNG_YH92X_PAD,%
                MPIO_HD200_PAD,MPIO_HD300_PAD,SANSA_FUZEPLUS_PAD}{Long \ButtonSelect}%
            \opt{IRIVER_H10_PAD}{Long \ButtonPlay}
            \opt{touchscreen}{Long \TouchCenter}
            \opt{PBELL_VIBE500_PAD}{\ButtonOK}
            \opt{SANSA_FUZEPLUS_PAD}{\ButtonBottomLeft}
            \opt{MPIO_HD200_PAD}{Long \ButtonFunc}
            \opt{MPIO_HD300_PAD}{Long \ButtonEnter}
            \opt{SAMSUNG_YH92X_PAD}{\ButtonPlay+\ButtonDown}%
                &
            \opt{HAVEREMOTEKEYMAP}{
                &}
            Context Menu
            \\
        }
        
        \opt{SANSA_E200_PAD,SANSA_C200_PAD,SANSA_FUZE_PAD,RECORDER_PAD,MROBE100_PAD%
            ,GIGABEAT_PAD,GIGABEAT_S_PAD,IRIVER_H100_PAD,IRIVER_H300_PAD,SANSA_CLIP_PAD%
            ,PBELL_VIBE500_PAD,MPIO_HD200_PAD,touchscreen,SANSA_FUZEPLUS_PAD,
            ,SAMSUNG_YH92X_PAD}{%
            \opt{SANSA_E200_PAD,SANSA_C200_PAD,IRIVER_H100_PAD,IRIVER_H300_PAD%
                ,MPIO_HD200_PAD,MPIO_HD300_PAD}{\ButtonRec}%
            \opt{SANSA_FUZE_PAD,SANSA_CLIP_PAD}{\ButtonHome}%
            \opt{RECORDER_PAD}{\ButtonOn}%
            \opt{MROBE100_PAD}{\ButtonPower}%
            \opt{GIGABEAT_PAD}{\ButtonA}%
            \opt{GIGABEAT_S_PAD}{\ButtonPlay}%
            \opt{touchscreen}{\TouchTopRight}%
            \opt{PBELL_VIBE500_PAD}{\ButtonCancel}%
            \opt{SANSA_FUZEPLUS_PAD}{\ButtonBottomRight}%
            \opt{SAMSUNG_YH92X_PAD}{\ButtonPlay+\ButtonUp}%
                &
            \opt{HAVEREMOTEKEYMAP}{
                &}
            Go to the next variation when at the first node in a branch
            \\
        }
    \end{btnmap}

\subsubsection{Menus}
\begin {description}
\item [Main Menu. ]
    The main menu for game setup and access to other menus.
    \begin{description}
        \item[New.] Create a new game with your choice of board size and handicaps.
        \item[Save.] Save the current state of the game.  It will be saved to
            \fname {/sgf/gbn\_def.sgf} unless otherwise set.
        \item[Save As.] Save to a specified file.
        \item[Game Info.] View and modify the metadata of the current game.
        \item[Playback Control.] Control the playback of the current playlist
            and modify the volume of your player.
        \item[Zoom Level.] Zoom in or out on the board.  If you set the zoom level,
            it will be saved and used again the next time you open this plugin.
        \item[Options.] Open the Options Menu.
        \item[Context Menu.] Open the Context Menu which allows you to set play
            modes and other tools.
        \item[Quit.] Leave the plugin.  Any unsaved changes are saved to
            \fname {/sgf/gbn\_def.sgf}.
    \end{description}
\item [Game Info. ]
    The menu for modifying game info (metadata) of the current game.  This
    information will be saved to the SGF file and can be viewed in almost all
    SGF readers.
    \begin{description}
        \item[Basic Info.] Shows a quick view of the basic game metadata, if any
            has been set (otherwise does nothing).  This option does not allow
            editing.
        \item[Time Limit.] The time limit of the current game.
        \item[Overtime.] The overtime settings of the current game.
        \item[Result.] The result of the current game. This text must follow the
            format specified at \url{http://www.red-bean.com/sgf/properties.html#RE}
            to be read by other SGF readers.  Some examples are
            \emph {B+R} (Black wins by resignation),
            \emph {B+5.5} (Black wins by 5.5 points),
            \emph {W+T} (White wins on Time).
        \item[Handicap.] The handicap of the current game.
        \item[Komi.] The komi of the current game (compensation to the white
            player for black having the first move).
        \item[Ruleset.] The name of the ruleset in use for this game.
            The \emph{NZ} and \emph{GOE} rulesets include suicide as a legal
            move (for multi-stone suicide only); the rest do not.
        \item[Black Player.] The name of the black player.
        \item[Black Rank.] Black's rank, in dan or kyu.
        \item[Black Team.] The name of black's team, if any.
        \item[White Player.] The name of the white player.
        \item[White Rank.]  White's rank, in dan or kyu.
        \item[White Team.] The name of white's team, if any.
        \item[Date.] The date that this game took place. This text must follow
            the format specified at \url{http://www.red-bean.com/sgf/properties.html#DT}
            to be read by other SGF readers.
        \item[Event.] The name of the event which this game was a part of, if any.
        \item[Place.] The place that this game took place.
        \item[Round.] If part of a tournament, the round number for this game.
        \item[Done.] Return to the previous menu.
    \end{description}

\item [Options. ]
    Customize the behavior of the plugin in certain ways.
    \begin{description}
        \item[Show Child Variations?] Enable this to mark child variations on 
            he board if there are more than one.  Note: variations which don't
            start with a move are not visible in this way.
        \item[Disable Idle Poweroff?] Enable this if you do not want the \dap{}
            to turn off after a certain period of inactivity (depends on your
            global Rockbox settings).
        \item[Idle Autosave Time.] Set the amount of idle time to wait before
            automatically saving any unsaved changes.  These autosaves go to
            the file \fname {/sgf/gbn\_def.sgf} regardless of if you have
            loaded a game or used \setting{Save As} to save the game before or
            not.  Set to \setting{Off} to disable this functionality completely.
        \item[Automatically Show Comments?] If this is enabled and you navigate
            to a node containing game comments, they will automatically be
            displayed.
    \end{description}

\item [Context Menu. ]
    The menu for choosing different play modes and tools, adding or editing
    comments, adding pass moves, or switching between sibling variations.
    \begin{description}
        \item[Play Mode.] Play moves normally on the board. If there are child
            moves from the current node, this mode will let you follow
            variations by simply playing the first move in the sequence.
            Unless it is following a variation, this mode will not allow you to
            play illegal moves. This is the default mode before another is set
            after loading a game or creating a new one.
        \item[Add Black Mode.] Add black stones to the board as desired. These
            stones are not moves and do not perform captures or count as ko threats.
        \item[Add White Mode.] Add white stones to the board as desired. These
            stones are not moves and do not perform captures or count as ko threats.
        \item[Erase Stone Mode.] Remove stones from the board as desired. These
            removed stones are not counted as captured, they are simply removed.
        \item[Pass.] Play a single pass move.  This does not change the mode of
            play.
        \item[Next Variation.] If the game is at the first move in a variation,
            this will navigate to the next variation after the current one. This
            is the only way to reach variations which start with adding or
            removing stones, as you cannot follow them by ``playing'' the same move.
        \item[Force Play Mode.] The same as Play Mode except that this mode will
            allow you to play illegal moves such as retaking a ko immediately
            without a ko threat, suicide on rulesets which don't allow it
            (including single stone suicide), and playing a move where there
            is already a stone.
        \item[Mark Mode.] Add generic marks to the board, or remove them.
        \item[Circle Mode.] Add circle marks to the board, or remove them.
        \item[Square Mode.] Add square marks to the board, or remove them.
        \item[Triangle Mode.] Add triangle marks to the board, or remove them.
        \item[Label Mode.] Add one character labels to the board. Each label
            starts at the letter `a' and each subsequent application of a label
            will increment the letter.  To remove a label, click on it until it
            cycles through the allowed letters and disappears.
        \item[Add/Edit Comment.] Add or edit a comment at the current node.
        \item[Done.] Go back to the previous screen.
    \end{description}
\end{description}



\opt{lcd_non-mono}{\nopt{iriverh10_5gb,ipodmini1g,c200,c200v2,mpiohd200,clipzip,samsungyh820}{
  \subsection{Invadrox}
\screenshot{plugins/images/ss-invadrox}{Invadrox}{img:invadrox}

Invadrox is a clone of the classic arcade game Space Invaders.
Kill those pesky aliens before they get to you. Remember, they
increase speed, drop down and reverse direction after every pass!

\begin{btnmap}
      \opt{IRIVER_H100_PAD,IRIVER_H300_PAD,IPOD_4G_PAD,IPOD_3G_PAD%
          ,IPOD_1G2G_PAD,IAUDIO_X5_PAD,GIGABEAT_PAD,SANSA_E200_PAD%
          ,SANSA_FUZE_PAD,GIGABEAT_S_PAD,IRIVER_H10_PAD,PBELL_VIBE500_PAD%
          ,SANSA_FUZEPLUS_PAD,SAMSUNG_YH92X_PAD}
          {\ButtonLeft}
      \opt{COWON_D2_PAD}{\TouchMidLeft{}, \TouchBottomLeft{} or \ButtonMinus}
      \opt{MPIO_HD300_PAD}{\ButtonRew}
       \opt{HAVEREMOTEKEYMAP}{& }
    & Move left \\
    %
      \opt{IRIVER_H100_PAD,IRIVER_H300_PAD,IPOD_4G_PAD,IPOD_3G_PAD%
          ,IPOD_1G2G_PAD,IAUDIO_X5_PAD,GIGABEAT_PAD,SANSA_E200_PAD%
          ,SANSA_FUZE_PAD,GIGABEAT_S_PAD,IRIVER_H10_PAD,PBELL_VIBE500_PAD%
          ,SANSA_FUZEPLUS_PAD,SAMSUNG_YH92X_PAD}
          {\ButtonRight}
      \opt{MPIO_HD300_PAD}{\ButtonFF}
      \opt{COWON_D2_PAD}{\TouchMidRight{}, \TouchBottomRight{} or \ButtonPlus}
       \opt{HAVEREMOTEKEYMAP}{& }
    & Move right \\
    %
    \opt{IRIVER_H100_PAD}{\ButtonOn}
    \opt{IRIVER_H300_PAD,IPOD_4G_PAD,IPOD_3G_PAD,IPOD_1G2G_PAD,IAUDIO_X5_PAD%
        ,GIGABEAT_PAD,SANSA_E200_PAD,SANSA_FUZE_PAD,GIGABEAT_S_PAD%
        ,SANSA_FUZEPLUS_PAD}{\ButtonSelect}
    \opt{IRIVER_H10_PAD,SAMSUNG_YH92X_PAD}{\ButtonPlay}
    \opt{COWON_D2_PAD}{\TouchBottomMiddle, \TouchCenter{} or \ButtonMenu}
    \opt{PBELL_VIBE500_PAD}{\ButtonOK}
    \opt{MPIO_HD300_PAD}{\ButtonEnter}
       \opt{HAVEREMOTEKEYMAP}{& }
    & Fire \\
    \opt{IRIVER_H100_PAD,IRIVER_H300_PAD}{\ButtonOff}
    \opt{IPOD_4G_PAD,IPOD_3G_PAD,IPOD_1G2G_PAD,MPIO_HD300_PAD}{\ButtonMenu}
    \opt{IRIVER_H10_PAD,IAUDIO_X5_PAD,GIGABEAT_PAD,%
        SANSA_E200_PAD,SANSA_FUZEPLUS_PAD}{\ButtonPower}
    \opt{SANSA_FUZE_PAD}{Long \ButtonHome}
    \opt{GIGABEAT_S_PAD}{\ButtonBack}
    \opt{COWON_D2_PAD}{\TouchTopLeft{} or \ButtonPower}
    \opt{PBELL_VIBE500_PAD,SAMSUNG_YH92X_PAD}{\ButtonRec}
       \opt{HAVEREMOTEKEYMAP}{& }
    & Quit\\
\end{btnmap}
}}

% $Id$ %
\subsection{Jackpot}
\screenshot{plugins/images/ss-jackpot}{Jackpot}{img:Jackpot}
This is a jackpot slot machine game. At the beginning of the game you
have 20\$.  Payouts are given when three matching symbols come up.

\begin{btnmap}
    \PluginSelect
      \opt{HAVEREMOTEKEYMAP}{& \PluginRCSelect}
    & Play\\

    \nopt{IPOD_4G_PAD,IPOD_3G_PAD}{\PluginCancel}
    \opt{IPOD_4G_PAD,IPOD_3G_PAD}{\ButtonMenu}
      \opt{HAVEREMOTEKEYMAP}{& \PluginRCCancel}
    & Exit the game \\
\end{btnmap}


\subsection{Jewels}
\screenshot{plugins/images/ss-jewels}{Jewels}{img:jewels}

Jewels is a simple yet addicting game which involves swapping
pairs of jewels in order to form connected segments of three
or more of the same type.

The goal of the game is to score as many points as possible
before running out of available moves. Higher points are
awarded to larger combos. The game advances to the next level
after every one hundred points and randomly clears several jewels.

In puzzle mode the aim of the game is to connect the puzzles, by skilful swapping pairs of jewels.

\begin{btnmap}
    \opt{RECORDER_PAD,IRIVER_H10_PAD,ONDIO_PAD,IRIVER_H100_PAD,IRIVER_H300_PAD%
        ,IPOD_4G_PAD,IPOD_3G_PAD,SANSA_E200_PAD,SANSA_FUZE_PAD,SANSA_C200_PAD,SANSA_CLIP_PAD%
        ,GIGABEAT_PAD,MROBE100_PAD,IAUDIO_X5_PAD,GIGABEAT_S_PAD,PBELL_VIBE500_PAD%
        ,SANSA_FUZEPLUS_PAD,SAMSUNG_YH92X_PAD,SAMSUNG_YH820_PAD}
    {\ButtonLeft/\ButtonRight/}
    \opt{RECORDER_PAD,ONDIO_PAD,IRIVER_H100_PAD,IRIVER_H300_PAD,SANSA_E200_PAD%
      ,SANSA_FUZE_PAD,SANSA_C200_PAD,SANSA_CLIP_PAD,GIGABEAT_PAD,MROBE100_PAD,IAUDIO_X5_PAD%
      ,GIGABEAT_S_PAD,PBELL_VIBE500_PAD,SANSA_FUZEPLUS_PAD,SAMSUNG_YH92X_PAD,SAMSUNG_YH820_PAD}
      {\ButtonUp/\ButtonDown}
    \opt{IPOD_4G_PAD}{\ButtonMenu/\ButtonPlay}
    \opt{IPOD_3G_PAD}{\ButtonScrollBack/\ButtonScrollFwd}
    \opt{IRIVER_H10_PAD}{\ButtonScrollUp/\ButtonScrollDown}
    \opt{COWON_D2_PAD}{\TouchMidLeft/\TouchMidRight/\TouchTopMiddle/\TouchBottomMiddle}
    \opt{MPIO_HD300_PAD}{\ButtonRew / \ButtonFF}
       \opt{HAVEREMOTEKEYMAP}{& }
    & Move the cursor around the jewels \\
    \opt{RECORDER_PAD,IRIVER_H10_PAD,SAMSUNG_YH92X_PAD,SAMSUNG_YH820_PAD}{\ButtonPlay}
    \opt{ONDIO_PAD}{\ButtonMenu}
    \opt{IRIVER_H100_PAD,IRIVER_H300_PAD,IPOD_4G_PAD,IPOD_3G_PAD,SANSA_E200_PAD%
      ,SANSA_FUZE_PAD,SANSA_C200_PAD,SANSA_CLIP_PAD,GIGABEAT_PAD,MROBE100_PAD,IAUDIO_X5_PAD%
      ,GIGABEAT_S_PAD,SANSA_FUZEPLUS_PAD}
      {\ButtonSelect}
    \opt{COWON_D2_PAD}{\TouchCenter}
    \opt{PBELL_VIBE500_PAD}{\ButtonOK}
    \opt{MPIO_HD300_PAD}{\ButtonEnter}
       \opt{HAVEREMOTEKEYMAP}{& }
    & Select a jewel \\
    \opt{IRIVER_H10_PAD,SANSA_E200_PAD,SANSA_C200_PAD,SANSA_CLIP_PAD,GIGABEAT_PAD,MROBE100_PAD%
      ,IAUDIO_X5_PAD,COWON_D2_PAD,SANSA_FUZEPLUS_PAD}{\ButtonPower}
    \opt{SANSA_FUZE_PAD}{Long \ButtonHome}
    \opt{GIGABEAT_S_PAD}{\ButtonBack}
    \opt{IPOD_3G_PAD,MPIO_HD300_PAD}{\ButtonMenu}
    \opt{IPOD_4G_PAD}{\ButtonSelect+ \ButtonMenu}
    \opt{RECORDER_PAD,ONDIO_PAD,IRIVER_H100_PAD,IRIVER_H300_PAD}{\ButtonOff}
    \opt{PBELL_VIBE500_PAD}{\ButtonRec}
    \opt{SAMSUNG_YH92X_PAD,SAMSUNG_YH820_PAD}{\ButtonRew}
       \opt{HAVEREMOTEKEYMAP}{& }
     & Menu \\
\end{btnmap}


\subsection{Maze}

This is a simple maze generator that creates perfect mazes that have only
one solution.

\begin{btnmap}
    \opt{IPOD_3G_PAD}{\ButtonScrollBack/\ButtonScrollFwd/\ButtonLeft/\ButtonRight}
    \nopt{IPOD_3G_PAD}{\PluginUp/\PluginDown/\PluginLeft/\PluginRight}
        &
    \opt{HAVEREMOTEKEYMAP}{
        \PluginRCUp, \PluginRCDown, \PluginRCLeft, \PluginRCRight
        &}
    Navigate maze
    \\

    \opt{IPOD_3G_PAD}{\ButtonMenu}
    \nopt{IPOD_3G_PAD}{\PluginCancel}
        &
    \opt{HAVEREMOTEKEYMAP}{\PluginRCCancel
        &}
    Exit plugin
    \\

    \opt{IPOD_3G_PAD}{Long \ButtonSelect}
    \nopt{IPOD_3G_PAD}{\PluginSelectRepeat}
        &
    \opt{HAVEREMOTEKEYMAP}{\PluginRCSelectRepeat
        &}
    New Maze
    \\

    \opt{IPOD_3G_PAD}{\ButtonSelect}
    \nopt{IPOD_3G_PAD}{\PluginSelect}
        &
    \opt{HAVEREMOTEKEYMAP}{\PluginRCSelect
        &}
    Display solution (toggle)
    \\
\end{btnmap}


\subsection{MazezaM}
\screenshot{plugins/images/ss-mazezam}{MazezaM}{fig:mazezam}

The goal of this puzzle game is to escape a dungeon consisting of ten
``mazezams''.
These are rooms containing rows of blocks which can be shifted left or
right.
You can move the rows only by pushing them and if you move the rows
carelessly, you will get stuck.
You can have another go by selecting ``retry level'' from the menu,
but this will cost you a life.
You start the game with three lives.
Luckily, there are checkpoints at levels four and eight.

\begin{btnmap}
    \opt{IPOD_4G_PAD,IPOD_3G_PAD}{\ButtonScrollBack, \ButtonScrollFwd, \ButtonLeft, \ButtonRight}
    \nopt{IPOD_4G_PAD,IPOD_3G_PAD}{\PluginUp, \PluginDown, \PluginLeft, \PluginRight}
        &
    \opt{HAVEREMOTEKEYMAP}{
        \PluginRCUp, \PluginRCDown, \PluginRCLeft, \PluginRCRight
        &}
    Move Character
    \\

    \opt{IPOD_4G_PAD,IPOD_3G_PAD}{\ButtonMenu}
    \nopt{IPOD_4G_PAD,IPOD_3G_PAD}{\PluginCancel}
        &
    \opt{HAVEREMOTEKEYMAP}{\PluginRCCancel
        &}
    Menu
    \\
\end{btnmap}


% $Id$ %
\subsection{Minesweeper}
\screenshot{plugins/images/ss-minesweeper}{Minesweeper plugin}{img:minesweeper}
The classic game of minesweeper.
The aim of the game is to uncover all of the squares on the board.  If a
mine is uncovered then the game is over. If a mine is not uncovered,
then the number of mines adjacent to the current square is revealed. 
The aim is to use the information you are given to work out where the
mines are and avoid them.  When the player is certain that they know
the location of a mine, it can be tagged to avoid accidentally
``stepping'' on it.

\begin{btnmap}
  \opt{IPOD_4G_PAD,IPOD_3G_PAD}{\ButtonMenu{} / \ButtonPlay{} / \ButtonLeft{} / \ButtonRight}
  \opt{IRIVER_H100_PAD,IRIVER_H300_PAD,IAUDIO_X5_PAD%
      ,SANSA_E200_PAD,SANSA_FUZE_PAD,SANSA_C200_PAD,SANSA_CLIP_PAD,GIGABEAT_PAD,GIGABEAT_S_PAD%
      ,MROBE100_PAD,PBELL_VIBE500_PAD,SANSA_FUZEPLUS_PAD,SAMSUNG_YH92X_PAD,SAMSUNG_YH820_PAD}
      {\ButtonUp{} / \ButtonDown{} / \ButtonLeft{} / \ButtonRight}
  \opt{IRIVER_H10_PAD}
      {\ButtonScrollUp{} / \ButtonScrollDown{} / \ButtonLeft{} / \ButtonRight}
  \opt{COWON_D2_PAD}
      {\TouchTopMiddle{} / \TouchBottomMiddle{} / \TouchMidLeft{} / \TouchMidRight}
  \opt{MPIO_HD300_PAD}{\ButtonScrollUp / \ButtonScrollDown / \ButtonRew / \ButtonFF}
       \opt{HAVEREMOTEKEYMAP}{& }
  & Move the cursor across the minefield \\
  %
  \opt{IPOD_4G_PAD,IPOD_3G_PAD,SANSA_E200_PAD,SANSA_FUZEPLUS_PAD}{
     \opt{IPOD_4G_PAD,IPOD_3G_PAD,SANSA_E200_PAD}{\ButtonScrollFwd{} / \ButtonScrollBack}
     \opt{SANSA_FUZEPLUS_PAD}{\ButtonBottomLeft / \ButtonBottomRight}
       \opt{HAVEREMOTEKEYMAP}{& }
  & Scroll through the entire minefield \\}%
  %
  \opt{IRIVER_H100_PAD,IRIVER_H300_PAD}{\ButtonOn{} / \ButtonRec}
  \opt{IPOD_4G_PAD,IPOD_3G_PAD,SANSA_FUZEPLUS_PAD}{\ButtonSelect}
  \opt{IAUDIO_X5_PAD,IRIVER_H10_PAD,GIGABEAT_S_PAD,PBELL_VIBE500_PAD,SAMSUNG_YH92X_PAD%
      ,SAMSUNG_YH820_PAD}{\ButtonPlay}
  \opt{SANSA_E200_PAD}{\ButtonRec}
  \opt{SANSA_FUZE_PAD}{\ButtonScrollFwd}
  \opt{SANSA_C200_PAD,SANSA_CLIP_PAD}{\ButtonSelect{} / \ButtonVolDown}
  \opt{GIGABEAT_PAD}{\ButtonA}
  \opt{MROBE100_PAD}{\ButtonDisplay}
  \opt{COWON_D2_PAD}{\TouchCenter}
  \opt{MPIO_HD300_PAD}{\ButtonEnter}
       \opt{HAVEREMOTEKEYMAP}{& }
  & Toggle flag on / off \\
  %
  \opt{IRIVER_H100_PAD,IRIVER_H300_PAD,IAUDIO_X5_PAD,SANSA_E200_PAD%
      ,SANSA_FUZE_PAD,GIGABEAT_PAD,GIGABEAT_S_PAD,MROBE100_PAD}{\ButtonSelect}
  \opt{IPOD_4G_PAD,IPOD_3G_PAD}{Long \ButtonSelect}
  \opt{IRIVER_H10_PAD}{\ButtonRew}
  \opt{SAMSUNG_YH92X_PAD,SAMSUNG_YH820_PAD}{\ButtonFF}
  \opt{SANSA_C200_PAD,SANSA_CLIP_PAD}{Long \ButtonSelect{} / \ButtonVolUp}
  \opt{COWON_D2_PAD}{\TouchBottomLeft}
  \opt{PBELL_VIBE500_PAD}{\ButtonOK}
  \opt{MPIO_HD300_PAD,SANSA_FUZEPLUS_PAD}{\ButtonPlay}
       \opt{HAVEREMOTEKEYMAP}{& }
  & Reveal the contents of the current square \\
  %
  \opt{IRIVER_H100_PAD,IRIVER_H300_PAD}{\ButtonMode}
  \opt{IPOD_4G_PAD,IPOD_3G_PAD}{\ButtonSelect+\ButtonPlay}
  \opt{IAUDIO_X5_PAD,SAMSUNG_YH820_PAD}{\ButtonRec}
  \opt{SANSA_C200_PAD,SANSA_CLIP_PAD}{\ButtonSelect+\ButtonUp}
  \opt{SANSA_E200_PAD}{Long \ButtonRec}
  \opt{SANSA_FUZE_PAD}{\ButtonScrollBack}
  \opt{IRIVER_H10_PAD}{\ButtonRew+\ButtonPlay}
  \opt{SANSA_FUZEPLUS_PAD}{\ButtonBack}
  \opt{GIGABEAT_PAD,GIGABEAT_S_PAD,MROBE100_PAD,PBELL_VIBE500_PAD}{\ButtonMenu}
  \opt{COWON_D2_PAD}{\TouchBottomRight}
  \opt{MPIO_HD300_PAD}{\ButtonMenu}
  \opt{SAMSUNG_YH92X_PAD}{Long \ButtonPlay}
       \opt{HAVEREMOTEKEYMAP}{& }
  & Display the current game status \\
  %
  \opt{IRIVER_H100_PAD,IRIVER_H300_PAD}{\ButtonOff}
  \opt{IAUDIO_X5_PAD,IRIVER_H10_PAD,SANSA_E200_PAD,SANSA_C200_PAD,SANSA_CLIP_PAD,GIGABEAT_PAD%
      ,MROBE100_PAD,COWON_D2_PAD,SANSA_FUZEPLUS_PAD}{\ButtonPower}
  \opt{SANSA_FUZE_PAD}{Long \ButtonHome}
  \opt{IPOD_4G_PAD,IPOD_3G_PAD}{\ButtonSelect+\ButtonMenu}
  \opt{GIGABEAT_S_PAD}{\ButtonBack}
  \opt{PBELL_VIBE500_PAD}{\ButtonRec}
  \opt{SAMSUNG_YH92X_PAD,SAMSUNG_YH820_PAD}{Long \ButtonRew}
  \opt{MPIO_HD300_PAD}{Long \ButtonMenu}
       \opt{HAVEREMOTEKEYMAP}{& 
          \opt{IRIVER_RC_H100_PAD}{\ButtonRCStop}
        }
  & Exit the game \\
\end{btnmap}


\opt{iriverh100,iaudiom5,lcd_color}{\nopt{c200,c200v2}{% $Id$ %
\subsection{Pacbox}
\screenshot{plugins/images/ss-pacbox}{Pacbox}{img:pacbox}
Pacbox is an emulator of the Pacman arcade machine hardware. It is a port of
\emph{PIE -- Pacman Instructional Emulator}
(\url{http://www.ascotti.org/programming/pie/pie.htm}).


\subsubsection{ROMs}
To use the emulator to play Pacman, you need a copy of ROMs for 
``Midway Pacman''.
\begin{table}
  \begin{rbtabular}{0.8\textwidth}{lX}{\textbf{Filename} & \textbf{MD5 checksum}}{}{}
    pacman.5e & 2791455babaf26e0b396c78d2b45f8f6\\
    pacman.5f & 9240f35d1d2beee0ff17195653b5e405\\
    pacman.6e & 290aa5eae9e2f63587b5dd5a7da932da\\
    pacman.6f & 19a886fcd8b5e88b0ed1b97f9d8659c0\\
    pacman.6h & d7cce8bffd9563b133ec17ebbb6373d4\\
    pacman.6j & 33c0e197be4c787142af6c3be0d8f6b0\\
  \end{rbtabular}
\end{table}

These need to be stored in the \fname{/.rockbox/pacman/} directory on your 
\dap. In the MAME ROMs collection the necessary files can be found in 
\fname{pacman.zip} and \fname{puckman.zip}. The MAME project itself can be 
found at \url{http://www.mame.net}.

\subsubsection{Keys}
\begin{btnmap}
    % 20GB H10 and 5/6GB H10 have different direction key mappings to match the
    % orientation of the playing field on their different displays - don't use *_PAD !
    \opt{IRIVER_H100_PAD,IRIVER_H300_PAD,IAUDIO_X5_PAD,IPOD_4G_PAD,%
        IPOD_3G_PAD,iriverh10,MROBE100_PAD,SANSA_FUZE_PAD}{\ButtonRight}
    \opt{GIGABEAT_PAD,GIGABEAT_S_PAD,SANSA_E200_PAD,PBELL_VIBE500_PAD%
         ,SANSA_FUZEPLUS_PAD,SAMSUNG_YH92X_PAD}{\ButtonUp}
    \opt{iriverh10_5gb}{\ButtonScrollUp}
    \opt{COWON_D2_PAD}{\TouchTopMiddle}
      \opt{HAVEREMOTEKEYMAP}{& }
    & Move Up\\
    \opt{IRIVER_H100_PAD,IRIVER_H300_PAD,IAUDIO_X5_PAD,IPOD_4G_PAD,%
    IPOD_3G_PAD,iriverh10,MROBE100_PAD,SANSA_FUZE_PAD}{\ButtonLeft}
    \opt{iriverh10_5gb}{\ButtonScrollDown}
    \opt{GIGABEAT_PAD,GIGABEAT_S_PAD,SANSA_E200_PAD,PBELL_VIBE500_PAD,SANSA_FUZEPLUS_PAD%
        ,SAMSUNG_YH92X_PAD}{\ButtonDown}
    \opt{COWON_D2_PAD}{\TouchBottomMiddle}
      \opt{HAVEREMOTEKEYMAP}{& }
    & Move Down\\
    \opt{IRIVER_H100_PAD,IRIVER_H300_PAD,IAUDIO_X5_PAD,MROBE100_PAD,SANSA_FUZE_PAD}{\ButtonUp}
    \opt{IPOD_4G_PAD,IPOD_3G_PAD}{\ButtonMenu}
    \opt{iriverh10}{\ButtonScrollUp}
    \opt{iriverh10_5gb,GIGABEAT_PAD,GIGABEAT_S_PAD,SANSA_E200_PAD,PBELL_VIBE500_PAD%
         ,SANSA_FUZEPLUS_PAD,SAMSUNG_YH92X_PAD}{\ButtonLeft}
    \opt{COWON_D2_PAD}{\TouchMidLeft}
      \opt{HAVEREMOTEKEYMAP}{& }
    & Move Left\\
    \opt{IRIVER_H100_PAD,IRIVER_H300_PAD,IAUDIO_X5_PAD,MROBE100_PAD,SANSA_FUZE_PAD}{\ButtonDown}
    \opt{IPOD_4G_PAD,IPOD_3G_PAD}{\ButtonPlay}
    \opt{iriverh10}{\ButtonScrollDown}
    \opt{iriverh10_5gb,GIGABEAT_PAD,GIGABEAT_S_PAD,SANSA_E200_PAD,PBELL_VIBE500_PAD%
         ,SANSA_FUZEPLUS_PAD,SAMSUNG_YH92X_PAD}{\ButtonRight}
    \opt{COWON_D2_PAD}{\TouchMidRight}
      \opt{HAVEREMOTEKEYMAP}{& }
    & Move Right\\
    \opt{IRIVER_H100_PAD,IRIVER_H300_PAD,IAUDIO_X5_PAD}{\ButtonRec}
    \opt{IPOD_4G_PAD,IPOD_3G_PAD}{\ButtonSelect}
    \opt{IRIVER_H10_PAD}{\ButtonFF}
    \opt{SANSA_E200_PAD,SANSA_FUZE_PAD}{\ButtonSelect+\ButtonDown}
    \opt{GIGABEAT_PAD}{\ButtonA}
    \opt{MROBE100_PAD}{\ButtonDisplay}
    \opt{GIGABEAT_S_PAD,SANSA_FUZEPLUS_PAD,SAMSUNG_YH92X_PAD}{\ButtonPlay}
    \opt{COWON_D2_PAD}{\TouchCenter}
    \opt{PBELL_VIBE500_PAD}{\ButtonOK}
      \opt{HAVEREMOTEKEYMAP}{& }
    & Insert Coin\\
    \opt{IRIVER_H100_PAD,IRIVER_H300_PAD,IAUDIO_X5_PAD,IPOD_4G_PAD,IPOD_3G_PAD%
        ,SANSA_E200_PAD,SANSA_FUZE_PAD,GIGABEAT_PAD,GIGABEAT_S_PAD%
        ,SANSA_FUZEPLUS_PAD}{\ButtonSelect}
    \opt{IRIVER_H10_PAD}{\ButtonRew}
    \opt{COWON_D2_PAD}{\TouchBottomLeft}
    \opt{PBELL_VIBE500_PAD}{\ButtonPlay}
    \opt{SAMSUNG_YH92X_PAD}{\ButtonFF}
      \opt{HAVEREMOTEKEYMAP}{& }
    & 1-Player Start\\
    \opt{IRIVER_H100_PAD,IRIVER_H300_PAD}{\ButtonOn}
    \opt{IPOD_4G_PAD,IPOD_3G_PAD}{n/a}
    \opt{IAUDIO_X5_PAD,IRIVER_H10_PAD,GIGABEAT_PAD,GIGABEAT_S_PAD}{\ButtonPower}
    \opt{SANSA_E200_PAD,PBELL_VIBE500_PAD}{\ButtonRec}
    \opt{MROBE100_PAD}{\ButtonMenu}
    \opt{SANSA_FUZEPLUS_PAD}{\ButtonBottomRight}
    \opt{COWON_D2_PAD}{\TouchBottomRight}
    \opt{SAMSUNG_YH92X_PAD}{\ButtonRew}
      \opt{HAVEREMOTEKEYMAP}{& }
    & 2-Player Start\\
    \opt{IRIVER_H100_PAD,IRIVER_H300_PAD}{\ButtonMode}
    \opt{IPOD_4G_PAD,IPOD_3G_PAD}{\ButtonSelect+\ButtonMenu}
    \opt{IAUDIO_X5_PAD,IRIVER_H10_PAD,MROBE100_PAD}{\ButtonPlay}
    \opt{SANSA_E200_PAD,SANSA_FUZEPLUS_PAD}{\ButtonPower}
    \opt{SANSA_FUZE_PAD}{\ButtonHome}
    \opt{GIGABEAT_PAD,GIGABEAT_S_PAD,COWON_D2_PAD,PBELL_VIBE500_PAD}{\ButtonMenu}
    \opt{SAMSUNG_YH92X_PAD}{\ButtonRec}
      \opt{HAVEREMOTEKEYMAP}{& }
    & Menu\\
\end{btnmap}
}}

% $Id: pegbox.tex 16693 2008-03-18 09:24:35Z roolku $ %
\subsection{Pegbox}
\screenshot{plugins/images/ss-pegbox}{pegbox}{img:pegbox}
To beat each level, you must destroy all of the pegs. If two like pegs are
pushed into each other they disappear except for triangles which form a solid 
block and crosses which allow you to choose a replacement block.

\begin{btnmap}
    \nopt{IPOD_4G_PAD,IPOD_3G_PAD,IRIVER_H10_PAD,touchscreen,IAUDIO_M3_PAD%
        ,MPIO_HD200_PAD,MPIO_HD300_PAD,XDUOO_X3_PAD}{%
        \ButtonUp, \ButtonDown, }%
    \opt{IPOD_4G_PAD,IPOD_3G_PAD}{\ButtonMenu, \ButtonPlay, }%
    \opt{IRIVER_H10_PAD,MPIO_HD300_PAD}{\ButtonScrollUp, \ButtonScrollDown, }%
    \opt{MPIO_HD200_PAD}{\ButtonRew, \ButtonFF, }%
    \opt{touchscreen}{\TouchTopMiddle, \TouchBottomMiddle, }%
    \opt{XDUOO_X3_PAD}{\ButtonPrev, \ButtonNext,}%
    %
    \nopt{touchscreen,MPIO_HD200_PAD,MPIO_HD300_PAD,IAUDIO_M3_PAD}{\ButtonLeft, \ButtonRight}
    \opt{MPIO_HD200_PAD}{\ButtonVolDown, \ButtonVolUp}
    \opt{MPIO_HD300_PAD}{\ButtonRew, \ButtonFF}
    \opt{touchscreen}{\TouchMidLeft, \TouchMidRight}
    \opt{XDUOO_X3_PAD}{\ButtonHome, \ButtonOption}
        &
    \opt{HAVEREMOTEKEYMAP}{
        &}
    to move around
        \\

    \nopt{IPOD_4G_PAD,IPOD_3G_PAD,IRIVER_H10_PAD%
        ,PBELL_VIBE500_PAD,touchscreen,IAUDIO_M3_PAD,MPIO_HD200_PAD%
        ,MPIO_HD300_PAD,SAMSUNG_YH92X_PAD,SAMSUNG_YH820_PAD}{\ButtonSelect}
    \opt{IPOD_4G_PAD,IPOD_3G_PAD}{\ButtonSelect{} + \ButtonRight}
    \opt{IRIVER_H10_PAD,PBELL_VIBE500_PAD,SAMSUNG_YH92X_PAD,SAMSUNG_YH820_PAD,XDUOO_X3_PAD}{\ButtonPlay}
    \opt{MPIO_HD200_PAD}{\ButtonFunc}
    \opt{MPIO_HD300_PAD}{\ButtonEnter}
    \opt{touchscreen}{\TouchCenter}
        &
    \opt{HAVEREMOTEKEYMAP}{
        &}
    to choose peg
        \\

    \opt{IRIVER_H100_PAD,IRIVER_H300_PAD}{\ButtonOn}
    \opt{IPOD_4G_PAD,IPOD_3G_PAD,SANSA_FUZE_PAD}
        {\ButtonSelect{} + \ButtonLeft}
    \opt{IAUDIO_X5_PAD,SANSA_C200_PAD,SANSA_E200_PAD}{\ButtonRec}
    \opt{SANSA_CLIP_PAD}{\ButtonHome}
    \opt{IRIVER_H10_PAD}{Long \ButtonFF}
    \opt{GIGABEAT_PAD}{\ButtonA}
    \opt{GIGABEAT_S_PAD,PBELL_VIBE500_PAD}{\ButtonMenu}
    \opt{MROBE100_PAD}{\ButtonPlay}
    \opt{MPIO_HD200_PAD}{\ButtonRec}
    \opt{MPIO_HD300_PAD}{\ButtonMenu}
    \opt{SANSA_FUZEPLUS_PAD}{\ButtonBack}
    \opt{SAMSUNG_YH92X_PAD,SAMSUNG_YH820_PAD}{\ButtonFF}
    \opt{touchscreen}{\TouchTopRight}
    \opt{XDUOO_X3_PAD}{\ButtonPower + \ButtonHome}
        &
    \opt{HAVEREMOTEKEYMAP}{
        &}
    to restart level
        \\

    \opt{IRIVER_H100_PAD,IRIVER_H300_PAD}{\ButtonMode}
    \opt{IPOD_4G_PAD,IPOD_3G_PAD}{\ButtonSelect{} + \ButtonMenu}
    \opt{IAUDIO_X5_PAD,MPIO_HD300_PAD}{\ButtonPlay}
    \opt{IRIVER_H10_PAD}{\ButtonFF{} + \ButtonScrollUp}
    \opt{SANSA_E200_PAD}{\ButtonScrollBack}
    \opt{GIGABEAT_PAD,GIGABEAT_S_PAD,SANSA_C200_PAD,SANSA_CLIP_PAD,SANSA_FUZEPLUS_PAD}{\ButtonVolUp}
    \opt{MROBE100_PAD}{\ButtonMenu}
    \opt{COWON_D2_PAD}{\TouchBottomLeft}
    \opt{PBELL_VIBE500_PAD}{\ButtonOK}
    \opt{MPIO_HD200_PAD}{\ButtonPlay + \ButtonFF}
    \opt{SAMSUNG_YH92X_PAD}{\ButtonPlay+\ButtonUp}
    \opt{SAMSUNG_YH820_PAD}{\ButtonRec+\ButtonUp}
    \opt{XDUOO_X3_PAD}{\ButtonVolUp}
        &
    \opt{HAVEREMOTEKEYMAP}{
        &}
    to go up a level
        \\

    \nopt{IPOD_4G_PAD,IPOD_3G_PAD,IAUDIO_X5_PAD}{
        \opt{IRIVER_H100_PAD,IRIVER_H300_PAD,MPIO_HD300_PAD}{\ButtonRec}
        \opt{IRIVER_H10_PAD}{\ButtonFF{} + \ButtonScrollDown}
        \opt{SANSA_E200_PAD}{\ButtonScrollFwd}
        \opt{SANSA_FUZE_PAD}{\ButtonSelect{} + \ButtonDown}
        \opt{SANSA_C200_PAD,SANSA_CLIP_PAD,GIGABEAT_PAD,GIGABEAT_S_PAD,SANSA_FUZEPLUS_PAD}{\ButtonVolDown}
        \opt{MROBE100_PAD}{\ButtonDisplay}
        \opt{touchscreen}{\TouchBottomRight}
        \opt{PBELL_VIBE500_PAD}{\ButtonCancel}
        \opt{MPIO_HD200_PAD}{\ButtonPlay{} + \ButtonRew}
        \opt{SAMSUNG_YH92X_PAD}{\ButtonPlay+\ButtonDown}
        \opt{SAMSUNG_YH820_PAD}{\ButtonRec+\ButtonDown}
        \opt{XDUOO_X3_PAD}{\ButtonVolDown}
            &
        \opt{HAVEREMOTEKEYMAP}{
            &}
        to go down a level
            \\
    }

    \opt{IRIVER_H100_PAD,IRIVER_H300_PAD}{\ButtonOff}
    \opt{IPOD_4G_PAD,IPOD_3G_PAD}{\ButtonSelect{} + \ButtonPlay}
    \opt{IAUDIO_X5_PAD,IRIVER_H10_PAD,SANSA_E200_PAD,GIGABEAT_PAD,MROBE100_PAD%
        ,SANSA_C200_PAD,SANSA_CLIP_PAD,COWON_D2_PAD,SANSA_FUZEPLUS_PAD}{\ButtonPower}
    \opt{SANSA_FUZE_PAD}{Long \ButtonHome}
    \opt{GIGABEAT_S_PAD}{\ButtonBack}
    \opt{PBELL_VIBE500_PAD}{\ButtonRec}
    \opt{SAMSUNG_YH92X_PAD,SAMSUNG_YH820_PAD}{\ButtonRew}
    \opt{MPIO_HD200_PAD}{\ButtonRec + \ButtonPlay}
    \opt{MPIO_HD300_PAD}{Long \ButtonMenu}
    \opt{XDUOO_X3_PAD}{\ButtonPower}
        &
    \opt{HAVEREMOTEKEYMAP}{
        &}
    to quit
        \\

\end{btnmap}


\opt{lcd_color}{\opt{large_plugin_buffer}{\subsection{Pixel Painter}
\screenshot{plugins/images/ss-pixelpainter}{Pixel Painter}{img:pixelpainter}
This game is written in LUA and based on the game of the same name by
Pavel Bakhilau (\url{http://js1k.com/2010-first/demo/453}).

Select a colour to flood-fill the board with that colour, starting from the
top-left pixel (meaning that any pixel which is connected to the top-left
through other pixels of the same colour will be changed to the selected colour).
Try to paint the entire board with as few moves as possible.

\begin{btnmap}
    \ifnum\dapdisplaywidth<\dapdisplayheight
        \PluginLeft{} / \PluginRight
    \else
        \PluginUp{} / \PluginDown
    \fi
    & Move colour selector\\

    \PluginSelect
    & Fill screen with selected colour\\

    \PluginCancel, \PluginExit
    & Enter game menu\\
\end{btnmap}
}}

\subsection{Pong}
\screenshot{plugins/images/ss-pong}{Pong}{img:pong}
Pong is a simple one or two player ``tennis game''. Whenever a player misses the ball the other scores. 

The game starts in demo mode, with the CPU controlling both sides.

As soon as a button to control one of the paddles is pressed, control of that paddle passes to the player,
so for a single player game, just press the appropriate buttons to control the side you want to play. For
a two player game, both players should just press the appropriate buttons for their side.

\begin{btnmap}
    \opt{RECORDER_PAD}{\ButtonFOne}
    \opt{ONDIO_PAD,SANSA_E200_PAD,SANSA_FUZE_PAD,SANSA_FUZE_PAD}{\ButtonLeft}
    \opt{IRIVER_H100_PAD,IRIVER_H300_PAD,IAUDIO_X5_PAD,GIGABEAT_PAD%
      ,GIGABEAT_S_PAD,SAMSUNG_YH92X_PAD}{\ButtonUp}
    \opt{IPOD_4G_PAD,IPOD_3G_PAD,PBELL_VIBE500_PAD}{\ButtonMenu}
    \opt{IRIVER_H10_PAD}{\ButtonScrollUp}
    \opt{SANSA_C200_PAD,SANSA_CLIP_PAD}{\ButtonVolUp}
    \opt{MROBE100_PAD}{\ButtonMenu}
    \opt{COWON_D2_PAD}{\TouchTopLeft}
    \opt{MPIO_HD300_PAD}{\ButtonRew}
    \opt{SANSA_FUZEPLUS_PAD}{\ButtonBack}
      \opt{HAVEREMOTEKEYMAP}{& }
    & Left player up\\
    \opt{RECORDER_PAD,IPOD_4G_PAD,IPOD_3G_PAD}{\ButtonLeft}
    \opt{ONDIO_PAD}{\ButtonMenu}
    \opt{IRIVER_H100_PAD,IRIVER_H300_PAD,IAUDIO_X5_PAD,SANSA_E200_PAD%
      ,SANSA_FUZE_PAD,GIGABEAT_PAD,GIGABEAT_S_PAD,SAMSUNG_YH92X_PAD}{\ButtonDown}
    \opt{IRIVER_H10_PAD}{\ButtonScrollDown}
    \opt{SANSA_C200_PAD,SANSA_CLIP_PAD}{\ButtonVolDown}
    \opt{MROBE100_PAD,PBELL_VIBE500_PAD}{\ButtonLeft}
    \opt{SANSA_FUZEPLUS_PAD}{\ButtonBottomLeft}
    \opt{COWON_D2_PAD}{\TouchBottomLeft}
    \opt{MPIO_HD300_PAD}{\ButtonRec}
      \opt{HAVEREMOTEKEYMAP}{& }
    & Left player down\\
    \opt{RECORDER_PAD}{\ButtonFThree}
    \opt{ONDIO_PAD,SANSA_E200_PAD,SANSA_FUZE_PAD,SANSA_C200_PAD,SANSA_CLIP_PAD}{\ButtonUp}
    \opt{IRIVER_H100_PAD,IRIVER_H300_PAD}{\ButtonOn}
    \opt{IPOD_4G_PAD,IPOD_3G_PAD}{\ButtonRight}
    \opt{IAUDIO_X5_PAD}{\ButtonRec}
    \opt{IRIVER_H10_PAD}{\ButtonRew}
    \opt{GIGABEAT_PAD,GIGABEAT_S_PAD}{\ButtonVolUp}
    \opt{MROBE100_PAD,PBELL_VIBE500_PAD,SANSA_FUZEPLUS_PAD}{\ButtonPlay}
    \opt{COWON_D2_PAD}{\TouchTopRight}
    \opt{MPIO_HD300_PAD,SAMSUNG_YH92X_PAD}{\ButtonFF}
      \opt{HAVEREMOTEKEYMAP}{& }
    & Right player up\\
    \opt{RECORDER_PAD,SANSA_E200_PAD,SANSA_FUZE_PAD}{\ButtonRight}
    \opt{ONDIO_PAD,SANSA_C200_PAD,SANSA_CLIP_PAD}{\ButtonDown}
    \opt{IRIVER_H100_PAD,IRIVER_H300_PAD}{\ButtonMode}
    \opt{IPOD_4G_PAD,IPOD_3G_PAD,IAUDIO_X5_PAD}{\ButtonPlay}
    \opt{IRIVER_H10_PAD}{\ButtonFF}
    \opt{GIGABEAT_PAD,GIGABEAT_S_PAD}{\ButtonVolDown}
    \opt{MROBE100_PAD,PBELL_VIBE500_PAD}{\ButtonRight}
    \opt{COWON_D2_PAD}{\TouchBottomRight}
    \opt{MPIO_HD300_PAD}{\ButtonPlay}
    \opt{SANSA_FUZEPLUS_PAD}{\ButtonBottomRight}
    \opt{SAMSUNG_YH92X_PAD}{\ButtonRew}
      \opt{HAVEREMOTEKEYMAP}{& }
    & Right player down\\
    \opt{RECORDER_PAD,ONDIO_PAD,IRIVER_H100_PAD,IRIVER_H300_PAD}{\ButtonOff}
    \opt{IPOD_4G_PAD,IPOD_3G_PAD}{\ButtonSelect}
    \opt{IAUDIO_X5_PAD,IRIVER_H10_PAD,SANSA_E200_PAD,SANSA_C200_PAD,SANSA_CLIP_PAD%
      ,GIGABEAT_PAD,MROBE100_PAD,COWON_D2_PAD,SANSA_FUZEPLUS_PAD}{\ButtonPower}
    \opt{SANSA_FUZE_PAD}{\ButtonHome}
    \opt{GIGABEAT_S_PAD}{\ButtonBack}
    \opt{PBELL_VIBE500_PAD,SAMSUNG_YH92X_PAD}{\ButtonRec}
    \opt{MPIO_HD300_PAD}{Long \ButtonMenu}
      \opt{HAVEREMOTEKEYMAP}{& }
    & Quit\\
\end{btnmap}


\opt{lcd_color}{\nopt{lowmem,iaudiox5,iriverh300}{\subsection{Quake}

This is id Software's \emph{Quake}, first released in 1996. The game
features a Gothic atmosphere and full 3D graphics.

\subsubsection{Installation}
Copy the game data files into \fname{/.rockbox/quake} on your
device. There should be an \fname{id1/} directory with a {.pak}
file. The shareware version is known to work.

\subsubsection{Configuration}
It is possible to customize the game controls through the
\emph{Options > Customize Controls} menu. Currently a limitation
exists which prevents one key from being bound to more than one
function.
}}

\input{plugins/reversi.tex}

% $Id$ %
\subsection{Robotfindskitten}
\screenshot{plugins/images/ss-robotfindskitten}{Robotfindskitten}{img:robotfindskitten}
In this game, you are robot (\#). Your job is to find kitten. This task
is complicated by the existence of various things which are not kitten.
Robot must touch items to determine if they are kitten or not. The game
ends when robotfindskitten.

\begin{btnmap}
    \PluginUp, \PluginDown, \PluginLeft, \PluginRight
      \opt{HAVEREMOTEKEYMAP}{& \PluginRCUp, \PluginRCDown, \PluginRCLeft, \PluginRCRight}
    & Move robot\\

    \nopt{IPOD_4G_PAD,IPOD_3G_PAD}{\PluginCancel}
    \opt{IPOD_4G_PAD,IPOD_3G_PAD}{Long \ButtonSelect}
      \opt{HAVEREMOTEKEYMAP}{& \PluginRCCancel}
    & Quit\\
\end{btnmap}


% $Id$ %
\subsection{Rockblox}
\screenshot{plugins/images/ss-rockblox}{Rockblox}{fig:rockblox}
Rockblox is a Rockbox version of the classic falling blocks game from Russia. 
The aim of the game is to make the falling blocks of different shapes 
form full rows. Whenever a row is completed,  it will be cleared away, and you
gain points. For every ten lines completed, the game level increases, making
the blocks fall faster. If the pile of blocks reaches the ceiling, the game is over.

\begin{btnmap}
    \nopt{SANSA_FUZEPLUS_PAD,SAMSUNG_YH92X_PAD}{
        \opt{IRIVER_H100_PAD,IRIVER_H300_PAD}{\ButtonOn}
        \opt{IPOD_4G_PAD,IPOD_3G_PAD}{\ButtonSelect+\ButtonPlay}
        \opt{IAUDIO_X5_PAD,IRIVER_H10_PAD,GIGABEAT_S_PAD}{\ButtonPlay}
        \opt{SANSA_E200_PAD,SANSA_C200_PAD}{\ButtonRec}
        \opt{SANSA_CLIP_PAD}{\ButtonHome}
        \opt{SANSA_FUZE_PAD}{\ButtonSelect+\ButtonUp}
        \opt{GIGABEAT_PAD}{\ButtonA}
        \opt{MROBE100_PAD}{\ButtonDisplay}
        \opt{COWON_D2_PAD}{\ButtonMenu}
        \opt{PBELL_VIBE500_PAD}{\ButtonCancel}
        \opt{MPIO_HD300_PAD}{\ButtonRec}
        \opt{SAMSUNG_YH820_PAD}{\ButtonRec}
           \opt{HAVEREMOTEKEYMAP}{& }
            & Restart game\\
    }
    \opt{IRIVER_H100_PAD,IRIVER_H300_PAD%
        ,IAUDIO_X5_PAD,SANSA_E200_PAD,SANSA_FUZE_PAD,SANSA_C200_PAD,SANSA_CLIP_PAD%
        ,GIGABEAT_PAD,GIGABEAT_S_PAD,MROBE100_PAD,IPOD_4G_PAD,IPOD_3G_PAD%
        ,IRIVER_H10_PAD,PBELL_VIBE500_PAD,SANSA_FUZEPLUS_PAD,SAMSUNG_YH92X_PAD%
        ,SAMSUNG_YH820_PAD}
        {\ButtonLeft}
    \opt{COWON_D2_PAD}{\TouchMidLeft}
    \opt{MPIO_HD300_PAD}{\ButtonRew}
       \opt{HAVEREMOTEKEYMAP}{& }
        & Move left\\
    \opt{IRIVER_H100_PAD,IRIVER_H300_PAD%
        ,IAUDIO_X5_PAD,SANSA_E200_PAD,SANSA_FUZE_PAD,SANSA_C200_PAD,SANSA_CLIP_PAD%
        ,GIGABEAT_PAD,GIGABEAT_S_PAD,MROBE100_PAD,IPOD_4G_PAD,IPOD_3G_PAD%
        ,IRIVER_H10_PAD,PBELL_VIBE500_PAD,SANSA_FUZEPLUS_PAD,SAMSUNG_YH92X_PAD%
        ,SAMSUNG_YH820_PAD}
        {\ButtonRight}
    \opt{COWON_D2_PAD}{\TouchMidRight}
    \opt{MPIO_HD300_PAD}{\ButtonFF}
       \opt{HAVEREMOTEKEYMAP}{& }
        & Move right\\
    \opt{IRIVER_H100_PAD,IRIVER_H300_PAD,IAUDIO_X5_PAD%
        ,SANSA_E200_PAD,SANSA_FUZE_PAD,SANSA_C200_PAD,SANSA_CLIP_PAD,GIGABEAT_PAD%
        ,GIGABEAT_S_PAD,MROBE100_PAD,PBELL_VIBE500_PAD}
        {\ButtonDown}
    \opt{SANSA_FUZEPLUS_PAD}{\ButtonSelect}
    \opt{IPOD_4G_PAD,IPOD_3G_PAD}{\ButtonPlay}
    \opt{IRIVER_H10_PAD}{\ButtonScrollDown}
    \opt{COWON_D2_PAD}{\TouchBottomMiddle}
    \opt{MPIO_HD300_PAD}{\ButtonEnter}
    \opt{SAMSUNG_YH92X_PAD}{\ButtonRew}
    \opt{SAMSUNG_YH820_PAD}{\ButtonFF}
       \opt{HAVEREMOTEKEYMAP}{& }
        & Move down\\
    \opt{IRIVER_H100_PAD,IRIVER_H300_PAD,IAUDIO_X5_PAD}{\ButtonSelect}
    \opt{scrollwheel}{\ButtonScrollBack}
    \opt{IAUDIO_X5_PAD}{\ButtonPower}
    \opt{IRIVER_H10_PAD}{\ButtonRew}
    \opt{SANSA_C200_PAD,SANSA_CLIP_PAD}{\ButtonVolDown}
    \opt{GIGABEAT_PAD,GIGABEAT_S_PAD}{\ButtonVolUp}
    \opt{SANSA_FUZEPLUS_PAD}{\ButtonVolDown{}; \ButtonBottomLeft}
    \opt{MROBE100_PAD,PBELL_VIBE500_PAD}{\ButtonMenu}
    \opt{COWON_D2_PAD}{\TouchBottomLeft}
    \opt{MPIO_HD300_PAD}{\ButtonScrollUp}
    \opt{SAMSUNG_YH92X_PAD,SAMSUNG_YH820_PAD}{\ButtonUp}
       \opt{HAVEREMOTEKEYMAP}{& }
        & Rotate anticlockwise\\
    \opt{IRIVER_H100_PAD,IRIVER_H300_PAD,IAUDIO_X5_PAD}
        {\ButtonUp}
    \opt{IPOD_4G_PAD,IPOD_3G_PAD}{\ButtonScrollFwd{} / \ButtonMenu}
    \opt{SANSA_E200_PAD,SANSA_FUZE_PAD}{\ButtonScrollFwd}
    \opt{IRIVER_H10_PAD}{\ButtonScrollUp}
    \opt{SANSA_C200_PAD,SANSA_CLIP_PAD}{\ButtonVolUp/\ButtonUp}
    \opt{GIGABEAT_PAD,GIGABEAT_S_PAD}{\ButtonVolDown}
    \opt{MROBE100_PAD,PBELL_VIBE500_PAD}{\ButtonPlay}
    \opt{COWON_D2_PAD}{\TouchBottomRight{} / \TouchTopMiddle }
    \opt{MPIO_HD300_PAD}{\ButtonScrollDown}
    \opt{SANSA_FUZEPLUS_PAD}{\ButtonVolUp{}; \ButtonBottomRight}
    \opt{SAMSUNG_YH92X_PAD,SAMSUNG_YH820_PAD}{\ButtonDown}
       \opt{HAVEREMOTEKEYMAP}{& }
        & Rotate clockwise\\
    \opt{IRIVER_H100_PAD,IRIVER_H300_PAD}{\ButtonMode}
    \opt{IPOD_4G_PAD,IPOD_3G_PAD,SANSA_E200_PAD,SANSA_FUZE_PAD,SANSA_C200_PAD,SANSA_CLIP_PAD%
        ,GIGABEAT_PAD,GIGABEAT_S_PAD,MROBE100_PAD}{\ButtonSelect}
    \opt{IAUDIO_X5_PAD}{\ButtonRec}
    \opt{IRIVER_H10_PAD}{\ButtonFF}
    \opt{COWON_D2_PAD}{\TouchCenter}
    \opt{PBELL_VIBE500_PAD}{\ButtonOK}
    \opt{MPIO_HD300_PAD,SAMSUNG_YH92X_PAD,SAMSUNG_YH820_PAD}{\ButtonPlay}
    \opt{SANSA_FUZEPLUS_PAD}{\ButtonDown}
       \opt{HAVEREMOTEKEYMAP}{& }
        & Drop\\
    \opt{hold_button}{
        \ButtonHold{} switch
       \opt{HAVEREMOTEKEYMAP}{& }
        & Pause\\
    }
    \opt{IRIVER_H100_PAD,IRIVER_H300_PAD}{\ButtonOff}
    \opt{IPOD_4G_PAD,IPOD_3G_PAD}{Long \ButtonSelect}
    \opt{IAUDIO_X5_PAD,IRIVER_H10_PAD,SANSA_E200_PAD,SANSA_C200_PAD,SANSA_CLIP_PAD,GIGABEAT_PAD,MROBE100_PAD}{\ButtonPower}
    \opt{SANSA_FUZE_PAD}{Long \ButtonHome}
    \opt{GIGABEAT_S_PAD}{\ButtonBack}
    \opt{COWON_D2_PAD,SANSA_FUZEPLUS_PAD}{\ButtonPower}
    \opt{PBELL_VIBE500_PAD}{\ButtonRec}
    \opt{SAMSUNG_YH92X_PAD}{\ButtonFF}
    \opt{SAMSUNG_YH820_PAD}{\ButtonRew}
    \opt{MPIO_HD300_PAD}{Long \ButtonMenu}
       \opt{HAVEREMOTEKEYMAP}{& 
          \opt{IRIVER_RC_H100_PAD}{\ButtonRCStop}
        }
        & Quit\\
\end{btnmap}


% $Id$ %
\subsection{Rockblox1d}

Rockblox1d is a game for people who find rockblox too hard. In this version the
second dimension is missing so the user only has to move the bricks down. No 
horizontal moving anymore and no need to rotate the brick!

\begin{btnmap}
      \PluginDown
      \opt{HAVEREMOTEKEYMAP}{& }
        & Move down faster\\

      \nopt{IPOD_4G_PAD,IPOD_3G_PAD}{\PluginCancel{} or \PluginExit}
      \opt{IPOD_4G_PAD,IPOD_3G_PAD}{\ButtonMenu}
      \opt{HAVEREMOTEKEYMAP}{& }
        & Quit\\
\end{btnmap}


\opt{lcd_color}{\subsection{Sgt-Puzzles}
\screenshot{plugins/images/ss-puzzles-cube}{``Cube'', a rolling solid puzzle}{fig:Cube}
\screenshot{plugins/images/ss-puzzles-map}{``Map'', a 4-coloring game}{fig:Map}

The games that begin with the ``sgt-'' prefix are ports of certain
puzzles from Simon Tatham's Portable Puzzle Collection, an open source
collection of single-player puzzle games.

\note{Certain puzzles may crash when run with demanding
  configurations. To prevent this, avoid setting extreme configuration
  values.}

\subsubsection{Puzzle Documentation}
For documentation on the games included, please see the ``Extensive
Help'' menu option from inside the plugin to read puzzle-specific
instructions or visit their official website at
\url{https://www.chiark.greenend.org.uk/~sgtatham/puzzles/}.

\subsubsection{Dynamic Font Sizing}
By default, each game will only use one of two fonts in drawing: the
hard-coded system font for fixed-width text, and the theme's UI font
for variable-width text. For improved puzzle rendering, each puzzle is
capable of using a special font pack when it is installed. This font
pack is available from
\url{https://download.rockbox.org/useful/sgt-fonts.zip}. To install,
simply extract the contents of this file to the
\fname{/.rockbox/fonts/} directory on your device. Once this has been
done, each game will dynamically load and use properly-sized fonts
whenever needed.

\note{On hard disk-based devices, this may cause a slight delay as the
  disk spins up to load the fonts when a puzzle is first started, and
  after using the ``Extensive Help'' feature.}

\subsubsection{``Zoom In'' Feature}
The ``Zoom In'' feature is available as an option from the pause
menu. It has two modes: viewing mode, and interaction mode. The
current mode is indicated in the title bar at the bottom of the
screen. This feature is most useful with low-resolution devices and
large puzzles.

Viewing mode is entered when the ``Zoom In'' option is selected, or
when {\PluginCancel} is pressed in interaction mode. It allows you to
pan around an enlarged version of the game. The directional keys pan
the image by a small amount in their respective directions, and
{\PluginSelect} should toggle interaction mode. To return to the pause
menu from viewing mode, press {\PluginCancel}.

In interaction mode, activated from viewer mode by pressing
{\PluginSelect}, your device's buttons all function as they do in the
normal gameplay mode, with the exception of {\PluginCancel}, which
returns the game to viewing mode, whereas in the normal gameplay mode
it would return directly to the pause menu. To return to the pause
menu from interaction mode, press {\PluginCancel} twice.

\note{Using certain features such as the ``Zoom In'' option may stop
  audio playback. This is normal, as the game requires additional
  memory from the system, which will automatically stop playback. The
  ``Playback Control'' menu will be hidden whenever this
  happens. Exiting the game will allow the resumption of audio
  playback.}
}

\subsection{Sliding Puzzle}
\screenshot{plugins/images/ss-sliding}{Sliding puzzle}{img:slidingpuzzle}

The classic sliding puzzle game.  Rearrange the pieces so that you can
see the whole picture, or switch to number tiles if you like it a little easier
Includes one picture puzzle\opt{albumart}{, but you can switch the puzzle picture to be the
album art of the currently playing music track, if one exists (see
\reference{ref:album_art})}.
You can also use the sliding puzzle plugin as a viewer for supported image types,
to turn your own pictures into a puzzle.

Key controls:

\begin{btnmap}
  \opt{IRIVER_H100_PAD,IRIVER_H300_PAD,IAUDIO_X5_PAD%
      ,SANSA_E200_PAD,SANSA_FUZE_PAD,SANSA_C200_PAD,SANSA_CLIP_PAD,GIGABEAT_PAD,GIGABEAT_S_PAD%
      ,MROBE100_PAD,PBELL_VIBE500_PAD,SANSA_FUZEPLUS_PAD,SAMSUNG_YH92X_PAD,SAMSUNG_YH820_PAD}
    {\ButtonLeft, \ButtonRight, \ButtonUp\ and \ButtonDown}
  \opt{IPOD_4G_PAD,IPOD_3G_PAD}
    {\ButtonLeft{} / \ButtonRight{} / \ButtonMenu{} / \ButtonPlay}
  \opt{IRIVER_H10_PAD}
    {\ButtonLeft{} / \ButtonRight{} / \ButtonScrollUp{} / \ButtonScrollDown}
  \opt{COWON_D2_PAD}
    {\TouchMidLeft{} / \TouchMidRight{} / \TouchTopMiddle{} / \TouchBottomMiddle}
  \opt{MPIO_HD300_PAD}{\ButtonRew / \ButtonFF / \ButtonScrollUp / \ButtonScrollDown}
       \opt{HAVEREMOTEKEYMAP}{& }
  & Move Tile \\
  %
  \opt{IRIVER_H100_PAD,IRIVER_H300_PAD,GIGABEAT_PAD,GIGABEAT_S_PAD,MROBE100_PAD}
      {\ButtonSelect}
  \opt{IAUDIO_X5_PAD,SANSA_E200_PAD,SANSA_C200_PAD}{\ButtonRec}
  \opt{SANSA_CLIP_PAD}{\ButtonHome}
  \opt{SANSA_FUZE_PAD}{\ButtonSelect+\ButtonDown}
  \opt{IPOD_4G_PAD,IPOD_3G_PAD}{\ButtonSelect+\ButtonLeft}
  \opt{IRIVER_H10_PAD}{\ButtonRew}
  \opt{SAMSUNG_YH92X_PAD,SAMSUNG_YH820_PAD}{\ButtonFF}
  \opt{COWON_D2_PAD}{\TouchBottomLeft}
  \opt{PBELL_VIBE500_PAD}{\ButtonCancel}
  \opt{SANSA_FUZEPLUS_PAD}{Long \ButtonPlay}
  \opt{MPIO_HD300_PAD}{\ButtonEnter}
       \opt{HAVEREMOTEKEYMAP}{& }
  & Shuffle \\
  %
  \opt{IRIVER_H100_PAD,IRIVER_H300_PAD}{\ButtonOn}
  \opt{IAUDIO_X5_PAD,IRIVER_H10_PAD}{\ButtonPlay}
  \opt{IPOD_4G_PAD,IPOD_3G_PAD}{\ButtonSelect+\ButtonRight}
  \opt{SANSA_E200_PAD,SANSA_FUZE_PAD,SANSA_C200_PAD,SANSA_CLIP_PAD}{\ButtonSelect}
  \opt{GIGABEAT_PAD}{\ButtonA}
  \opt{MROBE100_PAD}{\ButtonDisplay}
  \opt{GIGABEAT_S_PAD,PBELL_VIBE500_PAD}{\ButtonMenu}
  \opt{SANSA_FUZEPLUS_PAD}{Long \ButtonSelect}
  \opt{COWON_D2_PAD}{\TouchCenter}
  \opt{SAMSUNG_YH92X_PAD,SAMSUNG_YH820_PAD,MPIO_HD300_PAD}{\ButtonPlay}
       \opt{HAVEREMOTEKEYMAP}{& }
  & Switch between pictures (default puzzle\opt{albumart}{, album art}, and your own image if
launched via Open With), and numbered tiles \\
  %
  \opt{IRIVER_H100_PAD,IRIVER_H300_PAD}{\ButtonOff}
  \opt{IAUDIO_X5_PAD,IRIVER_H10_PAD,SANSA_E200_PAD,SANSA_C200_PAD,SANSA_CLIP_PAD,GIGABEAT_PAD%
      ,MROBE100_PAD,COWON_D2_PAD,SANSA_FUZEPLUS_PAD}{\ButtonPower}
  \opt{SANSA_FUZE_PAD}{Long \ButtonHome}
  \opt{IPOD_4G_PAD,IPOD_3G_PAD}{Long \ButtonSelect}
  \opt{GIGABEAT_S_PAD}{\ButtonBack}
  \opt{PBELL_VIBE500_PAD}{\ButtonRec}
  \opt{SAMSUNG_YH92X_PAD,SAMSUNG_YH820_PAD}{\ButtonRew}
  \opt{MPIO_HD300_PAD}{Long \ButtonMenu}
       \opt{HAVEREMOTEKEYMAP}{& 
          \opt{IRIVER_RC_H100_PAD}{\ButtonRCStop}
        }
  & Stop the game \\
\end{btnmap}


\subsection{Snake}
\screenshot{plugins/images/ss-snake}{Snake}{fig:snake}

This is the popular snake game. The aim is to grow your snake as large
as possible by eating the dots that appear on the screen. The game will
end when the snake touches either the borders of the screen or itself.

\begin{btnmap}
    \opt{IRIVER_H100_PAD,IRIVER_H300_PAD,IAUDIO_X5_PAD%
        ,SANSA_E200_PAD,SANSA_C200_PAD,SANSA_CLIP_PAD,SANSA_M200_PAD,GIGABEAT_PAD,GIGABEAT_S_PAD,MROBE100_PAD%
        ,SANSA_FUZE_PAD,PBELL_VIBE500_PAD,SANSA_FUZEPLUS_PAD,SAMSUNG_YH92X_PAD,SAMSUNG_YH820_PAD}
        {\ButtonUp{} / \ButtonDown{} / \ButtonLeft{} / \ButtonRight}
    \opt{IPOD_4G_PAD,IPOD_3G_PAD}{\ButtonMenu{} / \ButtonPlay{} / \ButtonLeft{} / \ButtonRight}
    \opt{IRIVER_H10_PAD}{\ButtonScrollUp{} / \ButtonScrollDown{} / \ButtonLeft{} / \ButtonRight}
    \opt{COWON_D2_PAD}{\TouchTopMiddle{} / \TouchBottomMiddle / \TouchMidLeft{} / \TouchMidRight}
    \opt{MPIO_HD300_PAD}{\ButtonRew / \ButtonFF / \ButtonScrollUp / \ButtonScrollDown}
  \opt{HAVEREMOTEKEYMAP}{& }
        & Move snake\\
    %
    \opt{IAUDIO_X5_PAD,IRIVER_H10_PAD,PBELL_VIBE500_PAD}{\ButtonPlay}
    \opt{IPOD_4G_PAD,IPOD_3G_PAD,SANSA_E200_PAD,SANSA_C200_PAD,SANSA_CLIP_PAD,GIGABEAT_PAD%
      ,GIGABEAT_S_PAD,MROBE100_PAD,SANSA_FUZE_PAD}
      {\ButtonSelect}
    \opt{IRIVER_H100_PAD,IRIVER_H300_PAD}{\ButtonOn}
    \opt{COWON_D2_PAD}{\TouchCenter}
    \opt{MPIO_HD300_PAD,SANSA_FUZEPLUS_PAD,SAMSUNG_YH92X_PAD,SAMSUNG_YH820_PAD}{\ButtonPlay}
       \opt{HAVEREMOTEKEYMAP}{& }
     & Toggle Play/Pause\\
    %
    \opt{SANSA_FUZEPLUS_PAD}{\ButtonPower}
    \opt{SAMSUNG_YH92X_PAD,SAMSUNG_YH820_PAD}{\ButtonRew}
       \opt{HAVEREMOTEKEYMAP}{& }
    & Go to the plugin's menu\\
\end{btnmap}


\subsection{Snake 2}
\screenshot{plugins/images/ss-snake2}
{Snake 2 {--} The Snake Strikes Back}{img:snake2}

Another version of the Snake game. Move the snake around, and eat the
apples that pop up on the screen. Each time an apple is eaten, the
snake gets longer. The game ends when the snake hits a wall, or runs
into itself.

\begin{btnmap}
    \opt{IRIVER_H100_PAD,IRIVER_H300_PAD,IAUDIO_X5_PAD%
        ,SANSA_E200_PAD,SANSA_C200_PAD,SANSA_CLIP_PAD,SANSA_M200_PAD,GIGABEAT_PAD,GIGABEAT_S_PAD,MROBE100_PAD%
        ,SANSA_FUZE_PAD,PBELL_VIBE500_PAD,SANSA_FUZEPLUS_PAD,SAMSUNG_YH92X_PAD,SAMSUNG_YH820_PAD}
        {\ButtonUp{} / \ButtonDown{} / \ButtonLeft{} / \ButtonRight}
    \opt{IPOD_4G_PAD,IPOD_3G_PAD}{\ButtonMenu{} / \ButtonPlay{} / \ButtonLeft{} / \ButtonRight}
    \opt{IRIVER_H10_PAD}{\ButtonScrollUp{} / \ButtonScrollDown{} / \ButtonLeft{} / \ButtonRight}
    \opt{COWON_D2_PAD}{\TouchTopMiddle{} / \TouchBottomMiddle / \TouchMidLeft{} / \TouchMidRight}
    \opt{MPIO_HD300_PAD}{\ButtonRew / \ButtonFF / \ButtonScrollUp / \ButtonScrollDown}
  \opt{HAVEREMOTEKEYMAP}{& }
        & Steer the snake\\
    \opt{PBELL_VIBE500_PAD,MPIO_HD300_PAD,SANSA_FUZEPLUS_PAD,SAMSUNG_YH92X_PAD%
        ,SAMSUNG_YH820_PAD}{\ButtonPlay}
    \opt{IRIVER_H100_PAD,IRIVER_H300_PAD}{\ButtonOn}
    \opt{IPOD_4G_PAD,IPOD_3G_PAD,IAUDIO_X5_PAD,SANSA_E200_PAD,SANSA_C200_PAD,SANSA_CLIP_PAD,SANSA_M200_PAD%
        ,GIGABEAT_PAD,GIGABEAT_S_PAD,MROBE100_PAD,SANSA_FUZE_PAD}
        {\ButtonSelect}
    \opt{IRIVER_H10_PAD}{\ButtonFF}
    \opt{COWON_D2_PAD}{\TouchCenter}
  \opt{HAVEREMOTEKEYMAP}{& }
        & Pause and resume the game\\
    \opt{IRIVER_H100_PAD,IRIVER_H300_PAD}{\ButtonOff}
    \opt{IPOD_4G_PAD,IPOD_3G_PAD}{\ButtonSelect+\ButtonMenu}
    \opt{IAUDIO_X5_PAD,IRIVER_H10_PAD,SANSA_E200_PAD,SANSA_C200_PAD,SANSA_CLIP_PAD,SANSA_M200_PAD,GIGABEAT_PAD%
        ,MROBE100_PAD,COWON_D2_PAD,SANSA_FUZEPLUS_PAD}{\ButtonPower}
    \opt{SANSA_FUZE_PAD}{Long \ButtonHome}
    \opt{GIGABEAT_S_PAD}{\ButtonBack}
    \opt{PBELL_VIBE500_PAD}{\ButtonRec}
    \opt{SAMSUNG_YH92X_PAD,SAMSUNG_YH820_PAD}{\ButtonRew}
    \opt{MPIO_HD300_PAD}{Long \ButtonMenu}
  \opt{HAVEREMOTEKEYMAP}{& }
        & Quit\\
\end{btnmap}

In game A, the maze stays the same, in game B
after an increasing number of apples eaten the maze is replaced by a
new one.


\subsection{Sokoban}
\screenshot{plugins/images/ss-sokoban}{Sokoban}{fig:sokoban}

The object of the game is to push boxes into their correct position in a
crowded warehouse with a minimal number of pushes and moves. The boxes
can only be pushed, never pulled, and only one can be pushed at a time.

Sokoban may be used as a viewer for viewing saved solutions and playing
external level sets with the \fname{.sok} extension. Level sets should be in
the standard Sokoban text format or RLE (Run Length Encoded). For more
information about the level format, see
\url{http://sokobano.de/wiki/index.php?title=Level_format}

\begin{btnmap}
\multicolumn{2}{c}{\textbf{In game}} \\
\hline
\opt{IRIVER_H100_PAD,IRIVER_H300_PAD%
    ,IAUDIO_X5_PAD,GIGABEAT_PAD,GIGABEAT_S_PAD,MROBE100_PAD,SANSA_E200_PAD%
    ,SANSA_FUZE_PAD,SANSA_C200_PAD,SANSA_CLIP_PAD,PBELL_VIBE500_PAD,SANSA_FUZEPLUS_PAD%
    ,SAMSUNG_YH92X_PAD,SAMSUNG_YH820_PAD}
    {\ButtonUp, \ButtonDown, }%
\opt{IPOD_4G_PAD,IPOD_3G_PAD}{\ButtonMenu, \ButtonPlay, }%
\opt{IRIVER_H10_PAD}{\ButtonScrollUp, \ButtonScrollDown, }%
\opt{IRIVER_H100_PAD,IRIVER_H300_PAD%
    ,IAUDIO_X5_PAD,GIGABEAT_PAD,GIGABEAT_S_PAD,MROBE100_PAD,SANSA_E200_PAD%
    ,SANSA_FUZE_PAD,SANSA_C200_PAD,SANSA_CLIP_PAD,IPOD_4G_PAD,IPOD_3G_PAD%
    ,IRIVER_H10_PAD,PBELL_VIBE500_PAD,SANSA_FUZEPLUS_PAD,SAMSUNG_YH92X_PAD%
    ,SAMSUNG_YH820_PAD}
    {\ButtonLeft, \ButtonRight}
\opt{COWON_D2_PAD}
    {\TouchTopMiddle, \TouchBottomMiddle, \TouchMidLeft, \TouchMidRight}
\opt{MPIO_HD300_PAD}{\ButtonScrollUp, \ButtonScrollDown, \ButtonRew, \ButtonFF}
       \opt{HAVEREMOTEKEYMAP}{& }
    & Move the ``sokoban'' up, down, left, or right\\
\opt{IRIVER_H100_PAD,IRIVER_H300_PAD}{\ButtonOff}
\opt{IPOD_4G_PAD,IPOD_3G_PAD}{\ButtonSelect+\ButtonMenu}
\opt{IAUDIO_X5_PAD,IRIVER_H10_PAD,GIGABEAT_PAD,MROBE100_PAD,SANSA_E200_PAD,SANSA_C200_PAD%
     ,SANSA_CLIP_PAD,SANSA_FUZEPLUS_PAD}{\ButtonPower}
\opt{SANSA_FUZE_PAD}{Long \ButtonHome}
\opt{GIGABEAT_S_PAD,COWON_D2_PAD}{\ButtonMenu}
\opt{MPIO_HD300_PAD}{Long \ButtonMenu}
\opt{PBELL_VIBE500_PAD}{\ButtonRec}
\opt{SAMSUNG_YH92X_PAD,SAMSUNG_YH820_PAD}{\ButtonPlay}
       \opt{HAVEREMOTEKEYMAP}{&
          \opt{IRIVER_RC_H100_PAD}{\ButtonRCStop}
         }
    & Menu \\
\nopt{IAUDIO_X5_PAD}{
    \opt{IRIVER_H100_PAD,IRIVER_H300_PAD}{\ButtonOn+\ButtonDown}
    \opt{IPOD_4G_PAD,IPOD_3G_PAD}{\ButtonSelect+\ButtonLeft}
    \opt{IRIVER_H10_PAD}{\ButtonPlay+\ButtonScrollDown}
    \opt{GIGABEAT_PAD,SANSA_C200_PAD,SANSA_CLIP_PAD,SANSA_FUZEPLUS_PAD}{\ButtonVolDown}
    \opt{MROBE10_PAD}{\ButtonDisplay}
    \opt{SANSA_E200_PAD,SANSA_FUZE_PAD}{\ButtonSelect+\ButtonDown}
    \opt{GIGABEAT_S_PAD}{\ButtonPrev}
    \opt{COWON_D2_PAD}{\ButtonMinus}
    \opt{PBELL_VIBE500_PAD}{\ButtonOK+\ButtonLeft}
    \opt{MPIO_HD300_PAD}{\ButtonPlay+\ButtonRew}
    \opt{SAMSUNG_YH92X_PAD}{\ButtonPlay+\ButtonDown}
    \opt{SAMSUNG_YH820_PAD}{\ButtonRec+\ButtonDown}
       \opt{HAVEREMOTEKEYMAP}{& }
    & Back to previous level \\
}
\nopt{IPOD_4G_PAD,IPOD_3G_PAD}{
    \opt{IRIVER_H100_PAD,IRIVER_H300_PAD}{\ButtonOn}
    \opt{IAUDIO_X5_PAD}{\ButtonRec}
    \opt{IRIVER_H10_PAD,SAMSUNG_YH92X_PAD}{\ButtonPlay+\ButtonRight}
    \opt{SAMSUNG_YH820_PAD}{\ButtonRec+\ButtonRight}
    \opt{GIGABEAT_PAD,MROBE100_PAD}{\ButtonMenu}
    \opt{SANSA_E200_PAD,SANSA_FUZE_PAD,SANSA_C200_PAD,SANSA_CLIP_PAD}{\ButtonSelect+\ButtonRight}
    \opt{GIGABEAT_S_PAD}{\ButtonPlay}
    \opt{COWON_D2_PAD}{\TouchTopRight}
    \opt{PBELL_VIBE500_PAD}{\ButtonOK+\ButtonCancel}
    \opt{MPIO_HD300_PAD}{\ButtonPlay+\ButtonEnter}
    \opt{SANSA_FUZEPLUS_PAD}{\ButtonBack}
       \opt{HAVEREMOTEKEYMAP}{& }
    & Restart level \\
}
\nopt{IAUDIO_X5_PAD}{
    \opt{IRIVER_H100_PAD,IRIVER_H300_PAD}{\ButtonOn+\ButtonUp}
    \opt{IPOD_4G_PAD,IPOD_3G_PAD}{\ButtonSelect+\ButtonRight}
    \opt{IRIVER_H10_PAD}{\ButtonPlay+\ButtonScrollUp}
    \opt{GIGABEAT_PAD,SANSA_C200_PAD,SANSA_CLIP_PAD,SANSA_FUZEPLUS_PAD}{\ButtonVolUp}
    \opt{GIGABEAT_S_PAD}{\ButtonNext}
    \opt{MROBE100_PAD}{\ButtonPlay}
    \opt{SANSA_E200_PAD,SANSA_FUZE_PAD}{\ButtonSelect+\ButtonUp}
    \opt{COWON_D2_PAD}{\ButtonPlus}
    \opt{PBELL_VIBE500_PAD}{\ButtonOK+\ButtonRight}
    \opt{MPIO_HD300_PAD}{\ButtonPlay+\ButtonFF}
    \opt{SAMSUNG_YH92X_PAD}{\ButtonPlay+\ButtonUp}
    \opt{SAMSUNG_YH820_PAD}{\ButtonRec+\ButtonUp}
       \opt{HAVEREMOTEKEYMAP}{& }
    & Go to next level \\
}
\opt{IRIVER_H100_PAD,IRIVER_H300_PAD}{\ButtonRec}
\opt{IPOD_4G_PAD,IPOD_3G_PAD,IAUDIO_X5_PAD,GIGABEAT_PAD,MROBE100_PAD%
     ,SANSA_E200_PAD,SANSA_FUZE_PAD,SANSA_C200_PAD,SANSA_CLIP_PAD}{\ButtonSelect}
\opt{IRIVER_H10_PAD,SAMSUNG_YH92X_PAD,SAMSUNG_YH820_PAD}{\ButtonRew}
\opt{GIGABEAT_S_PAD}{\ButtonVolUp}
\opt{COWON_D2_PAD}{\TouchTopRight}
\opt{SANSA_FUZEPLUS_PAD}{\ButtonBottomLeft}
\opt{PBELL_VIBE500_PAD}{\ButtonCancel}
\opt{MPIO_HD300_PAD}{\ButtonRec}
       \opt{HAVEREMOTEKEYMAP}{& }
    & Undo last movement \\

\opt{IAUDIO_X5_PAD}{\ButtonPlay}
\opt{IRIVER_H100_PAD,IRIVER_H300_PAD}{\ButtonMode}
\opt{IPOD_4G_PAD,IPOD_3G_PAD}{\ButtonSelect+\ButtonPlay}
\opt{IRIVER_H10_PAD,SAMSUNG_YH92X_PAD,SAMSUNG_YH820_PAD}{\ButtonFF}
\opt{GIGABEAT_PAD}{\ButtonA}
\opt{GIGABEAT_S_PAD}{\ButtonVolDown}
\opt{MROBE100_PAD}{\ButtonDisplay}
\opt{SANSA_E200_PAD,SANSA_C200_PAD}{\ButtonRec}
\opt{SANSA_CLIP_PAD}{\ButtonHome}
\opt{SANSA_FUZE_PAD}{\ButtonSelect+\ButtonLeft}
\opt{COWON_D2_PAD}{\TouchBottomLeft}
\opt{SANSA_FUZEPLUS_PAD}{\ButtonBottomRight}
\opt{PBELL_VIBE500_PAD}{\ButtonOK}
\opt{MPIO_HD300_PAD}{\ButtonPlay}
       \opt{HAVEREMOTEKEYMAP}{& }
    & Redo previously undone move \\
\hline
\multicolumn{2}{c}{\textbf{Solution playback}} \\
\hline
\opt{IAUDIO_X5_PAD,IRIVER_H10_PAD%
    ,MPIO_HD300_PAD,SANSA_FUZEPLUS_PAD,PBELL_VIBE500_PAD,SAMSUNG_YH92X_PAD%
    ,SAMSUNG_YH820_PAD}{\ButtonPlay}
\opt{IRIVER_H100_PAD,IRIVER_H300_PAD}{\ButtonOn}
\opt{IPOD_4G_PAD,IPOD_3G_PAD,GIGABEAT_PAD,GIGABEAT_S_PAD,MROBE100_PAD%
      ,SANSA_E200_PAD,SANSA_FUZE_PAD,SANSA_C200_PAD,SANSA_CLIP_PAD}{\ButtonSelect}
\opt{COWON_D2_PAD}{\TouchCenter}
       \opt{HAVEREMOTEKEYMAP}{& }
    & Pause/resume \\
\opt{IRIVER_H100_PAD,IRIVER_H300_PAD%
    ,IAUDIO_X5_PAD,GIGABEAT_PAD,GIGABEAT_S_PAD,MROBE100_PAD,SANSA_E200_PAD%
    ,SANSA_FUZE_PAD,SANSA_C200_PAD,SANSA_CLIP_PAD,SANSA_FUZEPLUS_PAD%
    ,PBELL_VIBE500_PAD,SAMSUNG_YH92X_PAD,SAMSUNG_YH820_PAD}
    {\ButtonUp/\ButtonDown}
\opt{IPOD_4G_PAD,IPOD_3G_PAD}{\ButtonMenu/\ButtonPlay}
\opt{IRIVER_H10_PAD,MPIO_HD300_PAD}{\ButtonScrollUp/\ButtonScrollDown}
\opt{COWON_D2_PAD}{\TouchTopMiddle/\TouchBottomMiddle}
       \opt{HAVEREMOTEKEYMAP}{& }
    & Increase/decrease playback speed \\
\opt{IAUDIO_X5_PAD,IRIVER_H10_PAD%
    ,IRIVER_H100_PAD,IRIVER_H300_PAD,IPOD_4G_PAD,IPOD_3G_PAD,GIGABEAT_PAD%
    ,GIGABEAT_S_PAD,MROBE100_PAD,SANSA_E200_PAD,SANSA_FUZE_PAD,SANSA_C200_PAD%
    ,SANSA_CLIP_PAD,SANSA_FUZEPLUS_PAD,PBELL_VIBE500_PAD,SAMSUNG_YH92X_PAD%
    ,SAMSUNG_YH820_PAD}
    {\ButtonLeft/\ButtonRight}
\opt{COWON_D2_PAD}{\TouchMidLeft/\TouchMidRight}
\opt{MPIO_HD300_PAD}{\ButtonRew/\ButtonFF}
       \opt{HAVEREMOTEKEYMAP}{& }
    & Go backward/forward (while paused) \\
\opt{IRIVER_H100_PAD,IRIVER_H300_PAD}{\ButtonOff}
\opt{IPOD_4G_PAD,IPOD_3G_PAD}{\ButtonSelect+\ButtonMenu}
\opt{IAUDIO_X5_PAD,IRIVER_H10_PAD,GIGABEAT_PAD,MROBE100_PAD,SANSA_E200_PAD,SANSA_C200_PAD,SANSA_CLIP_PAD%
    ,SANSA_FUZEPLUS_PAD}{\ButtonPower}
\opt{SANSA_FUZE_PAD}{Long \ButtonHome}
\opt{GIGABEAT_S_PAD,COWON_D2_PAD}{\ButtonMenu}
\opt{MPIO_HD300_PAD}{Long \ButtonMenu}
\opt{PBELL_VIBE500_PAD}{\ButtonRec}
\opt{SAMSUNG_YH92X_PAD,SAMSUNG_YH820_PAD}{\ButtonRew}
       \opt{HAVEREMOTEKEYMAP}{& }
    & Quit \\
\end{btnmap}

Some places where can you can find level sets:
\begin{itemize}
\item \url{http://www.sourcecode.se/sokoban/levels.php}
\item \url{http://sokobano.de/en/levels.php}
\end{itemize}
Note that some level sets may contain levels that are too large for this
version of Sokoban and are unplayable as a result.


\subsection{Solitaire}
\screenshot{plugins/images/ss-solitaire}{Klondike solitaire}{fig:solitaire}

This is the classic Klondike solitaire game for Rockbox.
This is probably the best-known solitaire in the world. Many people 
do not even realize that other games exist. Though the name may not 
be familiar, the game itself certainly is. This is due in no small 
part to Microsoft's inclusion of the the game in every version of 
Windows. Though popular, the odds of winning are rather low, perhaps 
one in thirty hands.

For the full set of rules to the game, and other interesting information
visit
\url{http://www.solitairecentral.com/rules/Klondike.html}

\begin{btnmap}
    \opt{RECORDER_PAD,ONDIO_PAD,IRIVER_H100_PAD,IRIVER_H300_PAD,IAUDIO_X5_PAD%
        ,GIGABEAT_PAD,GIGABEAT_S_PAD,MROBE100_PAD,SANSA_C200_PAD,SANSA_CLIP_PAD,PBELL_VIBE500_PAD%
        ,SANSA_FUZEPLUS_PAD,SAMSUNG_YH92X_PAD,SAMSUNG_YH820_PAD}
      {\ButtonUp{} / \ButtonDown}
    \opt{IPOD_4G_PAD,IPOD_3G_PAD,SANSA_E200_PAD,SANSA_FUZE_PAD}{\ButtonScrollFwd{} / \ButtonScrollBack}
    \opt{IRIVER_H10_PAD}{\ButtonScrollUp{} / \ButtonScrollDown}
    \opt{RECORDER_PAD,ONDIO_PAD,IRIVER_H100_PAD,IRIVER_H300_PAD,IAUDIO_X5_PAD%
        ,GIGABEAT_PAD,GIGABEAT_S_PAD,MROBE100_PAD,SANSA_C200_PAD,SANSA_CLIP_PAD,IPOD_4G_PAD%
        ,IPOD_3G_PAD,SANSA_E200_PAD,SANSA_FUZE_PAD,IRIVER_H10_PAD,PBELL_VIBE500_PAD%
        ,SANSA_FUZEPLUS_PAD,SAMSUNG_YH92X_PAD,SAMSUNG_YH820_PAD}
      {/ \ButtonLeft{} / \ButtonRight}
    \opt{COWON_D2_PAD}{\TouchTopMiddle{} / \TouchBottomMiddle{} / \TouchMidLeft{} / \TouchMidRight}
    \opt{MPIO_HD300_PAD}{\ButtonRew / \ButtonFF}
       \opt{HAVEREMOTEKEYMAP}{& }
      & Move Cursor around.\\
    %
    \opt{RECORDER_PAD}{\ButtonOn}
    \opt{ONDIO_PAD}{\ButtonMenu}
    \opt{IRIVER_H100_PAD,IRIVER_H300_PAD,IPOD_4G_PAD,IPOD_3G_PAD,IAUDIO_X5_PAD%
      ,SANSA_E200_PAD,SANSA_FUZE_PAD,SANSA_C200_PAD,SANSA_CLIP_PAD,GIGABEAT_PAD,GIGABEAT_S_PAD%
      ,MROBE100_PAD,SANSA_FUZEPLUS_PAD}
      {\ButtonSelect}
    \opt{IRIVER_H10_PAD,SAMSUNG_YH92X_PAD,SAMSUNG_YH820_PAD}{\ButtonPlay}
    \opt{COWON_D2_PAD}{\TouchCenter}
    \opt{PBELL_VIBE500_PAD}{\ButtonOK}
    \opt{MPIO_HD300_PAD}{\ButtonEnter}
       \opt{HAVEREMOTEKEYMAP}{& }
      & Select cards, move cards, reveal hidden cards...\\
    %
    \opt{RECORDER_PAD}{\ButtonFOne}
    \opt{ONDIO_PAD}{Long \ButtonMenu}
    \opt{IRIVER_H100_PAD,IRIVER_H300_PAD}{\ButtonMode}
    \opt{IPOD_4G_PAD,IPOD_3G_PAD,GIGABEAT_PAD,GIGABEAT_S_PAD,MROBE100_PAD}
      {\ButtonMenu}
    \opt{IAUDIO_X5_PAD}{\ButtonPlay}
    \opt{IRIVER_H10_PAD}{Long \ButtonLeft}
    \opt{SANSA_E200_PAD}{\ButtonRec}
    \opt{SANSA_FUZE_PAD}{\ButtonHome}
    \opt{SANSA_C200_PAD,SANSA_CLIP_PAD}{\ButtonVolDown}
    \opt{COWON_D2_PAD}{\TouchTopLeft}
    \opt{PBELL_VIBE500_PAD,MPIO_HD300_PAD}{\ButtonMenu}
    \opt{SANSA_FUZEPLUS_PAD}{\ButtonBack}
    \opt{SAMSUNG_YH92X_PAD,SAMSUNG_YH820_PAD}{\ButtonFF}
       \opt{HAVEREMOTEKEYMAP}{& }
      & If a card was selected -- unselect it, else\\
       \opt{HAVEREMOTEKEYMAP}{& }
      & Draw 3 new cards from the remains stack\\
    %
    \opt{RECORDER_PAD,IPOD_4G_PAD,IPOD_3G_PAD}{\ButtonPlay}
    \opt{ONDIO_PAD}{Long \ButtonDown}
    \opt{IRIVER_H100_PAD,IRIVER_H300_PAD}{\ButtonOn{} + \ButtonLeft}
    \opt{IAUDIO_X5_PAD}{Long \ButtonPlay}
    \opt{IRIVER_H10_PAD}{\ButtonFF}
    \opt{SANSA_E200_PAD,SANSA_FUZE_PAD}{\ButtonLeft}
    \opt{SANSA_C200_PAD}{\ButtonRec}
    \opt{SANSA_CLIP_PAD}{\ButtonHome}
    \opt{GIGABEAT_PAD}{\ButtonA{} + \ButtonLeft}
    \opt{GIGABEAT_S_PAD}{\ButtonSelect{} + \ButtonLeft}
    \opt{MROBE100_PAD}{\ButtonDisplay{} + \ButtonLeft}
    \opt{COWON_D2_PAD}{\TouchTopRight}
    \opt{PBELL_VIBE500_PAD}{\ButtonCancel}
    \opt{MPIO_HD300_PAD}{\ButtonPlay}
    \opt{SANSA_FUZEPLUS_PAD}{\ButtonBottomLeft}
    \opt{SAMSUNG_YH92X_PAD}{\ButtonPlay{} + \ButtonDown}
    \opt{SAMSUNG_YH820_PAD}{\ButtonRec{} + \ButtonDown}
       \opt{HAVEREMOTEKEYMAP}{& }
      & Put the card from the top of the remains stack on top of the cursor\\
    %
    \opt{RECORDER_PAD}{\ButtonFTwo}
    \opt{ONDIO_PAD}{Long \ButtonUp}
    \opt{IRIVER_H100_PAD,IRIVER_H300_PAD,GIGABEAT_PAD,GIGABEAT_S_PAD%
      ,MROBE100_PAD,SANSA_C200_PAD,SANSA_CLIP_PAD}{Long \ButtonSelect}
    \opt{IPOD_4G_PAD,IPOD_3G_PAD}{Long \ButtonMenu}
    \opt{IAUDIO_X5_PAD}{Long \ButtonSelect}
    \opt{IRIVER_H10_PAD}{\ButtonRew}
    \opt{SANSA_E200_PAD}{\ButtonRec{} + \ButtonRight}
    \opt{SANSA_FUZE_PAD}{\ButtonRight}
    \opt{COWON_D2_PAD}{\TouchBottomLeft}
    \opt{PBELL_VIBE500_PAD,SANSA_FUZEPLUS_PAD}{\ButtonPlay}
    \opt{MPIO_HD300_PAD}{\ButtonRec}
    \opt{SAMSUNG_YH92X_PAD}{\ButtonPlay{} + \ButtonUp}
    \opt{SAMSUNG_YH820_PAD}{\ButtonRec{} + \ButtonUp}
       \opt{HAVEREMOTEKEYMAP}{& }
      & Put the card under the cursor on one of the 4 final colour stacks.\\
    %
    \opt{RECORDER_PAD}{\ButtonFThree}
    \opt{IRIVER_H10_PAD,ONDIO_PAD,IPOD_4G_PAD,IPOD_3G_PAD}{Long \ButtonRight}
    \opt{IRIVER_H100_PAD,IRIVER_H300_PAD}{\ButtonOn{} + \ButtonRight}
    \opt{IAUDIO_X5_PAD}{\ButtonRec}
    \opt{SANSA_E200_PAD}{\ButtonRight}
    \opt{SANSA_FUZE_PAD}{Long \ButtonLeft}
    \opt{GIGABEAT_PAD}{\ButtonA{} + \ButtonRight}
    \opt{GIGABEAT_S_PAD}{\ButtonSelect{} + \ButtonRight}
    \opt{MROBE100_PAD}{\ButtonDisplay{} + \ButtonRight}
    \opt{SANSA_C200_PAD,SANSA_CLIP_PAD}{\ButtonVolUp}
    \opt{COWON_D2_PAD}{\TouchBottomRight}
    \opt{PBELL_VIBE500_PAD,MPIO_HD300_PAD,SANSA_FUZEPLUS_PAD}{Long \ButtonPlay}
    \opt{SAMSUNG_YH92X_PAD}{\ButtonPlay{} + \ButtonRight}
    \opt{SAMSUNG_YH820_PAD}{\ButtonRec{} + \ButtonRight}
       \opt{HAVEREMOTEKEYMAP}{& }
      & Put the card on top of the remains stack on one of the final colour stacks.\\
    %
    \opt{RECORDER_PAD,ONDIO_PAD,IRIVER_H300_PAD,IRIVER_H100_PAD}{\ButtonOff}
    \opt{IPOD_4G_PAD,IPOD_3G_PAD}{\ButtonMenu{} + \ButtonSelect}
    \opt{IAUDIO_X5_PAD,IRIVER_H10_PAD,SANSA_E200_PAD,SANSA_C200_PAD,SANSA_CLIP_PAD%
        ,GIGABEAT_PAD,MROBE100_PAD,COWON_D2_PAD,SANSA_FUZEPLUS_PAD}{\ButtonPower}
    \opt{SANSA_FUZE_PAD}{Long \ButtonHome}
    \opt{GIGABEAT_S_PAD}{\ButtonBack}
    \opt{PBELL_VIBE500_PAD}{\ButtonRec}
    \opt{SAMSUNG_YH92X_PAD,SAMSUNG_YH820_PAD}{\ButtonRew}
    \opt{MPIO_HD300_PAD}{\ButtonMenu}
       \opt{HAVEREMOTEKEYMAP}{& 
          \opt{IRIVER_RC_H100_PAD}{\ButtonRCStop}
        }
      & Show menu\\
\end{btnmap}


\subsection{Spacerocks}
\screenshot{plugins/images/ss-spacerocks}{Spacerocks}{img:spacerocks}
Spacerocks is a clone of the old arcade game Asteroids.  The goal of the game
is to blow up the asteroids and avoid being hit by them.  Once in a while, a
UFO will appear -- shoot this for extra points.

\begin{btnmap}
    %
    \opt{SAMSUNG_YH92X_PAD,SAMSUNG_YH820_PAD}{\ButtonPlay}
    \opt{IRIVER_H100_PAD,IRIVER_H300_PAD,IPOD_4G_PAD,IPOD_3G_PAD,IAUDIO_X5_PAD%
        ,SANSA_E200_PAD,SANSA_FUZE_PAD,SANSA_C200_PAD,SANSA_CLIP_PAD,GIGABEAT_PAD%
        ,GIGABEAT_S_PAD,MROBE100_PAD,SANSA_FUZEPLUS_PAD}
        {\ButtonSelect}
    \opt{IRIVER_H10_PAD}{\ButtonRew}
    \opt{COWON_D2_PAD}{\TouchBottomMiddle}
    \opt{PBELL_VIBE500_PAD}{\ButtonOK}
    \opt{MPIO_HD300_PAD}{\ButtonEnter}
       \opt{HAVEREMOTEKEYMAP}{& }
    & Shoot\\
    %
    \opt{IRIVER_H100_PAD,IRIVER_H300_PAD,IAUDIO_X5_PAD%
        ,SANSA_E200_PAD,SANSA_FUZE_PAD,SANSA_C200_PAD,SANSA_CLIP_PAD,GIGABEAT_PAD%
        ,GIGABEAT_S_PAD,MROBE100_PAD,PBELL_VIBE500_PAD,SANSA_FUZEPLUS_PAD%
        ,SAMSUNG_YH92X_PAD,SAMSUNG_YH820_PAD}
        {\ButtonUp}
    \opt{IPOD_4G_PAD,IPOD_3G_PAD}{\ButtonRight}
    \opt{IRIVER_H10_PAD}{\ButtonScrollUp}
    \opt{COWON_D2_PAD}{\TouchTopMiddle}
    \opt{MPIO_HD300_PAD}{\ButtonRec}
       \opt{HAVEREMOTEKEYMAP}{& }
    & Thrust\\
    %
    \opt{IRIVER_H100_PAD,IRIVER_H300_PAD,IAUDIO_X5_PAD%
        ,IRIVER_H10_PAD,SANSA_C200_PAD,SANSA_CLIP_PAD,GIGABEAT_PAD,GIGABEAT_S_PAD,MROBE100_PAD%
        ,PBELL_VIBE500_PAD,SANSA_FUZEPLUS_PAD,SAMSUNG_YH92X_PAD,SAMSUNG_YH820_PAD}
        {\ButtonLeft / \ButtonRight}
    \opt{IPOD_4G_PAD,IPOD_3G_PAD,SANSA_E200_PAD,SANSA_FUZE_PAD}{\ButtonScrollBack / \ButtonScrollFwd}
    \opt{COWON_D2_PAD}{\TouchMidLeft / \TouchMidRight}
    \opt{MPIO_HD300_PAD}{\ButtonScrollUp / \ButtonScrollDown}
       \opt{HAVEREMOTEKEYMAP}{& }
    & Turn left/right\\
    %
    \opt{IRIVER_H100_PAD,IRIVER_H300_PAD,IAUDIO_X5_PAD%
        ,SANSA_E200_PAD,SANSA_FUZE_PAD,SANSA_C200_PAD,SANSA_CLIP_PAD,GIGABEAT_PAD%
        ,GIGABEAT_S_PAD,MROBE100_PAD,PBELL_VIBE500_PAD,SAMSUNG_YH92X_PAD%
        ,SAMSUNG_YH820_PAD}
        {\ButtonDown}
    \opt{IPOD_4G_PAD,IPOD_3G_PAD}{\ButtonLeft}
    \opt{IRIVER_H10_PAD}{\ButtonScrollDown}
    \opt{COWON_D2_PAD}{\TouchTopRight}
    \opt{MPIO_HD300_PAD}{Long \ButtonPlay}
    \opt{SANSA_FUZEPLUS_PAD}{\ButtonBack}
       \opt{HAVEREMOTEKEYMAP}{& }
    & Teleport\\
    %
    \opt{IRIVER_H100_PAD,IRIVER_H300_PAD,SANSA_E200_PAD,SANSA_C200_PAD}{\ButtonRec}
    \opt{SANSA_CLIP_PAD}{\ButtonHome}
    \opt{SANSA_FUZE_PAD}{\ButtonSelect+\ButtonUp}
    \opt{IPOD_4G_PAD,IPOD_3G_PAD}{\ButtonPlay}
    \opt{IAUDIO_X5_PAD,IRIVER_H10_PAD,GIGABEAT_S_PAD,PBELL_VIBE500_PAD%
        ,MPIO_HD300_PAD,SANSA_FUZEPLUS_PAD}{\ButtonPlay}
    \opt{GIGABEAT_PAD}{\ButtonA}
    \opt{MROBE100_PAD}{\ButtonDisplay}
    \opt{COWON_D2_PAD}{\TouchCenter}
    \opt{SAMSUNG_YH92X_PAD,SAMSUNG_YH820_PAD}{\ButtonFF}
       \opt{HAVEREMOTEKEYMAP}{& }
    & Pause game\\
    %
    \opt{IRIVER_H100_PAD,IRIVER_H300_PAD}{\ButtonOff}
    \opt{IPOD_4G_PAD,IPOD_3G_PAD}{\ButtonMenu}
    \opt{IAUDIO_X5_PAD,IRIVER_H10_PAD,SANSA_E200_PAD,GIGABEAT_PAD,MROBE100_PAD%
        ,SANSA_C200_PAD,SANSA_CLIP_PAD,COWON_D2_PAD,SANSA_FUZEPLUS_PAD}{\ButtonPower}
    \opt{SANSA_FUZE_PAD}{Long \ButtonHome}
    \opt{GIGABEAT_S_PAD}{\ButtonBack}
    \opt{PBELL_VIBE500_PAD}{\ButtonRec}
    \opt{SAMSUNG_YH92X_PAD,SAMSUNG_YH820_PAD}{\ButtonRew}
    \opt{MPIO_HD300_PAD}{Long \ButtonMenu}
       \opt{HAVEREMOTEKEYMAP}{& 
          \opt{IRIVER_RC_H100_PAD}{\ButtonRCStop}
       }
    & Quit\\
\end{btnmap}


\subsection{Star}
\screenshot{plugins/images/ss-star}{Star game}{fig:star}

This is a puzzle game.  It is actually a rewrite of Star, a game written
by CDK designed for the hp48 calculator.

Rules: Take all of the ``o''s to go to the
next level.  You can switch control between the filled circle,
which can take ``o''s, and the filled square, which is used as a mobile
wall to allow your filled circle to get to places on the screen it
could not otherwise reach. The block cannot take ``o''s.

\begin{btnmap}
    \opt{IRIVER_H100_PAD,IRIVER_H300_PAD,IAUDIO_X5_PAD%
        ,SANSA_E200_PAD,SANSA_FUZE_PAD,SANSA_C200_PAD,SANSA_CLIP_PAD,GIGABEAT_PAD%
        ,GIGABEAT_S_PAD,MROBE100_PAD,IPOD_4G_PAD,IPOD_3G_PAD,IRIVER_H10_PAD%
        ,PBELL_VIBE500_PAD,SANSA_FUZEPLUS_PAD,SAMSUNG_YH92X_PAD,SAMSUNG_YH820_PAD}
        {\ButtonLeft}
    \opt{COWON_D2_PAD}{\TouchMidLeft}
    \opt{MPIO_HD300_PAD}{\ButtonRew}
    \opt{XDUOO_X3_PAD}{\ButtonPrev}
      \opt{HAVEREMOTEKEYMAP}{& }
        & Move Left\\
    \opt{IRIVER_H100_PAD,IRIVER_H300_PAD,IAUDIO_X5_PAD%
        ,SANSA_E200_PAD,SANSA_FUZE_PAD,SANSA_C200_PAD,SANSA_CLIP_PAD,GIGABEAT_PAD%
        ,GIGABEAT_S_PAD,MROBE100_PAD,IPOD_4G_PAD,IPOD_3G_PAD,IRIVER_H10_PAD%
        ,PBELL_VIBE500_PAD,SANSA_FUZEPLUS_PAD,SAMSUNG_YH92X_PAD,SAMSUNG_YH820_PAD}
        {\ButtonRight}
    \opt{MPIO_HD300_PAD}{\ButtonFF}
    \opt{COWON_D2_PAD}{\TouchMidRight}
    \opt{XDUOO_X3_PAD}{\ButtonNext}
      \opt{HAVEREMOTEKEYMAP}{& }
        & Move Right\\
    \opt{IRIVER_H100_PAD,IRIVER_H300_PAD,IAUDIO_X5_PAD%
        ,SANSA_E200_PAD,SANSA_FUZE_PAD,SANSA_C200_PAD,SANSA_CLIP_PAD,GIGABEAT_PAD%
        ,GIGABEAT_S_PAD,MROBE100_PAD,PBELL_VIBE500_PAD,SANSA_FUZEPLUS_PAD%
        ,SAMSUNG_YH92X_PAD,SAMSUNG_YH820_PAD}
        {\ButtonUp}
    \opt{IPOD_4G_PAD,IPOD_3G_PAD}{\ButtonMenu}
    \opt{IRIVER_H10_PAD}{\ButtonScrollUp}
    \opt{COWON_D2_PAD}{\TouchTopMiddle}
    \opt{MPIO_HD300_PAD}{\ButtonScrollUp}
    \opt{XDUOO_X3_PAD}{\ButtonHome}
      \opt{HAVEREMOTEKEYMAP}{& }
        & Move Up\\
    \opt{IRIVER_H100_PAD,IRIVER_H300_PAD,IAUDIO_X5_PAD%
        ,SANSA_E200_PAD,SANSA_FUZE_PAD,SANSA_C200_PAD,SANSA_CLIP_PAD,GIGABEAT_PAD%
        ,GIGABEAT_S_PAD,MROBE100_PAD,PBELL_VIBE500_PAD,SANSA_FUZEPLUS_PAD%
        ,SAMSUNG_YH92X_PAD,SAMSUNG_YH820_PAD}
        {\ButtonDown}
    \opt{IPOD_4G_PAD,IPOD_3G_PAD}{\ButtonPlay}
    \opt{IRIVER_H10_PAD}{\ButtonScrollDown}
    \opt{COWON_D2_PAD}{\TouchBottomMiddle}
    \opt{MPIO_HD300_PAD}{\ButtonScrollDown}
    \opt{XDUOO_X3_PAD}{\ButtonOption}
      \opt{HAVEREMOTEKEYMAP}{& }
        & Move Down\\
    \opt{IRIVER_H100_PAD,IRIVER_H300_PAD}{\ButtonMode}
    \opt{IPOD_4G_PAD,IPOD_3G_PAD,IAUDIO_X5_PAD,SANSA_E200_PAD,SANSA_FUZE_PAD%
        ,SANSA_C200_PAD,SANSA_CLIP_PAD,GIGABEAT_PAD,GIGABEAT_S_PAD,MROBE100_PAD}{\ButtonSelect}
    \opt{IRIVER_H10_PAD}{\ButtonRew}
    \opt{COWON_D2_PAD}{\TouchCenter}
    \opt{PBELL_VIBE500_PAD,SANSA_FUZEPLUS_PAD,SAMSUNG_YH92X_PAD,SAMSUNG_YH820_PAD,XDUOO_X3_PAD}{\ButtonPlay}
    \opt{MPIO_HD300_PAD}{\ButtonEnter}
      \opt{HAVEREMOTEKEYMAP}{& }
        & Switch between circle and square\\
    \opt{IRIVER_H100_PAD,IRIVER_H300_PAD}{\ButtonMode+\ButtonLeft}
    \opt{IPOD_4G_PAD,IPOD_3G_PAD,SANSA_E200_PAD,SANSA_FUZE_PAD,SANSA_C200_PAD,SANSA_CLIP_PAD}{\ButtonSelect+\ButtonLeft}
    \opt{IAUDIO_X5_PAD}{\ButtonPlay+\ButtonDown}
    \opt{IRIVER_H10_PAD}{\ButtonPlay+\ButtonScrollDown}
    \opt{GIGABEAT_PAD,GIGABEAT_S_PAD,SANSA_FUZEPLUS_PAD}{\ButtonVolDown}
    \opt{MROBE100_PAD}{\ButtonMenu}
    \opt{COWON_D2_PAD}{\TouchBottomLeft}
    \opt{PBELL_VIBE500_PAD}{\ButtonCancel}
    \opt{MPIO_HD300_PAD}{\ButtonPlay + \ButtonRew}
    \opt{SAMSUNG_YH92X_PAD}{\ButtonFF+\ButtonDown}
    \opt{SAMSUNG_YH820_PAD}{\ButtonRec+\ButtonDown}
    \opt{XDUOO_X3_PAD}{\ButtonPlay + \ButtonPrev}
      \opt{HAVEREMOTEKEYMAP}{& }
        & Previous level\\
    \opt{IRIVER_H100_PAD,IRIVER_H300_PAD}{\ButtonMode+\ButtonUp}
    \opt{IPOD_4G_PAD,IPOD_3G_PAD}{\ButtonSelect+\ButtonPlay}
    \opt{IAUDIO_X5_PAD,IRIVER_H10_PAD}{\ButtonPlay+\ButtonRight}
    \opt{SANSA_E200_PAD,SANSA_FUZE_PAD,SANSA_C200_PAD,SANSA_CLIP_PAD}{\ButtonSelect+\ButtonDown}
    \opt{GIGABEAT_PAD}{\ButtonA}
    \opt{GIGABEAT_S_PAD,PBELL_VIBE500_PAD}{\ButtonMenu}
    \opt{MROBE100_PAD}{\ButtonDisplay}
    \opt{COWON_D2_PAD}{\TouchBottomRight}
    \opt{MPIO_HD300_PAD}{Long \ButtonPlay}
    \opt{SANSA_FUZEPLUS_PAD}{Long \ButtonBack}
    \opt{SAMSUNG_YH92X_PAD,SAMSUNG_YH820_PAD}{Long \ButtonFF}
    \opt{XDUOO_X3_PAD}{\ButtonPlay + \ButtonOption}
      \opt{HAVEREMOTEKEYMAP}{& }
        & Reset level \\
    \opt{IRIVER_H100_PAD,IRIVER_H300_PAD}{\ButtonMode+\ButtonRight}
    \opt{IPOD_4G_PAD,IPOD_3G_PAD,SANSA_E200_PAD,SANSA_FUZE_PAD,SANSA_C200_PAD,SANSA_CLIP_PAD}{\ButtonSelect+\ButtonRight}
    \opt{IAUDIO_X5_PAD}{\ButtonPlay+\ButtonRight}
    \opt{IRIVER_H10_PAD}{\ButtonPlay+\ButtonScrollUp}
    \opt{GIGABEAT_PAD,GIGABEAT_S_PAD,SANSA_FUZEPLUS_PAD}{\ButtonVolUp}
    \opt{MROBE100_PAD}{\ButtonPlay}
    \opt{COWON_D2_PAD}{\TouchTopLeft}
    \opt{PBELL_VIBE500_PAD}{\ButtonOK}
    \opt{MPIO_HD300_PAD}{\ButtonPlay + \ButtonFF}
    \opt{SAMSUNG_YH92X_PAD}{\ButtonFF+\ButtonUp}
    \opt{SAMSUNG_YH820_PAD}{\ButtonRec+\ButtonUp}
    \opt{XDUOO_X3_PAD}{\ButtonPlay + \ButtonNext}
      \opt{HAVEREMOTEKEYMAP}{& }
        & Next level \\
    \opt{IRIVER_H100_PAD,IRIVER_H300_PAD}{\ButtonOff}
    \opt{IPOD_4G_PAD,IPOD_3G_PAD}{\ButtonSelect+\ButtonMenu}
    \opt{IAUDIO_X5_PAD,IRIVER_H10_PAD,SANSA_E200_PAD,SANSA_C200_PAD,SANSA_CLIP_PAD%
        ,GIGABEAT_PAD,COWON_D2_PAD,SANSA_FUZEPLUS_PAD,XDUOO_X3_PAD}{\ButtonPower}
    \opt{SANSA_FUZE_PAD}{Long \ButtonHome}
    \opt{GIGABEAT_S_PAD}{\ButtonBack}
    \opt{PBELL_VIBE500_PAD}{\ButtonRec}
    \opt{SAMSUNG_YH92X_PAD,SAMSUNG_YH820_PAD}{Long \ButtonRew}
    \opt{MPIO_HD300_PAD}{Long \ButtonMenu}
      \opt{HAVEREMOTEKEYMAP}{& 
          \opt{IRIVER_RC_H100_PAD}{\ButtonRCStop}
       }
        & Exit the game \\
\end{btnmap}


\subsection{\label{ref:Sudoku}Sudoku}
\screenshot{plugins/images/ss-sudoku}{Sudoku}{fig:sudoku}
Sudoku in Rockbox can act as both a plugin and a viewer.
When starting Sudoku from the \setting{Browse Plugins} menu, a 
random game will be generated automatically, and an estimate of its difficulty
(very easy, easy, medium, hard or fiendish) will be displayed on the screen.
New games can be generated from the \setting{Generate} menu option.
When ``playing'' an existing Sudoku game file from Rockbox' file browser
the plugin is invoked as viewer. The selected Sudoku will get loaded and you
can start solving it. The sudoku games need to be stored as text
files with the extension \fname{.ss} as single file per game.

You can create and save your own grids under the \setting{New} menu option.
Enter the menu (as described in the key table below) when you have finished and
enter the full path to save to including the \fname{.ss} extension 
(e.g. \fname{/sudoku/new.ss}).

\subsubsection{The scratchpad}
When you play Sudoku on paper most people like to mark numbers in cells that 
are possible candidates for the cells.
This can be done with the scratchpad, shown as separate column.
Change the number under the cursor to the number you want to put on the
scratchpad and press the scratchpad button, the number will then be added.
If the number was already on the scratchpad it will get removed again.
The column is stored separately for every cell on the board. The stored values
can be displayed inline as small dots by enabling the \setting{Show Markings}
settings.
\note{The scratchpad is \emph{not} saved when saving the game.}

\begin{btnmap}
    \opt{RECORDER_PAD,ONDIO_PAD,IRIVER_H100_PAD,IRIVER_H300_PAD,IAUDIO_X5_PAD%
        ,SANSA_E200_PAD,SANSA_FUZE_PAD,SANSA_C200_PAD,SANSA_CLIP_PAD,GIGABEAT_PAD%
        ,GIGABEAT_S_PAD,MROBE100_PAD,PBELL_VIBE500_PAD,SANSA_FUZEPLUS_PAD%
        ,SAMSUNG_YH92X_PAD}
        {\ButtonUp{} / \ButtonDown{} / \ButtonLeft{} / \ButtonRight}
    \opt{IPOD_4G_PAD,IPOD_3G_PAD}{\ButtonScrollFwd{} / \ButtonScrollBack}
    \opt{IRIVER_H10_PAD}{\ButtonScrollUp{} / \ButtonScrollDown{} / \ButtonLeft{} / \ButtonRight}
    \opt{COWON_D2_PAD}{\TouchTopMiddle{} / \TouchBottomMiddle{} / \TouchMidLeft{} / \TouchMidRight}
    \opt{MPIO_HD200_PAD}{\ButtonRew / \ButtonFF / \ButtonVolDown / \ButtonVolUp}
    \opt{MPIO_HD300_PAD}{\ButtonScrollUp / \ButtonScrollDown / \ButtonRew / \ButtonFF}
  \opt{HAVEREMOTEKEYMAP}{& }
    & Move the cursor\\
    %
    \opt{IPOD_4G_PAD,IPOD_3G_PAD}{\ButtonSelect & Change cursor move direction\\}
    %
    \opt{RECORDER_PAD}{\ButtonPlay}
    \opt{ONDIO_PAD}{\ButtonMenu}
    \opt{IRIVER_H100_PAD,IRIVER_H300_PAD}{\ButtonSelect{} / \ButtonOn}
    \opt{IPOD_4G_PAD,IPOD_3G_PAD}{\ButtonLeft{} / \ButtonRight}
    \opt{IAUDIO_X5_PAD,GIGABEAT_PAD,GIGABEAT_S_PAD,MROBE100_PAD%
         ,SANSA_FUZEPLUS_PAD}{\ButtonSelect}
    \opt{SANSA_FUZEPLUS_PAD}{; \ButtonBottomRight / \ButtonBottomLeft}
    \opt{IRIVER_H10_PAD}{\ButtonRew}
    \opt{SANSA_E200_PAD,SANSA_FUZE_PAD}{\ButtonScrollBack{} / \ButtonScrollFwd}
    \opt{SANSA_C200_PAD,SANSA_CLIP_PAD}{\ButtonSelect{} / \ButtonVolUp{} / \ButtonVolDown}
    \opt{COWON_D2_PAD}{\TouchCenter}
    \opt{PBELL_VIBE500_PAD}{\ButtonOK}
    \opt{MPIO_HD200_PAD}{\ButtonFunc}
    \opt{MPIO_HD300_PAD}{\ButtonEnter}
    \opt{SAMSUNG_YH92X_PAD}{\ButtonFF}
  \opt{HAVEREMOTEKEYMAP}{& }
    & Change number under the cursor\\
    %
    \opt{RECORDER_PAD}{Long \ButtonPlay}
    \opt{ONDIO_PAD}{Long \ButtonMenu+\ButtonDown}
    \opt{IRIVER_H100_PAD,IRIVER_H300_PAD}{Long \ButtonOn}
    \opt{IPOD_4G_PAD,IPOD_3G_PAD}{Long \ButtonLeft{} / \ButtonRight}
    \opt{IAUDIO_X5_PAD,GIGABEAT_PAD,GIGABEAT_S_PAD,MROBE100_PAD%
         ,SANSA_FUZEPLUS_PAD}{Long \ButtonSelect}
    \opt{SANSA_FUZEPLUS_PAD}{; Long \ButtonBottomRight{} / Long \ButtonBottomLeft}
    \opt{IRIVER_H10_PAD}{Long \ButtonRew}
    \opt{SANSA_E200_PAD,SANSA_FUZE_PAD}{Long \ButtonScrollBack{} / \ButtonScrollFwd}
    \opt{SANSA_C200_PAD,SANSA_CLIP_PAD}{Long \ButtonSelect{} / \ButtonVolUp{} / \ButtonVolDown}
    \opt{COWON_D2_PAD}{Long \TouchCenter}
    \opt{PBELL_VIBE500_PAD}{Long \ButtonOK}
    \opt{MPIO_HD200_PAD}{Long \ButtonFunc}
    \opt{MPIO_HD300_PAD}{Long \ButtonEnter}
    \opt{SAMSUNG_YH92X_PAD}{Long \ButtonFF}
  \opt{HAVEREMOTEKEYMAP}{& }
    & Constantly changing the number under the cursor\\
    %
    \opt{RECORDER_PAD}{\ButtonFOne}
    \opt{ONDIO_PAD}{Long \ButtonMenu}
    \opt{IRIVER_H100_PAD,IRIVER_H300_PAD}{\ButtonMode}
    \opt{IPOD_4G_PAD,IPOD_3G_PAD,GIGABEAT_PAD,GIGABEAT_S_PAD,MROBE100_PAD%
      ,COWON_D2_PAD,PBELL_VIBE500_PAD,MPIO_HD300_PAD}{\ButtonMenu}
    \opt{IAUDIO_X5_PAD,IRIVER_H10_PAD,SAMSUNG_YH92X_PAD}{\ButtonPlay}
    \opt{SANSA_E200_PAD}{\ButtonSelect}
    \opt{SANSA_FUZE_PAD}{\ButtonHome}
    \opt{SANSA_C200_PAD,SANSA_CLIP_PAD}{\ButtonPower}
    \opt{MPIO_HD200_PAD}{\ButtonRec}
    \opt{SANSA_FUZEPLUS_PAD}{\ButtonBack}
  \opt{HAVEREMOTEKEYMAP}{& }
    & Open Menu\\
    %
    \opt{RECORDER_PAD}{\ButtonFTwo}
    \opt{ONDIO_PAD}{\ButtonMenu+\ButtonLeft}
    \opt{IRIVER_H100_PAD,IRIVER_H300_PAD,IAUDIO_X5_PAD,SANSA_E200_PAD,SANSA_C200_PAD}{\ButtonRec}
    \opt{SANSA_CLIP_PAD}{\ButtonHome}
    \opt{SANSA_FUZE_PAD}{\ButtonSelect}
    \opt{IPOD_4G_PAD,IPOD_3G_PAD,GIGABEAT_S_PAD,PBELL_VIBE500_PAD,MPIO_HD200_PAD%
        ,MPIO_HD300_PAD,SANSA_FUZEPLUS_PAD}{\ButtonPlay}
    \opt{IRIVER_H10_PAD}{\ButtonFF}
    \opt{GIGABEAT_PAD}{\ButtonA}
    \opt{MROBE100_PAD}{\ButtonDisplay}
    \opt{COWON_D2_PAD}{Long \TouchBottomLeft}
    \opt{SAMSUNG_YH92X_PAD}{\ButtonRew}
  \opt{HAVEREMOTEKEYMAP}{& }
    & Add/Remove number to scratchpad\\
    %
    \opt{RECORDER_PAD,ONDIO_PAD,IRIVER_H100_PAD,IRIVER_H300_PAD}{\ButtonOff}
    \opt{IAUDIO_X5_PAD,IRIVER_H10_PAD,SANSA_E200_PAD,GIGABEAT_PAD,GIGABEAT_S_PAD%
      ,MROBE100_PAD}{\ButtonPower}
    \opt{SANSA_FUZE_PAD}{Long \ButtonHome}
    \opt{IPOD_4G_PAD,IPOD_3G_PAD}{Menu $\rightarrow$ Quit}
    \opt{SANSA_C200_PAD,SANSA_CLIP_PAD,COWON_D2_PAD}{Long \ButtonPower}
    \opt{PBELL_VIBE500_PAD,SAMSUNG_YH92X_PAD}{\ButtonRec}
    \opt{MPIO_HD200_PAD}{\ButtonRec + \ButtonPlay}
    \opt{MPIO_HD300_PAD}{Long \ButtonMenu}
  \opt{HAVEREMOTEKEYMAP}{& }
    & Quit\\
    %
\end{btnmap}

Some places where can you can find \fname{.ss} files:
\begin{itemize}
\item Simple Sudoku (Advanced Puzzle Packs 1 and 2 located near the bottom of that page):
\url{http://www.angusj.com/sudoku/}
\item Kjell's Sudoku generator/solver:
\url{http://kjell.haxx.se/sudoku/}
\end{itemize}


\opt{lcd_non-mono}{\nopt{iriverh10_5gb,ipodmini1g,c200,c200v2,mpiohd200,clipzip,samsungyh820}{
  \subsection{Superdom}

Superdom is a turn based strategy game, where the aim is to defeat the computer
player by overpowering them using your army.

When the game starts the player is given roughly 50% of the tiles on the map,
two farms, and two factories. To overpower the enemy, you must place resources
in adjacent tiles (diagonals do not count), such that your strength is greater
than the computers, then attack the square.

Each ``year'' is broken up into three phases: purchasing, movement, and war.
During the purchasing phase you are allocated money and food, may purchase
units/buildings, and may access the bank.

During the movement phase you can move your units only to adjacent squares
(except planes, which may move anywhere), at a cost of 1 move per unit you move.
(Men are considered to be one unit, no matter how many you move). You can change
the number of moves you receive by default in the settings, and you may purchase
additional moves for \$100 each. You may also launch nuclear weapons if you have
purchased any.

During the war phase you can attack the enemy. This is where the strengths come
into play. Each tile has its own strength for both the computer, and the human
player. If you attack a square owned by the computer player where your strength
is greater than the computer's, you will win the tile - and take control of any
building that were on it. If the strengths are equal, a victor is chosen at
random.

The bank (in the purchasing phase) is a place where you can store your money
and earn interest (usually about 10% p.a.), however, while the money is in the
bank, it is not accessible until the next purchasing phase.

\subsubsection{Notes on food}
Each year you are allocated an amount of food based on the number of farms you
control. Food is used to feed your men, but if you do not have enough food to
feed your population of men, some will die of starvation.

\subsubsection{Summary of units}
\begin{table}
    \begin{rbtabular}{\textwidth}{lccX}%
        {\textbf{Unit} & \textbf{Cost} & \textbf{Individual strength}%
         & \textbf{Special abilities}}{}{}
        Men       & \$1 each & 1.33 per 100 & No square population limit, however %
                                              require 1 food each per year\\
        Tanks     & \$300    & 3 & None\\
        Planes    & \$600    & 4 & Can move to any human controlled point on the map\\
        Farms     & \$1150   & 3 & Generates additional food at start of year\\
        Factories & \$1300   & 3 & Generates additional food at start of year\\
        Nukes     & \$2000   & 2 & During the movement phase you may lauch a nuke %
                                   to destroy all units on a given tile\\
    \end{rbtabular}
\end{table}
Also note that the colour of the adjacent tiles also count towards your strength.
}}

\opt{lcd_color}{\nopt{lowmem,iaudiox5,iriverh300}{\subsection{Wolf3D}

This is a port of Wolfenstein 3-D, derived from Wolf4SDL.

\subsubsection{Installation}
Copy the original data files (e.g. *.WL6) into the
\fname{/.rockbox/wolf3d/} directory. Sound should work by default.
}}

\subsection{Wormlet}
\screenshot{plugins/images/ss-wormlet}{Wormlet game}{img:wormlet}
Wormlet is a \opt{RECORDER_PAD}{multi{}-user }multi{}-worm game on a multi{}-threaded
multi{}-functional Rockbox console. You navigate a hungry little worm.
Help your worm to find food and to avoid poisoned argh{}-tiles. The
goal is to turn your tiny worm into a big worm for as long as possible.

\opt{RECORDER_PAD}{
For 2{}-player games a remote control is not necessary but recommended.
If you try to hold the \dap\ in the four hands of two players
you'll find out why. Games with three players are only
possible using a remote control.\\}


%The following table is only for the recorder version of the game, since the
%other versions do not support either multi player or different control modes.
%It is however prepared for the other targets should they ever support these
%features. Also some other parts of the text is "opted" out for these targets.

{\bfseries
Game controls:}

\opt{RECORDER_PAD}{
\renewcommand{\arraystretch}{1.8}
\begin{rbtabular}{\textwidth}{c X p{3cm} p{3cm} p{3cm}}%
{\textbf{Players} & \textbf{Modes} & \textbf{Player 1} & \textbf{Player 2}
                        & \textbf{Player 3}}{}{}
%
0 & Out of control & \multicolumn{3}{p{9cm}}{With no player taking part in the
    game all worms are out of control and steered by artificial stupidity.}\\
%
\multirow{2}{*}{1} & 2 key control & on \dap\ \ButtonLeft: turn left
                        \ButtonRight: turn right & {}- & {}-\\
                    & 4 key control & on \dap\
                        \ButtonLeft: turn  left
                        \ButtonUp: turn up
                        \ButtonRight: turn right
                        \ButtonDown: turn down & {}- & {}- \\
%
\multirow{2}{*}{2} & Remote control & on \dap\ \ButtonLeft: turn left
                        \ButtonRight: turn right & on remote control VOL DOWN:
                        turn left VOL UP: turn right & {}- \\
                    & No remote control & on \dap\ \ButtonLeft: turn left
                        \ButtonRight: turn right & on \dap\ \ButtonFTwo: turn
                        left \ButtonFThree: turn right & {}- \\
3 & Remote control & on \dap\ \ButtonLeft: turn left \ButtonRight: turn right
                        & on remote control VOL DOWN: turn left VOL UP: turn
                        right & on \dap\ \ButtonFTwo: turn left \ButtonFThree:
                        turn right \\
\end{rbtabular}
\renewcommand{\arraystretch}{1.0}
}

\nopt{RECORDER_PAD}{
    \begin{btnmap}
        \nopt{MPIO_HD200_PAD,MPIO_HD300_PAD,touchscreen}{\ButtonLeft}
        \opt{MPIO_HD200_PAD}{\ButtonVolDown}
        \opt{MPIO_HD300_PAD}{\ButtonRew}
        \opt{touchscreen}{\TouchMidLeft}
            &
        \opt{HAVEREMOTEKEYMAP}{
            &}
        Turn left
        \\
        
        \nopt{MPIO_HD200_PAD,MPIO_HD300_PAD,touchscreen}{\ButtonRight}
        \opt{MPIO_HD200_PAD}{\ButtonVolUp}
        \opt{MPIO_HD300_PAD}{\ButtonFF}
        \opt{touchscreen}{\TouchMidRight}
            &
        \opt{HAVEREMOTEKEYMAP}{
            &}
        Turn right
        \\
        
        \nopt{IPOD_3G_PAD,IPOD_4G_PAD,IRIVER_H10_PAD,MPIO_HD200_PAD%
            ,MPIO_HD300_PAD,touchscreen}{\ButtonUp}
        \opt{IPOD_3G_PAD,IPOD_4G_PAD}{\ButtonMenu}
        \opt{IRIVER_H10_PAD}{\ButtonScrollUp}
        \opt{MPIO_HD200_PAD}{\ButtonRew}
        \opt{MPIO_HD300_PAD}{\ButtonScrollUp}
        \opt{touchscreen}{\TouchTopMiddle}
            &
        \opt{HAVEREMOTEKEYMAP}{
            &}
        Turn Up
        \\
        
        \nopt{IPOD_3G_PAD,IPOD_4G_PAD,IRIVER_H10_PAD,MPIO_HD200_PAD%
            ,MPIO_HD300_PAD,touchscreen}{\ButtonDown}
        \opt{IPOD_3G_PAD,IPOD_4G_PAD}{\ButtonPlay}
        \opt{IRIVER_H10_PAD}{\ButtonScrollDown}
        \opt{MPIO_HD200_PAD}{\ButtonFF}
        \opt{MPIO_HD300_PAD}{\ButtonScrollDown}
        \opt{touchscreen}{\TouchBottomMiddle}
            &
        \opt{HAVEREMOTEKEYMAP}{
            &}
        Turn Down
        \\
    \end{btnmap}
}

\subsubsection{The game}
Use the control keys of your worm to navigate around obstacles and find
food. Worms do not stop moving except when dead. Dead worms are no fun.
Be careful as your worm will try to eat anything that you steer it
across. It won't distinguish whether it is edible or not.

\begin{description}
\item[Food.]
The small square hollow pieces are food. Move the worm over a food tile
to eat it. After eating the worm grows. Each time a piece of food has
been eaten a new piece of food will pop up somewhere. Unfortunately for
each new piece of food that appears two new ``argh'' pieces will
appear, too.
\item[Argh.]
An ``argh'' is a black square poisoned piece {}- slightly bigger than
food {}- that makes a worm say ``Argh!'' when
run into.  A worm that eats an ``argh'' is dead. Thus eating an
``argh'' must be avoided under any circumstances. ``Arghs'' have the
annoying tendency to accumulate. 
\item[Worms.]
Thou shall not eat worms. Neither other worms nor thyself. Eating worms
is blasphemous cannibalism, not healthy and causes instant
death. And it doesn't help anyway: the other worm
isn't hurt by the bite. It will go on creeping happily
and eat all the food you left on the table. 
\item[Walls.]
Don't crash into the walls. Walls are not edible.
Crashing a worm against a wall causes it a headache it
doesn't survive. 
\item[Game over.]
The game is over when all worms are dead. The longest worm wins the
game. 
\item [Pause the game.]
Press
\opt{RECORDER_PAD,IAUDIO_X5_PAD,PBELL_VIBE500_PAD,MPIO_HD200_PAD%
    ,MPIO_HD300_PAD,SAMSUNG_YH92X_PAD,SAMSUNG_YH820_PAD}{\ButtonPlay}%
\opt{ONDIO_PAD}{\ButtonMenu}%
\opt{IRIVER_H100_PAD,IRIVER_H300_PAD,IPOD_4G_PAD,IPOD_3G_PAD,SANSA_E200_PAD,SANSA_FUZE_PAD%
  ,GIGABEAT_PAD,GIGABEAT_S_PAD}{\ButtonSelect}
\opt{COWON_D2_PAD}{\TouchCenter}
to pause the game. Press it again to resume the game.

\item[Stop the game.]
There are two ways to stop a running game.

\begin{itemize}
\item If you want to quit Wormlet entirely simply hit
\opt{RECORDER_PAD,ONDIO_PAD,IRIVER_H100_PAD,IRIVER_H300_PAD}{\ButtonOff}%
\opt{IPOD_4G_PAD,IPOD_3G_PAD}{\ButtonMenu+\ButtonSelect}%
\opt{IAUDIO_X5_PAD,SANSA_E200_PAD,GIGABEAT_PAD}{\ButtonPower}%
\opt{SANSA_FUZE_PAD}{Long \ButtonHome}%
\opt{PBELL_VIBE500_PAD,SAMSUNG_YH92X_PAD,SAMSUNG_YH820_PAD}{\ButtonRec}%
\opt{MPIO_HD200_PAD}{\ButtonRec + \ButtonPlay}%
\opt{MPIO_HD300_PAD}{Long \ButtonMenu}%
\opt{GIGABEAT_S_PAD}{\ButtonBack}.
The game will stop immediately and you will return to the game menu. 
\item If you want to stop the game and still see the screen hit 
\opt{RECORDER_PAD,IRIVER_H100_PAD,IRIVER_H300_PAD}{\ButtonOn}%
\opt{ONDIO_PAD}{\ButtonOff+\ButtonMenu}%
\opt{IPOD_4G_PAD,IPOD_3G_PAD}{\ButtonSelect+\ButtonPlay}%
\opt{IAUDIO_X5_PAD,SANSA_E200_PAD}{\ButtonRec}%
\opt{SANSA_FUZE_PAD}{\ButtonSelect+\ButtonUp}%
\opt{GIGABEAT_PAD}{\ButtonA}%
\opt{PBELL_VIBE500_PAD}{\ButtonCancel}%
\opt{SAMSUNG_YH92X_PAD,SAMSUNG_YH820_PAD}{\ButtonRew}%
\opt{MPIO_HD200_PAD}{Long \ButtonFunc}%
\opt{MPIO_HD300_PAD}{Long \ButtonPlay}%
\opt{GIGABEAT_S_PAD}{\ButtonMenu}.
This freezes the game. If you hit
\opt{RECORDER_PAD,IRIVER_H100_PAD,IRIVER_H300_PAD}{\ButtonOn}%
\opt{ONDIO_PAD}{\ButtonOff+\ButtonMenu}%
\opt{IPOD_4G_PAD,IPOD_3G_PAD}{\ButtonSelect+\ButtonPlay}%
\opt{IAUDIO_X5_PAD,SANSA_E200_PAD}{\ButtonRec}%
\opt{SANSA_FUZE_PAD}{\ButtonSelect+\ButtonUp}%
\opt{GIGABEAT_PAD}{\ButtonA}%
\opt{PBELL_VIBE500_PAD}{\ButtonCancel}%
\opt{SAMSUNG_YH92X_PAD,SAMSUNG_YH820_PAD}{\ButtonRew}%
\opt{MPIO_HD200_PAD}{Long \ButtonFunc}%
\opt{MPIO_HD300_PAD}{Long \ButtonPlay}%
\opt{GIGABEAT_S_PAD}{\ButtonMenu}
button again a new game starts with the same configuration. To return to the
games menu you can hit
\opt{RECORDER_PAD,ONDIO_PAD,IRIVER_H100_PAD,IRIVER_H300_PAD}{\ButtonOff}%
\opt{IPOD_4G_PAD,IPOD_3G_PAD}{\ButtonMenu+\ButtonSelect}%
\opt{IAUDIO_X5_PAD,SANSA_E200_PAD,GIGABEAT_PAD}{\ButtonPower}%
\opt{SANSA_FUZE_PAD}{Long \ButtonHome}
\opt{PBELL_VIBE500_PAD,SAMSUNG_YH92X_PAD,SAMSUNG_YH820_PAD}{\ButtonRec}%
\opt{MPIO_HD200_PAD}{\ButtonRec + \ButtonPlay}%
\opt{MPIO_HD300_PAD}{Long \ButtonMenu}%
\opt{GIGABEAT_S_PAD}{\ButtonBack}. A stopped game can not be resumed. 
\end{itemize}
\end{description}

\subsubsection{The scoreboard}
On the right side of the game field is the score board. For each worm it
displays its status and its length. The top most entry displays the
state of worm 1, the second worm 2 and the third worm 3. When a worm
dies its entry on the score board turns black.

\begin{description}
\item[Len:]
Here the current length of the worm is displayed. When a worm is eating
food it grows by one pixel for each step it moves. 

\item[Hungry:]
That's the normal state of a worm. Worms are always
hungry and want to eat. It is good to have a hungry
worm since it means that your worm is alive. But it is
better to get your worm growing. 

\item[Growing:]
When a worm has eaten a piece of food it starts growing. For each step
it moves over food it can grow by one pixel. One piece of food lasts
for 7 steps. After your worm has moved 7 steps the food is used up. If
another piece of food is eaten while growing it will increase the size
of the worm for another 7 steps. 

\item[Crashed:]
This indicates that a worm has crashed against a wall.

\item[Argh:]
If the score board entry displays ``Argh!'' it
means the worm is dead because it tried to eat an ``argh''. Until we
can make the worm say ``Argh!'' it is your job to say ``Argh!'' aloud.

\item[Wormed:]
The worm tried to eat another worm or even itself.
That's why it is dead now.  Making traps for other players with a worm
is a good way to get them out of the game.
\end{description}

\subsubsection{Hints}

\begin{itemize}

\item Initially you will be busy with controlling your worm. Try to
avoid other worms and crawl far away from them. Wait until they curl up
themselves and collect the food afterwards. Don't worry if the other
worms grow longer than yours {}- you can catch up after they've died. 

\item When you are more experienced watch the tactics of other worms.
Those worms controlled by artificial stupidity head straight for the
nearest piece of food. Let the other worm have its next piece of food
and head for the food it would probably want next. Try to put yourself
between the opponent and that food. From now on you can `control' the
other worm by blocking it. You could trap it by making a 1 pixel wide
U{}-turn. You also could move from food to food and make sure you keep
between your opponent and the food. So you can always reach it before
your opponent. 

\opt{RECORDER_PAD}{
\item While playing the game the \dap\ can still play music. For
single player game use any music you like. For berserk games with 2 players use
hard rock and for 3 player games use heavy metal or X{}-Phobie
(\url{http://www.x-phobie.de/}).
Play fair and don't kick your opponent in the toe or
poke him in the eye. That would be bad manners.
}
\end{itemize}


% $Id$ %
\subsection{Xobox}
\screenshot{plugins/images/ss-xobox}{Xobox}{img:xobox}
Xobox is a simple clone of the well known arcade game Qix.
The aim of the game is to section off parts of the arena with your trail in
order to remove that section from the game. Be careful not to get in the way of
enemy balls because, if they hit you or your trail, you lose a life.
To finish a level you have to section off more than 75\%.

\begin{btnmap}
    \opt{IPOD_4G_PAD,IPOD_3G_PAD}{\ButtonMenu, \ButtonPlay,}
    \opt{IRIVER_H100_PAD,IRIVER_H300_PAD,IAUDIO_X5_PAD,SANSA_E200_PAD,SANSA_C200_PAD,SANSA_CLIP_PAD%
        ,GIGABEAT_PAD,GIGABEAT_S_PAD,MROBE100_PAD,PBELL_VIBE500_PAD%
        ,SANSA_FUZEPLUS_PAD,SAMSUNG_YH92X_PAD,SAMSUNG_YH820_PAD}%
        {\ButtonUp, \ButtonDown,}%
    \opt{IRIVER_H10_PAD,MPIO_HD300_PAD}{\ButtonScrollUp, \ButtonScrollDown,}
    \opt{IPOD_4G_PAD,IPOD_3G_PAD,IRIVER_H100_PAD,IRIVER_H300_PAD,IAUDIO_X5_PAD%
        ,SANSA_E200_PAD,SANSA_C200_PAD,SANSA_CLIP_PAD,GIGABEAT_PAD%
        ,GIGABEAT_S_PAD,MROBE100_PAD,IRIVER_H10_PAD,SANSA_FUZE_PAD,PBELL_VIBE500_PAD%
        ,SANSA_FUZEPLUS_PAD,SAMSUNG_YH92X_PAD,SAMSUNG_YH820_PAD}
        {\ButtonLeft, \ButtonRight}
    \opt{COWON_D2_PAD}{\TouchTopMiddle, \TouchBottomMiddle, \TouchMidLeft, \TouchMidRight}
    \opt{MPIO_HD300_PAD}{\ButtonRew, \ButtonFF}
    \opt{XDUOO_X3_PAD}{\ButtonPrev, \ButtonNext, \ButtonHome, \ButtonOption}
  \opt{HAVEREMOTEKEYMAP}{& }
    & Move around the arena\\
    \opt{IRIVER_H100_PAD,IRIVER_H300_PAD}{\ButtonMode}
    \opt{IPOD_4G_PAD,IPOD_3G_PAD,SANSA_FUZE_PAD}{\ButtonSelect}
    \opt{IAUDIO_X5_PAD,IRIVER_H10_PAD,GIGABEAT_S_PAD,PBELL_VIBE500_PAD%
        ,MPIO_HD300_PAD,SANSA_FUZEPLUS_PAD,SAMSUNG_YH92X_PAD,SAMSUNG_YH820_PAD,XDUOO_X3_PAD}{\ButtonPlay}
    \opt{SANSA_E200_PAD,SANSA_C200_PAD}{\ButtonRec}
    \opt{SANSA_CLIP_PAD}{\ButtonHome}
    \opt{GIGABEAT_PAD}{\ButtonA}
    \opt{MROBE100_PAD}{\ButtonDisplay}
    \opt{COWON_D2_PAD}{\TouchCenter}
  \opt{HAVEREMOTEKEYMAP}{& }
    & Pause\\
    \opt{IRIVER_H100_PAD,IRIVER_H300_PAD}{\ButtonOff}
    \opt{IPOD_4G_PAD,IPOD_3G_PAD}{Long \ButtonSelect}
    \opt{IAUDIO_X5_PAD,IRIVER_H10_PAD,SANSA_E200_PAD,SANSA_C200_PAD,SANSA_CLIP_PAD%
        ,GIGABEAT_PAD,MROBE100_PAD,COWON_D2_PAD,SANSA_FUZEPLUS_PAD,XDUOO_X3_PAD}{\ButtonPower}
    \opt{SANSA_FUZE_PAD}{Long \ButtonHome}
    \opt{GIGABEAT_S_PAD}{\ButtonBack}
    \opt{PBELL_VIBE500_PAD}{\ButtonRec}
    \opt{SAMSUNG_YH92X_PAD,SAMSUNG_YH820_PAD}{\ButtonRew}
    \opt{MPIO_HD300_PAD}{Long \ButtonMenu}
  \opt{HAVEREMOTEKEYMAP}{& }
    & Open menu\\
\end{btnmap}


\opt{lcd_color}{\subsection{XWorld}

In this cinematic, award winning platform game by Éric Chahi, you must evade capture
and do your best to escape an alien planet.  After an experiment goes awry the hero
must team up with an unlikely ally, when they both become fugitives on another world.
XWorld requires the data files, \fname{BANK*} and \fname{MEMLIST.BIN}, from the original ``Another World''
PC game to be copied into the \fname{.rockbox/xworld/} directory before the game can be played.

Additionally, ``extra'' data files that modify the in-game strings and font can be placed in the \fname{.rockbox/xworld/} directory with the names \fname{xworld.strings} and \fname{xworld.font}, respectively.

\begin{btnmap}
    %
    \opt{IRIVER_H100_PAD,IRIVER_H300_PAD,IAUDIO_X5_PAD%
        ,SANSA_E200_PAD,SANSA_FUZE_PAD,SANSA_C200_PAD,SANSA_CLIP_PAD,GIGABEAT_PAD%
        ,GIGABEAT_S_PAD,MROBE100_PAD,PBELL_VIBE500_PAD,SANSA_FUZEPLUS_PAD%
        ,SAMSUNG_YH92X_PAD,SAMSUNG_YH820_PAD}
        {\ButtonUp}
    \opt{IPOD_4G_PAD,IPOD_3G_PAD,IPOD_1G2G_PAD}{\ButtonMenu}
    \opt{IRIVER_H10_PAD}{\ButtonScrollUp}
    \opt{HAVE_TOUCHSCREEN}{\TouchTopMiddle}
    \opt{PBELL_VIBE500_PAD}{\ButtonOK}
       \opt{HAVEREMOTEKEYMAP}{& }
    & Up and Jump \\
    %
    \opt{IRIVER_H100_PAD,IRIVER_H300_PAD,IAUDIO_X5_PAD%
        ,SANSA_E200_PAD,SANSA_FUZE_PAD,SANSA_C200_PAD,SANSA_CLIP_PAD,GIGABEAT_PAD%
        ,GIGABEAT_S_PAD,MROBE100_PAD,PBELL_VIBE500_PAD,SANSA_FUZEPLUS_PAD%
        ,SAMSUNG_YH92X_PAD,SAMSUNG_YH820_PAD}
        {\ButtonDown}
    \opt{IPOD_4G_PAD,IPOD_3G_PAD,IPOD_1G2G_PAD}{\ButtonPlay}
    \opt{IRIVER_H10_PAD}{\ButtonScrollDown}
    \opt{HAVE_TOUCHSCREEN}{\TouchBottomMiddle}
    \opt{PBELL_VIBE500_PAD}{\ButtonCancel}
       \opt{HAVEREMOTEKEYMAP}{& }
    & Down and Crouch\\
    %
    \opt{IRIVER_H100_PAD,IRIVER_H300_PAD,IAUDIO_X5_PAD%
        ,SANSA_E200_PAD,SANSA_FUZE_PAD,SANSA_C200_PAD,SANSA_CLIP_PAD,GIGABEAT_PAD%
        ,GIGABEAT_S_PAD,MROBE100_PAD,PBELL_VIBE500_PAD,SANSA_FUZEPLUS_PAD%
        ,SAMSUNG_YH92X_PAD,SAMSUNG_YH820_PAD,IPOD_4G_PAD,IPOD_3G_PAD,IPOD_1G2G_PAD%
        ,IRIVER_H10_PAD}
        {\ButtonLeft{} / \ButtonRight}
    \opt{HAVE_TOUCHSCREEN}{\TouchMidLeft{} / \TouchMidRight}
    \opt{PBELL_VIBE500_PAD}{\ButtonMenu{} / \ButtonPlay}
       \opt{HAVEREMOTEKEYMAP}{& }
    & Move Left and Right\\
    %
    \opt{SANSA_FUZE_PAD}{\ButtonHome}
    \opt{SAMSUNG_YH92X_PAD}{\ButtonFF}
    \opt{IRIVER_H300_PAD,SANSA_E200_PAD,SAMSUNG_YH820_PAD,IAUDIO_X5M5_PAD}{\ButtonRec}
    \opt{IPOD_4G_PAD,IPOD_3G_PAD,IPOD_1G2G_PAD,CREATIVE_ZEN_PAD,SANSA_CLIP_PAD}{\ButtonSelect}
    \opt{SONY_NWZ_PAD,CREATIVEZVM_PAD}{\ButtonPlay}
    \opt{ONDAVX777_PAD,MROBE500_PAD,PBELL_VIBE500_PAD}{\ButtonPower}
    \opt{SAMSUNG_YPR0_PAD}{\ButtonUser}
    \opt{IRIVER_H10_PAD}{\ButtonRew}
    \opt{HM801_PAD}{\ButtonPrev}
    \opt{SONY_NWZ_PAD,CREATIVEZVM_PAD}{\ButtonPlay}
    \opt{MROBE500_PAD}{\ButtonPower}
    \opt{DX50_PAD,ONDAVX747_PAD,PHILIPS_HDD1630_PAD,PHILIPS_HDD6330_PAD,PHILIPS_SA9200_PAD%
        ,CREATIVE_ZENXFI2_PAD,CREATIVE_ZENXFI3_PAD,SANSA_CONNECT_PAD,SANSA_C200_PAD%
        ,SANSA_FUZEPLUS_PAD,GIGABEAT_PAD,GIGABEAT_S_PAD}{\ButtonVolUp}
    \opt{HAVE_TOUCHSCREEN}{\ButtonBottomLeft}
       \opt{HAVEREMOTEKEYMAP}{& }
    & Action and Fire\\
    %
    \opt{DX50_PAD,ONDAVX747_PAD,PHILIPS_HDD1630_PAD,PHILIPS_HDD6330_PAD,PHILIPS_SA9200_PAD%
        ,CREATIVE_ZENXFI2_PAD,CREATIVE_ZENXFI3_PAD,SANSA_CONNECT_PAD,SANSA_C200_PAD%
        ,SANSA_FUZEPLUS_PAD}{\ButtonVolDown}
    \opt{GIGABEAT_PAD,GIGABEAT_S_PAD}{\ButtonMenu}
    \opt{SANSA_FUZE_PAD}{\ButtonSelect}
    \opt{SAMSUNG_YH92X_PAD}{\ButtonRew}
    \opt{SAMSUNG_YH820_PAD,IAUDIO_X5M5_PAD}{\ButtonPlay}
    \opt{SANSA_E200_PAD,SANSA_CLIP_PAD}{\ButtonPower}
    \opt{CREATIVE_ZEN_PAD,SONY_NWZ_PAD}{\ButtonBack}
    \opt{CREATIVEZVM_PAD,SAMSUNG_YPR0_PAD}{\ButtonMenu}
    \opt{IRIVER_H300_PAD}{\ButtonMode}
    \opt{HM801_PAD}{\ButtonNext}
    \opt{PBELL_VIBE500_PAD}{\ButtonRec}
    \opt{IRIVER_H10_PAD}{\ButtonPlay}
    \opt{IPOD_4G_PAD,IPOD_3G_PAD,IPOD_1G2G_PAD}{\ButtonMenu{} / \ButtonSelect}
       \opt{HAVEREMOTEKEYMAP}{& }
    & Menu\\
\end{btnmap}
}

\section{Demos}

\subsection{Bounce}
\screenshot{plugins/images/ss-bounce}{Bounce}{img:bounce}
This demo is of the word ``Rockbox'' bouncing across the screen.
\opt{rtc}{There is also an analogue clock in the background.}
In \setting{Scroll mode} the bouncing text is replaced by a different one
scrolling from right to left.

\begin{btnmap}
    \nopt{IPOD_4G_PAD,IPOD_3G_PAD}{\PluginUp{} / \PluginDown}
    \opt{IPOD_4G_PAD,IPOD_3G_PAD}{\PluginLeft{} / \PluginRight}
        \opt{HAVEREMOTEKEYMAP}{& }
    & Moves to next/previous option\\
    \nopt{scrollwheel} {
       \PluginRight{} / \PluginLeft
    }
    \opt{scrollwheel} {
       \PluginScrollFwd{} / \PluginScrollBack
    }
       \opt{HAVEREMOTEKEYMAP}{& }
    & Increases/decreases option value\\
    \PluginSelect
       \opt{HAVEREMOTEKEYMAP}{& }
    & Toggles Scroll mode\\
    \nopt{IPOD_4G_PAD,IPOD_3G_PAD}{\PluginCancel{} or \PluginExit}
    \opt{IPOD_4G_PAD,IPOD_3G_PAD}{\PluginUp}
       \opt{HAVEREMOTEKEYMAP}{& }
    & Exits bounce demo\\
\end{btnmap}

Available options are:

\begin{description}
\item[Xdist/Ydist.] The distance to X axis and Y axis
respectively
\item[Xadd/Yadd.]How fast the code moves on the sine curve on
each axis
\item[Xsane/Ysane.] Changes the appearance of the bouncing.
\end{description}


\input{plugins/credits.tex}

\subsection{Cube}
\screenshot{plugins/images/ss-cube}{Cube}{img:cube}
This is a rotating cube screen saver in 3D.
\begin{btnmap}
    \opt{PLAYER_PAD,RECORDER_PAD}{\ButtonOn}
    \opt{ONDIO_PAD}{\ButtonMenu+\ButtonRight}
    \opt{IAUDIO_X5_PAD,SANSA_E200_PAD,SANSA_FUZE_PAD,SANSA_C200_PAD,SANSA_CLIP_PAD%
        ,IRIVER_H100_PAD,IRIVER_H300_PAD,GIGABEAT_S_PAD}{\ButtonSelect}
    \opt{IPOD_4G_PAD,IPOD_3G_PAD}{\ButtonSelect+\ButtonPlay}
    \opt{IRIVER_H10_PAD}{\ButtonFF}
    \opt{SAMSUNG_YH92X_PAD,SAMSUNG_YH820_PAD}{Long \ButtonFF}
    \opt{GIGABEAT_PAD}{\ButtonA}
    \opt{MROBE100_PAD}{\ButtonDisplay}
    \opt{COWON_D2_PAD}{\TouchBottomRight}
    \opt{PBELL_VIBE500_PAD}{\ButtonOK}
    \opt{MPIO_HD200_PAD}{\ButtonFunc}
    \opt{MPIO_HD300_PAD}{\ButtonEnter}
    \opt{SANSA_FUZEPLUS_PAD}{\ButtonBack}
       \opt{HAVEREMOTEKEYMAP}{& }
        & Display at maximum frame rate\\
    \opt{PLAYER_PAD,RECORDER_PAD,IPOD_4G_PAD,IPOD_3G_PAD,IAUDIO_X5_PAD%
        ,IRIVER_H10_PAD,GIGABEAT_S_PAD,PBELL_VIBE500_PAD,MPIO_HD200_PAD%
        ,MPIO_HD300_PAD,SANSA_FUZEPLUS_PAD,SAMSUNG_YH92X_PAD,SAMSUNG_YH820_PAD}{\ButtonPlay}
    \opt{ONDIO_PAD}{\ButtonMenu+\ButtonLeft}
    \opt{IRIVER_H100_PAD,IRIVER_H300_PAD}{\ButtonOn}
    \opt{GIGABEAT_PAD}{\ButtonSelect}
    \opt{SANSA_E200_PAD,SANSA_FUZE_PAD,SANSA_C200_PAD,SANSA_CLIP_PAD,MROBE100_PAD}{\ButtonUp}
    \opt{COWON_D2_PAD}{\TouchCenter}
       \opt{HAVEREMOTEKEYMAP}{& }
        & Pause\\
    \opt{PLAYER_PAD,ONDIO_PAD,GIGABEAT_PAD,GIGABEAT_S_PAD,MROBE100_PAD,PBELL_VIBE500_PAD}
        {\ButtonMenu}
    \opt{RECORDER_PAD}{\ButtonFThree}
    \opt{IRIVER_H100_PAD,IRIVER_H300_PAD}{\ButtonMode}
    \opt{IPOD_4G_PAD,IPOD_3G_PAD}{\ButtonSelect+\ButtonMenu}
    \opt{IAUDIO_X5_PAD,SANSA_FUZEPLUS_PAD}{\ButtonSelect}
    \opt{IRIVER_H10_PAD}{\ButtonRew}
    \opt{SAMSUNG_YH92X_PAD,SAMSUNG_YH820_PAD}{\ButtonFF}
    \opt{SANSA_E200_PAD,SANSA_FUZE_PAD,SANSA_C200_PAD,SANSA_CLIP_PAD}{\ButtonDown}
    \opt{COWON_D2_PAD}{\TouchTopRight}
    \opt{MPIO_HD200_PAD}{\ButtonRec}
    \opt{MPIO_HD300_PAD}{\ButtonMenu}
       \opt{HAVEREMOTEKEYMAP}{& }
        & Cycle draw mode\\
    \opt{ONDIO_PAD,GIGABEAT_PAD,GIGABEAT_S_PAD,MROBE100_PAD,RECORDER_PAD%
        ,IRIVER_H100_PAD,IRIVER_H300_PAD,IPOD_4G_PAD,IPOD_3G_PAD,IAUDIO_X5_PAD%
        ,IRIVER_H10_PAD,SANSA_E200_PAD,SANSA_FUZE_PAD,SANSA_C200_PAD,SANSA_CLIP_PAD%
        ,PBELL_VIBE500_PAD,SANSA_FUZEPLUS_PAD,SAMSUNG_YH92X_PAD,SAMSUNG_YH820_PAD}
        {\ButtonRight{} / \ButtonLeft}
    \opt{PLAYER_PAD}{\ButtonOn+\ButtonRight{} / \ButtonOn+\ButtonLeft}
    \opt{COWON_D2_PAD}{\TouchMidRight{} / \TouchMidLeft}
    \opt{MPIO_HD200_PAD,MPIO_HD300_PAD}{\ButtonRew / \ButtonFF}
       \opt{HAVEREMOTEKEYMAP}{& }
        & Select axis to adjust\\
    \opt{PLAYER_PAD}{\ButtonRight{} / \ButtonLeft}
    \opt{RECORDER_PAD,ONDIO_PAD,IRIVER_H100_PAD,IRIVER_H300_PAD,IAUDIO_X5_PAD%
        ,GIGABEAT_PAD,GIGABEAT_S_PAD,MROBE100_PAD,PBELL_VIBE500_PAD%
        ,SANSA_FUZEPLUS_PAD,SAMSUNG_YH92X_PAD,SAMSUNG_YH820_PAD}{\ButtonUp{} / \ButtonDown}
    \opt{IPOD_4G_PAD,IPOD_3G_PAD,SANSA_E200_PAD,SANSA_FUZE_PAD}{\ButtonScrollFwd{} / \ButtonScrollBack}
    \opt{IRIVER_H10_PAD}{\ButtonScrollUp{} / \ButtonScrollDown}
    \opt{SANSA_C200_PAD,SANSA_CLIP_PAD}{\ButtonVolDown{} / \ButtonVolUp}
    \opt{COWON_D2_PAD}{\TouchTopMiddle{} / \TouchBottomMiddle}
    \opt{MPIO_HD200_PAD}{\ButtonVolDown / \ButtonVolUp}
    \opt{MPIO_HD300_PAD}{\ButtonScrollDown / \ButtonScrollUp}
       \opt{HAVEREMOTEKEYMAP}{& }
        & Change speed/angle (speed can not be changed while paused)\\
    \opt{PLAYER_PAD}{\ButtonStop}
    \opt{RECORDER_PAD,ONDIO_PAD,IRIVER_H100_PAD,IRIVER_H300_PAD}{\ButtonOff}
    \opt{IPOD_4G_PAD,IPOD_3G_PAD}{\ButtonMenu}
    \opt{IAUDIO_X5_PAD,IRIVER_H10_PAD,SANSA_E200_PAD,SANSA_C200_PAD,SANSA_CLIP_PAD,GIGABEAT_PAD%
        ,MROBE100_PAD,COWON_D2_PAD,SANSA_FUZEPLUS_PAD}{\ButtonPower}
    \opt{SANSA_FUZE_PAD}{Long \ButtonHome}
    \opt{GIGABEAT_S_PAD}{\ButtonBack}
    \opt{PBELL_VIBE500_PAD}{\ButtonRec}
    \opt{SAMSUNG_YH92X_PAD,SAMSUNG_YH820_PAD}{\ButtonRew}
    \opt{MPIO_HD200_PAD}{\ButtonRec + \ButtonPlay}
    \opt{MPIO_HD300_PAD}{Long \ButtonMenu}
       \opt{HAVEREMOTEKEYMAP}{& 
          \opt{IRIVER_RC_H100_PAD}{\ButtonRCStop}
       }
        & Quit\\
\end{btnmap}


% $Id$ %
\subsection{Demystify}
\screenshot{plugins/images/ss-demystify}{Demystify}{img:demystify}
Demystify is a screen saver like demo.\\

\begin{btnmap}
    \nopt{scrollwheel}{\PluginLeft{} / \PluginRight}
    \opt{scrollwheel}{\PluginScrollBack{} / \PluginScrollFwd}
       \opt{HAVEREMOTEKEYMAP}{& \PluginRCLeft{} / \PluginRCRight}
    & Increase / decrease speed\\

    \nopt{scrollwheel}{\PluginUp{} / \PluginDown}
    \opt{scrollwheel}{\PluginLeft{} / \PluginRight}
       \opt{HAVEREMOTEKEYMAP}{& \PluginRCUp{} / \PluginRCDown}
    & Add / remove polygon\\

    \nopt{IPOD_4G_PAD,IPOD_3G_PAD}{\PluginCancel}
    \opt{IPOD_4G_PAD,IPOD_3G_PAD}{\PluginUp}
        \opt{HAVEREMOTEKEYMAP}{& \PluginRCCancel}
    & Quit\\
\end{btnmap}


% $Id$ %
\subsection{FFT}
This plugin is a basic frequency analyzer with 3 different frequency-amplitude
plots (lines, bars, and spectrogram).


% $Id$ %
\subsection{Fire}
\screenshot{plugins/images/ss-fire}{Fire}{img:fire}
Fire is a demo displaying a fire effect.
\begin{btnmap}
    \nopt{scrollwheel}{\PluginUp{} / \PluginDown}
    \opt{scrollwheel}{\PluginScrollFwd{} / \PluginScrollBack}
       \opt{HAVEREMOTEKEYMAP}{& \PluginRCUp{} / \PluginRCDown}
    & Increase / decrease number of flames\\

    \PluginLeft
       \opt{HAVEREMOTEKEYMAP}{& \PluginRCLeft}
    & Toggle flame type\\

    \PluginRight
       \opt{HAVEREMOTEKEYMAP}{& \PluginRCRight}
    & Toggle moving flames\\

    \nopt{IPOD_4G_PAD,IPOD_3G_PAD}{\PluginCancel}
    \opt{IPOD_4G_PAD,IPOD_3G_PAD}{\ButtonMenu}
       \opt{HAVEREMOTEKEYMAP}{& \PluginRCCancel}
    & Quit\\
\end{btnmap}


\subsection{Fractals}
\screenshot{plugins/images/ss-mandelbrot}{Fractals: Mandelbrot set}{img:mandelbrot}
This demonstration draws fractal images from the Mandelbrot set%
\nopt{lcd_color}{ using the greyscale engine}.
\begin{btnmap}
    \nopt{MPIO_HD200_PAD,MPIO_HD300_PAD}{Direction keys}
    \opt{MPIO_HD200_PAD}{\ButtonVolDown, \ButtonVolUp, \ButtonRew, \ButtonFF}
    \opt{MPIO_HD300_PAD}{\ButtonScrollDown, \ButtonScrollUp, \ButtonRew, \ButtonFF}
    \opt{HAVEREMOTEKEYMAP}{&}
    & Move about the image\\
    %
    \opt{IRIVER_H10_PAD,PBELL_VIBE500_PAD}{\ButtonPlay}
    \opt{IRIVER_H100_PAD,IRIVER_H300_PAD,IAUDIO_X5_PAD,GIGABEAT_PAD,MROBE100_PAD
        ,SANSA_FUZEPLUS_PAD}{\ButtonSelect}
    \opt{IPOD_4G_PAD,IPOD_3G_PAD,SANSA_E200_PAD,SANSA_FUZE_PAD}{\ButtonScrollFwd}
    \opt{SANSA_C200_PAD,SANSA_CLIP_PAD,GIGABEAT_S_PAD}{\ButtonVolUp}
    \opt{COWON_D2_PAD}{\TouchTopRight}
    \opt{MPIO_HD200_PAD}{\ButtonPlay + \ButtonFF}
    \opt{MPIO_HD300_PAD}{\ButtonPlay + \ButtonScrollUp}
    \opt{SAMSUNG_YH92X_PAD,SAMSUNG_YH820_PAD}{\ButtonFF}
       \opt{HAVEREMOTEKEYMAP}{& }
    & Zoom in\\
    %
    \opt{IRIVER_H100_PAD,IRIVER_H300_PAD}{\ButtonMode}
    \opt{IPOD_4G_PAD,IPOD_3G_PAD,SANSA_E200_PAD,SANSA_FUZE_PAD}{\ButtonScrollBack}
    \opt{IAUDIO_X5_PAD,GIGABEAT_PAD,MROBE100_PAD}{Long \ButtonSelect}
    \opt{IRIVER_H10_PAD}{Long \ButtonPlay}
    \opt{SANSA_FUZEPLUS_PAD}{\ButtonPlay}
    \opt{SANSA_C200_PAD,SANSA_CLIP_PAD,GIGABEAT_S_PAD}{\ButtonVolDown}
    \opt{COWON_D2_PAD}{\TouchTopLeft}
    \opt{PBELL_VIBE500_PAD}{\ButtonMenu}
    \opt{MPIO_HD200_PAD}{\ButtonPlay + \ButtonRew}
    \opt{MPIO_HD300_PAD}{\ButtonPlay + \ButtonScrollDown}
    \opt{SAMSUNG_YH92X_PAD,SAMSUNG_YH820_PAD}{\ButtonRew}
       \opt{HAVEREMOTEKEYMAP}{& }
    & Zoom out\\
    %
    \opt{IRIVER_H100_PAD,IRIVER_H300_PAD}{\ButtonOn+\ButtonLeft}
    \opt{IPOD_4G_PAD,IPOD_3G_PAD,SANSA_E200_PAD,SANSA_FUZE_PAD
        ,SANSA_C200_PAD,SANSA_CLIP_PAD}{\ButtonSelect+\ButtonLeft}
    \opt{IAUDIO_X5_PAD}{Long \ButtonPlay}
    \opt{IRIVER_H10_PAD}{\ButtonRew}
    \opt{GIGABEAT_PAD}{\ButtonVolDown}
    \opt{GIGABEAT_S_PAD}{\ButtonNext}
    \opt{MROBE100_PAD}{\ButtonPlay}
    \opt{COWON_D2_PAD}{\TouchBottomLeft}
    \opt{PBELL_VIBE500_PAD}{\ButtonCancel}
    \opt{MPIO_HD200_PAD}{\ButtonPlay + \ButtonVolDown}
    \opt{MPIO_HD300_PAD}{\ButtonPlay + \ButtonRew}
    \opt{SAMSUNG_YH92X_PAD}{\ButtonPlay{} + \ButtonDown}
    \opt{SAMSUNG_YH820_PAD}{\ButtonRec{} + \ButtonDown}
    \opt{SANSA_FUZEPLUS_PAD}{\ButtonBottomLeft}
       \opt{HAVEREMOTEKEYMAP}{& }
    & Decrease iteration depth (less detail)\\
    %
    \opt{IRIVER_H100_PAD,IRIVER_H300_PAD}{\ButtonOn+\ButtonRight}
    \opt{IPOD_4G_PAD,IPOD_3G_PAD,SANSA_E200_PAD,SANSA_FUZE_PAD,SANSA_C200_PAD,SANSA_CLIP_PAD}{\ButtonSelect+\ButtonRight}
    \opt{IAUDIO_X5_PAD}{\ButtonPlay}
    \opt{IRIVER_H10_PAD}{\ButtonFF}
    \opt{GIGABEAT_PAD}{\ButtonVolUp}
    \opt{GIGABEAT_S_PAD}{\ButtonPrev}
    \opt{MROBE100_PAD}{\ButtonMenu}
    \opt{COWON_D2_PAD}{\TouchBottomRight}
    \opt{PBELL_VIBE500_PAD}{\ButtonOK}
    \opt{MPIO_HD200_PAD}{\ButtonPlay + \ButtonVolUp}
    \opt{MPIO_HD300_PAD}{\ButtonPlay + \ButtonFF}
    \opt{SAMSUNG_YH92X_PAD}{\ButtonPlay{} + \ButtonUp}
    \opt{SAMSUNG_YH820_PAD}{\ButtonRec{} + \ButtonUp}
    \opt{SANSA_FUZEPLUS_PAD}{\ButtonBottomRight}
       \opt{HAVEREMOTEKEYMAP}{& }
    & Increase iteration depth (more detail)\\
    %
    \opt{IRIVER_H100_PAD,IRIVER_H300_PAD,IAUDIO_X5_PAD,SANSA_E200_PAD,SANSA_C200_PAD}{\ButtonRec}
    \opt{SANSA_CLIP_PAD}{\ButtonHome}
    \opt{SANSA_FUZE_PAD}{Long \ButtonSelect}
    \opt{IPOD_4G_PAD,IPOD_3G_PAD}{\ButtonSelect+\ButtonPlay}
    \opt{IRIVER_H10_PAD}{\ButtonPlay + \ButtonRew}
    \opt{GIGABEAT_PAD}{\ButtonA}
    \opt{GIGABEAT_S_PAD}{\ButtonMenu}
    \opt{MROBE100_PAD}{\ButtonDisplay}
    \opt{COWON_D2_PAD}{\TouchCenter}
    \opt{PBELL_VIBE500_PAD}{Long \ButtonCancel}
    \opt{MPIO_HD200_PAD,MPIO_HD300_PAD}{\ButtonRec}
    \opt{SAMSUNG_YH92X_PAD}{\ButtonPlay{} + \ButtonRight}
    \opt{SAMSUNG_YH820_PAD}{\ButtonPlay}
    \opt{SANSA_FUZEPLUS_PAD}{\ButtonBack}
       \opt{HAVEREMOTEKEYMAP}{& }
    & Reset and return to the default image\\
    %
    \opt{IRIVER_H100_PAD,IRIVER_H300_PAD}{\ButtonOff}
    \opt{IPOD_4G_PAD,IPOD_3G_PAD}{\ButtonSelect+\ButtonMenu}
    \opt{IAUDIO_X5_PAD,IRIVER_H10_PAD,SANSA_E200_PAD,SANSA_C200_PAD,SANSA_CLIP_PAD%
        ,GIGABEAT_PAD,MROBE100_PAD,COWON_D2_PAD,SANSA_FUZEPLUS_PAD}{\ButtonPower}
    \opt{SANSA_FUZE_PAD}{Long \ButtonHome}
    \opt{GIGABEAT_S_PAD}{\ButtonBack}
    \opt{PBELL_VIBE500_PAD}{\ButtonRec}
    \opt{SAMSUNG_YH92X_PAD,SAMSUNG_YH820_PAD}{Long \ButtonRew}
    \opt{MPIO_HD200_PAD}{\ButtonRec + \ButtonPlay}
    \opt{MPIO_HD300_PAD}{Long \ButtonMenu}
       \opt{HAVEREMOTEKEYMAP}{& 
          \opt{IRIVER_RC_H100_PAD}{\ButtonRCStop}
       }
    & Quit\\
\end{btnmap}


% $Id$ %
\subsection{Logo}
Demo showing the Rockbox logo bouncing around the screen.
\begin{btnmap}
        \PluginRight{} / \PluginLeft
        &
    \opt{HAVEREMOTEKEYMAP}{
        &}
    Increase / decrease speed on the x-axis
        \\

        \nopt{IPOD_4G_PAD,IPOD_3G_PAD}{\PluginUp{} / \PluginDown}
        \opt{IPOD_4G_PAD,IPOD_3G_PAD}{\ButtonScrollFwd{} / \ButtonScrollBack}
        &
    \opt{HAVEREMOTEKEYMAP}{
        &}
    Increase / decrease speed on the y-axis
        \\

    \nopt{IPOD_4G_PAD,IPOD_3G_PAD}{\PluginCancel{} or \PluginExit}
    \opt{IPOD_4G_PAD,IPOD_3G_PAD}{\ButtonMenu}
        &
    \opt{HAVEREMOTEKEYMAP}{
        &}
    Quit
        \\
\end{btnmap}


\nopt{xduoox3,clip,clipplus}{\subsection{Matrix}

This plugin is a visual demo resembling the scrolling code from ``The Matrix''
(\url{https://en.wikipedia.org/wiki/The_Matrix}).
}

% $Id$ %
%
%
\subsection{Mosaique}
\screenshot{plugins/images/ss-mosaic}{Mosaique}{img:mosaique}
This simple graphics demo draws a mosaic picture on the screen of the \dap.

\begin{btnmap}
    \PluginRight
    & Adjust the speed.\\
    \PluginSelect
    & Restart the drawing process.\\
    \nopt{IPOD_4G_PAD,IPOD_3G_PAD}{\PluginCancel{} or \PluginExit}
    \opt{IPOD_4G_PAD,IPOD_3G_PAD}{\ButtonMenu}
    & Exits Mosaique demo\\
\end{btnmap}


% $Id$ %
\subsection{Oscilloscope}
\screenshot{plugins/images/ss-oscilloscope}{Oscilloscope}{img:oscilloscope}

This demo shows the shape of the sound samples that make up the music
being played.
\opt{swcodec}{
  At faster speed rates, the \dap\ is less responsive
  to user input and music may start to skip.
}

\subsubsection{Keys}

\begin{btnmap}
    \opt{RECORDER_PAD}{\ButtonFOne}
    \opt{ONDIO_PAD,PBELL_VIBE500_PAD}{\ButtonMenu}
    \opt{IRIVER_H100_PAD,IRIVER_H300_PAD,IAUDIO_X5_PAD,SANSA_E200_PAD%
      ,SANSA_FUZE_PAD,SANSA_C200_PAD,SANSA_CLIP_PAD,GIGABEAT_PAD,MROBE100_PAD%
      ,SANSA_FUZEPLUS_PAD}{\ButtonSelect}
    \opt{IPOD_4G_PAD,IPOD_3G_PAD}{\ButtonSelect+\ButtonPlay}
    \opt{IRIVER_H10_PAD}{\ButtonRew}
    \opt{GIGABEAT_S_PAD}{\ButtonPrev}
    \opt{COWON_D2_PAD}{\TouchTopMiddle}
    \opt{MPIO_HD200_PAD}{\ButtonFunc}
    \opt{MPIO_HD300_PAD}{\ButtonEnter}
    \opt{SAMSUNG_YH92X_PAD,SAMSUNG_YH820_PAD}{\ButtonPlay+\ButtonLeft}
       \opt{HAVEREMOTEKEYMAP}{& }
    & Toggle filled / curve / plot \\
    \opt{RECORDER_PAD}{\ButtonFTwo}
    \opt{ONDIO_PAD}{\ButtonMenu+\ButtonRight}
    \opt{IRIVER_H100_PAD,IRIVER_H300_PAD}{\ButtonMode}
    \opt{IPOD_4G_PAD,IPOD_3G_PAD}{\ButtonSelect+\ButtonRight}
    \opt{IAUDIO_X5_PAD}{\ButtonRec}
    \opt{IRIVER_H10_PAD}{\ButtonFF}
    \opt{SANSA_E200_PAD,SANSA_C200_PAD,SANSA_CLIP_PAD,GIGABEAT_PAD,MROBE100_PAD}{\ButtonDown}
    \opt{SANSA_FUZE_PAD}{\ButtonSelect+\ButtonRight}
    \opt{GIGABEAT_S_PAD}{\ButtonNext}
    \opt{COWON_D2_PAD}{\TouchBottomMiddle}
    \opt{PBELL_VIBE500_PAD}{\ButtonCancel}
    \opt{SANSA_FUZEPLUS_PAD}{\ButtonBack}
    \opt{MPIO_HD200_PAD,MPIO_HD300_PAD}{\ButtonRec}
    \opt{SAMSUNG_YH92X_PAD,SAMSUNG_YH820_PAD}{\ButtonPlay+\ButtonRight}
       \opt{HAVEREMOTEKEYMAP}{& }
    & Toggle whether to scroll or not \\
    \opt{RECORDER_PAD}{\ButtonFThree}
    \opt{ONDIO_PAD}{\ButtonMenu+\ButtonLeft}
    \opt{IRIVER_H100_PAD,IRIVER_H300_PAD}{\ButtonRec}
    \opt{IPOD_4G_PAD,IPOD_3G_PAD}{\ButtonSelect+\ButtonLeft}
    \opt{IAUDIO_X5_PAD,SANSA_CLIP_PAD}{Long \ButtonSelect}
    \opt{IRIVER_H10_PAD}{Long \ButtonRew}
    \opt{SANSA_E200_PAD,SANSA_C200_PAD,GIGABEAT_PAD,MROBE100_PAD}{\ButtonUp}
    \opt{SANSA_FUZE_PAD}{\ButtonSelect+\ButtonLeft}
    \opt{GIGABEAT_S_PAD}{\ButtonMenu}
    \opt{COWON_D2_PAD}{\TouchBottomLeft}
    \opt{PBELL_VIBE500_PAD}{\ButtonOK}
    \opt{MPIO_HD200_PAD}{Long \ButtonFunc}
    \opt{MPIO_HD300_PAD}{\ButtonMenu}
    \opt{SANSA_FUZEPLUS_PAD}{\ButtonUp}
    \opt{SAMSUNG_YH92X_PAD,SAMSUNG_YH820_PAD}{\ButtonPlay+\ButtonUp}
       \opt{HAVEREMOTEKEYMAP}{& }
    & Toggle drawing orientation \\
    \opt{RECORDER_PAD,IPOD_4G_PAD,IPOD_3G_PAD,IAUDIO_X5_PAD,IRIVER_H10_PAD%
        ,GIGABEAT_S_PAD,PBELL_VIBE500_PAD,MPIO_HD200_PAD,MPIO_HD300_PAD%
        ,SANSA_FUZEPLUS_PAD}%
        {\ButtonPlay}
    \opt{ONDIO_PAD}{\ButtonMenu+\ButtonOff}
    \opt{IRIVER_H100_PAD,IRIVER_H300_PAD}{\ButtonOn}
    \opt{SANSA_E200_PAD,SANSA_C200_PAD}{\ButtonRec}
    \opt{SANSA_FUZE_PAD,SANSA_CLIP_PAD}{\ButtonUp}
    \opt{GIGABEAT_PAD}{\ButtonA}
    \opt{MROBE100_PAD}{\ButtonDisplay}
    \opt{COWON_D2_PAD}{\TouchCenter}
    \opt{SAMSUNG_YH92X_PAD,SAMSUNG_YH820_PAD}{\ButtonPlay+\ButtonDown}
       \opt{HAVEREMOTEKEYMAP}{& }
    & Pause the demo \\
    \opt{RECORDER_PAD,ONDIO_PAD,IRIVER_H100_PAD,IRIVER_H300_PAD,IAUDIO_X5_PAD,PBELL_VIBE500_PAD%
        ,SAMSUNG_YH92X_PAD,SAMSUNG_YH820_PAD}%
        {\ButtonUp{} / \ButtonDown}
    \opt{IPOD_4G_PAD,IPOD_3G_PAD,SANSA_E200_PAD,SANSA_FUZE_PAD}{\ButtonScrollFwd{} / \ButtonScrollBack}
    \opt{IRIVER_H10_PAD,MPIO_HD300_PAD}{\ButtonScrollUp{} / \ButtonScrollDown}
    \opt{SANSA_C200_PAD,SANSA_CLIP_PAD,GIGABEAT_PAD,GIGABEAT_S_PAD,MROBE100_PAD,MPIO_HD200_PAD%
        ,SANSA_FUZEPLUS_PAD}
        {\ButtonVolUp{} / \ButtonVolDown}
    \opt{COWON_D2_PAD}{\ButtonPlus{} / \ButtonMinus}
       \opt{HAVEREMOTEKEYMAP}{& }
    & Increase / decrease volume\\
    \opt{RECORDER_PAD,ONDIO_PAD,IRIVER_H100_PAD,IRIVER_H300_PAD,IAUDIO_X5_PAD%
        ,IPOD_4G_PAD,IPOD_3G_PAD,SANSA_E200_PAD,SANSA_FUZE_PAD,IRIVER_H10_PAD%
        ,SANSA_C200_PAD,SANSA_CLIP_PAD,GIGABEAT_PAD,GIGABEAT_S_PAD,MROBE100_PAD,PBELL_VIBE500_PAD%
        ,SANSA_FUZEPLUS_PAD,SAMSUNG_YH92X_PAD,SAMSUNG_YH820_PAD}
        {\ButtonRight{} / \ButtonLeft}
    \opt{COWON_D2_PAD}{\TouchMidRight\ /\ \TouchMidLeft}
    \opt{MPIO_HD200_PAD,MPIO_HD300_PAD}{\ButtonFF / \ButtonRew}
       \opt{HAVEREMOTEKEYMAP}{& }
    & Increase / decrease speed\\
    \opt{RECORDER_PAD,ONDIO_PAD,IRIVER_H100_PAD,IRIVER_H300_PAD}{\ButtonOff}
    \opt{IPOD_4G_PAD,IPOD_3G_PAD}{\ButtonSelect+\ButtonMenu}
    \opt{IAUDIO_X5_PAD,IRIVER_H10_PAD,SANSA_E200_PAD,SANSA_C200_PAD,SANSA_CLIP_PAD%
        ,GIGABEAT_PAD,MROBE100_PAD,COWON_D2_PAD,SANSA_FUZEPLUS_PAD}
        {\ButtonPower}
    \opt{SANSA_FUZE_PAD}{Long \ButtonHome}
    \opt{GIGABEAT_S_PAD}{\ButtonBack}
    \opt{PBELL_VIBE500_PAD,SAMSUNG_YH92X_PAD,SAMSUNG_YH820_PAD}{\ButtonRec}
    \opt{MPIO_HD200_PAD}{\ButtonRec + \ButtonPlay}
    \opt{MPIO_HD300_PAD}{Long \ButtonMenu}
       \opt{HAVEREMOTEKEYMAP}{&
          \opt{IRIVER_RC_H100_PAD}{\ButtonRCStop}
        }
    & Exit demo \\
\end{btnmap}


\opt{tagcache}{\subsection{PictureFlow}
\screenshot{plugins/images/ss-pictureflow}{PictureFlow}{img:pictureflow}
PictureFlow provides a visualisation of your albums with their associated cover
art. \opt{swcodec}{It is possible to start playback of the selected
album from PictureFlow. Playback will start from the selected track. The 
PictureFlow plugin will continue to run while your tracks are played.}

\opt{hwcodec}{
\note{PictureFlow is a visualisation only. It cannot be used to select and
play music.  Also, using this plugin will cause playback to stop.}
}

\subsubsection{Requirements}
PictureFlow uses both the album art (see \reference{ref:album_art}) and 
database (see \reference{ref:database}) features of Rockbox.
It is therefore important that these are working correctly before attempting
to use PictureFlow. In addition, there are some other points of which to be
aware:

  \begin{itemize}
    \item PictureFlow will accept album art larger than the dimensions of the
    screen, but the larger the dimensions, the longer they will take to scale.
  \end{itemize}

\subsubsection{Keys}
    \begin{btnmap}
        \opt{scrollwheel,IRIVER_H10_PAD,PBELL_VIBE500_PAD,MPIO_HD300_PAD,SAMSUNG_YH92X_PAD}{
            \ActionStdPrev{} / \ActionStdNext
                &
            \opt{HAVEREMOTEKEYMAP}{
                &}
            Scroll through albums / track list
                \\
        }
        
        % only scroll wheel and `strip' targets use the same action in album and track list
        \nopt{scrollwheel,IRIVER_H10_PAD,PBELL_VIBE500_PAD,MPIO_HD300_PAD,SAMSUNG_YH92X_PAD}{%
            % currently the M3 does not use buttons of the main unit which has no display
            \nopt{IAUDIO_M3_PAD,MPIO_HD200_PAD,touchscreen}{\ButtonLeft{} / \ButtonRight}
            \opt{MPIO_HD200_PAD}{FIXME}
            \opt{touchscreen}{\TouchMidLeft{} / \TouchMidRight}
                &
            \opt{HAVEREMOTEKEYMAP}{
                \opt{IAUDIO_M3_PAD,GIGABEAT_RC_PAD}{\ActionRCStdPrev{} / \ActionRCStdNext}
                &}
            Scroll through albums
                \\

            \nopt{IAUDIO_M3_PAD}{\ActionStdPrev{} / \ActionStdNext}
                &
            \opt{HAVEREMOTEKEYMAP}{
                % even though the M3 uses an Iaudio remote, mapping differs when used with M/X5
                \opt{IAUDIO_M3_PAD}{\ButtonRCLeft{} / \ButtonRCRight}
                \opt{GIGABEAT_RC_PAD}{\ButtonRCVolUp{} / \ButtonRCVolDown}
                &}
            Scroll through track list
                \\
        }

        \nopt{IAUDIO_M3_PAD}{%
            \nopt{ONDIO_PAD,IRIVER_H10_PAD,RECORDER_PAD,touchscreen,PBELL_VIBE500_PAD%
                 ,SANSA_FUZE_PAD,MPIO_HD200_PAD,MPIO_HD300_PAD,SAMSUNG_YH92X_PAD}
                 {\ButtonSelect}
            \opt{ONDIO_PAD}{\ButtonUp} 
            \opt{IRIVER_H10_PAD,PBELL_VIBE500_PAD,SAMSUNG_YH92X_PAD}{\ButtonRight} 
            \opt{RECORDER_PAD}{\ButtonOn} 
            \opt{touchscreen}{\TouchCenter}
            \opt{SANSA_FUZE_PAD}{\ButtonRight}
            \opt{MPIO_HD200_PAD}{\ButtonFunc}
            \opt{MPIO_HD300_PAD}{\ButtonEnter}
        }
            &
        \opt{HAVEREMOTEKEYMAP}{
            \opt{IAUDIO_M3_PAD}{\ButtonRCPlay}
            \opt{GIGABEAT_RC_PAD}{\ButtonRCFF}
            &}
        Enter track list
            \nopt{ONDIO_PAD}{/ Play album from selected track}
            \\
            
        % Ondio uses a different button in album list and track list
        \opt{ONDIO_PAD}{
            \ButtonMenu
                &
            Play album from selected track in track list
                \\
        }
        
        \nopt{IAUDIO_M3_PAD,MPIO_HD200_PAD,MPIO_HD300_PAD,touchscreen,SANSA_FUZEPLUS_PAD}{\ButtonLeft}
        \opt{MPIO_HD200_PAD}{\ButtonRec}
        \opt{MPIO_HD300_PAD}{\ButtonMenu}
        \opt{SANSA_FUZEPLUS_PAD}{\ButtonLeft{} or \ButtonBack}
        \opt{touchscreen}{
            \opt{COWON_D2_PAD}{\ButtonPower{} or}
            \TouchBottomRight}
            &
        \opt{HAVEREMOTEKEYMAP}{
            \opt{IAUDIO_M3_PAD,GIGABEAT_RC_PAD}{\ActionRCStdCancel}
            &}
        Exit track list
            \\

        \nopt{IAUDIO_M3_PAD,SANSA_FUZEPLUS_PAD}{\ActionStdMenu}
            \opt{SANSA_FUZEPLUS_PAD}{long \ButtonSelect}
            &
        \opt{HAVEREMOTEKEYMAP}{
            \opt{IAUDIO_M3_PAD,GIGABEAT_RC_PAD}{\ActionRCStdMenu}
            &}
        Enter menu
            \\

        \nopt{IAUDIO_M3_PAD}{%
            \opt{IRIVER_H100_PAD,IRIVER_H300_PAD,RECORDER_PAD,ONDIO_PAD}{\ButtonOff}
            \opt{IAUDIO_X5_PAD,GIGABEAT_PAD,GIGABEAT_S_PAD,SANSA_E200_PAD,SANSA_CLIP_PAD%
                ,MROBE100_PAD,SANSA_FUZEPLUS_PAD}{\ButtonPower}
            \opt{SANSA_C200_PAD,IRIVER_H10_PAD}{Long \ButtonPower}
            \opt{IPOD_4G_PAD,IPOD_3G_PAD}{Long \ButtonMenu}
            \opt{SANSA_FUZE_PAD}{Long \ButtonHome} 
            \opt{PBELL_VIBE500_PAD,SAMSUNG_YH92X_PAD}{\ButtonRec}
            \opt{MPIO_HD200_PAD}{FIXME}
            \opt{MPIO_HD300_PAD}{Long \ButtonMenu}
            \opt{touchscreen}{
                \opt{COWON_D2_PAD}{Long \ButtonPower{} or}
                \TouchBottomRight{} (in album view)}
        }
            &
        \opt{HAVEREMOTEKEYMAP}{
            \opt{IAUDIO_M3_PAD}{\ButtonRCRec}
            \opt{GIGABEAT_RC_PAD}{\ButtonRCRew}
            &}
        Exit PictureFlow
            \\
            
    \end{btnmap}

\subsubsection{Main Menu}
\begin{description}
  \item[Go to WPS.] Leave PictureFlow and enter the while playing screen.
  \opt{swcodec}{\item[Playback Control.] Control music playback from within the plugin.}
  \item[Settings.] Enter the settings menu.
  \item[Return.] Exit menu.
  \item[Quit.] Exit PictureFlow plugin.
\end{description}

\subsubsection{Settings Menu}

\begin{description}
  \item[Show FPS.] Displays frames per second on screen.
  \item[Spacing.] The distance between the front edges of the side slides, i.e. changes
  the degree of overlap of the side slides. A larger number means less overlap. Scales with zoom.
  \item[Centre margin.] The distance, in screen pixels, with zoom at 100, between
  the centre and side slides. Scales with zoom.
  \item[Number of slides.] Sets the number of slides at each side, including the
  centre slide. Therefore if set to 4, there will be 3 slides on the left,
  the centre slide, and then 3 slides on the right.
  \item[Zoom.] Changes the distance at which slides are rendered from the ``camera''.
  \item[Show album title.] Allows setting the album title to be shown above or
  below the cover art, or not at all.
  \item[Resize Covers.] Set whether to automatically resize the covers or to leave
  them at their original size.
  \item[Rebuild cache.] Rebuild the PictureFlow cache. This is needed in order
  for PictureFlow to pick up new albums, and may occasionally be needed if albums
  are removed.
\end{description}
}

% $Id$ %
\subsection{Plasma}
\screenshot{plugins/images/ss-plasma}{Plasma}{img:plasma}
Plasma is a demo displaying a 80's style retro plasma effect.

      \begin{btnmap}
          \nopt{IPOD_4G_PAD,IPOD_3G_PAD}{\PluginUp{} / \PluginDown}
          \opt{scrollwheel}{/ \PluginScrollFwd{} / \PluginScrollBack}
              &
          \opt{HAVEREMOTEKEYMAP}{\PluginRCUp{} / \PluginRCDown
              &}
          Increase / decrease Frequency
              \\
       
          \opt{lcd_color}{%
              \PluginSelect
                  &
              \opt{HAVEREMOTEKEYMAP}{\PluginRCSelect
                  &}
              Change Color
                  \\
          }%
       
          \nopt{IPOD_4G_PAD,IPOD_3G_PAD}{\PluginCancel{} / \PluginExit}
          \opt{IPOD_4G_PAD,IPOD_3G_PAD}{\ButtonMenu}
              &
          \opt{HAVEREMOTEKEYMAP}{\PluginRCCancel
              \opt{IAUDIO_RC_PAD}{ / \PluginRCExit}
              &}
          Exit
              \\

\end{btnmap}

\subsection{Rocklife}

This an implementation of J. H. Conway's Game of Life (see
\url{http://en.wikipedia.org/wiki/Conway%27s_Game_of_Life} for a detailed
description).

Rockbox can open files with a configuration description (\fname{.cells} files).
Just ``play'' such file and the game configuration stored in it will be loaded
into this plugin.

A \fname{.cells} file is a text file. A capital `O' marks a live cell, a dot
marks a dead cell, all other characters are ignored. Everything on a line
starting with an exclamation sign (and including it) is a comment and is
ignored.

\begin{btnmap}
    \PluginSelect
      \opt{HAVEREMOTEKEYMAP}{& \PluginRCSelect}
    & Play/pause\\

    \PluginDown
      \opt{HAVEREMOTEKEYMAP}{& \PluginRCDown}
    & Change growth mode\\

    \PluginRight
      \opt{HAVEREMOTEKEYMAP}{& \PluginRCRight}
    & Next generation\\

    \PluginLeft
      \opt{HAVEREMOTEKEYMAP}{& \PluginRCLeft}
    & Status (only when paused)\\

    \PluginCancel
      \opt{HAVEREMOTEKEYMAP}{& \PluginRCCancel}
    & Exit\\
\end{btnmap}

\subsection{Snow}
\screenshot{plugins/images/ss-snow}{Have you ever seen snow falling?}{img:snow}
This demo replicates snow falling on your screen. If you love winter,
you will love this demo.  Or maybe not.
Press
\nopt{IPOD_4G_PAD,IPOD_3G_PAD}{\PluginCancel{} or \PluginExit{}}
\opt{IPOD_4G_PAD,IPOD_3G_PAD}{\ButtonMenu}
to quit.



% $Id$ %
\subsection{Starfield}
\screenshot{plugins/images/ss-starfield}{Starfield}{img:starfield}
Starfield simulation (like the classic screensaver).

\begin{btnmap}
    \PluginRight{} / \PluginLeft
  \opt{HAVEREMOTEKEYMAP}{& }
    & Increase / decrease number of stars\\
    
  \nopt{IPOD_4G_PAD,IPOD_3G_PAD}{\PluginUp{} / \PluginDown}
  \opt{IPOD_4G_PAD,IPOD_3G_PAD}{\ButtonScrollFwd{} / \ButtonScrollBack}
  \opt{HAVEREMOTEKEYMAP}{& }
    & Increase / decrease speed\\
    \opt{lcd_color}{%
         \PluginSelect%
       \opt{HAVEREMOTEKEYMAP}{& }
        & Change colours\\%
    }%
  \nopt{IPOD_4G_PAD,IPOD_3G_PAD}{\PluginCancel{} or \PluginExit}
  \opt{IPOD_4G_PAD,IPOD_3G_PAD}{\ButtonMenu}
  \opt{HAVEREMOTEKEYMAP}{& }
    & Quit\\
\end{btnmap}


\subsection{VU meter}
\screenshot{plugins/images/ss-vumeter}{VU-Meter}{}

This is a VU meter, which displays the volume of the left and right
audio channels. There are 3 types of meter selectable.  The analogue
meter is a classic needle style.  The digital meter is modelled after
LED volume displays, and the mini{}-meter option allows for the display
of small meters in addition to the main display (as above).  From the
settings menu the decay time for the meter (its memory), the meter type
and the meter scale can be changed.

\begin{btnmap}
\opt{RECORDER_PAD,ONDIO_PAD,IRIVER_H100_PAD,IRIVER_H300_PAD}{\ButtonOff}
\opt{IPOD_4G_PAD,IPOD_3G_PAD}{\ButtonMenu}
\opt{IAUDIO_X5_PAD,IRIVER_H10_PAD,SANSA_E200_PAD,SANSA_C200_PAD,SANSA_CLIP_PAD,GIGABEAT_PAD%
     ,MROBE100_PAD,COWON_D2_PAD}{\ButtonPower}
\opt{SANSA_FUZE_PAD}{Long \ButtonHome}
\opt{GIGABEAT_S_PAD,SANSA_FUZEPLUS_PAD}{\ButtonBack}
\opt{PBELL_VIBE500_PAD,SAMSUNG_YH92X_PAD,SAMSUNG_YH820_PAD}{\ButtonRec}
\opt{MPIO_HD200_PAD}{\ButtonRec + \ButtonPlay}
\opt{MPIO_HD300_PAD}{Long \ButtonMenu}
  \opt{HAVEREMOTEKEYMAP}{&
          \opt{IRIVER_RC_H100_PAD}{\ButtonRCStop}
  }
    & Save settings and quit\\
\opt{RECORDER_PAD,IRIVER_H100_PAD,IRIVER_H300_PAD}{\ButtonOn}
\opt{ONDIO_PAD}{\ButtonMenu}
\opt{IPOD_4G_PAD,IPOD_3G_PAD,IAUDIO_X5_PAD,IRIVER_H10_PAD,PBELL_VIBE500_PAD%
    ,MPIO_HD200_PAD,MPIO_HD300_PAD,SANSA_FUZEPLUS_PAD,SAMSUNG_YH92X_PAD%
    ,SAMSUNG_YH820_PAD}{\ButtonPlay}
\opt{SANSA_E200_PAD,SANSA_C200_PAD}{\ButtonRec}
\opt{SANSA_CLIP_PAD}{\ButtonHome}
\opt{SANSA_FUZE_PAD}{Long \ButtonSelect}
\opt{GIGABEAT_PAD}{\ButtonA}
\opt{MROBE100_PAD}{\ButtonDisplay}
\opt{GIGABEAT_S_PAD}{\ButtonNext}
\opt{COWON_D2_PAD}{\TouchCenter}
  \opt{HAVEREMOTEKEYMAP}{& }
    & Help\\
\opt{RECORDER_PAD}{\ButtonFOne}
\opt{ONDIO_PAD}{Long \ButtonMenu}
\opt{IRIVER_H100_PAD,IRIVER_H300_PAD,IPOD_4G_PAD,IPOD_3G_PAD,IAUDIO_X5_PAD%
     ,SANSA_E200_PAD,SANSA_FUZE_PAD,SANSA_C200_PAD,SANSA_CLIP_PAD}{\ButtonSelect}
\opt{SANSA_FUZEPLUS_PAD}{Long \ButtonSelect}
\opt{IRIVER_H10_PAD,SAMSUNG_YH92X_PAD,SAMSUNG_YH820_PAD}{\ButtonRew}
\opt{GIGABEAT_PAD,GIGABEAT_S_PAD,MROBE100_PAD,COWON_D2_PAD,PBELL_VIBE500_PAD%
    ,MPIO_HD300_PAD}{\ButtonMenu}
\opt{MPIO_HD200_PAD}{\ButtonFunc}
  \opt{HAVEREMOTEKEYMAP}{& }
    & Settings\\
\opt{RECORDER_PAD,ONDIO_PAD,IRIVER_H100_PAD,IRIVER_H300_PAD,IAUDIO_X5_PAD%
     ,GIGABEAT_PAD,GIGABEAT_S_PAD,MROBE100_PAD,PBELL_VIBE500_PAD,SAMSUNG_YH92X_PAD%
     ,SAMSUNG_YH820_PAD}{\ButtonUp}
\opt{IPOD_4G_PAD,IPOD_3G_PAD,SANSA_E200_PAD,SANSA_FUZE_PAD}{\ButtonScrollFwd}
\opt{SANSA_C200_PAD,SANSA_CLIP_PAD,MPIO_HD200_PAD,SANSA_FUZEPLUS_PAD}{\ButtonVolUp}
\opt{IRIVER_H10_PAD,MPIO_HD300_PAD}{\ButtonScrollUp}
\opt{COWON_D2_PAD}{\TouchTopMiddle}
  \opt{HAVEREMOTEKEYMAP}{& }
    & Raise Volume\\
\opt{RECORDER_PAD,ONDIO_PAD,IRIVER_H100_PAD,IRIVER_H300_PAD,IAUDIO_X5_PAD%
     ,GIGABEAT_PAD,GIGABEAT_S_PAD,MROBE100_PAD,PBELL_VIBE500_PAD,SAMSUNG_YH92X_PAD%
     ,SAMSUNG_YH820_PAD}{\ButtonDown}
\opt{IPOD_4G_PAD,IPOD_3G_PAD,SANSA_E200_PAD,SANSA_FUZE_PAD}{\ButtonScrollBack}
\opt{SANSA_C200_PAD,SANSA_CLIP_PAD,MPIO_HD200_PAD,SANSA_FUZEPLUS_PAD}{\ButtonVolDown}
\opt{IRIVER_H10_PAD,MPIO_HD300_PAD}{\ButtonScrollDown}
\opt{COWON_D2_PAD}{\TouchBottomMiddle}
  \opt{HAVEREMOTEKEYMAP}{& }
    & Lower Volume\\
\end{btnmap}


\section{\label{ref:Viewersplugins}Viewers}

Viewers are plugins which are associated with specific file extensions.
They cannot be run directly but are started by ``playing'' the associated file.
Viewers are stored in the \fname{/.rockbox/rocks/viewers/} directory.
\par
\note{
Some viewer plugins can only be used by selecting the \setting{Open With...}
option from the \setting{Context Menu} (see \reference{ref:Contextmenu}).}
\begin{table}
  \begin{rbtabular}{.92\textwidth}{Xlc}%
      {\textbf{Viewer Plugin}& \textbf{Associated filetype(s)} & \textbf{Context Menu only}}%
      {}{}
    Shortcuts             & \fname{.link}                    &   \\
    MS Windows shortcuts  & \fname{.lnk}                     &   \\
    Chip-8 Emulator       & \fname{.ch8}                     &   \\
    Frotz                 & \fname{.z1} to \fname{.z8}       &   \\
    Image Viewer          & \fname{.bmp, .jpg, .jpeg, .png\opt{lcd_color}{, .ppm}}  &   \\
    Lua scripting language& \fname{.lua}                     &   \\
    \nopt{lowmem}{
        Midiplay          & \fname{.mid, .midi}              &   \\
        Mikmod            & \fname{.669, .amf, .asy, .dsm,}  &   \\
                          & \fname{.far, .gdm, imf, .it,}    &  \\
                          & \fname{.m15, .med, .mod, .okt,}   &  \\
                          & \fname{.s3m, .stm, .stx, .ult,}  &  \\
                          & \fname{.uni, .umx, .xm}         &   \\
        MPEG Player       & \fname{.mpg, .mpeg, .mpv, .m2v}  &   \\
    }
        MP3 Encoder       & \fname{.wav}                     & x \\
        Rockboy       & \fname{.gb, .gbc}                &   \\
    Search                & \fname{.m3u, .m3u8}              & x \\
    Shopping list         & \fname{.shopper}                 &   \\
    Sort                  & \fname{.*}                       & x \\
    Text Viewer           & \fname{.txt,.nfo, .*}            &   \\
    VBRfix                & \fname{.mp3}                     & x \\
    ZXBox                 & \fname{.tap, .tax, .sna, .z80}   &   \\
  \end{rbtabular}
\end{table}

\subsection{Shortcuts}
\label{ref:Shortcutsplugin}

The Shortcuts Plugin allows you to jump to places within the file browser
without having to navigate there manually. The plugin works with
\fname{.link} files. A \fname{.link} file is just a text file with every
line containing the name of the file or the directory you want to quickly
jump to. All names should be full absolute names, i.e. they should start
with a \fname{/}. Directory names should also end with a \fname{/}.

\note{This plugin cannot read Microsoft Windows shortcuts (\fname{.lnk}
files). These are handled by a separate plugin; see
\reference{ref:Winshortcutsplugin}.}

\subsubsection{How to create \fname{.link} files}

You can use your favourite text editor to create a \fname{.link} file on the
PC an then copy the file to the \dap{}. Or you can use the context menu on
either a file or a directory in the file browser tree, and use the ``Add to
shortcuts'' menu option. This will append a line with the full name of the
file or the directory to the \fname{shortcuts.link} file in the root
directory of the \dap{}. (The file will be created if it does not exist
yet.) You can later rename the automatically created \fname{shortcuts.link}
file or move it to another directory if you wish. Subsequent calls of the
context menu will create it again.


\subsubsection{How to use \fname{.link} files, i.e. jump to desired places}

To use a \fname{.link} file just ``play'' it from the file browser. This will
show you a list with the entries in the file. Selecting one of them will
then exit the plugin and leave you within the directory selected, or with
the file selected in the file browser. You can then play the file or do
with it whatever you want. The file will not be ``played'' automatically.

If the \fname{.link} file contains only one entry no list will be shown, you
will directly jump to that location. The file \fname{shortcuts.link} in the
root directory is an exception. After ``playing'' it, the list will be shown
even if the file contains just one entry.

If the list you are seeing is from \fname{shortcuts.link} in the root
directory, you can delete the selected entry by pressing \ActionStdMenu.
Deleting entries from other \fname{.link} files is not possible.


\subsubsection{Advanced Usage}

Placing the line ``\#Display last path segments=n'' (where n is a number) in
the beginning of a \fname{.link} file will leave just the last n segments of
the entries when they are shown. For example, if n is chosen to be 1, then
the entry \fname{/MyMusic/collection/song.mp3} will be shown as
\fname{song.mp3}. This allows you to hide common path prefixes.

You can also provide a custom display name for each entry individually. To
do so, append a tabulator character after the entry's path followed by your
custom name. That name will then be used for showing the entry. For example:
\begin{example}
    /MyMusic/collection/song.mp3<TAB>My favourite song\symbol{33}
\end{example}


\subsection{Windows Shortcuts}
\label{ref:Winshortcutsplugin}

This plugin follows Microsoft Windows Explorer shortcuts (\fname{.lnk} files).
In Rockbox, these types of shortcuts will show up as \fname{.lnk} files. To
follow a shortcut, just ``play'' a \fname{.lnk} file from the file browser.
The plugin will navigate the file browser to the linked file (which
will be highlighted) or directory (which will be opened). Linked files will
not be automatically opened; you must do this manually.

Only relative links across the same volume are supported.

\note{You may like to use native Rockbox shortcuts instead. These can be
    created from within Rockbox itself and have advanced capabilities.
    See \reference{ref:Shortcutsplugin}.}


\subsection{\label{ref:Chip8emulator}Chip{}-8 Emulator}
Chip8 is a kind of assembly language for a long-gone architecture.
This plugin runs games written using the chip8 instructions.
To start a game open a \fname{.ch8} file in the \setting{File Browser}

There are lots of tiny Chip8 games (usually only about 256 bytes to a
couple of KB) which were made popular by the HP48
calculator's emulator for them. The original Chip8 had
64$\times$32 pixel graphics, and the new superchip emulator supports 128$\times$64
graphics.

The only problem is that they are based on a 4$\times$4 keyboard, but since most
games do not use all of the buttons, this can easily be worked around.

To do this, one may put a \fname{.c8k} file with the same name as the
original program which contains new key mappings (for \fname{BLINKY.ch8}, one
writes a \fname{BLINKY.c8k} file). That \fname{.c8k} file contains 16
characters describing the mapping from the Chip8 keyboard to the default key
mapping (that way, several Chip8 keys can be pressed using only one
Rockbox key). For example, a file containing the single line:
\begin{code}
    0122458469ABCDEF
\end{code}
would correspond to the following non-default mappings:

3 $\rightarrow$ 2, 6 $\rightarrow$ 8, 7 $\rightarrow$4, 8 $\rightarrow$ 6.

The default keymappings are:
\begin{table}
    \begin{center}
    \begin{tabularx}{.9\textwidth}{c|ccccccccccccccccc}
    \toprule 
    Chip8 & Off & 0 & 1 & 2 & 3 & 4 & 5 & 6 & 7 & 8 & 9 
        & A & B & C & D & E & F\\
    \midrule
    \begin{sideways}Key\end{sideways}
        &
        % key "off"
        \begin{sideways}
        \opt{RECORDER_PAD,ONDIO_PAD,IRIVER_H100_PAD,IRIVER_H300_PAD}{\ButtonOff}
        \opt{IPOD_4G_PAD,IPOD_3G_PAD}{\ButtonMenu}
        \opt{IAUDIO_X5_PAD,IRIVER_H10_PAD,SANSA_E200_PAD,SANSA_C200_PAD%
          ,GIGABEAT_PAD,MROBE100_PAD,SANSA_FUZE_PAD,SANSA_FUZEPLUS_PAD}
            {\ButtonPower}
        \opt{GIGABEAT_S_PAD}{\ButtonBack}
        \opt{PBELL_VIBE500_PAD,SAMSUNG_YH92X_PAD}{\ButtonRec}
        \opt{MPIO_HD200_PAD}{\ButtonRec + \ButtonPlay}
        \opt{MPIO_HD300_PAD}{Long \ButtonMenu}
        \end{sideways}
        &
        % Key "0" 
        \begin{sideways}
        \opt{SANSA_FUZEPLUS_PAD}{\ButtonVolDown}
        \end{sideways}
        &
        % "Key "1"
        \begin{sideways}
        \opt{SANSA_FUZEPLUS_PAD}{\ButtonBack}
        \opt{RECORDER_PAD}{\ButtonFOne}
        \opt{ONDIO_PAD,SANSA_FUZEPLUS_PAD}{\ButtonUp}
        \opt{GIGABEAT_PAD,GIGABEAT_S_PAD,MROBE100_PAD}{\ButtonMenu}
        \opt{MPIO_HD200_PAD,MPIO_HD300_PAD}{\ButtonRew}
        \end{sideways}
        &
        % Key "2"
        \begin{sideways}
        \opt{IPOD_4G_PAD,IPOD_3G_PAD,SANSA_E200_PAD,SANSA_FUZE_PAD}
            {\ButtonScrollBack}
        \opt{IRIVER_H100_PAD,IRIVER_H300_PAD,IAUDIO_X5_PAD,GIGABEAT_PAD%
          ,RECORDER_PAD,ONDIO_PAD,SANSA_FUZEPLUS_PAD,GIGABEAT_S_PAD%
          ,MROBE100_PAD,PBELL_VIBE500_PAD,SAMSUNG_YH92X_PAD}{\ButtonUp}
        \opt{IRIVER_H10_PAD}{\ButtonScrollUp}
        \opt{SANSA_C200_PAD}{\ButtonVolUp}
        \opt{MPIO_HD200_PAD,MPIO_HD300_PAD}{\ButtonFF}
        \end{sideways}
        &
        % Key "3"
        \begin{sideways}
        \opt{RECORDER_PAD}{\ButtonFThree}
        \opt{MROBE100_PAD,SANSA_FUZEPLUS_PAD}{\ButtonPlay}
        \opt{GIGABEAT_PAD,GIGABEAT_S_PAD}{\ButtonVolDown}
        \opt{MPIO_HD200_PAD}{\ButtonFunc}
        \opt{MPIO_HD300_PAD}{\ButtonMenu}
        \end{sideways}
        &
        % Key "4"
        \begin{sideways}
        \opt{RECORDER_PAD,ONDIO_PAD,IRIVER_H100_PAD,IRIVER_H300_PAD,IPOD_4G_PAD%
            ,IPOD_3G_PAD,IAUDIO_X5_PAD,IRIVER_H10_PAD,SANSA_E200_PAD%
            ,SANSA_C200_PAD,GIGABEAT_PAD,GIGABEAT_S_PAD,MROBE100_PAD%
            ,SANSA_FUZE_PAD,PBELL_VIBE500_PAD,SANSA_FUZEPLUS_PAD,SAMSUNG_YH92X_PAD}
            {\ButtonLeft}
        \opt{MPIO_HD200_PAD}{\ButtonRec}
        \opt{MPIO_HD300_PAD}{\ButtonEnter}
        \end{sideways}
        &
        % Key "5"
        \begin{sideways}
        \opt{RECORDER_PAD}{\ButtonPlay}
        \opt{ONDIO_PAD}{\ButtonMenu}
        \opt{IRIVER_H100_PAD,IRIVER_H300_PAD,IAUDIO_X5_PAD,SANSA_E200_PAD%
            ,SANSA_C200_PAD,GIGABEAT_PAD,GIGABEAT_S_PAD,MROBE100_PAD%
            ,SANSA_FUZE_PAD,SANSA_FUZEPLUS_PAD}
            {\ButtonSelect}
        \opt{IPOD_4G_PAD,IPOD_3G_PAD,IRIVER_H10_PAD,SAMSUNG_YH92X_PAD}{\ButtonPlay}
        \opt{PBELL_VIBE500_PAD}{\ButtonOK}
        \opt{MPIO_HD200_PAD}{\ButtonPlay}
        \opt{MPIO_HD300_PAD}{\ButtonRec}
        \end{sideways}
        &
        % Key "6"
        \begin{sideways}
        \opt{RECORDER_PAD,ONDIO_PAD,IRIVER_H100_PAD,IRIVER_H300_PAD,IPOD_4G_PAD%
            ,IPOD_3G_PAD,IAUDIO_X5_PAD,IRIVER_H10_PAD,SANSA_E200_PAD%
            ,SANSA_C200_PAD,GIGABEAT_PAD,GIGABEAT_S_PAD,MROBE100_PAD%
            ,SANSA_FUZE_PAD,PBELL_VIBE500_PAD,SANSA_FUZEPLUS_PAD,SAMSUNG_YH92X_PAD}
            {\ButtonRight}
        \opt{MPIO_HD200_PAD}{\ButtonVolDown}
        \opt{MPIO_HD300_PAD}{\ButtonPlay}
        \end{sideways}
        &
        % Key "7"
        \begin{sideways}
        \opt{RECORDER_PAD}{\ButtonFTwo}
        \opt{MROBE100_PAD}{\ButtonDisplay}
        \opt{GIGABEAT_PAD,GIGABEAT_S_PAD}{\ButtonVolUp}
        \opt{MPIO_HD200_PAD}{\ButtonVolUp}
        \opt{MPIO_HD300_PAD}{\ButtonScrollUp}
        \opt{SANSA_FUZEPLUS_PAD}{\ButtonBottomLeft}
        \end{sideways}
        &
        % Key "8"
        \begin{sideways}
        \opt{RECORDER_PAD,ONDIO_PAD,IRIVER_H100_PAD,IRIVER_H300_PAD%
            ,GIGABEAT_PAD,GIGABEAT_S_PAD,MROBE100_PAD,PBELL_VIBE500_PAD,SAMSUNG_YH92X_PAD%
            }{\ButtonDown}
        \opt{IPOD_4G_PAD,IPOD_3G_PAD,SANSA_E200_PAD,SANSA_FUZE_PAD}
            {\ButtonScrollFwd}
        \opt{IAUDIO_X5_PAD,SANSA_FUZEPLUS_PAD}{\ButtonDown}
        \opt{IRIVER_H10_PAD,MPIO_HD300_PAD}{\ButtonScrollDown}
        \opt{SANSA_C200_PAD}{\ButtonVolDown}
        \end{sideways}
        &
        % Key "9"
        \begin{sideways}
        \opt{RECORDER_PAD}{\ButtonOn}
        \opt{GIGABEAT_PAD}{\ButtonA}
        \opt{GIGABEAT_S_PAD}{\ButtonPlay}
        \opt{SANSA_FUZEPLUS_PAD}{\ButtonBottomRight}
        \end{sideways}
        &
        % Key "A"
        \begin{sideways}
          \opt{SANSA_FUZEPLUS_PAD}{\ButtonVolUp}
        \end{sideways}
        &
        % Key "B"
        &
        % Key "C"
        &
        % Key "D"
        &
        % Key "E"
        &
        % Key "F"
    \\\bottomrule
    \end{tabularx}
    \end{center}
\end{table}

Some places where can you can find \fname{.ch8} files:
\begin{itemize}
\item The PluginChip8 page on www.rockbox.org has several attached:
\wikilink{PluginChip8}
\item Check out the HP48 chip games section:
\url{http://www.hpcalc.org/hp48/games/chip/}
\item PC emulator by the guy who wrote the HP48 emulator:
\url{http://www.pdc.kth.se/~lfo/chip8/CHIP8.htm}
\item Links to other chip8 emulators: 
\url{http://www.zophar.net/chip8.html}
\end{itemize}


% $Id$ %
\subsection{\label{ref:Frotz}Frotz}
Frotz is a Z-Machine interpreter for playing Infocom's interactive fiction
games, and newer games using the same format. To start a game open a
\fname{.z1 - .z8} file in the \setting{File Browser}. Most modern games are
in the \fname{.z5} or \fname{.z8} format but the older formats used by
Infocom are supported.

Z-Machine games are text based and most depend heavily on typed commands.
The virtual keyboard is used for text entry, both for typing entire lines
and for typing single characters when the game requires single character
input.

Sounds, pictures, colour and Unicode are not currently supported, but
the interpreter informs the game of this and almost all games will
adapt so that they are still playable. This port of Frotz attempts to be
compliant with the Z-Machine Specification version 1.0.

Some places where you can find Z-Machine games, and information about
interactive fiction:
\begin{itemize}
\item The Interactive Fiction Archive, where many free modern works
can be downloaded:
\url{http://www.ifarchive.org/}
\item The specific folder on the if-archive containing Z-Machine games:
\url{http://www.ifarchive.org/indexes/if-archiveXgamesXzcode.html}
\item A copy of the Infocom homepage, with information about the
classic commercial Infocom games:
\url{http://www.vaxdungeon.com/Infocom/}
\item The Frotz homepage (for the original Unix port):
\url{http://frotz.sourceforge.net/}
\item A Beginner's Guide to Playing Interactive Fiction:
\url{http://www.microheaven.com/IFGuide/}
\end{itemize}

\begin{btnmap}
    \PluginUp
       \opt{HAVEREMOTEKEYMAP}{& \PluginRCUp}
        & Display keyboard to enter text\\

    \PluginSelect
       \opt{HAVEREMOTEKEYMAP}{& \PluginRCSelect}
        & Press enter\\

    \PluginCancel
       \opt{HAVEREMOTEKEYMAP}{& \PluginRCCancel}
        & Open Frotz menu (not available at MORE prompts)\\

    \PluginExit
        \opt{HAVEREMOTEKEYMAP}{& }
        & Quit\\
\end{btnmap}


% $Id$ %
\subsection{Image Viewer}
This plugin opens image files from the \setting{File Browser} to display them\nopt{lcd_color}{ using Rockbox's greyscale library}. Supported formats are as follows.

\begin{table}
  \begin{rbtabular}{.60\textwidth}{lX}%
      {\textbf{Format}& \textbf{File-extension(s)}}%
      {}{}
    BMP         & \fname{.bmp}                  \\
    JPEG        & \fname{.jpg, .jpe, .jpeg}     \\
    PNG         & \fname{.png}                  \\
    GIF         & \fname{.gif}                  \\
    \opt{lcd_color}{
      PPM         & \fname{.ppm}                  \\
    }
  \end{rbtabular}
\end{table}

\opt{large_plugin_buffer}{
\par
\note{
When an audio file is playing the size of the image is limited as 
the decoding process needs to share memory with audio tracks. To be able to
view a bigger file you may need to stop playback.}
}
\nopt{large_plugin_buffer}{%
\note{This plugin will cause playback to stop.}%
}%

\begin{btnmap}
    \opt{RECORDER_PAD,ONDIO_PAD,IRIVER_H100_PAD,IRIVER_H300_PAD,IAUDIO_X5_PAD%
      ,SANSA_E200_PAD,SANSA_FUZE_PAD,SANSA_C200_PAD,SANSA_CLIP_PAD,GIGABEAT_PAD,GIGABEAT_S_PAD%
      ,MROBE100_PAD,PBELL_VIBE500_PAD,SANSA_FUZEPLUS_PAD,SAMSUNG_YH92X_PAD%
      ,SAMSUNG_YH820_PAD}{\ButtonUp\ / \ButtonDown}%
    \opt{IPOD_4G_PAD,IPOD_3G_PAD}{\ButtonMenu\ / \ButtonPlay}%
    \opt{IRIVER_H10_PAD}{\ButtonScrollUp\ / \ButtonScrollDown} %
    \opt{RECORDER_PAD,ONDIO_PAD,IRIVER_H100_PAD,IRIVER_H300_PAD,IAUDIO_X5_PAD%
      ,SANSA_E200_PAD,SANSA_FUZE_PAD,SANSA_C200_PAD,SANSA_CLIP_PAD,GIGABEAT_PAD,GIGABEAT_S_PAD%
      ,MROBE100_PAD,IPOD_4G_PAD,IPOD_3G_PAD,IRIVER_H10_PAD,PBELL_VIBE500_PAD%
      ,SANSA_FUZEPLUS_PAD,SAMSUNG_YH92X_PAD,SAMSUNG_YH820_PAD}
      {/ \ButtonLeft\ / \ButtonRight}
    \opt{MPIO_HD200_PAD}{\ButtonVolDown / \ButtonVolUp /%
                         \ButtonRec + \ButtonRew / \ButtonRec + \ButtonFF}
    \opt{MPIO_HD300_PAD}{\ButtonRew / \ButtonFF /%
                         \ButtonPlay + \ButtonScrollUp / \ButtonPlay + \ButtonScrollDown}
    \opt{touchscreen}{\TouchTopMiddle{} / \TouchBottomMiddle{}/ \TouchMidLeft{} / \TouchMidRight}
       \opt{HAVEREMOTEKEYMAP}{& }
    & Move around in zoomed in image\\
    \opt{RECORDER_PAD}{\ButtonPlay}
    \opt{ONDIO_PAD}{\ButtonMenu}
    \opt{IRIVER_H100_PAD,IRIVER_H300_PAD,IAUDIO_X5_PAD,SANSA_E200_PAD%
        ,SANSA_FUZE_PAD,SANSA_C200_PAD,SANSA_CLIP_PAD,MROBE100_PAD}{\ButtonSelect}
    \opt{IPOD_4G_PAD,IPOD_3G_PAD}{\ButtonScrollFwd}
    \opt{IRIVER_H10_PAD}{\ButtonPlay}
    \opt{GIGABEAT_PAD,GIGABEAT_S_PAD,SANSA_FUZEPLUS_PAD}{\ButtonVolUp}
    \opt{PBELL_VIBE500_PAD,SAMSUNG_YH820_PAD}{\ButtonRec+\ButtonUp}
    \opt{MPIO_HD200_PAD,MPIO_HD300_PAD}{\ButtonPlay}
    \opt{SAMSUNG_YH92X_PAD}{\ButtonPlay+\ButtonUp}
    \opt{touchscreen}{\TouchTopRight}
       \opt{HAVEREMOTEKEYMAP}{& }
    & Zoom in\\
    \opt{RECORDER_PAD}{\ButtonOn}
    \opt{ONDIO_PAD}{\ButtonMenu+\ButtonDown}
    \opt{IRIVER_H100_PAD,IRIVER_H300_PAD}{\ButtonMode}
    \opt{IPOD_4G_PAD,IPOD_3G_PAD}{\ButtonScrollBack}
    \opt{IAUDIO_X5_PAD,SANSA_E200_PAD,SANSA_FUZE_PAD,SANSA_C200_PAD,SANSA_CLIP_PAD}{Long \ButtonSelect}
    \opt{IRIVER_H10_PAD}{Long \ButtonPlay}
    \opt{GIGABEAT_PAD,GIGABEAT_S_PAD,SANSA_FUZEPLUS_PAD}{\ButtonVolDown}
    \opt{MROBE100_PAD}{\ButtonPlay}
    \opt{PBELL_VIBE500_PAD,SAMSUNG_YH820_PAD}{\ButtonRec+\ButtonDown}
    \opt{MPIO_HD200_PAD,MPIO_HD300_PAD}{\ButtonRec}
    \opt{SAMSUNG_YH92X_PAD}{\ButtonPlay+\ButtonDown}
    \opt{touchscreen}{\TouchTopLeft}
       \opt{HAVEREMOTEKEYMAP}{& }
    & Zoom out\\
    \opt{RECORDER_PAD}{\ButtonFThree}
    \opt{ONDIO_PAD}{\ButtonMenu+\ButtonRight}
    \opt{IRIVER_H100_PAD}{\ButtonOn}
    \opt{IRIVER_H300_PAD}{\ButtonRec}
    \opt{IPOD_4G_PAD,IPOD_3G_PAD}{\ButtonSelect+\ButtonRight}
    \opt{IAUDIO_X5_PAD}{\ButtonPlay}
    \opt{SANSA_FUZEPLUS_PAD}{\ButtonBottomRight}
    \opt{IRIVER_H10_PAD,SAMSUNG_YH92X_PAD,SAMSUNG_YH820_PAD}{\ButtonFF}
    \opt{SANSA_E200_PAD,SANSA_FUZE_PAD}{\ButtonScrollFwd}
    \opt{SANSA_C200_PAD,SANSA_CLIP_PAD}{\ButtonVolUp}
    \opt{GIGABEAT_PAD}{\ButtonA+\ButtonRight}
    \opt{GIGABEAT_S_PAD}{\ButtonNext}
    \opt{MROBE100_PAD}{\ButtonDisplay+\ButtonRight}
    \opt{PBELL_VIBE500_PAD}{\ButtonRec+\ButtonRight}
    \opt{MPIO_HD200_PAD}{\ButtonFF}
    \opt{MPIO_HD300_PAD}{\ButtonScrollDown}
    \opt{touchscreen}{\TouchBottomRight}
       \opt{HAVEREMOTEKEYMAP}{& }
    & Next image in directory\\
    \opt{RECORDER_PAD}{\ButtonFTwo}
    \opt{ONDIO_PAD}{\ButtonMenu+\ButtonLeft}
    \opt{IRIVER_H100_PAD,IAUDIO_X5_PAD}{\ButtonRec}
    \opt{IRIVER_H300_PAD}{\ButtonOn}
    \opt{IPOD_4G_PAD,IPOD_3G_PAD}{\ButtonSelect+\ButtonLeft}
    \opt{IRIVER_H10_PAD,SAMSUNG_YH92X_PAD,SAMSUNG_YH820_PAD}{\ButtonRew}
    \opt{SANSA_E200_PAD,SANSA_FUZE_PAD}{\ButtonScrollBack}
    \opt{SANSA_C200_PAD,SANSA_CLIP_PAD}{\ButtonVolDown}
    \opt{SANSA_FUZEPLUS_PAD}{\ButtonBottomLeft}
    \opt{GIGABEAT_PAD}{\ButtonA+\ButtonLeft}
    \opt{GIGABEAT_S_PAD}{\ButtonPrev}
    \opt{MROBE100_PAD}{\ButtonDisplay+\ButtonLeft}
    \opt{PBELL_VIBE500_PAD}{\ButtonRec+\ButtonLeft}
    \opt{MPIO_HD200_PAD}{\ButtonRew}
    \opt{MPIO_HD300_PAD}{\ButtonScrollUp}
    \opt{touchscreen}{\TouchBottomLeft}
       \opt{HAVEREMOTEKEYMAP}{& }
    & Previous image in directory\\
    \opt{SANSA_E200_PAD,SANSA_C200_PAD,SANSA_CLIP_PAD,SANSA_FUZEPLUS_PAD%
      ,SAMSUNG_YH92X_PAD,SAMSUNG_YH820_PAD}{%currently only defined for the sansa pads and samsung yh*
       \opt{SANSA_E200_PAD,SANSA_C200_PAD}{\ButtonRec}
       \opt{SANSA_CLIP_PAD}{\ButtonHome}
       \opt{SANSA_FUZEPLUS_PAD}{\ButtonPlay}
       \opt{SAMSUNG_YH92X_PAD}{\ButtonRec\ switch}
       \opt{SAMSUNG_YH820_PAD}{Long \ButtonPlay}
         \opt{HAVEREMOTEKEYMAP}{& }
        & Toggle slide show mode\\
    }
    \opt{RECORDER_PAD,ONDIO_PAD,IRIVER_H100_PAD,IRIVER_H300_PAD}{\ButtonOff}
    \opt{IPOD_4G_PAD,IPOD_3G_PAD}{\ButtonSelect+\ButtonMenu}
    \opt{IAUDIO_X5_PAD,IRIVER_H10_PAD,SANSA_E200_PAD,SANSA_C200_PAD,SANSA_CLIP_PAD}{\ButtonPower}
    \opt{SANSA_FUZE_PAD}{Long \ButtonHome}
    \opt{GIGABEAT_PAD,GIGABEAT_S_PAD,MROBE100_PAD,PBELL_VIBE500_PAD}{\ButtonMenu}
    \opt{MPIO_HD200_PAD}{\ButtonFunc}
    \opt{MPIO_HD300_PAD}{\ButtonEnter}
    \opt{touchscreen}{\TouchCenter}
    \opt{SANSA_FUZEPLUS_PAD}{\ButtonSelect}
    \opt{SAMSUNG_YH92X_PAD,SAMSUNG_YH820_PAD}{\ButtonPlay}
       \opt{HAVEREMOTEKEYMAP}{&
          \opt{IRIVER_RC_H100_PAD}{\ButtonRCStop}
       }
    & Show menu / Abort \\
    \opt{IPOD_4G_PAD,IPOD_3G_PAD,GIGABEAT_PAD,GIGABEAT_S_PAD,MROBE100_PAD,PBELL_VIBE500_PAD%
        ,SAMSUNG_YH820_PAD}{
        \opt{IPOD_4G_PAD,IPOD_3G_PAD}{\ButtonSelect+\ButtonPlay}
        \opt{GIGABEAT_PAD,MROBE100_PAD}{\ButtonPower}
        \opt{GIGABEAT_S_PAD}{\ButtonBack}
        \opt{PBELL_VIBE500_PAD}{\ButtonCancel}
        \opt{MPIO_HD200_PAD}{\ButtonRec + \ButtonPlay}
        \opt{MPIO_HD300_PAD}{Long \ButtonMenu}
        \opt{SANSA_FUZEPLUS_PAD}{\ButtonBack}
        \opt{SAMSUNG_YH820_PAD}{\ButtonRec+\ButtonPlay}
            \opt{HAVEREMOTEKEYMAP}{& }
        & Quit the viewer \\
    }
\end{btnmap}

The menu has the following entries.
\begin{description}
\item[Return.] Returns you to the image
\item[Toggle Slideshow Mode.] Enables or disables the slideshow mode.
\item[Change Slideshow Timeout.] You can set the timeout for the slideshow
  between 1 second and 20 seconds.
\opt{large_plugin_buffer}{
\item[Show Playback Menu.] From the playback menu you can control the
playback of the currently loaded playlist and change the volume of your \dap.
}
\opt{lcd_color}{
\item[Display Options.] From this menu you can force the viewer to render the
image in greyscale using the \setting{Greyscale} option or set the method of
dithering used in the \setting{Dithering} submenu. These settings only take effect
for JPEG images.
}
\item[Quit.] Quits the viewer and returns to the \setting{File Browser}.
\end{description}

\note{
Progressive scan and other unusual JPEG files are not supported, and will
result in various ``unsupported xx'' messages. Processing could also fail if the
image is too big to decode which will be explained by a respective message.

\opt{lcd_color}{
  Supported PPM files are both ASCII PPM (P3) and raw PPM (P6).
  For more information about PPM files, see
  \url{http://netpbm.sourceforge.net/doc/ppm.html}
}
}


\opt{large_plugin_buffer}{\input{plugins/lua.tex}}

\nopt{lowmem}{\subsection{Midiplay}

To get MIDI file playback, a patchset is required. This file contains the
instruments required to synthesize the music. A sample patchset is available
through the wiki at \wikilink{PluginMidiPlay}, and needs to be extracted
to the \fname{.rockbox} directory in the root of your player. There should
now be a \fname{/.rockbox/patchset/} directory, with the patchset directory
containing several \fname{.pat} files and two \fname {.cfg} files. Just
select a MIDI file with either the \fname{.mid} or \fname{.midi} extension
in the file browser to start playback.
% portalplayer targets
\opt{ipod,sansa,iriverh10,iriverh10_5gb,mrobe100}{
\note{Currently playing MIDI files is still in its early stages and you
might experience ``Buffer miss!'' with many files, except simple ones.}
}

\begin{btnmap}
    \opt{IRIVER_H100_PAD,IRIVER_H300_PAD,GIGABEAT_PAD,GIGABEAT_S_PAD%
        ,IAUDIO_X5_PAD,MROBE100_PAD,PBELL_VIBE500_PAD,SANSA_FUZEPLUS_PAD,SAMSUNG_YH92X_PAD%
        ,SAMSUNG_YH820_PAD}{\ButtonUp/ \ButtonDown}
    \opt{IPOD_3G_PAD,IPOD_4G_PAD,SANSA_E200_PAD,SANSA_FUZE_PAD}{\ButtonScrollFwd/ \ButtonScrollBack}
    \opt{IRIVER_H10_PAD,MPIO_HD300_PAD}{\ButtonScrollUp/ \ButtonScrollDown}
    \opt{SANSA_C200_PAD,SANSA_CLIP_PAD,MPIO_HD200_PAD}{\ButtonVolUp/ \ButtonVolDown}
    \opt{COWON_D2_PAD}{\TouchTopMiddle{} / \TouchBottomMiddle}
       \opt{HAVEREMOTEKEYMAP}{& }
    & Volume up/ Volume down\\
    %
    \opt{IRIVER_H100_PAD,IRIVER_H300_PAD,GIGABEAT_PAD,GIGABEAT_S_PAD%
        ,IAUDIO_X5_PAD,MROBE100_PAD,IPOD_3G_PAD,IPOD_4G_PAD,SANSA_E200_PAD%
        ,SANSA_FUZE_PAD,IRIVER_H10_PAD,SANSA_C200_PAD,SANSA_CLIP_PAD,PBELL_VIBE500_PAD%
        ,SANSA_FUZEPLUS_PAD,SAMSUNG_YH92X_PAD,SAMSUNG_YH820_PAD}
        {\ButtonRight/ \ButtonLeft}
    \opt{MPIO_HD200_PAD,MPIO_HD300_PAD}{\ButtonFF / \ButtonRew}
    \opt{COWON_D2_PAD}{\TouchMidRight{} / \TouchMidLeft}
       \opt{HAVEREMOTEKEYMAP}{& }
    & Skip 3 seconds forward/ backward\\
    %
    \opt{IRIVER_H100_PAD,IRIVER_H300_PAD}{\ButtonOn}
    \opt{IPOD_3G_PAD,IPOD_4G_PAD,GIGABEAT_S_PAD,IAUDIO_X5_PAD,IRIVER_H10_PAD,PBELL_VIBE500_PAD%
        ,MPIO_HD200_PAD,MPIO_HD300_PAD,SANSA_FUZEPLUS_PAD,SAMSUNG_YH92X_PAD,SAMSUNG_YH820_PAD}
        {\ButtonPlay}
    \opt{SANSA_E200_PAD,SANSA_FUZE_PAD,SANSA_C200_PAD,SANSA_CLIP_PAD}{\ButtonUp}
    \opt{GIGABEAT_PAD}{\ButtonA}
    \opt{MROBE100_PAD}{\ButtonDisplay}
    \opt{COWON_D2_PAD}{\TouchCenter}
       \opt{HAVEREMOTEKEYMAP}{& }
    & Pause/Resume playback\\
    %
    \opt{IRIVER_H100_PAD,IRIVER_H300_PAD}{\ButtonOff}
    \opt{IPOD_3G_PAD,IPOD_4G_PAD}{\ButtonSelect+\ButtonMenu}
    \opt{GIGABEAT_PAD,GIGABEAT_S_PAD,SANSA_E200_PAD,SANSA_C200_PAD,SANSA_CLIP_PAD,IAUDIO_X5_PAD%
        ,IRIVER_H10_PAD,MROBE100_PAD,COWON_D2_PAD,SANSA_FUZEPLUS_PAD}{\ButtonPower}
    \opt{SANSA_FUZE_PAD}{Long \ButtonHome}
    \opt{PBELL_VIBE500_PAD,SAMSUNG_YH92X_PAD,SAMSUNG_YH820_PAD}{\ButtonRec}
    \opt{MPIO_HD200_PAD}{\ButtonRec + \ButtonPlay}
    \opt{MPIO_HD300_PAD}{Long \ButtonMenu}
       \opt{HAVEREMOTEKEYMAP}{& }
    & Stop playback and quit\\
\end{btnmap}
}
\nopt{lowmem}{\subsection{Mikmod}

Mikmod plays most common tracker music formats.

\begin{btnmap}
    \opt{IRIVER_H100_PAD,IRIVER_H300_PAD,GIGABEAT_PAD,GIGABEAT_S_PAD%
        ,IAUDIO_X5_PAD,MROBE100_PAD,PBELL_VIBE500_PAD,SANSA_FUZEPLUS_PAD,SAMSUNG_YH92X_PAD%
        ,SAMSUNG_YH820_PAD}{\ButtonUp/ \ButtonDown}
    \opt{IPOD_3G_PAD,IPOD_4G_PAD,SANSA_E200_PAD,SANSA_FUZE_PAD}{\ButtonScrollFwd/ \ButtonScrollBack}
    \opt{IRIVER_H10_PAD,MPIO_HD300_PAD}{\ButtonScrollUp/ \ButtonScrollDown}
    \opt{SANSA_C200_PAD,SANSA_CLIP_PAD,MPIO_HD200_PAD}{\ButtonVolUp/ \ButtonVolDown}
    \opt{COWON_D2_PAD}{\TouchTopMiddle{} / \TouchBottomMiddle}
    \opt{XDUOO_X3_PAD}{\ButtonVolUp/ \ButtonVolDown}
       \opt{HAVEREMOTEKEYMAP}{& }
    & Volume up/ Volume down\\
    %
    \opt{IRIVER_H100_PAD,IRIVER_H300_PAD,GIGABEAT_PAD,GIGABEAT_S_PAD%
        ,IAUDIO_X5_PAD,MROBE100_PAD,IPOD_3G_PAD,IPOD_4G_PAD,SANSA_E200_PAD%
        ,SANSA_FUZE_PAD,IRIVER_H10_PAD,SANSA_C200_PAD,SANSA_CLIP_PAD,PBELL_VIBE500_PAD%
        ,SANSA_FUZEPLUS_PAD,SAMSUNG_YH92X_PAD,SAMSUNG_YH820_PAD}
        {\ButtonRight/ \ButtonLeft}
    \opt{MPIO_HD200_PAD,MPIO_HD300_PAD}{\ButtonFF / \ButtonRew}
    \opt{COWON_D2_PAD}{\TouchMidRight{} / \TouchMidLeft}
    \opt{XDUOO_X3_PAD}{\ButtonNext/ \ButtonPrev}
       \opt{HAVEREMOTEKEYMAP}{& }
    & Skip to next/prev file\\
    %
    \opt{IRIVER_H100_PAD,IRIVER_H300_PAD,GIGABEAT_PAD,GIGABEAT_S_PAD%
        ,IAUDIO_X5_PAD,MROBE100_PAD,IPOD_3G_PAD,IPOD_4G_PAD,SANSA_E200_PAD%
        ,SANSA_FUZE_PAD,IRIVER_H10_PAD,SANSA_C200_PAD,SANSA_CLIP_PAD,PBELL_VIBE500_PAD%
        ,SANSA_FUZEPLUS_PAD,SAMSUNG_YH92X_PAD,SAMSUNG_YH820_PAD}
        {Long \ButtonRight/ Long \ButtonLeft}
    \opt{MPIO_HD200_PAD,MPIO_HD300_PAD}{Long \ButtonFF / Long \ButtonRew}
    \opt{COWON_D2_PAD}{Long \TouchMidRight{} / Long \TouchMidLeft}
    \opt{XDUOO_X3_PAD}{Long \ButtonNext/ Long \ButtonPrev}
       \opt{HAVEREMOTEKEYMAP}{& }
    & Skip to next/prev sequence\\
    %
    \opt{IRIVER_H100_PAD,IRIVER_H300_PAD}{\ButtonOn}
    \opt{IPOD_3G_PAD,IPOD_4G_PAD,GIGABEAT_S_PAD,IAUDIO_X5_PAD,IRIVER_H10_PAD,PBELL_VIBE500_PAD%
        ,MPIO_HD200_PAD,MPIO_HD300_PAD,SANSA_FUZEPLUS_PAD,SAMSUNG_YH92X_PAD,SAMSUNG_YH820_PAD}
        {\ButtonPlay}
    \opt{SANSA_E200_PAD,SANSA_FUZE_PAD,SANSA_C200_PAD,SANSA_CLIP_PAD}{\ButtonUp}
    \opt{GIGABEAT_PAD}{\ButtonA}
    \opt{MROBE100_PAD}{\ButtonDisplay}
    \opt{COWON_D2_PAD}{\TouchCenter}
    \opt{XDUOO_X3_PAD}{\ButtonPlay}
       \opt{HAVEREMOTEKEYMAP}{& }
    & Pause/Resume playback\\
    %
    \opt{XDUOO_X3_PAD}{\ButtonOption}
       \opt{HAVEREMOTEKEYMAP}{& }
    & Enter configuration menu\\
    %
    \opt{IRIVER_H100_PAD,IRIVER_H300_PAD}{\ButtonOff}
    \opt{IPOD_3G_PAD,IPOD_4G_PAD}{\ButtonSelect+\ButtonMenu}
    \opt{GIGABEAT_PAD,GIGABEAT_S_PAD,SANSA_E200_PAD,SANSA_C200_PAD,SANSA_CLIP_PAD,IAUDIO_X5_PAD%
        ,IRIVER_H10_PAD,MROBE100_PAD,COWON_D2_PAD,SANSA_FUZEPLUS_PAD}{\ButtonPower}
    \opt{SANSA_FUZE_PAD}{Long \ButtonHome}
    \opt{PBELL_VIBE500_PAD}{\ButtonRec}
    \opt{SAMSUNG_YH92X_PAD,SAMSUNG_YH820_PAD}{Long \ButtonPlay}
    \opt{MPIO_HD200_PAD}{\ButtonRec + \ButtonPlay}
    \opt{MPIO_HD300_PAD}{Long \ButtonMenu}
    \opt{XDUOO_X3_PAD}{\ButtonPower}
       \opt{HAVEREMOTEKEYMAP}{& }
    & Stop playback and quit\\
\end{btnmap}

Mikmod is highly configurable, and has many setting that affect output quality.  However, 
less powerful \dap{}s may lack the cpu resources that higher quality settings require, expecially
on more complex modules.  Settings are saved and re-loaded automatically.

\subsubsection{Configuration}
\begin {description}
\item [Panning Separation]
    Specifies how far to separate Left and Right channels.  128 is full separation, 0 is none, effectively resulting in mono sound.
\item [Reverberation]
    Specifies the amount of reberb (echo) to add to the mix.  0 is disabled.
\item [Interpolation]
    Enabling this improves the sound of lower-quality instruments.  Recommended.
\item [Swap Channels]
    Swaps the left and right channels.
\item [Surround]
    Enables a faux-surround effect.
\item [HQ Mixer]
    Enables a high quality audio mixer that is very CPU intensive.  Must restart the plugin to take effect.
\item [Sample Rate]
    Select the sample/mixing rate for playback.  44.1KHz or higher is recommended, but you can select from any rate the \dap{} natively supports.
\end {description}
}

\input{plugins/mp3_encoder.tex}

\opt{iriverh300,iriverh100,SANSA_FUZE_PAD,SANSA_E200_PAD,IPOD_4G_PAD,IPOD_3G_PAD%
    ,IPOD_1G2G_PAD,SAMSUNG_YPR0_PAD}{
  \subsection{PDBox}

PDBox is a Pure Data audio environment with small-size GUI. Those who do not know
what Pure Data is are advised to look at the book \href{http://aspress.co.uk/ds/}
{``Designing Sound''}, both at the abridged text and at the sample chapters.
Another good Pure Data tutorial can be found at \url{http://www.pd-tutorial.com/}.

\subsubsection{Prerequisites for using the plugin}
To test the abilities of PDBox get the file
\href{https://www.rockbox.org/wiki/pub/Main/PureDataOnRockbox/PureData.zip}
{\fname{PureData.zip}}. See \wikilink{PluginPdbox} for more information.
}

\nopt{%
  ipod1g2g,ipod3g,ipod4g,ipodmini% horizontal pixelformat not implemented
  ,iaudiom3,mpiohd200, % vertical interleaved pixelformat not implemented
  ,mrobe100% lcd size/depth not implemented
  }{
  \subsection{\label{ref:Rockboy}Rockboy}
\screenshot{plugins/images/ss-rockboy}{Rockboy}{img:rockboy}
Rockboy is a Nintendo Game Boy and Game Boy Color emulator for Rockbox based on
the gnuboy emulator. To start a game, open a ROM file saved as \fname{.gb} or
\fname{.gbc} in the file browser.\\

\opt{ipod}{
    Within Rockboy the wheel is used as a touchpad. It is split into 8 sections
     that when tapped correspond to 8 buttons as detailed in the table below.
}

\subsubsection{Default keys}
\begin{btnmap}
    \opt{RECORDER_PAD,IRIVER_H100_PAD,IRIVER_H300_PAD,IAUDIO_X5_PAD%
        ,SANSA_E200_PAD,SANSA_FUZE_PAD,SANSA_C200_PAD,GIGABEAT_PAD%
        ,GIGABEAT_S_PAD,SANSA_CLIP_PAD,SANSA_FUZEPLUS_PAD}{\ButtonUp{} / \ButtonDown}
    \opt{IPOD_4G_PAD}{Tap \ButtonPlay{} / \ButtonMenu}
    \opt{IRIVER_H10_PAD}{\ButtonScrollUp{} / \ButtonScrollDown}
    \opt{PBELL_VIBE500_PAD}{\ButtonOK{} / \ButtonCancel}
    \opt{RECORDER_PAD,IRIVER_H100_PAD,IRIVER_H300_PAD,IAUDIO_X5_PAD%
        ,SANSA_E200_PAD,SANSA_FUZE_PAD,SANSA_C200_PAD,GIGABEAT_PAD%
        ,GIGABEAT_S_PAD,IPOD_4G_PAD,IRIVER_H10_PAD,PBELL_VIBE500_PAD%
        ,SANSA_CLIP_PAD,SANSA_FUZEPLUS_PAD}
        {\ButtonLeft{} / \ButtonRight}
    \opt{MPIO_HD300_PAD}{\ButtonRew / \ButtonFF / \ButtonScrollUp / \ButtonScrollDown}
    \opt{COWON_D2_PAD}{\TouchTopMiddle{} / \TouchBottomMiddle{} / \TouchMidLeft{} / \TouchMidRight}
       \opt{HAVEREMOTEKEYMAP}{& }
    & Direction keys\\
    %
    \opt{RECORDER_PAD}{\ButtonFOne}
    \opt{IRIVER_H100_PAD}{\ButtonOn}
    \opt{IRIVER_H300_PAD}{\ButtonRec}
    \opt{IPOD_4G_PAD}{Tap \btnfnt{Top-Right}}
    \opt{IAUDIO_X5_PAD,IRIVER_H10_PAD,MPIO_HD300_PAD}{\ButtonPlay}
    \opt{SANSA_E200_PAD,SANSA_FUZE_PAD,SANSA_C200_PAD,SANSA_CLIP_PAD}{\ButtonSelect}
    \opt{GIGABEAT_PAD,GIGABEAT_S_PAD}{\ButtonVolUp}
    \opt{COWON_D2_PAD}{\ButtonPlus}
    \opt{PBELL_VIBE500_PAD}{\ButtonPower}
    \opt{SANSA_FUZEPLUS_PAD}{\ButtonVolUp}
       \opt{HAVEREMOTEKEYMAP}{& }
    & A button\\
    %
    \opt{RECORDER_PAD}{\ButtonFTwo}
    \opt{IRIVER_H100_PAD}{\ButtonOff}
    \opt{IRIVER_H300_PAD}{\ButtonMode}
    \opt{IPOD_4G_PAD}{Tap \btnfnt{Top-Left}}
    \opt{IAUDIO_X5_PAD,SANSA_E200_PAD,SANSA_C200_PAD,PBELL_VIBE500_PAD}{\ButtonRec}
    \opt{SANSA_FUZE_PAD,SANSA_CLIP_PAD}{\ButtonHome}
    \opt{IRIVER_H10_PAD}{\ButtonFF}
    \opt{GIGABEAT_PAD,GIGABEAT_S_PAD}{\ButtonVolDown}
    \opt{COWON_D2_PAD}{\ButtonMinus}
    \opt{MPIO_HD300_PAD}{\ButtonRec}
    \opt{SANSA_FUZEPLUS_PAD}{\ButtonVolDown}
       \opt{HAVEREMOTEKEYMAP}{& }
    & B button\\
    %
    \opt{RECORDER_PAD}{\ButtonFThree}
    \opt{IRIVER_H100_PAD}{\ButtonRec}
    \opt{IRIVER_H300_PAD}{\ButtonOn}
    \opt{IPOD_4G_PAD}{Tap \btnfnt{Bottom-Right} /~Press \ButtonSelect}
    \opt{IAUDIO_X5_PAD,SANSA_FUZEPLUS_PAD}{\ButtonSelect}
    \opt{SANSA_E200_PAD,SANSA_FUZE_PAD}{\ButtonScrollBack}
    \opt{SANSA_C200_PAD,SANSA_CLIP_PAD}{\ButtonVolDown}
    \opt{IRIVER_H10_PAD}{\ButtonRew}
    \opt{GIGABEAT_PAD}{\ButtonA}
    \opt{GIGABEAT_S_PAD,PBELL_VIBE500_PAD}{\ButtonPlay}
    \opt{COWON_D2_PAD}{\TouchTopRight}
    \opt{MPIO_HD300_PAD}{Long \ButtonPlay}
       \opt{HAVEREMOTEKEYMAP}{& }
    & Start\\
    %
    \nopt{IAUDIO_X5_PAD,IRIVER_H10_PAD}{
        \opt{RECORDER_PAD,SANSA_FUZEPLUS_PAD}{\ButtonPlay}
        \opt{IRIVER_H100_PAD,IRIVER_H300_PAD}{\ButtonSelect}
        \opt{IPOD_4G_PAD}{Tap \btnfnt{Bottom-Left}}
        \opt{SANSA_E200_PAD,SANSA_FUZE_PAD}{\ButtonScrollFwd}
        \opt{SANSA_C200_PAD,SANSA_CLIP_PAD}{\ButtonVolUp}
        \opt{GIGABEAT_PAD,GIGABEAT_S_PAD}{\ButtonSelect}
        \opt{COWON_D2_PAD}{\TouchCenter}
        \opt{PBELL_VIBE500_PAD}{\ButtonUp}
        \opt{MPIO_HD300_PAD}{\ButtonEnter}
               \opt{HAVEREMOTEKEYMAP}{& }
        & Select\\
        %
    }
    \opt{RECORDER_PAD,IRIVER_H100_PAD,iaudiom5,MPIO_HD300_PAD}{
        \opt{RECORDER_PAD}{\ButtonOn}
        \opt{IRIVER_H100_PAD,IAUDIO_X5_PAD,MPIO_HD300_PAD}{\ButtonHold{} switch}
              \opt{HAVEREMOTEKEYMAP}{& }
        & Cycle display scaling modes\\
    %  defined for the Recorders and targets with 160x128x2 displays (H100, M5)
    }
    \opt{RECORDER_PAD,IRIVER_H300_PAD}{\ButtonOff}
    \opt{IRIVER_H100_PAD}{\ButtonMode}
    \opt{IPOD_4G_PAD}{\ButtonHold{} switch}
    \opt{IAUDIO_X5_PAD,SANSA_E200_PAD,SANSA_FUZE_PAD,SANSA_C200_PAD%
    ,IRIVER_H10_PAD,SANSA_CLIP_PAD}{\ButtonPower}
    \opt{GIGABEAT_PAD,GIGABEAT_S_PAD,COWON_D2_PAD,PBELL_VIBE500_PAD%
        ,MPIO_HD300_PAD}{\ButtonMenu}
    \opt{SANSA_FUZEPLUS_PAD}{\ButtonBack}
       \opt{HAVEREMOTEKEYMAP}{& }
    & Open Rockboy menu\\
\end{btnmap}

\subsubsection{Rockboy menu}
\begin{description}
\item[Load Game\ldots] Loads a previously saved game.
\item[Save Game\ldots] Saves your current state.
\item[Options\ldots]
    \begin{description}
    \item[Max Frameskip.]
    Change frameskip setting to improve speed.
    \item[Sound.]
    Toggle sound on or off.
    \item[Stats.]
    Toggle showing fps and current frameskip.
    \item[Set Keys (BUGGY)]
    Select this option to set a new keymapping.
    \opt{lcd_color}{
        \note{The direction keys are set for the normal screen orientation,
        not the rotated orientation.}
        \item[Screen Size.]
        Choose whether the original aspect ratio should be kept when scaling
        the picture to the screen%
        % targets with bigger displays than the original gameboy
        \opt{gigabeat,iriverh300,ipodcolor,ipodvideo,e200,e200v2}{
            or whether it should be displayed unscaled%
        }.
        \item[Screen Rotate.]
        Rotate the displayed picture and direction keys by 90 degrees.
        \item[Set Palette.]
        Pick one of a few predefined colour palettes.
    }
    \end{description}
\item[Reset.] Resets the Emulator.
\item[Quit RockBoy.] Quits the Rockboy plugin.
\end{description}
}

\input{plugins/search.tex}

% $Id$
\subsection{Shopper}
\label{ref:Shopperplugin}

Shopper is a plugin which allows you to maintain reusable shopping lists.
To create a list, use a text editor to write down a list of items (one per
line; note that the line length should not exceed 40 characters) and save the
file as \fname{<name>.shopper}. If you want to separate the items you can do
so by creating categories, which are prepended with `\#'. To open a
\fname{.shopper} file just ``play'' it from the file browser.

\begin{example}
    #groceries
    bananas
    cucumber
    4 apples
    6 apples
    #dairy
    milk
    cheese
\end{example}
Note that it isn't possible to choose exact quantities, but you can create a
number of entries with different quantities in the name of the item, such as
for the apples in the above example.

There are two modes, \emph{edit mode} and \emph{view mode}. The edit mode
shows all the items, and it allows you to select which of the items you want
to buy. When you have finished selecting the items, use the menu to go to the
view mode, and you will see only the items you wish to buy. If you `select'
an item in view mode then that item will be removed from the list.

When you exit Shopper the last view is saved, including which items you have
selected, so if you re-open the shopping list it will be as you left it. There
are additional menu options for clearing the list, selecting all items, showing
and hiding the categories, toggling the categories, and displaying the playback
menu.

\subsubsection{Shopper Keys}
\begin{btnmap}
    \ActionStdOk{}
        \opt{HAVEREMOTEKEYMAP}{& \ActionRCStdOk}
     & Select or clear an item\\

    \ActionStdMenu{} or \ActionStdContext{}
      \opt{HAVEREMOTEKEYMAP}{& \ActionRCStdMenu{} or \ActionRCStdContext}
     & Show menu\\

    \ActionStdCancel{}
        \opt{HAVEREMOTEKEYMAP}{& \ActionRCStdCancel}
    & Exit\\
\end{btnmap}


\input{plugins/sort.tex}

\opt{lcd_non-mono}{\subsection{Speedread}
\screenshot{plugins/images/ss-speedread}{speedread}{fig:Speedread in action}

This plugin is designed for reading plain-text files such as
ebooks. It works by using a form of Rapid Serial Visual Presentation
(RSVP) that has optimized word placement to reduce or eliminate eye
movement (saccades) when reading.
}

\subsection{Text Viewer}
\screenshot{plugins/images/ss-text_viewer-main}{Text Viewer}{img:text_viewer-main}
This is a Viewer for text files with word wrap. Just open a \fname{.txt} or
\fname{.nfo} file to display it. The text viewer features controls to handle
various styles of text formatting and has top{}-of{}-file and bottom{}-of{}-file
buttons. You can view files without a \fname{.txt} or \fname{.nfo} extension
by using \emph{Open with} from the \emph{Context Menu}
(see \reference{ref:Contextmenu}). You can also bookmark pages.

\subsubsection{Default keys}
\begin{btnmap}
    \opt{IRIVER_H100_PAD,IRIVER_H300_PAD,IAUDIO_X5_PAD%
        ,SANSA_E200_PAD,SANSA_FUZE_PAD,GIGABEAT_PAD,MROBE100_PAD,SANSA_FUZEPLUS_PAD%
        ,SAMSUNG_YH92X_PAD,SAMSUNG_YH820_PAD}{\ButtonUp}
    \opt{IPOD_4G_PAD,IPOD_3G_PAD}{\ButtonScrollBack}
    \opt{IRIVER_H10_PAD,MPIO_HD300_PAD}{\ButtonScrollUp}
    \opt{SANSA_C200_PAD,SANSA_CLIP_PAD}{\ButtonVolUp}
    \opt{GIGABEAT_S_PAD}{\ButtonPrev}
    \opt{COWON_D2_PAD}{\ButtonMinus{} / }%
    \opt{touchscreen}{\TouchTopMiddle}
    \opt{PBELL_VIBE500_PAD}{\ButtonOK}
    \opt{MPIO_HD200_PAD}{\ButtonRew}
    \opt{HAVEREMOTEKEYMAP}{&
        \opt{IAUDIO_M3_PAD}{\ButtonRCUp}
    }
        & Scroll{}-up\\

    \opt{IRIVER_H100_PAD,IRIVER_H300_PAD,IAUDIO_X5_PAD%
        ,SANSA_E200_PAD,SANSA_FUZE_PAD,GIGABEAT_PAD,MROBE100_PAD,SANSA_FUZEPLUS_PAD%
        ,SAMSUNG_YH92X_PAD,SAMSUNG_YH820_PAD}{\ButtonDown}
    \opt{IPOD_4G_PAD,IPOD_3G_PAD}{\ButtonScrollFwd}
    \opt{IRIVER_H10_PAD,MPIO_HD300_PAD}{\ButtonScrollDown}
    \opt{SANSA_C200_PAD,SANSA_CLIP_PAD}{\ButtonVolDown}
    \opt{GIGABEAT_S_PAD}{\ButtonNext}
    \opt{COWON_D2_PAD}{\ButtonPlus{} / }%
    \opt{touchscreen}{\TouchBottomMiddle}
    \opt{PBELL_VIBE500_PAD}{\ButtonCancel}
    \opt{MPIO_HD200_PAD}{\ButtonFF}
    \opt{HAVEREMOTEKEYMAP}{&
        \opt{IAUDIO_M3_PAD}{\ButtonRCDown}
    }
        & Scroll{}-down\\

    \opt{GIGABEAT_S_PAD}{\ButtonPlay+\ButtonLeft}
    \opt{MPIO_HD200_PAD}{\ButtonVolDown}
    \opt{MPIO_HD300_PAD}{\ButtonRew}
    \opt{touchscreen}{\TouchMidLeft}
    \nopt{GIGABEAT_S_PAD,touchscreen,IAUDIO_M3_PAD,MPIO_HD200_PAD%
         ,MPIO_HD300_PAD,SAMSUNG_YH92X_PAD,SAMSUNG_YH820_PAD}{\ButtonLeft}
    \opt{HAVEREMOTEKEYMAP}{&
        \opt{IAUDIO_M3_PAD}{\ButtonRCLeft}
    }
        & Top of file (Narrow mode) /
        One screen left (Wide mode)\\

    \opt{GIGABEAT_S_PAD}{\ButtonPlay+\ButtonRight}
    \opt{MPIO_HD200_PAD}{\ButtonVolUp}
    \opt{MPIO_HD300_PAD}{\ButtonFF}
    \opt{touchscreen}{\TouchMidRight}
    \nopt{GIGABEAT_S_PAD,touchscreen,IAUDIO_M3_PAD,MPIO_HD200_PAD%
         ,MPIO_HD300_PAD,SAMSUNG_YH92X_PAD,SAMSUNG_YH820_PAD}{\ButtonRight}
    \opt{HAVEREMOTEKEYMAP}{&
        \opt{IAUDIO_M3_PAD}{\ButtonRCRight}
    }
        & Bottom of file (Narrow mode) /
        One screen right (Wide mode)\\

    \opt{IRIVER_H100_PAD,IRIVER_H300_PAD,SANSA_E200_PAD%
        ,SANSA_FUZE_PAD,SANSA_C200_PAD,SANSA_CLIP_PAD,GIGABEAT_S_PAD%
        ,GIGABEAT_PAD,PBELL_VIBE500_PAD,SANSA_FUZEPLUS_PAD,SAMSUNG_YH92X_PAD%
        ,SAMSUNG_YH820_PAD}{%
        \opt{IRIVER_H100_PAD,IRIVER_H300_PAD}{\ButtonOn+\ButtonUp}
        \opt{SANSA_E200_PAD,SANSA_FUZE_PAD}{\ButtonScrollBack}
        \opt{SANSA_C200_PAD,SANSA_CLIP_PAD,GIGABEAT_S_PAD,PBELL_VIBE500_PAD}{\ButtonUp}
        \opt{GIGABEAT_PAD}{\ButtonA+\ButtonUp}
        \opt{SANSA_FUZEPLUS_PAD}{\ButtonVolUp}
        \opt{SAMSUNG_YH92X_PAD}{\ButtonFF+\ButtonUp}
        \opt{SAMSUNG_YH820_PAD}{\ButtonRec+\ButtonUp}
    \opt{HAVEREMOTEKEYMAP}{& }
        & One line up\\
    }

    \opt{IRIVER_H100_PAD,IRIVER_H300_PAD,SANSA_E200_PAD%
        ,SANSA_FUZE_PAD,SANSA_C200_PAD,SANSA_CLIP_PAD,GIGABEAT_S_PAD%
        ,GIGABEAT_PAD,PBELL_VIBE500_PAD,SANSA_FUZEPLUS_PAD,SAMSUNG_YH92X_PAD%
        ,SAMSUNG_YH820_PAD}{%
        \opt{IRIVER_H100_PAD,IRIVER_H300_PAD}{\ButtonOn+\ButtonDown}
        \opt{SANSA_E200_PAD,SANSA_FUZE_PAD}{\ButtonScrollFwd}
        \opt{SANSA_C200_PAD,SANSA_CLIP_PAD,GIGABEAT_S_PAD,PBELL_VIBE500_PAD}{\ButtonDown}
        \opt{GIGABEAT_PAD}{\ButtonA+\ButtonDown}
        \opt{SANSA_FUZEPLUS_PAD}{\ButtonVolDown}
        \opt{SAMSUNG_YH92X_PAD}{\ButtonFF+\ButtonDown}
        \opt{SAMSUNG_YH820_PAD}{\ButtonRec+\ButtonDown}
    \opt{HAVEREMOTEKEYMAP}{& }
        & One line down\\
    }

    \opt{IRIVER_H100_PAD,IRIVER_H300_PAD,GIGABEAT_PAD,%
        GIGABEAT_S_PAD,SANSA_FUZEPLUS_PAD,SAMSUNG_YH92X_PAD,SAMSUNG_YH820_PAD}{
        \opt{IRIVER_H100_PAD,IRIVER_H300_PAD}{\ButtonOn+\ButtonLeft}
        \opt{GIGABEAT_S_PAD}{\ButtonLeft}
        \opt{GIGABEAT_PAD}{\ButtonA+\ButtonLeft}
        \opt{SANSA_FUZEPLUS_PAD}{\ButtonBottomLeft}
        \opt{SAMSUNG_YH92X_PAD}{\ButtonFF+\ButtonLeft}
        \opt{SAMSUNG_YH820_PAD}{\ButtonRec+\ButtonLeft}
    \opt{HAVEREMOTEKEYMAP}{& }
        & One column left\\
    }

    \opt{IRIVER_H100_PAD,IRIVER_H300_PAD,GIGABEAT_PAD,%
        GIGABEAT_S_PAD,SANSA_FUZEPLUS_PAD,SAMSUNG_YH92X_PAD,SAMSUNG_YH820_PAD}{
        \opt{IRIVER_H100_PAD,IRIVER_H300_PAD}{\ButtonOn+\ButtonRight}
        \opt{GIGABEAT_S_PAD}{\ButtonRight}
        \opt{GIGABEAT_PAD}{\ButtonA+\ButtonRight}
        \opt{SANSA_FUZEPLUS_PAD}{\ButtonBottomRight}
        \opt{SAMSUNG_YH92X_PAD}{\ButtonFF+\ButtonRight}
        \opt{SAMSUNG_YH820_PAD}{\ButtonRec+\ButtonRight}
    \opt{HAVEREMOTEKEYMAP}{& }
        & One column right\\
    }

    \opt{IPOD_4G_PAD,IPOD_3G_PAD,IAUDIO_X5_PAD%
        ,IRIVER_H10_PAD,GIGABEAT_S_PAD,PBELL_VIBE500_PAD%
        ,MPIO_HD200_PAD,MPIO_HD300_PAD,SANSA_FUZEPLUS_PAD}{\ButtonPlay}
    \opt{IRIVER_H100_PAD,IRIVER_H300_PAD}{\ButtonSelect}
    \opt{GIGABEAT_PAD}{\ButtonA}
    \opt{SANSA_C200_PAD,SANSA_E200_PAD}{\ButtonRec}
    \opt{SANSA_CLIP_PAD}{\ButtonHome}
    \opt{SANSA_FUZE_PAD}{\ButtonDown+\ButtonSelect}
    \opt{MROBE100_PAD}{\ButtonDisplay}
    \opt{SAMSUNG_YH820_PAD}{\ButtonFF}
    \opt{SAMSUNG_YH92X_PAD}{\ButtonRecOn{} or \ButtonRecOff}
    \opt{MPIO_HD200_PAD}{FIXME}
    \opt{touchscreen}{\TouchCenter}
    \opt{HAVEREMOTEKEYMAP}{&
        \opt{IAUDIO_M3_PAD}{\ButtonRCMode}
    }
        & Toggle autoscroll\\

    \opt{IRIVER_H100_PAD,IRIVER_H300_PAD}{\ButtonOn+\ButtonSelect}
    \opt{IPOD_4G_PAD,IPOD_3G_PAD,GIGABEAT_PAD,GIGABEAT_S_PAD,MROBE100_PAD}{\ButtonSelect}
    \opt{IRIVER_H10_PAD}{\ButtonFF}
    \opt{IAUDIO_X5_PAD}{\ButtonRec}
    \opt{SANSA_FUZEPLUS_PAD}{Long \ButtonSelect}
    \opt{SANSA_E200_PAD,SANSA_C200_PAD,SANSA_CLIP_PAD}{\ButtonDown+\ButtonSelect}
    \opt{COWON_D2_PAD}{\ButtonMenu+\ButtonPlus}
    \opt{SANSA_FUZE_PAD}{\ButtonUp+\ButtonSelect}
    \opt{SAMSUNG_YH92X_PAD,SAMSUNG_YH820_PAD}{\ButtonPlay}
    \opt{PBELL_VIBE500_PAD}{\ButtonPower}
    \opt{MPIO_HD200_PAD}{\ButtonRec}
    \opt{MPIO_HD300_PAD}{\ButtonEnter}
    \opt{HAVEREMOTEKEYMAP}{&
        \opt{IAUDIO_M3_PAD}{\ButtonRCPlay+\ButtonRCMode}
    }
        & Set/Reset bookmarks\\

    \opt{IPOD_4G_PAD,IPOD_3G_PAD,GIGABEAT_PAD,GIGABEAT_S_PAD%
        ,MROBE100_PAD,PBELL_VIBE500_PAD,MPIO_HD300_PAD}{\ButtonMenu}
    \opt{IRIVER_H100_PAD,IRIVER_H300_PAD}{\ButtonMode}
    \opt{IAUDIO_X5_PAD,SANSA_C200_PAD,SANSA_CLIP_PAD,SANSA_E200_PAD}{\ButtonSelect}
    \opt{SANSA_FUZE_PAD}{Long \ButtonSelect}
    \opt{SANSA_FUZEPLUS_PAD}{\ButtonBack}
    \opt{IRIVER_H10_PAD,SAMSUNG_YH92X_PAD,SAMSUNG_YH820_PAD}{\ButtonRew}
    \opt{COWON_D2_PAD}{\ButtonMenu{} / }%
    \opt{MPIO_HD200_PAD}{\ButtonFunc}
    \opt{touchscreen}{\TouchTopRight}
    \opt{HAVEREMOTEKEYMAP}{&
        \opt{IAUDIO_M3_PAD}{\ButtonRCPlay}
    }
        & Enter menu\\

    \opt{IRIVER_H100_PAD,IRIVER_H300_PAD}{\ButtonOff}
    \opt{IPOD_4G_PAD,IPOD_3G_PAD}{\ButtonSelect+\ButtonMenu}
    \opt{IAUDIO_X5_PAD,IRIVER_H10_PAD,SANSA_E200_PAD,SANSA_C200_PAD,SANSA_CLIP_PAD%
        ,GIGABEAT_PAD,MROBE100_PAD,SANSA_FUZEPLUS_PAD}{\ButtonPower}
    \opt{SANSA_FUZE_PAD}{Long \ButtonHome}
    \opt{GIGABEAT_S_PAD}{\ButtonBack}
    \opt{COWON_D2_PAD}{\ButtonPower{} / }%
    \opt{touchscreen}{\TouchTopLeft}
    \opt{IAUDIO_M3_PAD,PBELL_VIBE500_PAD}{\ButtonRec}
    \opt{SAMSUNG_YH92X_PAD,SAMSUNG_YH820_PAD}{Long \ButtonRew}
    \opt{MPIO_HD200_PAD}{\ButtonRec + \ButtonPlay}
    \opt{MPIO_HD300_PAD}{Long \ButtonMenu}
    \opt{HAVEREMOTEKEYMAP}{&
        \opt{IAUDIO_M3_PAD}{\ButtonRCRec}
        \opt{IRIVER_RC_H100_PAD}{\ButtonRCStop}
    }
        & Exit text viewer\\

\end{btnmap}

\subsubsection{Menu}

\begin{description}
\item[Return] Return to the file being viewed.
\item[Viewer Options] Change settings for the current file.
    \begin{description}
    \item[Encoding] Set the codepage in the text viewer.
    Available settings:
    \setting{ISO-8859-1} (Latin 1).
    \setting{ISO-8859-7} (Greek),
    \opt{lcd_bitmap}{
        \setting{ISO-8859-8} (Hebrew),
    }
    \setting{CP1251} (Cyrillic),
    \opt{lcd_bitmap}{
        \setting{ISO-8859-11} (Thai),
        \setting{CP1256} (Arabic),
    }
    \setting{ISO-8859-9} (Turkish),
    \setting{ISO-8859-2} (Latin Extended),
    \setting{CP1250} (Central European),
    \opt{lcd_bitmap}{
        \setting{SJIS} (Japanese),
        \setting{GB-2312} (Simple Chinese),
        \setting{KSX-1001} (Korean),
        \setting{BIG5} (Traditional Chinese),
    }
    \setting{UTF-8} (Unicode),
    This setting only applies to the plugin and is independent from the
    \setting{Default Codepage} setting (see \reference{ref:Defaultcodepage}).
    \item[Word Wrap] Toggle word wrap mode.
        \begin{description}
            \item[On] Break lines at the maximum column limit.
            \item[Off (Chop Words)] Break lines at white space or hyphen.
        \end{description}
    \item[Line Mode] Change how lines are displayed.
        \begin{description}
            \item[Normal] Break lines at newline characters.
            \item[Join] Join lines together.
            \item[Expand] Add a blank line at newlines. Useful for making paragraphs
            clearer in some book style text files.
            \opt{lcd_bitmap}{
            \item[Reflow Lines] Justify the text.
            }
        \end{description}
    \item[Screens Per Page] Set the number of screens per page. Available
      options are \setting{1} to \setting{5} screens per page.
    \item[Alignment] Set the text alignment.
        \begin{description}
            \item[Right] Set the text alignment to the right.
            (Useful for displaying right-to-left languages, such as Arabic or Hebrew)
            \item[Left] Set the text alignment to the left.
        \end{description}
    \opt{lcd_bitmap}{
    \item[Show Header] Select whether to show the header. The header displays the file path. 
        \begin{description}
            \item[No] Do not display the header.
            \item[Yes] Display the header.
        \end{description}
    \item[Show Footer] Select whether to show the footer. The footer dispays the page number.
        \begin{description}
            \item[No] Do not display the footer.
            \item[Yes] Display the footer.
        \end{description}

    \item[Font] Select the font to be used by the Text Viewer.
    \item[Show Statusbar] Select whether to show the status bar. If you select
         a theme settings that the status bar does not display (see
         \reference{ref:configure_rockbox_themes}), the status bar is not
         displayed even if you select \setting{Yes}.
        \begin{description}
            \item[No] Do not display the status bar.
            \item[Yes] Display the status bar.
        \end{description}
    }
    \item[Scroll Settings] The scrolling settings submenu.
        \begin{description}
            \item[Horizontal] Submenu for horizontal scrolling settings.
                \begin{description}
                    \item[Scrollbar] Toggle the horizontal scrollbar for the
                    current mode. If the file fits on one screen, there is no
                    scrollbar and this setting has no effect.
                        \begin{description}
                            \item[No] Do not display the horizontal scroll bar.
                            \item[Yes] Display the horizontal scroll bar.
                        \end{description}
                    \item[Scroll Mode] Change the function of the ``Left'' and
    ``Right'' buttons.
                        \begin{description}
                            \item[Scroll by Screen] Move to the previous/next
                            screen.
                            \item[Scroll by Column] Move to the previous/next
                            column.
                        \end{description}
                \end{description}
            \item[Vertical] Submenu for vertical scrolling settings.
                \begin{description}
                    \item[Scrollbar] Toggle the vertical scrollbar for the
                    current mode. If the file fits on one screen, there is no
                    scrollbar and this setting has no effect.
                        \begin{description}
                            \item[No] Do not display the vertical scroll bar.
                            \item[Yes] Display the vertical scroll bar.
                        \end{description}
                    \item[Scroll Mode] Change the function of the ``Scroll-up''
                    and ``Scroll-down'' buttons.
                        \begin{description}
                            \item[Scroll by Page] Scroll up or down one full screen.
                            \item[Scroll by Line] Scroll up or down one line.
                        \end{description}
                    \opt{lcd_bitmap}{
                    \item[Overlap Pages] Set whether the last line from the
                    previous screen is retained when scrolling pages.
                        \begin{description}
                           \item[No] Do not retain previous line.
                           \item[Yes] Retain previous line.
                        \end{description}
                    }
                    \item[Auto-scroll Speed] Control the speed of auto-scrolling
                    in number of lines per second.  Available options are
                    \setting{1} to \setting{10} lines per second. As an example,
                    \setting{4} will scroll the text at four lines per second.
                    \item[Left/Right Key (Narrow mode)] Change the function of
                    the ``Left'' and ``Right'' buttons when the screen is in
                    narrow mode (i.e. one screen per page).
                    \begin{description}
                        \item[Previous/Next Page] Scroll up or down one full screen.
                        \item[Top/Bottom Page] Move to the top or bottom page.
                    \end{description}
                \end{description}
        \end{description}
    \item[Indent Spaces] Set the number of spaces to indent the text when line
      mode is set to \setting{Reflow Lines}. Available options are \setting{0}
      to \setting{5} spaces. If you select \setting{0}, a blank line is
      displayed as an indent.
    \end{description}

\item[Show Playback Menu] Display the playback menu to allow control of the
currently playing music without leaving the plugin.

\item[Select Bookmark] Select a saved bookmark. In the screenshot below, the
``*'' denotes the current page.

\screenshot{plugins/images/ss-text_viewer-sel_bk_menu}{The select bookmark menu}{img:text_viewer-sel_bk}

\item[Global Settings] Set the default settings for the text viewer.
The setting items are the same as \setting{Viewer Options}. The global
settings are stored in
\fname{.rockbox/rocks/viewers/viewer.dat}.

\item[Quit] Exits the plugin. The text viewer automatically
stores its settings, the current position and bookmarks in
\fname{.rockbox/rocks/viewers/viewer\_file.dat}.
\end{description}

\subsubsection{Bookmarks}
    To add a bookmark, press 
    \opt{IRIVER_H100_PAD,IRIVER_H300_PAD}{\ButtonOn+\ButtonSelect}%
    \opt{IPOD_4G_PAD,IPOD_3G_PAD,GIGABEAT_PAD,GIGABEAT_S_PAD,MROBE100_PAD}{\ButtonSelect}%
    \opt{IRIVER_H10_PAD}{\ButtonFF}\opt{IAUDIO_X5_PAD}{\ButtonRec}%
    \opt{SANSA_E200_PAD,SANSA_C200_PAD,SANSA_CLIP_PAD}{\ButtonDown+\ButtonSelect}%
    \opt{COWON_D2_PAD}{\ButtonMenu+\ButtonPlus}%
    \opt{SANSA_FUZE_PAD}{\ButtonUp+\ButtonSelect}%
    \opt{IAUDIO_M3_PAD}{\ButtonRCPlay+\ButtonRCMode}.
    \opt{MPIO_HD200_PAD}{FIXME}
    The bookmark will be displayed as shown below. To delete the bookmark
    press the same button again.

\screenshot{plugins/images/ss-text_viewer-bookmark}{A bookmark}{img:text_viewer-bookmark}



% $Id$ %
\subsection{\label{ref:ThemeRemove}Theme Remove}
This plugin offers a way to remove a theme. Open the \setting{Context Menu} (see \reference{ref:Contextmenu}) 
upon a theme\fname{.cfg} file and select \setting{Open With... $\rightarrow$ theme\_remove}.
Some files are not removed regardless of the \setting{Remove Options} such as
\fname{rockbox\_default.wps} and the font file currently in use.

\subsubsection{Theme Remove menu}
\begin{description}
  \item[Remove Theme.]
  Selecting this will delete the files specified in the \setting{Remove Options}.
  After a theme has been successfully removed, a log message is displayed listing 
  which items have been deleted and which are being kept. Exit this screen by 
  pressing any key. A file called \fname{theme\_remove\_log.txt} is created in 
  the root directory of your \dap{} listing all the changes.

  \item[Remove Options.]
  This menu specifies which items are removed if
  \setting{Remove Theme} is selected in the menu.

  One of the following options can be chosen for each setting.
  \begin{description}
    \item[Ask for Removal.]
    Selecting this option brings up a dialogue with two options:
    press \ActionYesNoAccept{} to confirm deletion or any other key to cancel.
    \item[Remove if not Used.]
    Selecting this option will remove the file automatically, if it is not 
    used by another theme in the theme directory and not currently used.
    \item[Never Remove.]
    Selecting this option will always skip deleting the file.
    \item[Always Remove.]
    Selecting this option will remove the file with no regard to
    whether it's used by another theme or not.
  \end{description}

  \begin{description}
    \item[Font.]
    Specifies how the \fname{.fnt} file belonging to a theme \fname{.cfg} file is handled.
    If this option is set to \setting{Remove if not Used}, the fonts came from rockbox-fonts.zip
    will not be removed as themes may depend on those fonts.
    \item[WPS.]
    Specifies how the \fname{.wps} file belonging to a theme \fname{.cfg} file is handled.
    \item[Statusbar Skin.]
    Specifies how the \fname{.sbs} file belonging to a theme \fname{.cfg} file is handled.
\opt{HAVE_REMOTE_LCD}{
    \item[Remote WPS.]
    Specifies how the \fname{.rwps} file belonging to a theme \fname{.cfg} file is handled.
    \item[Remote Statusbar Skin.]
    Specifies how the \fname{.rsbs} file belonging to a theme \fname{.cfg} file is handled.
}%
\opt{lcd_non-mono}{
    \item[Backdrop.]
    Specifies how the backdrop \fname{.bmp} file belonging to a theme \fname{.cfg} file is handled.
}%
    \item[Iconset.]
    Specifies how the iconset \fname{.bmp} file belonging to a theme \fname{.cfg} file is handled.
    \item[Viewers Iconset.]
    Specifies how the viewers iconset \fname{.bmp} file belonging to a theme \fname{.cfg} file is handled.
\opt{HAVE_REMOTE_LCD}{
    \item[Remote Iconset.]
    Specifies how the remote iconset \fname{.bmp} file belonging to 
    a theme \fname{.cfg} file is handled.
    \item[Remote Viewers Iconset.]
    Specifies how the remote viewers iconset \fname{.bmp} file belonging to 
    a theme \fname{.cfg} file is handled.
}% HAVE_REMOTE_LCD
\opt{lcd_color}{
    \item[Filetype Colours.]
     Specifies how the colours \fname{.colours} file belonging to a theme \fname{.cfg} file is handled.
}%

    \item[Create Log File.]
    Setting this to \setting{No} prevents the log file from being created.
  \end{description}
  \item[Quit.]
  Exits this plugin.
\end{description}


\input{plugins/vbrfix.tex}

\subsection{\label{ref:ZXBox}ZXBox}
\screenshot{plugins/images/ss-zxbox}{ZXBox}{img:zxbox}
ZXBox is a port of the ``Spectemu'' ZX Spectrum 48k emulator for Rockbox
(\Pointinghand\href{http://kempelen.iit.bme.hu/~mszeredi/spectemu/spectemu.html}
{project's homepage}). To start a game open a tape file or snapshot saved as
\fname{.tap}, \fname{.tzx}, \fname{.z80} or \fname{.sna} in the file browser.\\
\note{As ZXBox is a 48k emulator only loading of 48k z80 snapshots is possible.}

\subsubsection{Default keys}
The emulator is set up for 5 different buttons: Up, Down, Left, Right and
Jump/Fire. Each one of these can be mapped to one key of the Spectrum Keyboard
or they can be used like a ``Kempston'' joystick. Per default the buttons,
including an additional but fixed menu button, are assigned as follows:

\begin{btnmap}
    \opt{IPOD_3G_PAD,IPOD_4G_PAD}{\ButtonMenu/\ButtonPlay/}
    \opt{RECORDER_PAD,ONDIO_PAD,IRIVER_H100_PAD,IRIVER_H300_PAD,GIGABEAT_PAD,GIGABEAT_S_PAD%
        ,IAUDIO_X5_PAD,SANSA_C200_PAD,SANSA_CLIP_PAD,SANSA_E200_PAD,SANSA_FUZE_PAD,MROBE100_PAD%
        ,PBELL_VIBE500_PAD,SANSA_FUZEPLUS_PAD,SAMSUNG_YH92X_PAD,SAMSUNG_YH820_PAD}%
        {\ButtonUp/\ButtonDown/}
    \opt{IRIVER_H10_PAD}{\ButtonScrollUp/\ButtonScrollDown/}
    \opt{IPOD_3G_PAD,IPOD_4G_PAD,RECORDER_PAD,ONDIO_PAD,IRIVER_H100_PAD%
        ,IRIVER_H300_PAD,GIGABEAT_PAD,GIGABEAT_S_PAD,IAUDIO_X5_PAD%
        ,SANSA_C200_PAD,SANSA_CLIP_PAD,SANSA_E200_PAD,SANSA_FUZE_PAD,MROBE100_PAD%
        ,IRIVER_H10_PAD,PBELL_VIBE500_PAD,SANSA_FUZEPLUS_PAD,SAMSUNG_YH92X_PAD%
        ,SAMSUNG_YH820_PAD}{\ButtonLeft/\ButtonRight}
    \opt{COWON_D2_PAD}{\TouchTopMiddle{}/\TouchBottomMiddle{}/\TouchMidLeft{}/\TouchMidRight}
    \opt{MPIO_HD200_PAD}{\ButtonVolDown / \ButtonVolUp / \ButtonRew / \ButtonFF}
    \opt{MPIO_HD300_PAD}{\ButtonRew / \ButtonFF / \ButtonScrollUp / \ButtonScrollDown}
  \opt{HAVEREMOTEKEYMAP}{& }
    & Directional movement\\
    %
    \opt{IPOD_3G_PAD,IPOD_4G_PAD,GIGABEAT_PAD,GIGABEAT_S_PAD,IAUDIO_X5_PAD%
        ,SANSA_C200_PAD,SANSA_CLIP_PAD,SANSA_E200_PAD,SANSA_FUZE_PAD,MROBE100_PAD
        ,SANSA_FUZEPLUS_PAD}{\ButtonSelect}
    \opt{RECORDER_PAD,SAMSUNG_YH92X_PAD,SAMSUNG_YH820_PAD}{\ButtonPlay}
    \opt{IRIVER_H100_PAD,IRIVER_H300_PAD}{\ButtonOn}
    \opt{ONDIO_PAD}{\ButtonMenu}
    \opt{IRIVER_H10_PAD}{\ButtonRew}
    \opt{COWON_D2_PAD}{\TouchCenter}
    \opt{PBELL_VIBE500_PAD}{\ButtonOK}
    \opt{MPIO_HD200_PAD}{\ButtonFunc}
    \opt{MPIO_HD300_PAD}{\ButtonEnter}
  \opt{HAVEREMOTEKEYMAP}{& }
    & Jump/Fire\\
    %
    \opt{RECORDER_PAD}{\ButtonFOne}
    \opt{ONDIO_PAD}{\ButtonOff}
    \opt{IPOD_3G_PAD,IPOD_4G_PAD}{\ButtonHold{} switch}
    \opt{IRIVER_H100_PAD,IRIVER_H300_PAD}{\ButtonMode}
    \opt{GIGABEAT_PAD,GIGABEAT_S_PAD,COWON_D2_PAD}{\ButtonMenu}
    \opt{SANSA_C200_PAD,SANSA_CLIP_PAD,SANSA_E200_PAD,MROBE100_PAD}{\ButtonPower}
    \opt{SANSA_FUZE_PAD}{Long \ButtonHome}
    \opt{IAUDIO_X5_PAD}{\ButtonPlay}
    \opt{IRIVER_H10_PAD}{\ButtonFF}
    \opt{PBELL_VIBE500_PAD}{\ButtonCancel}
    \opt{MPIO_HD200_PAD}{\ButtonRec + \ButtonPlay}
    \opt{MPIO_HD300_PAD}{Long \ButtonMenu}
    \opt{SANSA_FUZEPLUS_PAD}{\ButtonBack}
    \opt{SAMSUNG_YH92X_PAD,SAMSUNG_YH820_PAD}{\ButtonRew}
  \opt{HAVEREMOTEKEYMAP}{& }
    & Open ZXBox menu\\
\end{btnmap}

\subsubsection{ZXBox menu}
\begin{description}
\item[ Vkeyboard.]
    This is a virtual keyboard representing the Spectrum keyboard. Controls are
    the same as in standard Rockbox, but you just press one key instead of
    entering a phrase.
\item[Play/Pause Tape.] Toggles playing of the tape (if it is loaded).
\item[Save Quick Snapshot.] Saves snapshot into \fname{/.rockbox/zxboxq.z80}.
\item[Load Quick Snapshot.] Loads snapshot from \fname{/.rockbox/zxboxq.z80}.
\item[Save Snapshot.]
    Saves a snapshot of the current state. You would enter the full path and
    desired name - for example \fname{/games/zx/snapshots/chuckie.sna}. The
    snapshot format will be chosen after the extension you specified, per
    default \fname{.z80} will be taken in case you leave it open.
\item[Toggle Fast Mode.]
    Toggles fastest possible emulation speed (no sound, maximum frameskip etc.).
    This is Useful when loading tapes with some specific loaders.
\item[Options.]
    \begin{description}
    \item[Map Keys To Kempston.]
        Controls whether the \daps{} buttons should simulate a ``Kempston''
        joystick or some assigned keys of the Spectrum keyboard.
    \item[Display Speed.]Toggle displaying the emulation speed (in percent).
    \item[Invert Colours.]
        Inverts the Spectrum colour palette, sometimes helps visibility.
    \item[Frameskip]
        Sets the number of frames to skip before displaying one. With zero
        frameskip ZXBox tries to display 50 frames per second.
    \item[Sound.]Turns sound on or off.
    \item[Volume.]Controls volume of sound output.
    \item[Predefined Keymap]
        Select one of the predefined keymaps. For example \setting{2w90z} means:
        map ZXBox's \btnfnt{Up} to \setting{2}, \btnfnt{Down} to \setting{w},
        \btnfnt{Left} to \setting{9}, \btnfnt{Right} to \setting{0} and
        \btnfnt{Jump/Fire} to \setting{z}. This example keymap is used in the
        ``Chuckie Egg'' game.
    \item[Custom Keymap]
        This menu allows you to map one of the Spectrum keys accessible through the 
        plugin's virtual keyboard to each one of the buttons.
    \item[Quit.] Quits the emulator..
    \end{description}
\end{description}

\nopt{ipodvideo}{% no scaling for here, still include it?
\subsubsection{Hacking graphics}
Due to ZXBox's simple (but fast) scaling to the screen by dropping lines and
columns some games can become unplayable. It is possible to hack graphics to
make them better visible with the help of an utility such as the ``Spectrum
Graphics Editor''. Useful tools can be found at the ``World of Spectrum'' site
(\url{http://www.worldofspectrum.org/utilities.html}).}


\section{Applications}

\opt{rtc}{\subsection{Alarm Clock}

This plugin is an alarm clock, which resumes a paused song at a given time.

\subsubsection{Key configuration}
\begin{btnmap}
    \PluginLeft{} / \PluginRight
       \opt{HAVEREMOTEKEYMAP}{& \PluginRCLeft{} / \PluginRCRight}
    & Switch between hours/minutes selection \\

    \opt{scrollwheel}{\PluginScrollBack{} / \PluginScrollFwd{} or}
    \nopt{IPOD_4G_PAD,IPOD_3G_PAD}{\PluginUp{} / \PluginDown}
    \opt{IPOD_4G_PAD,IPOD_3G_PAD}{\PluginDown}
       \opt{HAVEREMOTEKEYMAP}{& \PluginRCUp{} / \PluginRCDown}
    & Increase/Decrease hours/minutes \\

    \PluginSelect
       \opt{HAVEREMOTEKEYMAP}{& \PluginRCSelect}
    & Set the alarm \\

    \nopt{IPOD_4G_PAD,IPOD_3G_PAD}{\PluginCancel}
    \opt{IPOD_4G_PAD,IPOD_3G_PAD}{\ButtonMenu}
       \opt{HAVEREMOTEKEYMAP}{& \PluginRCCancel}
    & Exit \\
\end{btnmap}

\subsubsection{Setting an alarm}
First select a track and play it, then launch the ``alarmclock'' plugin. The 
plugin pauses the playback. Enter a 24h-time (e.g. 13:58) and set the alarm. 
Music playback will resume when the set time is reached.
}

\input{plugins/batterybenchmark.tex}

% $Id$ %
\subsection{Calculator}
\screenshot{plugins/images/ss-calculator}{Calculator}{img:calculator}

This is a simple scientific calculator for use on the \dap. It works like a
standard calculator. Pressing the ``1st'' and ``2nd'' buttons will toggle between
 other available math functions.

\begin{btnmap}
    \opt{RECORDER_PAD,ONDIO_PAD,IRIVER_H100_PAD,IRIVER_H300_PAD,IPOD_4G_PAD%
      ,IPOD_3G_PAD,IAUDIO_X5_PAD,SANSA_E200_PAD,SANSA_C200_PAD,SANSA_CLIP_PAD,GIGABEAT_PAD%
      ,GIGABEAT_S_PAD,MROBE100_PAD,IRIVER_H10_PAD,SANSA_FUZE_PAD,PBELL_VIBE500_PAD%
      ,SANSA_FUZEPLUS_PAD,SAMSUNG_YH92X_PAD,SAMSUNG_YH820_PAD}
      {\ButtonLeft{} / \ButtonRight{} /}
    \opt{RECORDER_PAD,ONDIO_PAD,IRIVER_H100_PAD,IRIVER_H300_PAD,IAUDIO_X5_PAD%
      ,SANSA_E200_PAD,SANSA_C200_PAD,SANSA_CLIP_PAD,GIGABEAT_PAD,GIGABEAT_S_PAD,MROBE100_PAD%
      ,SANSA_FUZE_PAD,PBELL_VIBE500_PAD,SANSA_FUZEPLUS_PAD,SAMSUNG_YH92X_PAD,SAMSUNG_YH820_PAD}
      {\ButtonUp{} / \ButtonDown}
    \opt{SANSA_E200_PAD,SANSA_FUZE_PAD}{/}
    \opt{scrollwheel}{\ButtonScrollFwd{} / \ButtonScrollBack}
    \opt{IRIVER_H10_PAD}{\ButtonScrollUp{} / \ButtonScrollDown}
    \opt{MPIO_HD300_PAD}{\ButtonRew / \ButtonFF / \ButtonScrollUp / \ButtonScrollDown}
    \opt{MPIO_HD200_PAD}{\ButtonRew / \ButtonFF}
    \opt{COWON_D2_PAD}{\TouchMidLeft{} / \TouchMidRight / \TouchTopMiddle / \TouchBottomMiddle}
       \opt{HAVEREMOTEKEYMAP}{& }
    & Move around the keypad\\
    %
    \opt{RECORDER_PAD}{\ButtonPlay}
    \opt{ONDIO_PAD}{\ButtonMenu}
    \opt{IRIVER_H100_PAD,IRIVER_H300_PAD,IPOD_4G_PAD,IPOD_3G_PAD,IAUDIO_X5_PAD%
        ,SANSA_E200_PAD,SANSA_C200_PAD,SANSA_CLIP_PAD,GIGABEAT_PAD,GIGABEAT_S_PAD,MROBE100_PAD%
        ,SANSA_FUZE_PAD,SANSA_FUZEPLUS_PAD}
        {\ButtonSelect}
    \opt{IRIVER_H10_PAD,SAMSUNG_YH92X_PAD,SAMSUNG_YH820_PAD}{\ButtonPlay}
    \opt{COWON_D2_PAD}{\TouchCenter}
    \opt{PBELL_VIBE500_PAD}{\ButtonOK}
    \opt{MPIO_HD200_PAD}{\ButtonFunc}
    \opt{MPIO_HD300_PAD}{\ButtonEnter}
       \opt{HAVEREMOTEKEYMAP}{& }
    & Select a button\\
    %
    \nopt{ONDIO_PAD,IPOD_4G_PAD,IPOD_3G_PAD}{
        \opt{RECORDER_PAD}{\ButtonFOne}
        \opt{IRIVER_H100_PAD,IRIVER_H300_PAD,IAUDIO_X5_PAD,SANSA_E200_PAD%
            ,SANSA_C200_PAD,MPIO_HD200_PAD}{\ButtonRec}
        \opt{SANSA_CLIP_PAD,SANSA_FUZE_PAD}{\ButtonHome}
        \opt{IRIVER_H10_PAD,SAMSUNG_YH92X_PAD,SAMSUNG_YH820_PAD}{\ButtonRew}
        \opt{GIGABEAT_PAD}{\ButtonA}
        \opt{GIGABEAT_S_PAD}{\ButtonPlay}
        \opt{MROBE100_PAD}{\ButtonDisplay}
        \opt{COWON_D2_PAD,MPIO_HD300_PAD}{\ButtonMenu}
        \opt{PBELL_VIBE500_PAD}{\ButtonCancel}
        \opt{SANSA_FUZEPLUS_PAD}{\ButtonBack}
       \opt{HAVEREMOTEKEYMAP}{& }
        & Delete last entered digit or clear after calculation\\
        %
    }
    \opt{RECORDER_PAD,IRIVER_H100_PAD,IRIVER_H300_PAD,SAMSUNG_YH820_PAD}{
        \opt{RECORDER_PAD}{\ButtonFTwo}
        \opt{IRIVER_H100_PAD,IRIVER_H300_PAD}{\ButtonMode}
        \opt{SAMSUNG_YH820_PAD}{\ButtonRec}
       \opt{HAVEREMOTEKEYMAP}{& }
        & Cycle through the 4 basic operators\\
        %
    }
    \opt{RECORDER_PAD}{\ButtonFThree}
    \opt{ONDIO_PAD}{Long \ButtonMenu}
    \opt{IRIVER_H100_PAD,IRIVER_H300_PAD}{\ButtonOn}
    \opt{IPOD_4G_PAD,IPOD_3G_PAD,IRIVER_H10_PAD,IAUDIO_X5_PAD,IRIVER_H10_PAD
        ,PBELL_VIBE500_PAD,MPIO_HD200_PAD,MPIO_HD300_PAD,SANSA_FUZEPLUS_PAD}
        {\ButtonPlay}
    \opt{SANSA_E200_PAD,SANSA_C200_PAD,SANSA_CLIP_PAD,SANSA_FUZE_PAD}{Long \ButtonSelect}
    \opt{GIGABEAT_PAD,GIGABEAT_S_PAD,MROBE100_PAD}{\ButtonMenu}
    \opt{COWON_D2_PAD}{\TouchBottomRight}
    \opt{SAMSUNG_YH92X_PAD,SAMSUNG_YH820_PAD}{\ButtonFF}
       \opt{HAVEREMOTEKEYMAP}{& }
    & Calculate\\
    %
    \opt{RECORDER_PAD,ONDIO_PAD,IRIVER_H100_PAD,IRIVER_H300_PAD}{\ButtonOff}
    \opt{IPOD_4G_PAD,IPOD_3G_PAD}{\ButtonMenu}
    \opt{IAUDIO_X5_PAD,IRIVER_H10_PAD,SANSA_E200_PAD,SANSA_C200_PAD,SANSA_CLIP_PAD,GIGABEAT_PAD%
        ,MROBE100_PAD,COWON_D2_PAD}{\ButtonPower}
    \opt{SANSA_FUZE_PAD}{Long \ButtonHome}
    \opt{GIGABEAT_S_PAD}{\ButtonBack}
    \opt{PBELL_VIBE500_PAD}{\ButtonRec}
    \opt{SAMSUNG_YH92X_PAD,SAMSUNG_YH820_PAD}{Long \ButtonRew}
    \opt{MPIO_HD200_PAD}{\ButtonRec + \ButtonPlay}
    \opt{MPIO_HD300_PAD}{Long \ButtonMenu}
    \opt{SANSA_FUZEPLUS_PAD}{Long \ButtonBack}
       \opt{HAVEREMOTEKEYMAP}{&
          \opt{IRIVER_RC_H100_PAD}{\ButtonRCStop}
       }
    & Quit\\
\end{btnmap}


\opt{rtc}{% $Id$ %
\subsection{Calendar}
\screenshot{plugins/images/ss-calendar}{Calendar}{img:calendar}
This is a small and simple calendar application with memo saving function.
Dots indicate dates with memos. The available memo types are: one off,
yearly, monthly, and weekly memos.

You can select what day is first day of week by the setting \setting{First Day of Week} in the menu.

\begin{btnmap}
    \opt{IRIVER_H100_PAD,IRIVER_H300_PAD,IAUDIO_X5_PAD%
        ,SANSA_C200_PAD,SANSA_CLIP_PAD,GIGABEAT_PAD,MROBE100_PAD,GIGABEAT_S_PAD,IPOD_4G_PAD%
        ,IPOD_3G_PAD,SANSA_E200_PAD,IRIVER_H10_PAD,SANSA_FUZE_PAD,PBELL_VIBE500_PAD
        ,SANSA_FUZEPLUS_PAD,SAMSUNG_YH92X_PAD,SAMSUNG_YH820_PAD}
        {\ButtonLeft{} / \ButtonRight{} /}
    \opt{IRIVER_H100_PAD,IRIVER_H300_PAD,IAUDIO_X5_PAD,SANSA_C200_PAD,SANSA_CLIP_PAD%
        ,GIGABEAT_PAD,MROBE100_PAD,GIGABEAT_S_PAD,PBELL_VIBE500_PAD,SANSA_FUZEPLUS_PAD%
        ,SAMSUNG_YH92X_PAD,SAMSUNG_YH820_PAD}
        {\ButtonUp{} / \ButtonDown}
    \opt{IPOD_4G_PAD,IPOD_3G_PAD,SANSA_E200_PAD,SANSA_FUZE_PAD}
        {\ButtonScrollFwd{} / \ButtonScrollBack}
    \opt{IRIVER_H10_PAD,MPIO_HD300_PAD}{\ButtonScrollUp{} / \ButtonScrollDown}
    \opt{COWON_D2_PAD}{\TouchMidLeft{} / \TouchMidRight{} / \TouchTopMiddle{} / \TouchBottomMiddle}
       \opt{HAVEREMOTEKEYMAP}{& }
    & Move the selector\\
    %
    \opt{IRIVER_H100_PAD,IRIVER_H300_PAD,IPOD_4G_PAD,IPOD_3G_PAD,IAUDIO_X5_PAD%
        ,SANSA_E200_PAD,SANSA_C200_PAD,SANSA_CLIP_PAD,GIGABEAT_PAD,MROBE100_PAD,GIGABEAT_S_PAD%
        ,SANSA_FUZE_PAD,SANSA_FUZEPLUS_PAD}
        {\ButtonSelect}
    \opt{IRIVER_H10_PAD,SAMSUNG_YH92X_PAD,SAMSUNG_YH820_PAD}{\ButtonPlay}
    \opt{PBELL_VIBE500_PAD}{\ButtonOK}
    \opt{MPIO_HD300_PAD}{\ButtonEnter}
       \opt{HAVEREMOTEKEYMAP}{& }
    & Show memos for the selected day\\
    %
    \opt{MPIO_HD300_PAD}{\ButtonRew / \ButtonFF
    & Previous / Next week\\}
    %
    \opt{MROBE100_PAD}{\ButtonMenu{} + \ButtonUp{} / \ButtonDown}
    \opt{IRIVER_H100_PAD,IRIVER_H300_PAD}{\ButtonMode{} / \ButtonRec}
    \opt{IPOD_4G_PAD,IPOD_3G_PAD}{\ButtonPlay{} / \ButtonMenu}
    \opt{IAUDIO_X5_PAD}{\ButtonRec{} / \ButtonPlay}
    \opt{GIGABEAT_PAD,SANSA_C200_PAD,SANSA_CLIP_PAD}{\ButtonVolUp{} / \ButtonVolDown}
    \opt{GIGABEAT_S_PAD}{\ButtonNext{} / \ButtonPrev}
    \opt{SANSA_E200_PAD,SANSA_FUZE_PAD}{\ButtonUp{} / \ButtonDown}
    \opt{IRIVER_H10_PAD}{\ButtonRew{} / \ButtonFF}
    \opt{COWON_D2_PAD}{\TouchBottomLeft{} / \TouchBottomRight}
    \opt{PBELL_VIBE500_PAD}{\ButtonMenu{} / \ButtonPlay}
    \opt{SAMSUNG_YH92X_PAD}{\ButtonFF{} + \ButtonUp{} / \ButtonDown}
    \opt{SAMSUNG_YH820_PAD}{\ButtonRec{} + \ButtonUp{} / \ButtonDown}
    \opt{MPIO_HD300_PAD}{\ButtonRec{} / \ButtonPlay}
    \opt{SANSA_FUZEPLUS_PAD}{\ButtonBack{} / \ButtonPlay}
       \opt{HAVEREMOTEKEYMAP}{& }
    & Previous / Next month\\
    %
    \opt{IRIVER_H100_PAD,IRIVER_H300_PAD}{\ButtonOff}
    \opt{IPOD_4G_PAD,IPOD_3G_PAD}{\ButtonMenu{} + \ButtonSelect}
    \opt{GIGABEAT_S_PAD}{\ButtonBack}
    \opt{IAUDIO_X5_PAD,IRIVER_H10_PAD,SANSA_E200_PAD,SANSA_C200_PAD,SANSA_CLIP_PAD,GIGABEAT_PAD,MROBE100_PAD}{\ButtonPower}
    \opt{SANSA_FUZE_PAD}{Long \ButtonHome}
    \opt{COWON_D2_PAD}{\ButtonPower}
    \opt{PBELL_VIBE500_PAD}{\ButtonRec}
    \opt{SAMSUNG_YH92X_PAD,SAMSUNG_YH820_PAD}{\ButtonRew}
    \opt{MPIO_HD300_PAD}{Long \ButtonMenu}
       \opt{HAVEREMOTEKEYMAP}{& }
    & Quit\\
\end{btnmap}
}

% $Id$ %
\subsection{Chess Clock}
\screenshot{plugins/images/ss-chess_clock}{Chess Clock}{img:chessclock}
The chess clock plugin is designed  to simulate a chess clock, but it can be
used in any kind of game with up to ten players.

\subsubsection{Setup}
  \begin{btnmap}
    \opt{PLAYER_PAD,IRIVER_H100_PAD,IRIVER_H300_PAD,IAUDIO_X5_PAD,IRIVER_H10_PAD%
        ,SANSA_E200_PAD,SANSA_C200_PAD,SANSA_CLIP_PAD,SANSA_FUZE_PAD,SANSA_FUZEPLUS_PAD}
        {\ButtonRight{} / \ButtonLeft}
    \opt{RECORDER_PAD,ONDIO_PAD,GIGABEAT_PAD,GIGABEAT_S_PAD,MROBE100_PAD,PBELL_VIBE500_PAD}
        {\ButtonUp{} / \ButtonDown}
    \opt{IPOD_4G_PAD,IPOD_3G_PAD}{\ButtonScrollBack{} / \ButtonScrollFwd}
    \opt{MPIO_HD200_PAD}{\ButtonVolUp / \ButtonVolDown}
    \opt{MPIO_HD300_PAD}{\ButtonScrollUp / \ButtonScrollDown}
       \opt{HAVEREMOTEKEYMAP}{& }
      & Increase / decrease displayed Value\\
    %
    \opt{PLAYER_PAD,RECORDER_PAD,IRIVER_H10_PAD}{\ButtonPlay}
    \opt{IRIVER_H100_PAD,IRIVER_H300_PAD}{\ButtonOn}
    \opt{ONDIO_PAD}{\ButtonRight}
    \opt{IPOD_4G_PAD,IPOD_3G_PAD,IAUDIO_X5_PAD,SANSA_E200_PAD,SANSA_C200_PAD,SANSA_CLIP_PAD%
        ,GIGABEAT_PAD,GIGABEAT_S_PAD,MROBE100_PAD,SANSA_FUZE_PAD,SANSA_FUZEPLUS_PAD}{\ButtonSelect}
    \opt{PBELL_VIBE500_PAD}{\ButtonOK}
    \opt{MPIO_HD200_PAD}{\ButtonFunc}
    \opt{MPIO_HD300_PAD}{\ButtonEnter}
       \opt{HAVEREMOTEKEYMAP}{& }
      & Move to next screen\\
    %
    \opt{PLAYER_PAD}{\ButtonStop}
    \opt{ONDIO_PAD,IPOD_4G_PAD,IPOD_3G_PAD}{\ButtonMenu}
    \opt{RECORDER_PAD,IRIVER_H100_PAD,IRIVER_H300_PAD}{\ButtonOff}
    \opt{IAUDIO_X5_PAD,MPIO_HD200_PAD,MPIO_HD300_PAD}{\ButtonRec}
    \opt{IRIVER_H10_PAD,SANSA_E200_PAD,SANSA_C200_PAD,SANSA_CLIP_PAD,GIGABEAT_PAD,MROBE100_PAD}{\ButtonPower}
    \opt{SANSA_FUZE_PAD}{Long \ButtonHome}
    \opt{GIGABEAT_S_PAD,SANSA_FUZEPLUS_PAD}{\ButtonBack}
    \opt{PBELL_VIBE500_PAD}{\ButtonCancel}
       \opt{HAVEREMOTEKEYMAP}{& }
      & Move to previous screen\\
  \end{btnmap}

\begin{itemize}
  \item First enter the number of players (1--10)
  \item Then set the total game time in mm:ss
  \item Then the maximum round time is entered.  For example, this could
  be used to play Scrabble for a maximum of 15 minutes each, with each
  round taking no longer than one minute.
  \item Done. Player 1 starts in paused mode.
\end{itemize}

\subsubsection{While playing}
The number of the current player is displayed on the top line. The time
below is the time remaining for that round (and possibly also the total
time left if different).

Keys are as follows:

  \begin{btnmap}
    \opt{PLAYER_PAD}{\ButtonOn}
    \opt{RECORDER_PAD,ONDIO_PAD}{\ButtonOff}
    \opt{IRIVER_H100_PAD,IRIVER_H300_PAD}{\ButtonSelect}
    \opt{IPOD_4G_PAD,IPOD_3G_PAD}{\ButtonPlay}
    \opt{IAUDIO_X5_PAD}{\ButtonRec}
    \opt{IRIVER_H10_PAD,SANSA_E200_PAD,SANSA_C200_PAD,SANSA_CLIP_PAD,GIGABEAT_PAD,MROBE100_PAD}{\ButtonPower}
    \opt{SANSA_FUZE_PAD,SANSA_FUZEPLUS_PAD}{\ButtonPower}
    \opt{GIGABEAT_S_PAD}{\ButtonBack}
    \opt{PBELL_VIBE500_PAD}{\ButtonRec}
    \opt{MPIO_HD200_PAD}{\ButtonRec + \ButtonPlay}
    \opt{MPIO_HD300_PAD}{Long \ButtonMenu}
       \opt{HAVEREMOTEKEYMAP}{& }
      & Exit plugin \\
    %
    \opt{PLAYER_PAD}{\ButtonStop}
    \opt{RECORDER_PAD,ONDIO_PAD,IPOD_4G_PAD,IPOD_3G_PAD,PBELL_VIBE500_PAD}{\ButtonLeft}
    \opt{IRIVER_H100_PAD,IRIVER_H300_PAD}{\ButtonOff}
    \opt{IAUDIO_X5_PAD}{\ButtonPower}
    \opt{IRIVER_H10_PAD}{\ButtonFF}
    \opt{SANSA_E200_PAD,SANSA_C200_PAD,SANSA_CLIP_PAD,SANSA_FUZE_PAD}{\ButtonDown}
    \opt{GIGABEAT_PAD}{\ButtonA}
    \opt{GIGABEAT_S_PAD}{\ButtonPrev}
    \opt{MROBE100_PAD}{\ButtonDisplay}
    \opt{SANSA_FUZEPLUS_PAD}{\ButtonBack}
       \opt{HAVEREMOTEKEYMAP}{& }
      & Restart round for the current player \\
    %
    \opt{PLAYER_PAD,RECORDER_PAD,IAUDIO_X5_PAD,IRIVER_H10_PAD,GIGABEAT_S_PAD}{\ButtonPlay}
    \opt{IRIVER_H100_PAD,IRIVER_H300_PAD}{\ButtonOn}
    \opt{ONDIO_PAD}{\ButtonRight}
    \opt{IPOD_4G_PAD,IPOD_3G_PAD,SANSA_E200_PAD,SANSA_C200_PAD,SANSA_CLIP_PAD,GIGABEAT_PAD%
        ,MROBE100_PAD,SANSA_FUZE_PAD}
        {\ButtonSelect}
    \opt{PBELL_VIBE500_PAD,MPIO_HD200_PAD,MPIO_HD300_PAD,SANSA_FUZEPLUS_PAD}{\ButtonPlay}
       \opt{HAVEREMOTEKEYMAP}{& }
      & Pause the time (press again to continue) \\
    %
    \opt{PLAYER_PAD,IRIVER_H100_PAD,IRIVER_H300_PAD,IAUDIO_X5_PAD,IRIVER_H10_PAD%
        ,SANSA_E200_PAD,SANSA_C200_PAD,SANSA_CLIP_PAD,SANSA_FUZE_PAD}{\ButtonRight}
    \opt{RECORDER_PAD,ONDIO_PAD,GIGABEAT_PAD,GIGABEAT_S_PAD,MROBE100_PAD,PBELL_VIBE500_PAD,SANSA_FUZEPLUS_PAD}
        {\ButtonUp}
    \opt{IPOD_4G_PAD,IPOD_3G_PAD}{\ButtonScrollBack}
       \opt{HAVEREMOTEKEYMAP}{& }
      & Switch to next player \\
    %
    \opt{PLAYER_PAD,IRIVER_H100_PAD,IRIVER_H300_PAD,IAUDIO_X5_PAD,IRIVER_H10_PAD%
        ,SANSA_E200_PAD,SANSA_C200_PAD,SANSA_CLIP_PAD,SANSA_FUZE_PAD}{\ButtonLeft}
    \opt{RECORDER_PAD,ONDIO_PAD,GIGABEAT_PAD,GIGABEAT_S_PAD,MROBE100_PAD,PBELL_VIBE500_PAD,SANSA_FUZEPLUS_PAD}
        {\ButtonDown}
    \opt{IPOD_4G_PAD,IPOD_3G_PAD}{\ButtonScrollFwd}
    \opt{MPIO_HD200_PAD}{\ButtonVolDown}
    \opt{MPIO_HD300_PAD}{\ButtonScrollDown}
       \opt{HAVEREMOTEKEYMAP}{& }
      & Switch to previous player \\
    %
    \opt{PLAYER_PAD,ONDIO_PAD,IPOD_4G_PAD,IPOD_3G_PAD,GIGABEAT_PAD,GIGABEAT_S_PAD%
         ,MROBE100_PAD,PBELL_VIBE500_PAD,MPIO_HD300_PAD}{\ButtonMenu}
    \opt{RECORDER_PAD}{\ButtonFOne}
    \opt{IRIVER_H100_PAD,IRIVER_H300_PAD}{\ButtonRec}
    \opt{IAUDIO_X5_PAD}{\ButtonSelect}
    \opt{SANSA_FUZEPLUS_PAD}{Long \ButtonSelect}
    \opt{IRIVER_H10_PAD}{\ButtonRew}
    \opt{SANSA_E200_PAD,SANSA_C200_PAD,SANSA_CLIP_PAD,SANSA_FUZE_PAD}{\ButtonUp}
    \opt{MPIO_HD200_PAD}{Long \ButtonFunc}
       \opt{HAVEREMOTEKEYMAP}{& }
      & Open menu %
        \opt{PLAYER_PAD,RECORDER_PAD,IRIVER_H10_PAD}{(\ButtonPlay\ to select.)}%
        \opt{IRIVER_H100_PAD,IRIVER_H300_PAD}{(\ButtonOn\ to select.)}%
        \opt{ONDIO_PAD}{(\ButtonRight\ to select.)}%
        \opt{IPOD_4G_PAD,IPOD_3G_PAD,SANSA_E200_PAD,SANSA_C200_PAD,SANSA_CLIP_PAD,GIGABEAT_PAD%
            ,GIGABEAT_S_PAD,MROBE100_PAD,SANSA_FUZE_PAD,SANSA_FUZEPLUS_PAD}
            {(\ButtonSelect\ to select.)}%
        \opt{IAUDIO_X5_PAD}{(press again to select.)}\\
  \end{btnmap}
From the menu it is possible to delete a player, modify the round time
for the current player or set the total time for the game. When the round time
is up for a player the message ``ROUND UP!'' is shown (press  NEXT to
continue). When the total time is up for a player the message ``TIME UP!''is
shown. The player will then be removed from the timer.


\opt{rtc}{\subsection{Clock}
\screenshot{plugins/images/ss-clock}{Clock}{img:clock}

This is a fully featured analogue and digital clock plugin.  

\subsubsection{Key configuration}

\begin{btnmap}
    \PluginLeft{} / \PluginRight
       \opt{HAVEREMOTEKEYMAP}{& \PluginRCLeft{} / \PluginRCRight}
    & Cycle through modes \\

    \nopt{IPOD_4G_PAD,IPOD_3G_PAD}{\PluginUp{} / \PluginDown}
    \opt{IPOD_4G_PAD,IPOD_3G_PAD}{\PluginDown}
       \opt{HAVEREMOTEKEYMAP}{& \PluginRCUp{} / \PluginRCDown}
    & Cycle through skins \\

    \nopt{IPOD_4G_PAD,IPOD_3G_PAD}{\PluginCancel}
    \opt{IPOD_4G_PAD,IPOD_3G_PAD}{\ButtonMenu}
       \opt{HAVEREMOTEKEYMAP}{& \PluginRCCancel}
    & Main Menu \\

    \PluginSelect
       \opt{HAVEREMOTEKEYMAP}{& \PluginRCSelect}
    & Start / Stop Counter \\

    \PluginSelectRepeat
       \opt{HAVEREMOTEKEYMAP}{& \PluginRCSelectRepeat}
    & Reset Counter \\
\end{btnmap}

\textbf{Clock Menu}
\begin{description}
\item[View Clock] Exits the menu and returns to the current clock mode display.
\item[Mode Selector] Opens a menu from which you can select a clock mode to view.
\item[Counter Settings] Opens a menu from which you can adjust settings
    pertaining to the counter.
\item[Mode Settings] Opens a menu from which you can adjust settings pertaining to
    the current clock mode (analog, digital, binary).
\item[General Settings]
    \begin{description}
    \item[Reset Settings]
    Reset all settings to their default values.
    \item[Save Settings]
    Save all settings to disk.
    \item[Show Counter]
    Toggle Counter display.
    \item[Save]
    Choose whether to disable automatic saving, saving to disk on exit, or 
    saving to disk every settings change.
    \item[Backlight]
    Choose whether to disable the backlight, use the user's timeout setting,
    or keep the backlight on.
    \item[Idle Poweroff]
    Toggle Idle Poweroff.
    \note{This setting is not saved to disk.}
    \end{description}
\item[Help] Opens a brief help screen with key mappings and functionality.
\item[Credits] Displays a credits roll.
\end{description}

\subsubsection{Analog mode}
Small, round, analog clock is displayed in the middle of the LCD.
Time readout, if enabled, is displayed at the upper left.
If Time readout is in 12-hour (``12h'') mode, AM or PM will be displayed at the
upper right.  The Date readout, if enabled, is displayed at the lower left.
The Counter, if enabled, is displayed at the lower right.
The second hand, if enabled, is displayed along with the hour and minute hands.
Digit display, if enabled, places ``12'', ``3'', ``6'', and ``9'' around the
face of the clock in their respective positions. 

\subsubsection{Digital mode}
An imitation of an LCD, this mode shows a Clock comprised of digital ``segments''.
The Date readout, if enabled, is displayed at the bottom, center.
The Second readout, if in ``Text'' mode, is displayed at the top, center; if in
``Bar'' mode, is displayed as a progress bar at the top of the LCD; if in
``Invert'' mode, will invert the LCD left-to-right as the seconds pass (a
fully-inverted LCD means the entire minute has passed).
The Counter, if displayed, is shown at the upper left.
The Blinking Colon, if enabled, blinks the colon once every second.
12-hour mode, if enabled, will display the time in a 12-hour format.

\subsubsection{LCD mode}
Based on the Digital Mode, the LCD mode is another imitation of an LCD.
The settings available in this mode are exactly the same as Digital Mode, but
they are independent of Digital Mode. For example, you can have the Date
Readout enabled in Digital Mode and disabled in LCD Mode.

\subsubsection{Fullscreen}
A Fullscreen clock is displayed. Show Border, if enabled, will draw a small
box at every hour position (1 to 12 inclusive).
Invert Seconds, if enabled, will invert the LCD as the seconds pass.
Second Hand, if enabled, will draw a second hand among the hour and minute hands.

\subsubsection{Binary mode}
This mode shows a Binary clock.
The hour is displayed on the top line, the minute is displayed on the middle
line, and the seconds are on the last line.
Circle mode, if enabled, draws empty and full circles, instead of zeros and ones.
For help on reading binary, please visit:
\url{http://en.wikipedia.org/wiki/Binary_numeral_system}

\subsubsection{Plain mode}
This mode shows a ``plain'' clock in large text that takes up nearly the whole LCD.
}

\subsection{Dict}

\subsubsection{Prerequisites for using the plugin}
To use the plugin, firstly you need to have the dictionary files which contain
the words (index) and their description -- \fname{dict.index} and
\fname{dict.desc}, respectively -- on \fname{/.rockbox/rocks/apps/} folder.

The dictionary files can be created by yourself, or you can get them crafted
from the web. More information can be found at \wikilink{PluginDict}.

\subsubsection{Using the plugin}
Now that you already have the two necessary files in place, you can launch the
dict plugin (under Applications on the Browse plugins menu). The first thing
you will see is the text input screen.

Type part of a word (or the whole word) or anything the dict files have a
definition to and accept the text input. The plugin will search for matching
entries on the \fname{dict.index} file and display according description/meaning
contained in the \fname{dict.desc} file.

If no matches are found on the dictionary, a ``Not found'' message will be
displayed and the plugin will exit. You can do another search by relaunching
the plugin.


\input{plugins/disktidy.tex}

\input{plugins/keybox.tex}

\opt{HAVE_BACKLIGHT}{% $Id$ %
\subsection{Lamp}
Lamp is a simple plugin to use your player as a lamp (flashlight, torch).
You get an empty screen with maximum brightness.
\begin{btnmap}
  \opt{lcd_color}{
    \PluginLeft{} / \PluginRight
        &
    \opt{HAVEREMOTEKEYMAP}{
        & }
    Toggle between colours
        \\
  }
  \opt{backlight_brightness}{
    \nopt{scrollwheel}{\PluginUp{} / \PluginDown}
    \opt{scrollwheel}{\PluginScrollFwd{} / \PluginScrollBack}
        &
    \opt{HAVEREMOTEKEYMAP}{
        & }
    Change the brightness
        \\
  }
  \opt{HAVE_BUTTON_LIGHTS}{
    \PluginSelect{}
        &
    \opt{HAVEREMOTEKEYMAP}{
        & }
    Toggle the button light
        \\
  }
    \nopt{IPOD_4G_PAD,IPOD_3G_PAD}{\PluginCancel{} or \PluginExit}
    \opt{IPOD_4G_PAD,IPOD_3G_PAD}{\ButtonMenu}
    & Exit to menu\\
\end{btnmap}
}

% $Id$ %
\subsection{Lrcplayer}
% \screenshot{plugins/images/ss-lrcplayer}{Lrcplayer}{}
This plugin displays lyrics in \fname{.lrc} files (and some other formats)
synchronized with the song being played.

\subsubsection{Supported file types}
\begin{enumerate}
\item \fname{.lrc}
\item \fname{.lrc8}
\item \fname{.snc}
\item \fname{.txt}
\item id3v2 SYLT or USLT tags in mp3 files
\end{enumerate}

\fname{.lrc8} files are the same as \fname{.lrc} files except that they are UTF8
encoded. The Lyrics3 tag is not supported.

\subsubsection{Supported tags and formats for \fname{.lrc} files}
The following tags are supported:
\begin{verbatim}
[ti:title]
[ar:artist]
[offset:offset (msec)]
\end{verbatim}

Each line should resemble one of the following:
\begin{verbatim}
[time tag]line
[time tag]...[time tag]line
[time tag]<word time tag>word<word time tag>...<word time tag>
\end{verbatim}

The time tag must be in the form [mm:ss], [mm:ss.xx], or [mm:ss.xxx] where mm is
minutes, ss is seconds, xx is tenth of milliseconds, and xxx is milliseconds.
Any other tags and lines without time tags are ignored.

\subsubsection{Location of lyrics files}
The plugin checks the following directories for lyrics files.
\opt{swcodec}{If no lyrics file is found and the audio file is a \fname{.mp3},
  it also checks for SYLT and USLT tags in the id3v2 tags.}

\begin{enumerate}
\item The directory containing the audio file and its parent directories.
\item For each of the above directories, the plugin searches for a subdirectory
      named ``Lyrics''.
\item Finally, the plugin will search as above, but within a directory called
      ``/Lyrics''. The name of this directory can be customized, see below.
\end{enumerate}

If the audio file currently playing is \fname{/Music/Artist/Album/Title.mp3},
then the following files will be searched for, in this order. \fname{.ext} is one
of the supported extensions from the list above, and will be searched for in the
same order as in that list.

\begin{verbatim}
/Music/Artist/Album/Title.ext
/Music/Artist/Title.ext
/Music/Title.ext
/Title.ext
/Music/Artist/Album/Lyrics/Title.ext
/Music/Artist/Lyrics/Title.ext
/Music/Lyrics/Title.ext
/Lyrics/Title.ext
/Lyrics/Musics/Artist/Album/Title.ext
/Lyrics/Musics/Artist/Title.ext
/Lyrics/Musics/Title.ext
/Lyrics/Title.ext
\end{verbatim}

\subsubsection{Controls}
\begin{table}
  \begin{btnmap}{}{}
    \ActionWpsVolUp{} / \ActionWpsVolDown
    \opt{HAVEREMOTEKEYMAP}{& \ActionRCWpsVolUp{} / \ActionRCWpsVolDown}
    & Volume up/down.\\
    %
    \ActionWpsSkipPrev
    \opt{HAVEREMOTEKEYMAP}{& \ActionRCWpsSkipPrev}
    & Go to beginning of track, or if pressed while in the
      first seconds of a track, go to the previous track.\\
    %
    \ActionWpsSeekBack
    \opt{HAVEREMOTEKEYMAP}{& \ActionRCWpsSeekBack}
    & Rewind in track.\\
    %
    \ActionWpsSkipNext
    \opt{HAVEREMOTEKEYMAP}{& \ActionRCWpsSkipNext}
    & Go to the next track.\\
    %
    \ActionWpsSeekFwd
    \opt{HAVEREMOTEKEYMAP}{& \ActionRCWpsSeekFwd}
    & Fast forward in track.\\
    %
    \ActionWpsPlay
    \opt{HAVEREMOTEKEYMAP}{& \ActionRCWpsPlay}
    & Toggle play/pause.\\
    %
    \ActionWpsStop{} or \ActionWpsBrowse
    \opt{HAVEREMOTEKEYMAP}{& \ActionRCWpsStop{} or \ActionRCWpsBrowse}
    & Exit the plugin.\\
    %
    \ActionWpsContext
    \opt{HAVEREMOTEKEYMAP}{& \ActionRCWpsContext}
    & Enter timetag editor.\\
    %
    \ActionWpsMenu%
    \opt{HAVEREMOTEKEYMAP}{& \ActionRCWpsMenu}
    & Enter \setting{Lrcplayer Menu}.\\
    %
  \end{btnmap}
\end{table}

\subsubsection{Lrcplayer Menu}

\begin{description}
  \item[Theme settings.] Change theme related settings.
  \begin{description}
    \opt{lcd_bitmap}{%
      \item[Show Statusbar.] Show / hide the statusbar.
      \item[Display Title.] Show / hide the track title.
    }%
    \item[Display Time.] Show / hide the current time.
    \opt{lcd_color}{%
      \item[Inactive Colour.] Set the colour of the inactive part of the lyrics.
    }%
    \item[Backlight Force On.] Do not turn off the backlight while displaying
         the lyrics.
  \end{description}
  \opt{lcd_bitmap}{%
    \item[Display Settings.] Change how the lyrics are displayed.
    \begin{description}
      \item[Wrap.] Breaks lines at white space.
      \item[Wipe.] Wipes the text.
      \item[Alignment.] Align text to the left, centre, or right.
      \item[Activate Only Current Line.]
          Activate only the current line, or the current and previous lines.
    \end{description}
  }%
  \item[Lyrics Settings.] Change how the lyrics files are loaded.
  \begin{description}
    \item[Encoding.] Sets the codepage used in the plugin.
    \opt{swcodec}{%
      \item[Read ID3 tag.] Read lyrics from id3 tags in mp3 files.
    }%
    \item[Lrc Directory.] Set the directory where lyrics files are stored,
      must be a maximum of 63 bytes.
  \end{description}
  \item[Playback Control.] Show the playback control menu.
  \item[Time Offset.] Set an offset for the time tags for the lyrics currently in use.
  \item[Timetag Editor.] Enter the timetag editor.
  \item[Quit.] Exit the plugin.
\end{description}

\subsubsection{Editing the time tags}

The display time for each line can be changed with the timetag editor.
Selecting a line changes its time to the current position of the track.
To set a specific time or to adjust the time, press \ActionStdContext{} to
bring up a screen to adjust the time.
Changes will be saved automatically when the song is changed.
Editing words in lyrics is not supported.


\subsection{Main Menu Configuration}
{\label{ref:main_menu_config}}

This plugin helps you customizing the main menu (i.e. reorder or hide menu
items). It changes the appropriate configuration file as described in
\reference{ref:CustomisingTheMainMenu}.

When you start the plugin, the available main menu items will be displayed.
By pressing \ActionStdOk{} you open a menu with the following options:
\begin{description}
\item[Toggle Item] Hide the selected menu item or make it visible again
\item[Move Item Up] Swap the selected menu item with the previous one
\item[Move Item down] Swap the selected menu item with the next one
\item[Load Default Configuration] Discards all customization
\item[Exit] Save your changes to the configuration file and exit the plugin
\end{description}
You can leave the plugin without saving by pressing \ActionStdCancel{}.


\input{plugins/md5sum.tex}

\subsection{Metronome}

This plugin can be used as a metronome to keep time during music
practice. It supports two modes of operation, depending on it being
started from the plugin menu or as viewer for tempomap (\verb:.tempo:)
files.

The sound is a piercing square wave that can be heard well also
through loud music from a band.
In addition, the display also indicates the beats while playing
so that you can discreetly place the device
in your sight for checking the tempo instead of wearing
headphones at a concert.

\subsubsection{Simple Interactive Mode}

This is the mode of operation that is active when starting the
plugin directly from the menu. It offers a uniform metronome sound at
a constant tempo.
You can adjust the tempo through the interface or by tapping it out
on the appropriate button.

\begin{btnmap}
    \PluginExit
       \opt{HAVEREMOTEKEYMAP}{& }
        & Exit plugin \\

    \PluginCancel
       \opt{HAVEREMOTEKEYMAP}{& \PluginRCCancel}
        & Stop \\
        
    \PluginSelectRepeat
       \opt{HAVEREMOTEKEYMAP}{& \PluginRCCancel}
       & Start \\

    \PluginSelect
       \opt{HAVEREMOTEKEYMAP}{& \PluginRCSelect}
        & Tap tempo \\

    \PluginLeft{} / \PluginRight
       \opt{HAVEREMOTEKEYMAP}{& \PluginRCLeft{} / \PluginRCRight}
        & Adjust tempo \\

    \opt{scrollwheel}{\PluginScrollFwd{} / \PluginScrollBack}
    \nopt{scrollwheel}{\PluginUp{} / \PluginDown}
        \opt{HAVEREMOTEKEYMAP}{& \PluginRCUp{} / \PluginRCDown}
        & Adjust volume \\

    \opt{IRIVER_H100_PAD,IRIVER_H300_PAD,SANSA_E200_PAD}{
      \ButtonRec
        \opt{HAVEREMOTEKEYMAP}{& }
        & Sync tap \\}
\end{btnmap}


\subsubsection{Programmed Track Mode}

When starting the plugin as a viewer for tempomap files
(ending in \verb:.tempo:), it starts in the track mode that offers
playback of a preprogrammed metronome track consisting out of
multiple parts, each with possibly different properties.

In contrast to the simple mode, there exists the notion of
meter and bars, along with emphasis on certain beats.
Parts can have these properties:

\begin{itemize}
    \item finite or infinite duration in bars (navigation only jumps
        to the beginning of infinite parts),
    \item differing meters (4/4, 3/4, 6/8, etc., default 4/4),
    \item differing tempo (always in quarter beats per minute,
        default 120) with
    \begin{itemize}
        \item one tempo per bar or even one tempo per beat, or
        \item smooth tempo changes with configurable acceleration, and
    \end{itemize}
    \item custom beat patterns (tick/tock/silence on each beat),
        default being emphasis (tick) on first beat, normal sound
        (tock) on others.
\end{itemize}

\paragraph{The button mapping}
is different to enable navigation in the programmed track.
\begin{btnmap}
    \PluginExit
       \opt{HAVEREMOTEKEYMAP}{& }
        & Exit plugin \\

    \PluginCancel
       \opt{HAVEREMOTEKEYMAP}{& \PluginRCCancel}
        & Stop (stay at position) \\
        
    \PluginSelect
       \opt{HAVEREMOTEKEYMAP}{& \PluginRCSelect}
        & Start from / Stop at current position \\

    \PluginLeft{} / \PluginRight
       \opt{HAVEREMOTEKEYMAP}{& \PluginRCLeft{} / \PluginRCRight}
        & Seek in track \\

    \opt{scrollwheel}{\PluginScrollFwd{} / \PluginScrollBack}
    \nopt{scrollwheel}{\PluginUp{} / \PluginDown}
        \opt{HAVEREMOTEKEYMAP}{& \PluginRCUp{} / \PluginRCDown}
        & Adjust volume \\

    \opt{IRIVER_H100_PAD,IRIVER_H300_PAD,SANSA_E200_PAD}{
      \ButtonRec
        \opt{HAVEREMOTEKEYMAP}{& }
        & Sync tap \\}
\end{btnmap}

\paragraph{Navigation}
The display indicates the part properties and position in track as such:
\begin{verbatim}
    Metronome Track
    ---------------
       "Interlude"
    3/4@120 V-25
    P2/13: B1/5+2
\end{verbatim}
In this example, the part label is ``Interlude'', the meter is 3/4 and
the tempo 120 quarter beats per minute (bpm). The volume setting is at -25
and this is the second part of a track with 13 total. In that part,
the position is at the second beat of the first bar of five.

\paragraph{The syntax of programmed tracks}
in tempomap files follows the format defined by
\url{http://das.nasophon.de/klick/}.
Actually, the goal is to keep compatibility between klick and this
Rockbox metronome.
The parts of a track are specified one line each in this scheme
(pieces in [] optional):
\begin{verbatim}
[name:] bars [meter ]tempo[-tempo2[*accel|/accel] [pattern] [volume]
\end{verbatim}
The bar count and tempo always have to be specified, the rest is optional.

One example is
\begin{verbatim}
part I: 12 3/4 133
\end{verbatim}
for a part named ``part I'' , 12 bars long, in 3/4 meter with
a tempo of 133 quarter beats per minute.
Tempo changes are indicated by specifying a tempo range and the
acceleration in one of these ways:
\begin{verbatim}
0 4/4 90-150*0.25
0 4/4 150-90/4
16 4/4 100-200
\end{verbatim}
The first one goes from 90 to 150 bpm in an endless part with 0.25 bpm
increase per bar. The second one goes down from 150 to 90 with
4 bars per bpm change, which is the same acceleration as in the first line.
The last one is a part of 16 bars length that changes tempo from 100 to 200
smoothly during its whole lifetime (6.25 bpm/bar). For details on how the
acceleration works, see
\url{http://thomas.orgis.org/science/metronome-tempomath/tempomath.html}.

It is also possible to provide a tempo for each individual beat in a part
by separating values with a comma (no spaces),
\begin{verbatim}
varibeat: 3 4/4 135,90,78,100,120,120,99,100,43,94,120,133
\end{verbatim}
where the beat duration is first according to 135 bpm, then 90 bpm,
and so forth. You are required to provide a value for each beat
in all bars of the part.

You can provide a pattern that controls how the beats are played:
\begin{center}
\begin{tabular}{c|l}
    Symbol & Meaning \\
    \hline
    X & emphasized beat (Tick) \\
    x & normal beat (Tock) \\
    . & silent beat
\end{tabular}
\end{center}

Some examples:
\begin{verbatim}
default: 0 4/4 120 Xxxx
rockon2: 0 4/4 120 xXxX
  solea: 0 12/4 180 xxXxxXxXxXxX
shuffle: 0 12/12 120 x.xX.xx.xX..
  funky: 0 16/16 120 x.x.X..X.Xx.X..X
\end{verbatim}
The 12/12 for the shuffle create 1/4 triplets. Just do a bit of math;-)
This is still a metronome, not a drum machine, but it can act like a basic
one, helping you to figure out a certain rhythm within the meter.

The UI is developed so that it fits into the display of a Sansa Clip+ and
that is the hardware device it is tested on. It seems to work reasonably
on some other models in the simulator.

At last, a more complete tempomap file:
\begin{verbatim}
# An example track exercising the programmable Rockbox metronome
# or also http://das.nasophon.de/klick/.
     lead-in: 1 4/4 120 XXXX 0.5 # 4 emphasized but less loud ticks
       intro: 4 4/4 120          # standard beat
tearing down: 4     120-90       # changing tempo from 120 to 90
       break: 2 1/4 90           # 2 1/4 bars at 90
     rolling: 2 6/8 90           # 2 6/8 at same tempo (quarters!)
    rumbling: 4 3/4 90 X.x       # 3/4, first (tick) and last (tock)
     ramp-up: 8 2/4 90-150       # speeding up to 150 bpm again
        flow: 4     150          # steady 4/4 at 150 bpm
       death: 8     150-60       # going down to 60
       final: 1 1/1 60           # one last hit
\end{verbatim}


% $Id$ %
\subsection{One-Time Password Client}
This plugin provides the ability to generate one-time passwords (OTPs)
for authentication purposes. It implements an HMAC-based One-Time
Password Algorithm (RFC 4226), and on targets which support it, a
Time-based One-Time Password Algorithm (RFC 6238).

\subsubsection{Adding Accounts}
The plugin supports two methods of adding accounts: URI import, and
manual entry.

\opt{rtc}{ It is important to note that for TOTP (time-based) accounts
  to work properly, the clock on your device MUST be accurate to no
  less than 30 seconds from the time on the authentication server, and
  the correct time zone must be configured in the plugin.  See
  \reference{ref:Timeanddateactual} for more information.  }

\subsubsection{URI Import}
This method of adding an account reads a list of URIs from a file. It
expects each URI to be on a line by itself in the following format:

\begin{verbatim}
otpauth://[hotp OR totp]/[account name]?secret=[Base32 secret][&counter=X][&period=X][&digits=X]
\end{verbatim}

An example is shown below, provisioning a TOTP key for an account called ``bob'':

\begin{verbatim}
otpauth://totp/bob?secret=JBSWY3DPEHPK3PXP
\end{verbatim}

Any other URI options are not supported and will be ignored.

Most services will provide a scannable QR code that encodes a OTP
URI. In order to use those, first scan the QR code separately and save
the URI to a file on your device. If necessary, rewrite the URI so it
is in the format shown above. For example, GitHub's URI has a slash
after the provider. In order for this URI to be properly parsed, you
must rewrite the account name so that it does not contain a slash.

\subsubsection{Manual Import}
If direct URI import is not possible, the plugin supports the manual
entry of data associated with an account. After you select the
``Manual Entry'' option, it will prompt you for an account name. You
may type anything you wish, but it should be memorable. It will then
prompt you for the Base32-encoded secret. Most services will provide
this to you directly, but some may only provide you with a QR code. In
these cases, you must scan the QR code separately, and then enter the
string following the ``secret='' parameter on your Rockbox device
manually.

On devices with a real-time clock, \opt{rtc}{like yours,} the plugin
will ask whether the account is a time-based account
(TOTP). \opt{rtc}{If you answer ``yes'' to this question, it will ask
  for further information regarding the account. Usually it is safe to
  accept the defaults here. } However, if your device lacks a
real-time clock, the plugin's functionality will be restricted to
HMAC-based (HOTP) accounts only. If this is the case, the plugin will
prompt you for information regarding the HOTP setup.

\opt{rtc} {
  \subsection{Advanced Settings}
  \subsubsection{Time Zone Configuration}
  In order for TOTP accounts to work properly, the plugin must be able
  to determine the current UTC time. This means that, first, your
  device's clock must be synchronized with UTC time, and second, that
  the plugin knows what time zone the clock is using. The plugin will
  prompt you on its first run for this piece of information. However,
  should this setting need changing at a later time, possibly due to
  Daylight Saving Time adjustment, it is located under the
  ``Advanced'' submenu. NOTE: in the UI simulator, use the ``UTC''
  setting no matter what the clock may read. }


\subsection{Periodic Table}

The periodic table plugin allows easy browsing and viewing of details of elements, giving a detailed output for each selection. Navigate the table using the directional keys, pressing back or exit exits the plugin.

\begin{btnmap}
  \PluginUp, \PluginDown, \PluginLeft, \PluginRight
  \opt{HAVEREMOTEKEYMAP}{& \PluginRCUp, \PluginRCDown, \PluginRCLeft, \PluginRCRight}
  & Move cursor\\

  \PluginCancel
  \opt{HAVEREMOTEKEYMAP}{& \PluginRCCancel}
  & Quit\\
\end{btnmap}


\opt{recording_mic}{\input{plugins/pitch_detector.tex}}

\input{plugins/random_folder_advance_config.tex}

\input{plugins/resistor.tex}

\opt{lcd_color}{% $Id$ %
\subsection{Rockpaint}
\screenshot{plugins/images/ss-rockpaint}{Rockpaint}%
{img:Rockpaint}
Rockpaint is a bitmap (\fname{.bmp}) editor for Rockbox. It can open any \fname{.bmp} file
whose dimensions are the same size as your device's screen or smaller; it can
also create empty bitmaps for you to work with.\\

\subsubsection{Opening A File}
To open a file, you may use either the context menu option ``Open With'' in the
File Browser, or you may enter Rockpaint first using the Plugins menu and open
a file from there. To perform the latter, simply press Rockpaint's Menu button
or move the cursor beyond the bottom of the screen; then move the cursor onto
``Menu'' and select it. Finally, select ``Load'' and navigate to the image you
wish to open.\\

\subsubsection{Tools}
Rockpaint offers several tools to aid you in editing; you can view them by
either pressing Rockpaint's ``Menu'' key or by attempting to move the cursor
beyond the bottom of the screen. From top to bottom and left to right, and
by section, they are as follows:
\begin{description}
    \item[Colour Picker]
        The top left tool shows your colours that are at the ready. To
        swap them, ``click'' on the background colour. To edit the foreground colour,
        click on it.
    \item[Preset Palette]
        Several preset colours are available. Clicking on one changes
        the foreground of the Colour Picker to the selected colour.
    \item[Pencil]
        Draws as you move the cursor. You can change the brush size with the
        Menu option ``Brush Size''. Use the Select key to toggle the tool while editing
        the image.
    \item[Selection tool]
        Allows you to select a rectangular region; once you do, you
        will be shown a menu of options (including ``cancel'' if you make a mistake).
    \item[Line tool]
        Draws a straight line.
    \item[Curve tool]
        Allows you to draw a line and curve it.
    \item[Rectangle tool]
        Draws an unfilled rectangle.
    \item[Circle tool]
        Draws an unfilled circle.
    \item[Gradient fill]
        To use this tool, click at the starting and ending points.
        Starting with the background and going to the foreground colour, Rockpaint
        will fill the region with a gradual colour change.
    \item[Bucket fill]
        Fills an same-colour or empty region with a colour.
    \item[Dropper]
        Click on a colour in the image to change the foreground colour to it.
    \item[Eraser]
        The opposite of the pencil; it changes painted pixels to white.
    \item[Text tool]
        ``Draws'' text on the image.
    \item[Filled rectangle]
        Same as the Rectangle tool, but fills it with colour.
    \item[Filled circle]
        Same as the Circle tool, but fills it with colour.
    \item[Curved Gradient Fill]
        Same as Gradient fill, but you must draw two lines.
        Rockpaint will draw a curved, gradual change of colour in the region.
    \item[Menu]
        This opens the Main Menu. You can also press the Menu key to open it.
\end{description}

\subsubsection{Main Menu}
The main menu consists of the following:
\begin{description}
    \item[Resume]
        Closes the Main Menu.
    \item[New]
        Creates a new canvas and discards the current file. BE CAREFUL.
        You will lose any unsaved changes in the file that is currently open.
    \item[Load]
        Loads a bitmap file. Simply navigate to the file as you
        would in the file browser.
    \item[Save]
        Saves the current file. If it has not been saved before,
        you will be given a chance to name it and choose the saving location.
    \item[Set Width]
        Allows you to change the width of the image.
        Border to indicate the width will be shown but it doesn't affect drawing.
    \item[Set Height]
        Allows you to change the height of the image.
        Border to indicate the height will be shown but it doesn't affect drawing.
    \item[Brush speed]
        Changes the speed at which the selection cursor
        moves when you hold down a movement button.
    \item[Brush size]
        Allows you to adjust the drawing size of the pencil tool.
    \item[Choose colour]
        Allows you to manually edit the foreground colour.
        You can edit the RBG and/or the HSV values.
    \item[Grid size]
        Allows you to show or hide a grid over the canvas,
        and to specify its size.
    \item[Exit]
        Exits Rockpaint.
\end{description}

\warn{BE CAREFUL. Rockpaint will NOT prompt you to save
if you select Exit, so any unsaved changes will be lost.}

\begin{btnmap}
    \nopt{IPOD_4G_PAD,IPOD_3G_PAD,IPOD_1G2G_PAD}{
        \nopt{IRIVER_H300_PAD,GIGABEAT_S_PAD,SANSA_FUZE_PAD,PBELL_VIBE500_PAD%
             ,SAMSUNG_YH92X_PAD,SAMSUNG_YH820_PAD}
            {\ButtonPower}
        \opt{IRIVER_H300_PAD}{\ButtonOff}
        \opt{GIGABEAT_S_PAD}{\ButtonBack}
        \opt{SANSA_FUZE_PAD}{Long \ButtonHome}
        \opt{PBELL_VIBE500_PAD,SAMSUNG_YH92X_PAD,SAMSUNG_YH820_PAD}{\ButtonRec}
          \opt{HAVEREMOTEKEYMAP}{& }
        & Quits Rockpaint immediately.\\
    }

    \nopt{touchscreen}{\ButtonLeft{} / \ButtonRight{} /}
    \opt{touchscreen}{\TouchMidLeft{} / \TouchMidRight{} /}
    \nopt{IPOD_4G_PAD,IPOD_3G_PAD,IPOD_1G2G_PAD,IRIVER_H10_PAD,touchscreen}
        {\ButtonUp{} / \ButtonDown}
    \opt{IPOD_4G_PAD,IPOD_3G_PAD,IPOD_1G2G_PAD}{\ButtonMenu{} / \ButtonPlay}
    \opt{IRIVER_H10_PAD}{\ButtonScrollUp{} / \ButtonScrollDown}
    \opt{touchscreen}{\TouchTopMiddle{} / \TouchBottomMiddle}
      \opt{HAVEREMOTEKEYMAP}{& }
    & Moves the cursor around.\\

    \opt{IRIVER_H300_PAD}{\ButtonOn}%
    \opt{IPOD_4G_PAD,IPOD_3G_PAD,IPOD_1G2G_PAD}{\ButtonSelect+\ButtonMenu}%
    \opt{SANSA_C200_PAD,SANSA_E200_PAD}{\ButtonSelect+\ButtonPower}%
    \opt{SANSA_CLIP_PAD}{\ButtonHome}%
    \opt{SANSA_FUZE_PAD}{\ButtonSelect+\ButtonDown}%
    \opt{IAUDIO_X5_PAD}{\ButtonPlay}%
    \opt{GIGABEAT_PAD,GIGABEAT_S_PAD,COWON_D2_PAD,PBELL_VIBE500_PAD}{\ButtonMenu}%
    \opt{IRIVER_H10_PAD}{\ButtonPlay}%
    \opt{SAMSUNG_YH92X_PAD,SAMSUNG_YH820_PAD}{\ButtonFF}%
      \opt{HAVEREMOTEKEYMAP}{& }
    & Displays the Main Menu.\\

    \opt{IRIVER_H300_PAD,IAUDIO_X5_PAD,SANSA_C200_PAD,SANSA_E200_PAD}{\ButtonRec}
    \opt{SANSA_CLIP_PAD}{\ButtonVolUp}
    \opt{SANSA_FUZE_PAD}{\ButtonSelect+\ButtonLeft}
    \opt{IPOD_4G_PAD,IPOD_3G_PAD,IPOD_1G2G_PAD}{\ButtonMenu+\ButtonLeft}
    \opt{GIGABEAT_PAD}{\ButtonA}
    \opt{IRIVER_H10_PAD,SAMSUNG_YH92X_PAD,SAMSUNG_YH820_PAD}{\ButtonRew}
    \opt{GIGABEAT_S_PAD}{\ButtonPlay}
    \opt{SANSA_FUZEPLUS_PAD}{\ButtonBottomLeft{} or \ButtonBottomRight}
    \opt{touchscreen}{\TouchBottomLeft}
    \opt{PBELL_VIBE500_PAD}{\ButtonOK}
      \opt{HAVEREMOTEKEYMAP}{& }
    & Displays the toolbar.\\

    \nopt{IRIVER_H10_PAD,touchscreen,PBELL_VIBE500_PAD,SANSA_FUZEPLUS_PAD%
         ,SAMSUNG_YH92X_PAD,SAMSUNG_YH820_PAD}{\ButtonSelect}%
    \opt{IRIVER_H10_PAD}{\ButtonFF}%
    \opt{touchscreen}{\TouchCenter}
    \opt{PBELL_VIBE500_PAD,SAMSUNG_YH92X_PAD,SAMSUNG_YH820_PAD}{\ButtonPlay}
    \opt{SANSA_FUZEPLUS_PAD}{\ButtonVolUp}
      \opt{HAVEREMOTEKEYMAP}{& }
    & Toggles the brush and selects objects.\\

\end{btnmap}
}

\subsection{Stats}
\screenshot{plugins/images/ss-stats}{The stats-plugin}{}

The stats plugin counts the directories and files
(the total number as well as the number
of audio, playlist, image and video files)
on your \dap{}.
Press \ActionStdCancel{} to exit the plugin.


\subsection{Stopwatch}
\screenshot{plugins/images/ss-stopwatch}{Stopwatch}{fig:stopwatch}

A simple stopwatch program with support for saving times.

\begin{btnmap}
    \opt{PLAYER_PAD}{\ButtonMenu}
    \opt{RECORDER_PAD,ONDIO_PAD,IRIVER_H100_PAD,IRIVER_H300_PAD}{\ButtonOff}
    \opt{IPOD_4G_PAD,IPOD_3G_PAD}{\ButtonMenu}
    \opt{IAUDIO_X5_PAD,IRIVER_H10_PAD,SANSA_E200_PAD,SANSA_C200_PAD,GIGABEAT_PAD%
        ,MROBE100_PAD,SANSA_CLIP_PAD}{\ButtonPower}
    \opt{SANSA_FUZE_PAD}{Long \ButtonHome}
    \opt{GIGABEAT_S_PAD}{\ButtonBack}
    \opt{PBELL_VIBE500_PAD}{\ButtonRec}
    \opt{SAMSUNG_YH92X_PAD,SAMSUNG_YH820_PAD}{\ButtonRew}
    \opt{MPIO_HD200_PAD}{\ButtonRec + \ButtonPlay}
    \opt{MPIO_HD300_PAD}{Long \ButtonMenu}
       \opt{HAVEREMOTEKEYMAP}{& 
          \opt{IRIVER_RC_H100_PAD}{\ButtonRCStop}
        }
    & Quit Plugin \\
    %
    \opt{PLAYER_PAD,RECORDER_PAD,IAUDIO_X5_PAD,IRIVER_H10_PAD,GIGABEAT_S_PAD%
        ,PBELL_VIBE500_PAD,MPIO_HD200_PAD,MPIO_HD300_PAD,SAMSUNG_YH92X_PAD%
        ,SAMSUNG_YH820_PAD}{\ButtonPlay}
    \opt{ONDIO_PAD,SANSA_E200_PAD,SANSA_FUZE_PAD,SANSA_C200_PAD%
        ,SANSA_CLIP_PAD}{\ButtonRight}
    \opt{IRIVER_H100_PAD,IRIVER_H300_PAD,GIGABEAT_PAD,MROBE100_PAD,IPOD_4G_PAD%
        ,IPOD_3G_PAD}{\ButtonSelect}
  \opt{HAVEREMOTEKEYMAP}{& }
    & Start / stop \\
    %
    \opt{PLAYER_PAD}{\ButtonStop}
    \opt{RECORDER_PAD,ONDIO_PAD,IPOD_4G_PAD,IPOD_3G_PAD,SANSA_E200_PAD%
        ,SANSA_FUZE_PAD,SANSA_C200_PAD,SANSA_CLIP_PAD,SAMSUNG_YH92X_PAD%
        ,SAMSUNG_YH820_PAD}{\ButtonLeft}
    \opt{IRIVER_H100_PAD,IRIVER_H300_PAD}{\ButtonDown}
    \opt{IRIVER_H10_PAD,MPIO_HD200_PAD,MPIO_HD300_PAD}{\ButtonRew}
    \opt{IAUDIO_X5_PAD}{\ButtonRec}
    \opt{GIGABEAT_PAD}{\ButtonA}
    \opt{GIGABEAT_S_PAD}{\ButtonMenu}
    \opt{MROBE100_PAD}{\ButtonDisplay}
    \opt{PBELL_VIBE500_PAD}{\ButtonOK}
  \opt{HAVEREMOTEKEYMAP}{& }
    & Reset timer (only when timer is stopped)\\
    %
    \opt{PLAYER_PAD,RECORDER_PAD,IRIVER_H100_PAD,IRIVER_H300_PAD}{\ButtonOn}
    \opt{ONDIO_PAD,GIGABEAT_PAD,MROBE100_PAD,PBELL_VIBE500_PAD}{\ButtonMenu}
    \opt{IPOD_4G_PAD,IPOD_3G_PAD,SAMSUNG_YH92X_PAD,SAMSUNG_YH820_PAD}{\ButtonRight}
    \opt{IRIVER_H10_PAD,MPIO_HD200_PAD,MPIO_HD300_PAD}{\ButtonFF}
    \opt{IAUDIO_X5_PAD,SANSA_E200_PAD,SANSA_FUZE_PAD,SANSA_C200_PAD,GIGABEAT_S_PAD%
        ,SANSA_CLIP_PAD}{\ButtonSelect}
  \opt{HAVEREMOTEKEYMAP}{& }
    & Take lap time \\
    %
    \opt{PLAYER_PAD,IRIVER_H100_PAD,IRIVER_H300_PAD}{\ButtonLeft{} / \ButtonRight}
    \opt{RECORDER_PAD,ONDIO_PAD,IAUDIO_X5_PAD,SANSA_E200_PAD,SANSA_FUZE_PAD%
        ,SANSA_C200_PAD,GIGABEAT_PAD,GIGABEAT_S_PAD,MROBE100_PAD,PBELL_VIBE500_PAD%
        ,SANSA_CLIP_PAD,SAMSUNG_YH92X_PAD,SAMSUNG_YH820_PAD}{\ButtonUp{} / \ButtonDown}
    \opt{IPOD_4G_PAD,IPOD_3G_PAD}{\ButtonScrollFwd{} / \ButtonScrollBack}
    \opt{IRIVER_H10_PAD,MPIO_HD300_PAD}{\ButtonScrollUp{} / \ButtonScrollDown}
    \opt{MPIO_HD200_PAD}{\ButtonVolUp / \ButtonVolDown}
  \opt{HAVEREMOTEKEYMAP}{& }
    & Scroll through lap times \\
\end{btnmap}


\subsection{\label{sec:text_editor}Text Editor}
This plugin allows you to view and edit simple text documents on your DAP.
You can view files by using \setting{Open with} from the
\setting{Context Menu} (see \reference{ref:Contextmenu}).

\subsubsection{Usage}
If you start the Text Editor from the plugin browser you will be greeted with
a blank screen. When started from the \setting{Open with} menu item your file 
should be shown on the screen. You can now edit the file.
The Text Editor is line based. This means you can edit one line at a time using
the \setting{Virtual Keyboard} (see \reference{sec:virtual_keyboard}).

\begin{itemize}
  \item Move the selection bar to the line you want to edit.
  \item Edit the highlighted text line or insert a new one using the Item Menu.
  \item When finished editing exit the Text Editor. You'll be shown a list of
        save options. 
\end{itemize}
\note{When you have not changed the file the Text Editor will quit immediately.}

\begin{btnmap}
    \ActionStdOk 
      \opt{HAVEREMOTEKEYMAP}{& \ActionRCStdOk}
    & Edit Line / Select Character\\

    \ActionStdCancel 
      \opt{HAVEREMOTEKEYMAP}{& \ActionRCStdCancel} 
    & Exit / Abort Editing\\

    \opt{SAMSUNG_YH92X_PAD,SAMSUNG_YH820_PAD}{\ActionStdContext}
    \nopt{SAMSUNG_YH92X_PAD,SAMSUNG_YH820_PAD}{\ActionStdMenu}
      \opt{HAVEREMOTEKEYMAP}{& \ActionRCStdMenu} 
    & Show Item Menu\\

    \nopt{SAMSUNG_YH92X_PAD,SAMSUNG_YH820_PAD}{
        \ActionStdContext
          \opt{HAVEREMOTEKEYMAP}{& \ActionRCStdContext}
        & Delete Line\\
    }
\end{btnmap}



% $Id$ %
\opt{hotkey}{
    \section{\label{ref:Hotkeys}Hotkeys}
    Hotkeys are shortcut keys for use in the \nopt{touchscreen}{\setting{File Browser},
    \setting{Database}, \setting{Playlist Viewer}, and }\setting{WPS} screen.  To use one, press
    \nopt{touchscreen}{\ActionTreeHotkey{} within the \setting{File Browser},
    \setting{Database}, or \setting{Playlist Viewer}, or}
    \ActionWpsHotkey{} within the \setting{WPS}
    screen.\nopt{touchscreen}{ The assigned function will launch with reference
    to the current file or directory, if applicable.  Each screen has its own
    assignment.} If there is no assignment for a given screen,
    the hotkey is ignored.

    The default assignment for the \nopt{touchscreen}{File Browser hotkey is
    \setting{Off}, while the default for the }WPS hotkey is
    \setting{View Playlist}.

    The hotkey assignments are changed in the Hotkey menu (see
    \reference{ref:HotkeySettings}) under \setting{General Settings}.
}


