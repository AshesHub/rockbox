
\subsubsection{Bootloader installation from Windows}

Manual installation under Windows is not supported. Please use \caps{Rockbox
Utility} for bootloader and Rockbox installation.

\subsubsection{Bootloader installation from Mac OS X}

\warn{Please make sure that your iPod is formatted using FAT32
      (a.k.a. WinPod) before attempting to install the bootloader!
      Installation will not work on HFS/HFS+ iPods (a.k.a. MacPods).}

\begin{enumerate}

\item Download the bootloader in .ipod format from
\url{https://files.freemyipod.org/~user890104/bootloader-ipodclassic-v1_0/bootloader-ipod6g.ipod}

\item Download mks5lboot for your operating system from
\url{https://files.freemyipod.org/~user890104/bootloader-ipodclassic.html\#download_stable}

\item You need to have package libusb installed using Homebrew (brew install
libusb) or MacPorts (port install libusb) in order to run mks5lboot.

\item Start mks5lboot from a terminal with the following command-line:
mks5lboot --dfuscan -l It should scan for DFU devices every second.

\item It is important to stop iTunes (dock icon -> Quit) and iTunesHelper
(using Activity monitor, find the process and select Quit or Force
quit if it keeps restarting) BEFORE continuing to the next
step. Otherwise iTunes will put your iPod in wrong mode, and you will
not be able to proceed with the installation.

\item Put your iPod in DFU mode.

\item When the device is detected, press CTRL+C to terminate the scan
process, and proceed to the next step.

\item Start mks5lboot from a terminal with the following command-line:
mks5lboot --bl-inst path/to/bootloader-ipod6g.ipod, providing the
correct path to bootloader-ipod6g.ipod that you downloaded earlier.

\item When the installation is complete, you should have Rockbox up and
running!

\end{enumerate}

\subsubsection{Bootloader installation from Linux}

\begin{enumerate}

\item Connect your iPod in normal mode (iTunes/file transfer).
\item Download \wikilink{RockboxUtility} for your operating system.
\item When \caps{Rockbox Utility} opens, select the checkbox named Show disabled targets, and point the installer to your iPod's mount point.
\item On the installation screen make sure that Rockbox is selected and Bootloader is not selected. You can install themes or the game files if you want.
\item Start the Rockbox installation.
\item Download the \href{https://files.freemyipod.org/~user890104/bootloader-ipodclassic-v1_0/bootloader-ipod6g.ipod}{bootloader in .ipod format}.
\item Download \href{https://files.freemyipod.org/~user890104/bootloader-ipodclassic.html#download_stable}{mks5lboot} for your operating system. Alternatively, you can \href{https://files.freemyipod.org/~user890104/bootloader-ipodclassic.html#build_mks5lboot}{build it} from the source code.
\item You need to have package libusb-1.0.0 installed in order to run mks5lboot.
\item To make sure the installer is marked as executable, start the following command in the terminal: chmod +x mks5lboot.
\item Start mks5lboot from a terminal with the following command-line: ./mks5lboot --dfuscan -l. It should scan for DFU devices every second.
\item Put your iPod in \href{https://files.freemyipod.org/~user890104/bootloader-ipodclassic.html#dfu}{DFU mode}.
\item When the device is detected, press CTRL+C to terminate the scan process, and proceed to the next step.
\item Start mks5lboot from a terminal with the following command-line: ./mks5lboot --bl-inst path/to/bootloader-ipod6g.ipod, providing the correct path to bootloader-ipod6g.ipod that you downloaded earlier.
\item When the installation is complete, you should have Rockbox up and running!

\end{enumerate}
