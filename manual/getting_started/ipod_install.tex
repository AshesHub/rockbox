\subsubsection{Bootloader installation from Windows}

\begin{enumerate}

\item Download ipodpatcher.exe from
\download{bootloader/ipod/ipodpatcher/win32/ipodpatcher.exe}
and run it whilst logged in with an administrator account.

\item If all has gone well, you should see some information displayed about
your \dap{} and a message asking you if you wish to install the Rockbox
bootloader. Press i followed by ENTER, and ipodpatcher will now
install the bootloader. After a short time you should see the message
``[INFO] Bootloader installed successfully.'' Press ENTER again to exit
ipodpatcher.

\item \note{If ipodpatcher fails to install the bootloader for you, please
be certain that you do indeed have a supported iPod model and are logged in
as an administrator. If you do, run
ipodpatcher once more and try again. If you don't, then do not attempt to
install again.}

\end{enumerate}

\subsubsection{Bootloader installation from Mac OS X}

\begin{enumerate}

\item Attach your \dap{} to your Mac and wait for its icon to appear in
Finder.

\item Download and open ipodpatcher.dmg from
\download{bootloader/ipod/ipodpatcher/macosx/ipodpatcher.dmg}
and then double-click on the ipodpatcher icon inside. You can also
drag the ipodpatcher icon to a location on your hard drive and launch
it from the Terminal.

\item If all has gone well, you should see some
information displayed about your \dap{} and a message asking you if you
wish to install the Rockbox bootloader. Press i followed by ENTER, and
ipodpatcher will now install the bootloader. After a short time you
should see the message ``[INFO] Bootloader installed successfully.'' Press
ENTER again to exit ipodpatcher and then quit the Terminal application.

\item \note{If ipodpatcher fails to install the bootloader for you, please
be certain that you do indeed have a supported iPod model. If you do, run
ipodpatcher once more and try again. If you don't, then do not attempt to
install again.}

\item Your \dap{} will now automatically reconnect itself to your Mac.
Wait for it to connect, and then eject and unplug it in the normal way.
\note{You should unplug your ipod immediately after ejecting it to
prevent Rockbox immediately rebooting your \dap{} into disk mode when it
detects that your \dap{} is attached to a computer. }

\end{enumerate}

\subsubsection{Bootloader installation from Linux}

\begin{enumerate}

\item Download ipodpatcher from
\download{bootloader/ipod/ipodpatcher/linux32x86/ipodpatcher} (32-bit x86
binary) or \download{bootloader/ipod/ipodpatcher/linux64amd64/ipodpatcher}
(64-bit amd64 binary). You can save this anywhere you wish, but the next
steps will assume you have saved it in your home directory.

\item Attach your \dap{} to your computer.

\item Open up a terminal window and type the following commands:

\begin{code}
    cd $HOME
    chmod +x ipodpatcher
    ./ipodpatcher
\end{code}

\note{You need to be the root user in order for ipodpatcher to have
sufficient permission to perform raw disk access to your \dap{}.}

\item If all has gone well, you should see some information displayed about
your \dap{} and a message asking you if you wish to install the Rockbox
bootloader. Press i followed by ENTER, and ipodpatcher will now install the
bootloader. After a short time you should see the message ``[INFO] Bootloader
installed successfully.'' Press ENTER again to exit ipodpatcher.

\end{enumerate}
