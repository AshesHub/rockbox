\chapter{Installation}\label{sec:installation}

Installing Rockbox is generally a quick and easy procedure. However
before beginning there are a few important things to know.

\section{Before Starting}

\opt{e200}{\fixme{NOTE: These instructions will not work on the
``Rhapsody'' version of the E200 series (also known as E200R). Please
follow the instructions at
\url{http://www.rockbox.org/wiki/SansaE200RInstallation}.}}

\opt{ipodnano,ipodnano2g,ipodvideo,ipod6g,e200,c200,c200v2,e200v2,clipv1,clipv2,cowond2,fuze,fuzev2}{
\begin{description}
\item[Supported hardware versions.]
    \opt{ipodnano,ipodnano2g}{
    The \playertype{} is available in multiple versions, not
    all of which run Rockbox.  Rockbox presently runs only on
    the first and second generation Ipod Nano. Rockbox does
    \emph{not} run on the third, fourth or fifth generation Ipod Nano.
    For information on identifying which Ipod you own, see this page on
    Apple's website: \url{http://www.info.apple.com/kbnum/n61688}.
  }
  \opt{ipodvideo}{
    The \playertype{} is the 5th/5.5th generation \playerman{} only.
    For information on identifying which Ipod you own, see this page on Apple's
    website: \url{http://www.info.apple.com/kbnum/n61688}.
  }
  \opt{ipod6g}{
    The \playertype{} refers to the 6th generation model of the
    \playerman{}. It comes with disk sizes of 80GB, 120GB, and 160GB
    in ``thick'' and ``slim'' versions.
  }
  \opt{c200,c200v2,e200,e200v2}{
    The \playertype{} is available in multiple versions, and you need to make
    sure which you have by checking the Sandisk firmware version number under
    Settings $\rightarrow$ Info. The v1 firmware is named 01.xx.xx, while the
    v2 firmware begins with 03. Make sure that you are following the
    instructions from the correct manual.
}
  \opt{clipv1,clipv2,fuze,fuzev2}{
    The \playertype{} is available in multiple versions, and you need to make
    sure which you have by checking the Sandisk firmware version number under
    Settings $\rightarrow$ \opt{fuze,fuzev2}{System Settings $\rightarrow$}
    Info. The v1 firmware is named 01.xx.xx, while the v2 firmware begins with
    02. Make sure that you are following the instructions from the correct
    manual.
}
  \opt{cowond2}{
    Rockbox runs on all \playerman{} \playertype{} and \playertype{}+ variants
    (2 / 4 / 8 / 16~GB, with or without DAB/DMB).
    \note{Newer \playertype{}+ hardware revisions use an updated power
    management chip, and some functionality is not yet implemented on these
    players (e.g. touchscreen support).}
}
\end{description}
}

\opt{cowond2}{
\begin{description}
  \item[Current limitations.] Most Rockbox functions are usable on the
    \playertype{}/\playertype{}+, including music playback and most plugins, but
    there are a number of shortcomings that prevent it from being a fully
    supported target:
    \begin{itemize}
      \item An SD card is required to use many features, since the internal
        flash memory is read-only in Rockbox.
        \warn{The SD driver is still in development and may contain bugs.
          There have been reports of SD cards becoming unusable after being used
          with Rockbox on \playerman{} \playertype{}. Only use old, low-capacity
          cards until we are satisfied the driver is safe to use.\\}
      \item There is only basic touchscreen support. Further work is
        required to make the UI more usable with a touchscreen in general.
        \note{The touchscreen can be used in two modes, either a 3$\times$3 grid mode
          which divides the screen into areas to emulate a set of physical
          buttons (the default setting), or ``absolute point'' mode where the
          touchscreen is used to point to items on the screen.\\}
    \end{itemize}
\end{description}
}
\opt{ondavx777}{
\begin{description}
  \item[Current limitations.] Most Rockbox functions are usable on the
    \playertype{}, including music playback and most plugins, but
    there are a number of shortcomings that prevent it from being a fully
    supported target:
    \begin{itemize}
      \item A MicroSD card is required to run Rockbox, since the internal
        flash memory hasn't been figured out yet.
      \item There is only basic touchscreen support. Further work is
        required to make the UI more usable with a touchscreen in general.
        \note{The touchscreen can be used in two modes, either a 3$\times$3 grid mode
          which divides the screen into areas to emulate a set of physical
          buttons (the default setting), or ``absolute point'' mode where the
          touchscreen is used to point to items on the screen.\\}
    \end{itemize}
\end{description}
}
\opt{iriverh300}{
\begin{description}
  \item[DRM capability.] If your \dap{} has a US firmware, then by installing Rockbox you will
  \emph{permanently} lose the ability to playback files with DRM.
\end{description}
}

\opt{sansaAMS,fuzeplus}{
\begin{description}
  \item[DRM capability.] It is possible that installation of the bootloader
  may lead to you \emph{permanently} losing the ability to playback files
  with DRM.
\end{description}
}

\nopt{gigabeats}{
\begin{description}

\nopt{ipod1g2g}{
  \item[USB connection.]
}
\opt{ipod1g2g}{
  \item[Firewire connection.]
}
  To transfer Rockbox to your \dap{} you need to
  connect it to your computer. For manual installation/uninstallation, or
  should autodetection fail during automatic installation, you need to know
  where to access the \dap{}. On Windows this means you need to know
  the drive letter associated with the \dap{}. On Linux you need to know
  the mount point of your \dap{}. On Mac OS X you need to know the volume
  name of your \dap{}.

  \opt{ipod}{
    If you have Itunes installed and it is configured to open automatically
    when your \dap{} is attached (the default behaviour), then wait for it to
    open and then quit it. You also need to ensure the ``Enable use as disk''
    option is enabled for your \dap{} in Itunes. Your \dap{} should then enter
    disk mode automatically when connected to a computer via
    \nopt{ipod1g2g}{USB.}\opt{ipod1g2g}{Firewire.} If your computer does not
    recognise your \dap{}, you may need to enter disk mode manually. Disconnect
    your \dap{} from the computer. Hard reset the \dap{} by pressing and
    holding the \ButtonMenu{} and \nopt{IPOD_3G_PAD}{\ButtonSelect{}}%
    \opt{IPOD_3G_PAD}{\ButtonPlay{}} buttons simultaneously. As soon as the
    \dap{} resets, press and hold the \nopt{IPOD_3G_PAD}{\ButtonSelect{} and
    \ButtonPlay{}}\opt{IPOD_3G_PAD}{\ButtonLeft{} and \ButtonRight{}} buttons
    simultaneously. Your \dap{} should enter disk mode and you can try
    reconnecting to the computer.
  }
  \opt{iaudiox5}{
    When instructed to connect/disconnect the USB cable, always use
    the USB port through the subpack, not the side `USB Host' port. The side port
    is intended to be used for USB OTG connections only (digital cameras, memory
    sticks, etc.).
  }
  \opt{sansa,e200v2,clipv1,clipv2,fuzeplus,c200v2}{
    \note{The following steps require you to change the setting in
    \setting{Settings $\rightarrow$ USB Mode} to \setting{MSC} from within the
    original firmware.}

    \nopt{sansaAMS,fuzeplus}{
        \warn{Never extract files to your \dap{} while it
        is in recovery mode.}
    }
  }

  \opt{fuze,fuzev2,clipplus,clipzip}{
    \note{The following steps require you to change the setting in
    \setting{Settings $\rightarrow$ System Settings $\rightarrow$ USB Mode} to
    \setting{MSC} from within the original firmware.  Further note that
     all original firmware settings will be lost immediately after patching
     the bootloader, so you may need to correct this setting again after
     installing rockbox.}
  }

  \opt{iriverh10,iriverh10_5gb}{
    The installation requires you to use UMS mode and so
    may require use of the UMS trick, whereby it is possible to force a MTP
    \playertype{} to start up in UMS mode as follows:
      \begin{enumerate}
        \item Ensure the \dap{} is fully powered off by \opt{iriverh10}{using
          a pin to push the small reset button inside the hole between the
          \ButtonHold{} switch and remote control connector.}\opt{iriverh10_5gb}{removing
          the battery and putting it back in again.}
        \item Connect your \playertype{} to the computer using the data cable.
        \item Hold \ButtonRight{} and push \ButtonPower{} to turn the \dap{} on.
        \item Continue holding \ButtonRight{} until the USB Connected screen appears.
        \item The \dap{} will now appear as a regular disk on your computer.
      \end{enumerate}
      \note{Once Rockbox has been installed, when you shut down your \dap{} from Rockbox it will totally
       power the player off so step 1 is no longer necessary.}
  }
  \opt{gigabeatf}{The installation requires you to change a setting in the
      original firmware.  Make sure the option under \setting{Setup
      $\rightarrow$ Connections $\rightarrow$ PC Connections} is set to
      \setting{gigabeat room}. Also, during installation, do not connect your
      \dap{} using the cradle but plug the USB cable directly to the \dap{}.
  }
  \opt{vibe500,samsungyh}{
    The installation requires you to use UMS mode.
    In order to start up your \playertype{} in UMS mode you need to:
      \begin{enumerate}
        \item Turn off the \dap{} (Original Firmware).
        \item Connect your \playertype{} to the computer using the data cable.
        \opt{vibe500}{
        \item Push and hold \ButtonPower{} (for about 2 seconds) until the ``USB'' screen appears.
        }
      \end{enumerate}
    The \dap{} will now appear as a regular disk on your computer.
  }
\end{description}
}

\opt{ipod,sansa}{
\begin{description}
  \item[Administrator/Root rights.] Installing the bootloader portion of Rockbox
  requires you to have administrative (Windows) or root (Linux) rights.
  Consequently when doing either the automatic or manual bootloader install,
  please ensure that you are logged in with an administrator account or have root rights.
\end{description}
}

\opt{ipod}{
\begin{description}
  \item[File system format.] Rockbox only works on Ipods formatted with
  the FAT32 filesystem (i.e. Ipods initialised by Itunes
  for Windows). It does not work with the HFS+ filesystem (i.e. Ipods
  initialised by Itunes for the Mac). More information and instructions for
  converting an Ipod to FAT32 can be found on the
  \wikilink{IpodConversionToFAT32} wiki
  page on the Rockbox website. Note that after conversion, you can still use
  a FAT32 Ipod with a Mac.
\end{description}
}

\section{Installing Rockbox}\label{sec:installing_rockbox}\index{Installation}
There are two ways to install Rockbox: automated and manual. The automated
way is the preferred method of installing Rockbox for the majority of
people. Rockbox Utility is a graphical application that does almost everything
for you. However, should you encounter a problem, then the manual way is
still available to you.\\

\opt{gigabeats}{\note{The automated install is not yet available for the
  \playerlongtype{}. For now you can use the manual method to install Rockbox.
  Please still read the section on the automatic install as it explains
  various important aspects of Rockbox, such as the different versions
  available.\\}}

  \opt{HAVE_RB_BL_ON_DISK}{There are three separate components,
    two of which need to be installed in order to run Rockbox:}
  \opt{HAVE_RB_BL_IN_FLASH}{There are two separate components
    which need to be installed in order to run Rockbox:}

\begin{description}
\opt{HAVE_RB_BL_ON_DISK}{
\item[The \playerman{} bootloader.]
  The \playerman{} bootloader is the program that tells your \dap{} how to load
  and start the original firmware. It is also responsible for any emergency,
  recovery, or disk modes on your \dap{}. This bootloader is stored in special flash
  memory in your \playerman{} and comes factory-installed. It is not necessary
  to modify this in order to install Rockbox.}

\item[The Rockbox bootloader.] \index{Bootloader}
  \opt{HAVE_RB_BL_ON_DISK}{The Rockbox bootloader is loaded from disk by
  the \playerman{} bootloader. It is responsible for loading the Rockbox
  firmware and for providing the dual boot function. It directly replaces the
  \playerman{} firmware in the \daps{} boot sequence.
  \opt{gigabeatf}{\note{Dual boot does not currently work on the Gigabeat.}}}

  \opt{HAVE_RB_BL_IN_FLASH}{
  The bootloader is the program that tells your
  \dap{} how to load and start other components of Rockbox and for providing
  the dual boot function. This is the component of Rockbox that is installed
  to the flash memory of your \playerman.
  \opt{iaudiom3,iaudiom5,iaudiox5}{\note{Dual boot does not currently work on the \playertype.}}}

\item[The Rockbox firmware.]
  \opt{HAVE_RB_BL_IN_FLASH}{Unlike the \playerman{} firmware, which runs
  entirely from flash memory,}
  \opt{HAVE_RB_BL_ON_DISK}{Similar to the \playerman{} firmware,}
  most of the Rockbox code is contained in a
  ``build'' that resides on your \daps{} drive. This makes it easy to
  update Rockbox. The build consists of a directory called
  \fname{.rockbox} which contains all of the Rockbox files, and is
  located in the root of your \daps{} drive.
\end{description}

Apart from the required parts there are some addons you might be interested
in installing.
\begin{description}
\item[Fonts.] Rockbox can load custom fonts. The fonts are
    distributed as a separate package and thus need to be installed
    separately. They are not required to run Rockbox itself but
    a lot of themes require the fonts package to be installed.

\item[Themes.] The appearance of Rockbox can be customised by themes. Depending
    on your taste you might want to install additional themes to change
    the look of Rockbox.
\end{description}

\subsection{Automated Installation}

To automatically install Rockbox, download the official installer and
housekeeping tool \caps{Rockbox Utility}. It allows you to:
\begin{itemize}
\item Automatically install all needed components for using Rockbox
        (``Minimal Installation'').
\item Automatically install all suggested components (``Complete Installation'').
\item Selectively install optional components.
\item Install additional fonts and themes.
\item Install voice files and generate talk clips.
\item Uninstall all components you installed using Rockbox Utility.
\end{itemize}

Prebuilt binaries for Windows, Linux and Mac OS X are
available at the \wikilink{RockboxUtility} wiki page.\\

\opt{gigabeats,ondavx777}{\note{Rockbox Utility does not currently support the
\playertype{} and you will therefore need to follow the manual install
instructions below.\\}}

When first starting \caps{Rockbox Utility} run ``Autodetect'',
found in the configuration dialog (File $\rightarrow$ Configure). Autodetection
can detect most player types. If autodetection fails or is unable to detect
the mountpoint, make sure to enter the correct values. The mountpoint indicates
the location of the \dap{} in your filesystem. On Windows, this is the drive
letter the \dap{} gets assigned, on other systems this is a path in the
filesystem.\\*


\opt{iriverh100,iriverh300}{
  Rockbox Utility will ask you for a compatible copy of the original
  firmware. This is because for legal reasons we cannot distribute
  the bootloader directly. Instead, we have to patch the Iriver firmware
  with the Rockbox bootloader.

  Download a supported version of the Iriver firmware for your
  \playername{} from the Iriver website, links can be found on
  \wikilink{IriverBoot}.

  Supported Iriver firmware versions currently include
  \opt{iriverh100}{1.63US, 1.63EU, 1.63K, 1.65US, 1.65EU, 1.65K, 1.66US,
    1.66EU and 1.66K. Note that the H140 uses the same firmware as the H120;
    H120 and H140 owners should use the firmware called \fname{ihp\_120.hex}.
    Likewise, the iHP110 and iHP115 use the same firmware, called
    \fname{ihp\_100.hex}. Be sure to use the correct firmware file for
    your player.}
    \opt{iriverh300}{1.28K, 1.28EU, 1.28J, 1.29K, 1.29J, 1.30EU and 1.31K.
    \note{The US \playername{} firmware is not supported and cannot be
    patched to be used with the bootloader. If you wish to install Rockbox
    on a US \playername{}, you must first install a non-US version of the
    original firmware and then install one of the supported versions patched
    with the Rockbox bootloader.}
    \warn{Installing a non-US firmware on a US \playername{} will
    \emph{permanently} remove DRM support from the player.}}

  If the file that you downloaded is a \fname{.zip} file, use an unzip
  utility like mentioned in the prerequisites section to extract
  the \fname{.hex} from the \fname{.zip} file
  to your desktop. Likewise, if the file that you downloaded is an
  \fname{.exe} file, double-click on the \fname{.exe} file to extract
  the \fname{.hex} file to your desktop.
  When running Linux you should be able to extract \fname{.exe}
  files using \fname{unzip}.
}

\opt{mpiohd200,mpiohd300}{
  Rockbox Utility will ask you for a compatible copy of the original
  firmware. This is because for legal reasons we cannot distribute
  the bootloader directly. Instead, we have to patch the MPIO firmware
  with the Rockbox bootloader.

  Download a supported version of the MPIO firmware for your
  \playername{} from the MPIO website, links can be found on
  \opt{mpiohd200}{\wikilink{MPIOHD200Port}}%
  \opt{mpiohd300}{\wikilink{MPIOHD300Port}}.

  \warn{The only tested version of the original firmware is
  \opt{mpiohd200}{1.30.05}\opt{mpiohd300}{1.30.06}
  and as such is the only supported version}

  If the file that you downloaded is a \fname{.zip} file, use an unzip
  utility like mentioned in the prerequisites section to extract
  the \fname{.SYS} from the \fname{.zip} file
  to your desktop. Likewise, if the file that you downloaded is an
  \fname{.exe} file, double-click on the \fname{.exe} file to extract
  the \fname{.SYS} file to your desktop.
  When running Linux you should be able to extract \fname{.exe}
  files using \fname{unzip}.
}

\opt{fuzeplus}{
  Rockbox Utility will ask you for a compatible copy of the original
  firmware. This is because for legal reasons we cannot distribute
  the bootloader directly. Instead, we have to patch the \playerman{}
  firmware with the Rockbox bootloader.

  Download a supported version of the \playerman{} firmware for your
  \playername{} from the \playerman{} website, links can be found on
  \wikilink{SansaFuzePlusPort}.
  \note{Although the only tested version of the original firmware is 02.38.6,
    Rockbox Utility should be able to patch any newer version.}
  \note{If the file that you downloaded is a \fname{.zip} file, use an unzip
    utility like mentioned in the prerequisites section to extract the
    \fname{firmware.sb} from the \fname{.zip} file to your desktop.}
}
\subsubsection{Choosing a Rockbox version}\label{sec:choosing_version}

There are three different versions of Rockbox available from the
Rockbox website:
\label{Version}
Release version, current build and archived daily build. You need to decide which one
you want to install and get the appropriate version for your \dap{}. If you
select either ``Minimal Installation'' or ``Complete Installation'' from the
``Quick Start'' tab, then Rockbox Utility will automatically install the
release version of Rockbox. Using the ``Installation'' tab will allow you
to select which version you wish to install.

\begin{description}

\item[Release.] The release version is the latest stable release, free
   of known critical bugs. For a manual install, the current stable release of Rockbox is
   available at \url{http://www.rockbox.org/download/}.

\item[Development Build.] The development build is built at each change to
  the Rockbox source code repository and represents the current state of Rockbox
  development. This means that the build could contain bugs but most of
  the time is safe to use. For a manual install, you can download the current build from
  \url{http://build.rockbox.org/}.

\item[Archived Build.] In addition to the release version and the current build,
  there is also an archive of daily builds available for download. These are
  built once a day from the latest source code in the repository. For a manual install,
  you can download archived builds from \url{http://www.rockbox.org/daily.shtml}.

\end{description}

\note{Because current and archived builds are development versions that
      change frequently, they may behave differently than described in this manual,
      or they may introduce new (and potentially annoying) bugs. Unless you wish to
      try the latest and greatest features at the price of possibly greater instability,
      or you wish to help with development, you should stick with the release.\\*}

Please now go to \reference{ref:finish_install} to complete the installation procedure.

\subsection{Manual Installation}

The manual installation method is still available to you, should you need or desire it
by following the instructions below. If you have used Rockbox Utility
to install Rockbox, then you do not need to follow the next section and can skip
straight to \reference{ref:finish_install}

\opt{gigabeats}{\subsubsection{Installing the bootloader}
    \input{getting_started/gigabeats_install.tex}
}

\subsubsection{Installing the firmware}\label{sec:installing_firmware}

\opt{gigabeats}{\note{When your \dap{} is in the Rockbox USB or bootloader
USB mode, you will see two visible partitions -- the 150~MB firmware
partition (containing at least a file called \fname{nk.bin}) and
the main data partition. Rockbox \emph{must} be installed onto the main
data partition.}}

\begin{enumerate}
\item Download your chosen version of Rockbox from the links in the
  previous section.

\item Connect your \dap{} to the computer via USB
  \opt{sansa,sansaAMS,iriverh10,iriverh10_5gb,vibe500,fuzeplus,samsungyh}
  { in MSC mode }
  \opt{ipod3g,ipod4g,ipodmini,ipodcolor}{ or Firewire }as described in
  the manual that came with your \dap{}.

\item Take the \fname{.zip} file that you downloaded and use
 the ``Extract all'' command of your unzip program to extract
 the files onto
 \opt{gigabeats,fuzeplus}{the main data partition of }
 \opt{cowond2}{either an SD card or the internal memory of }
 \opt{ondavx777}{the MicroSD of }
 your \dap{}.
 \opt{cowond2}{\note{If you have chosen to extract to the internal memory, it
     will not be possible to save settings.}}
\end{enumerate}

\note{The entire contents of the \fname{.zip} file should be extracted
directly to the root of your \daps{} drive. Do not try to
create a separate directory on your \dap{} for the Rockbox
files! The \fname{.zip} file already contains the internal
structure that Rockbox needs.\\}

% This has nothing to do with swcodec, just that these players need our own
% bootloader so we can decide where we want the main binary.
If the contents of the \fname{.zip} file are extracted correctly, you will
have a directory called \fname{.rockbox}, which contains all the files needed
by Rockbox, in the main directory of your \daps{} drive.

\opt{swcodec}{%
    \nopt{gigabeats}{%
    \subsubsection{Installing the bootloader}
        \opt{iriverh100,iriverh300}{% $Id$ %

  Installing the bootloader is the trickiest part of the installation.
  As explained above, we cannot distribute the bootloader directly, and
  thus need to patch a compatible version of the Iriver firmware, which
  can be downloaded as described above.


\begin{enumerate}
  \item Download the firmware patcher \fname{fwpatcher.exe} from 
  \download{bootloader/iriver/} and save it to your desktop.
    \note{The firmware patcher contains Unicode support, which is not supported by 
    all versions of Windows. If you have difficulty with the firmware patcher, try 
    downloading the alternate firmware patcher \fname{fwpatchernu.exe}, which is 
    built without Unicode support.}
  %
  \item Go to your desktop and double-click on the firmware patcher.
  %
  \item In the firmware patcher dialog box, click on the \setting{Browse}
  button and navigate
  to the \fname{.hex} file that you previously downloaded.
  %
  \item Click \setting{Patch}. The firmware patcher will patch the
    original firmware to include the Rockbox bootloader. The \fname{.hex}
    file on your desktop is now a modified version of the original
    \fname{.hex} file.
  %
  \item Copy \fname{.hex} to the root folder (the highest level) of your \dap{}
\end{enumerate}
}
        \opt{mpiohd200,mpiohd300}{% $Id$ %

  Installing the bootloader is the trickiest part of the installation.
  As explained above, we cannot distribute the bootloader directly, and
  thus need to patch a compatible version of the MPIO firmware, which
  can be downloaded as described above.


\begin{enumerate}
   \item Download official Rockbox bootloader for MPIO \playertype{} from
   \opt{mpiohd200}{
       \download{bootloader/mpio/hd200}}%
   \opt{mpiohd300}{
       \download{bootloader/mpio/hd300}}
   and save it to your desktop. The archive contains three files: bootloader.mpio, 
   bootloader.map and rockbox-info.txt. The first file is actual bootloader, two
   others can be used for debugging and are irrelevant for end user.
   \item Download mkmpioboot utility from \download{bootloader/mpio/mkmpioboot}
   \item Process previously downloaded official firmware to include rockbox bootloader.
   Open terminal window and type the following command:
     \begin{code}[firstline=\opt{mpiohd200}{1}\opt{mpiohd300}{2},
                  lastline=\opt{mpiohd200}{1}\opt{mpiohd300}{2}]
       mkmpioboot HD200_UPG.SYS bootloader.mpio HD200_UPG.rb
       mkmpioboot HD300_UPG.SYS bootloader.mpio HD300_UPG.rb
     \end{code}
   \item Copy \opt{mpiohd200}{\fname{HD200\_UPG.rb}}%
              \opt{mpiohd300}{\fname{HD300\_UPG.rb}} 
   to the SYSTEM folder of your \dap{} and rename back to
   \opt{mpiohd200}{\fname{HD200\_UPG.SYS}}\opt{mpiohd300}{\fname{HD300\_UPG.SYS}}
   \item Safe eject your \dap{}
   \item Connect Wall charger and turn on the device. This should trigger firmware
   upgrade process which will install rockbox bootloader to the flash memory of the
   player.
\end{enumerate}
}
        \opt{ipod}{
          \nopt{ipod6g} {
            % $Id$ %

\nopt{ipod6g}{

\subsubsection{Bootloader installation from Windows}

\begin{enumerate}

\item Download ipodpatcher.exe from
\download{bootloader/ipod/ipodpatcher/win32/ipodpatcher.exe}
and run it whilst logged in with an administrator account.

\item If all has gone well, you should see some information displayed about
your \dap{} and a message asking you if you wish to install the Rockbox
bootloader. Press i followed by ENTER, and ipodpatcher will now
install the bootloader. After a short time you should see the message
``[INFO] Bootloader installed successfully.'' Press ENTER again to exit
ipodpatcher.

\item \note{If ipodpatcher fails to install the bootloader for you, please
be certain that you do indeed have a supported iPod model and are logged in
as an administrator. If you do, run
ipodpatcher once more and try again. If you don't, then do not attempt to
install again.}

\end{enumerate}

\subsubsection{Bootloader installation from Mac OS X}

\begin{enumerate}

\item Attach your \dap{} to your Mac and wait for its icon to appear in
Finder.

\item Download and open ipodpatcher.dmg from
\download{bootloader/ipod/ipodpatcher/macosx/ipodpatcher.dmg}
and then double-click on the ipodpatcher icon inside. You can also
drag the ipodpatcher icon to a location on your hard drive and launch
it from the Terminal.

\item If all has gone well, you should see some
information displayed about your \dap{} and a message asking you if you
wish to install the Rockbox bootloader. Press i followed by ENTER, and
ipodpatcher will now install the bootloader. After a short time you
should see the message ``[INFO] Bootloader installed successfully.'' Press
ENTER again to exit ipodpatcher and then quit the Terminal application.

\item \note{If ipodpatcher fails to install the bootloader for you, please
be certain that you do indeed have a supported iPod model. If you do, run
ipodpatcher once more and try again. If you don't, then do not attempt to
install again.}

\item Your \dap{} will now automatically reconnect itself to your Mac.
Wait for it to connect, and then eject and unplug it in the normal way.
\note{You should unplug your ipod immediately after ejecting it to
prevent Rockbox immediately rebooting your \dap{} into disk mode when it
detects that your \dap{} is attached to a computer. }

\end{enumerate}

\subsubsection{Bootloader installation from Linux}

\begin{enumerate}

\item Download ipodpatcher from
\download{bootloader/ipod/ipodpatcher/linux32x86/ipodpatcher} (32-bit x86
binary) or \download{bootloader/ipod/ipodpatcher/linux64amd64/ipodpatcher}
(64-bit amd64 binary). You can save this anywhere you wish, but the next
steps will assume you have saved it in your home directory.

\item Attach your \dap{} to your computer.

\item Open up a terminal window and type the following commands:

\begin{code}
    cd $HOME
    chmod +x ipodpatcher
    ./ipodpatcher
\end{code}

\note{You need to be the root user in order for ipodpatcher to have
sufficient permission to perform raw disk access to your \dap{}.}

\item If all has gone well, you should see some information displayed about
your \dap{} and a message asking you if you wish to install the Rockbox
bootloader. Press i followed by ENTER, and ipodpatcher will now install the
bootloader. After a short time you should see the message ``[INFO] Bootloader
installed successfully.'' Press ENTER again to exit ipodpatcher.

\end{enumerate}}

\opt{ipod6g}{

\subsubsection{Bootloader installation from Windows}

Manual installation under Windows is not supported. Please use \caps{Rockbox
Utility} for bootloader and Rockbox installation.

\subsubsection{Bootloader installation from Mac OS X}

\warn{Please make sure that your iPod is formatted using FAT32
      (a.k.a. WinPod) before attempting to install the bootloader!
      Installation will not work on HFS/HFS+ iPods (a.k.a. MacPods).}

\begin{enumerate}

\item Download the bootloader in .ipod format from
\url{https://files.freemyipod.org/~user890104/bootloader-ipodclassic-v1_0/bootloader-ipod6g.ipod}

\item Download mks5lboot for your operating system from
\url{https://files.freemyipod.org/~user890104/bootloader-ipodclassic.html\#download_stable}

\item You need to have package libusb installed using Homebrew (brew install
libusb) or MacPorts (port install libusb) in order to run mks5lboot.

\item Start mks5lboot from a terminal with the following command-line:
mks5lboot --dfuscan -l It should scan for DFU devices every second.

\item It is important to stop iTunes (dock icon -> Quit) and iTunesHelper
(using Activity monitor, find the process and select Quit or Force
quit if it keeps restarting) BEFORE continuing to the next
step. Otherwise iTunes will put your iPod in wrong mode, and you will
not be able to proceed with the installation.

\item Put your iPod in DFU mode.

\item When the device is detected, press CTRL+C to terminate the scan
process, and proceed to the next step.

\item Start mks5lboot from a terminal with the following command-line:
mks5lboot --bl-inst path/to/bootloader-ipod6g.ipod, providing the
correct path to bootloader-ipod6g.ipod that you downloaded earlier.

\item When the installation is complete, you should have Rockbox up and
running!

\end{enumerate}

\subsubsection{Bootloader installation from Linux}

\begin{enumerate}

\item Connect your iPod in normal mode (iTunes/file transfer).
\item Download \wikilink{RockboxUtility} for your operating system.
\item When \caps{Rockbox Utility} opens, select the checkbox named Show disabled targets, and point the installer to your iPod's mount point.
\item On the installation screen make sure that Rockbox is selected and Bootloader is not selected. You can install themes or the game files if you want.
\item Start the Rockbox installation.
\item Download the \href{https://files.freemyipod.org/~user890104/bootloader-ipodclassic-v1_0/bootloader-ipod6g.ipod}{bootloader in .ipod format}.
\item Download \href{https://files.freemyipod.org/~user890104/bootloader-ipodclassic.html#download_stable}{mks5lboot} for your operating system. Alternatively, you can \href{https://files.freemyipod.org/~user890104/bootloader-ipodclassic.html#build_mks5lboot}{build it} from the source code.
\item You need to have package libusb-1.0.0 installed in order to run mks5lboot.
\item To make sure the installer is marked as executable, start the following command in the terminal: chmod +x mks5lboot.
\item Start mks5lboot from a terminal with the following command-line: ./mks5lboot --dfuscan -l. It should scan for DFU devices every second.
\item Put your iPod in \href{https://files.freemyipod.org/~user890104/bootloader-ipodclassic.html#dfu}{DFU mode}.
\item When the device is detected, press CTRL+C to terminate the scan process, and proceed to the next step.
\item Start mks5lboot from a terminal with the following command-line: ./mks5lboot --bl-inst path/to/bootloader-ipod6g.ipod, providing the correct path to bootloader-ipod6g.ipod that you downloaded earlier.
\item When the installation is complete, you should have Rockbox up and running!

\end{enumerate}
}
}
          \opt{ipod6g} {
            
\subsubsection{Bootloader installation from Windows}

Manual installation under Windows is not supported. Please use \caps{Rockbox
Utility} for bootloader and Rockbox installation.

\subsubsection{Bootloader installation from Mac OS X}

\warn{Please make sure that your iPod is formatted using FAT32
      (a.k.a. WinPod) before attempting to install the bootloader!
      Installation will not work on HFS/HFS+ iPods (a.k.a. MacPods).}

\begin{enumerate}

\item Download the bootloader in .ipod format from
\url{https://files.freemyipod.org/~user890104/bootloader-ipodclassic-v1_0/bootloader-ipod6g.ipod}

\item Download mks5lboot for your operating system from
\url{https://files.freemyipod.org/~user890104/bootloader-ipodclassic.html\#download_stable}

\item You need to have package libusb installed using Homebrew (brew install
libusb) or MacPorts (port install libusb) in order to run mks5lboot.

\item Start mks5lboot from a terminal with the following command-line:
mks5lboot --dfuscan -l It should scan for DFU devices every second.

\item It is important to stop iTunes (dock icon -> Quit) and iTunesHelper
(using Activity monitor, find the process and select Quit or Force
quit if it keeps restarting) BEFORE continuing to the next
step. Otherwise iTunes will put your iPod in wrong mode, and you will
not be able to proceed with the installation.

\item Put your iPod in DFU mode.

\item When the device is detected, press CTRL+C to terminate the scan
process, and proceed to the next step.

\item Start mks5lboot from a terminal with the following command-line:
mks5lboot --bl-inst path/to/bootloader-ipod6g.ipod, providing the
correct path to bootloader-ipod6g.ipod that you downloaded earlier.

\item When the installation is complete, you should have Rockbox up and
running!

\end{enumerate}

\subsubsection{Bootloader installation from Linux}

\begin{enumerate}

\item Connect your iPod in normal mode (iTunes/file transfer).
\item Download \wikilink{RockboxUtility} for your operating system.
\item When \caps{Rockbox Utility} opens, select the checkbox named Show disabled targets, and point the installer to your iPod's mount point.
\item On the installation screen make sure that Rockbox is selected and Bootloader is not selected. You can install themes or the game files if you want.
\item Start the Rockbox installation.
\item Download the \href{https://files.freemyipod.org/~user890104/bootloader-ipodclassic-v1_0/bootloader-ipod6g.ipod}{bootloader in .ipod format}.
\item Download \href{https://files.freemyipod.org/~user890104/bootloader-ipodclassic.html#download_stable}{mks5lboot} for your operating system. Alternatively, you can \href{https://files.freemyipod.org/~user890104/bootloader-ipodclassic.html#build_mks5lboot}{build it} from the source code.
\item You need to have package libusb-1.0.0 installed in order to run mks5lboot.
\item To make sure the installer is marked as executable, start the following command in the terminal: chmod +x mks5lboot.
\item Start mks5lboot from a terminal with the following command-line: ./mks5lboot --dfuscan -l. It should scan for DFU devices every second.
\item Put your iPod in \href{https://files.freemyipod.org/~user890104/bootloader-ipodclassic.html#dfu}{DFU mode}.
\item When the device is detected, press CTRL+C to terminate the scan process, and proceed to the next step.
\item Start mks5lboot from a terminal with the following command-line: ./mks5lboot --bl-inst path/to/bootloader-ipod6g.ipod, providing the correct path to bootloader-ipod6g.ipod that you downloaded earlier.
\item When the installation is complete, you should have Rockbox up and running!

\end{enumerate}
} }
        \opt{iaudio}{The \playername{} has a built-in bootloader which performs the
firmware update and can also access the hard drive via USB.  The
Rockbox bootloader can therefore be very minimalistic, as it does not require
 its own USB mode.  This makes it less dangerous to install the Rockbox bootloader
 as you can always restore it using the \playerman{} bootloader.

\note{The Rockbox bootloader overwrites the original firmware, making it
   impossible to dual-boot.}

\subsubsection{Installation}
\begin{itemize}
\item Download the Rockbox bootloader binary from 
\download{bootloader/iaudio/}.
  \opt{iaudiox5}{Use the \fname{x5v\_fw.bin} file if your \dap{} is an X5V (without radio).
      If it is an X5 or X5L, use the \fname{x5\_fw.bin} file.}
  \opt{iaudiom5}{Use the \fname{m5\_fw.bin} file.}
  \opt{iaudiom3}{Use the \fname{cowon\_m3.bin} file.}
\item Copy it to the \fname{FIRMWARE} directory on your \dap{}.
\end{itemize}
}
        \opt{iriverh10,iriverh10_5gb}{\subsubsection{Installation}
\begin{enumerate}
  \item Download 
  \opt{iriverh10}{\download{bootloader/iriver/H10_20GC.mi4}}
  \opt{iriverh10_5gb}{
    \begin{itemize}
        \item \download{bootloader/iriver/H10.mi4} if your \dap{} is UMS or
        \item \download{bootloader/iriver/H10_5GB-MTP/H10.mi4} if it is MTP.
    \end{itemize}}
    \item Connect your \playertype{} to the computer using UMS mode and the UMS trick%
        \opt{iriverh10_5gb}{ if necessary}.
    \item Rename the \opt{iriverh10}{\fname{H10\_20GC.mi4}}\opt{iriverh10_5gb}{\fname{H10.mi4}}
        file to \fname{OF.mi4} in the \fname{System} directory on your \playertype{}.
        \opt{iriverh10_5gb}{\note{If you have a Pure model \playertype{} (which
            does not have an FM radio) it is possible that this file will be
            called \fname{H10EMP.mi4} instead. If so, rename the \fname{H10.mi4}
            you downloaded in step 1 to \fname{H10EMP.mi4}.}}
    \note{You should keep a safe backup of this file for use if you ever wish
        to switch back to the \playerman{} firmware.}
    \note{If you cannot see the \fname{System} directory, you will need to make
        sure your operating system is configured to show hidden files and
        directories.}

  \item Copy the \opt{iriverh10}{\fname{H10\_20GC.mi4}}\opt{iriverh10_5gb}{\fname{H10.mi4}
      (or \fname{H10EMP.mi4} if you have a \playertype{} Pure)} file you
      downloaded to the System directory on your \dap{}.
\end{enumerate}
}
        \opt{gigabeatf}{% $Id$

\begin{itemize}
\item Download the Rockbox bootloader from
  \download{bootloader/gigabeat/}
\item Starting at the root directory of your player browse into the directory
  \fname{GBSYSTEM} and from that into the subdirectory \fname{FWIMG}.
  These directories are hidden. Make sure that you have configured your browser
  to show hidden files or you may be unable to see \fname{FWIMG}.
\item In that directory you'll find a file called \fname{FWIMG01.DAT}. This too
  may be hidden. Rename the file to \fname{FWIMG01.DAT.ORIG}. Make sure you
  spelled that name  correctly as it is needed for booting the \playerman{} firmware.
  \warn{If you do not complete this step then you will be unable to uninstall Rockbox
  without a copy of the original firmware from the original install CD.}
\item Now copy the file \fname{FWIMG01.DAT} you downloaded to that directory.
  Make sure the spelling is correct.
\end{itemize}
}
        \opt{sansa}{\input{getting_started/sansa_install.tex}}
        \opt{sansaAMS}{% $Id: %

In order to install the bootloader, you will need to download the following
files:

\begin{itemize}
  \item An original firmware (OF) from
    \opt{e200v2}{
      \url{http://forums.sandisk.com/t5/User-guides-Firmware-Software/e200v2-03-01-16-Firmware-Download-amp-Instructions/td-p/36562}}%
    \opt{fuze,fuzev2}{
      \url{http://forums.sandisk.com/t5/Fuze/Sansa-Fuze-Firmware-Update-01-02-31-amp-02-03-33/td-p/139175}}%
    \opt{clipv1,clipv2}{
      \url{http://forums.sandisk.com/t5/Clip-Clip/Sansa-Clip-Firmware-01-01-35-amp-02-01-35-Release/td-p/144965}}%
    \opt{clipplus}{
      \url{http://forums.sandisk.com/t5/Clip-Clip/Sansa-Clip-Firmware-Update-01-02-15/td-p/150227}}%
    \opt{c200v2}{
      \url{http://forums.sandisk.com/t5/User-guides-Firmware-Software/c200-version2-Latest-Firmware-Installation-Instructions-03-02-05/td-p/34097}}
    \opt{clipzip}{
      \url{http://forums.sandisk.com/t5/Clip-Zip/Sansa-Clip-Zip-Firmware-01-01-18/td-p/254122}}

  \item The bootloader:
    \opt{e200v2}{\download{bootloader/sandisk-sansa/e200v2/bootloader-e200v2.sansa}}%
    \opt{fuze}{\download{bootloader/sandisk-sansa/fuze/bootloader-fuze.sansa}}%
    \opt{fuzev2}{\download{bootloader/sandisk-sansa/fuzev2/bootloader-fuzev2.sansa}}%
    \opt{clipv1}{\download{bootloader/sandisk-sansa/clip/bootloader-clip.sansa}}%
    \opt{clipv2}{\download{bootloader/sandisk-sansa/clipv2/bootloader-clipv2.sansa}}%
    \opt{clipplus}{\download{bootloader/sandisk-sansa/clipplus/bootloader-clipplus.sansa}}%
    \opt{c200v2}{\download{bootloader/sandisk-sansa/c200v2/bootloader-c200v2.sansa}}%
    \opt{clipzip}{\download{bootloader/sandisk-sansa/clipzip/bootloader-clipzip.sansa}}

  \item The version of mkamsboot for your operating system from
    \download{bootloader/sandisk-sansa/mkamsboot/}
\end{itemize}

\subsection{Bootloader installation from Windows}

\begin{enumerate}

\item Create a folder called ``rbinstall'' on your desktop, and then extract
  the OF, the bootloader and mkamsboot into it.

\item Open a new command prompt, and navigate to the ``rbinstall'' folder.

\item Run mkamsboot.exe, passing the name of the OF and bootloader files that
  you downloaded, in addition to a name for the output file.

  % The following arguments to the code environment select the correct line out
  % of the code block. This is necessary as using \opt{} in the code block would
  % create unwanted empty lines.
  \begin{code}[firstline=\opt{fuze}{1}\opt{fuzev2}{2}\opt{e200v2}{3}%
            \opt{clipv1}{4}\opt{clipv2}{5}\opt{clipplus}{6}\opt{c200v2}{7}%
            \opt{clipzip}{8},%
               lastline=\opt{fuze}{1}\opt{fuzev2}{2}\opt{e200v2}{3}%
            \opt{clipv1}{4}\opt{clipv2}{5}\opt{clipplus}{6}\opt{c200v2}{7}%
            \opt{clipzip}{8}]
    mkamsboot.exe fuzea.bin bootloader-fuze.sansa patched.bin
    mkamsboot.exe fuzpa.bin bootloader-fuzev2.sansa patched.bin
    mkamsboot.exe e200pe.bin bootloader-e200v2.sansa patched.bin
    mkamsboot.exe m300a.bin bootloader-clip.sansa patched.bin
    mkamsboot.exe m30pa.bin bootloader-clipv2.sansa patched.bin
    mkamsboot.exe clppa.bin bootloader-clipplus.sansa patched.bin
    mkamsboot.exe c200pa.bin bootloader-c200v2.sansa patched.bin
    mkamsboot.exe clpza.bin bootloader-clipzip.sansa patched.bin
  \end{code}
 
  If mkamsboot.exe does not report success, then retry or abort.

\item Copy the output file (``patched.bin'' in the example given above) to the
  root of the device and rename it to the same name as the OF file you downloaded earlier 
  \opt{fuze}{(``fuzea.bin''} \opt{e200v2}{(``e200pe.bin''}
  \opt{clipv1}{(``m300a.bin''} \opt{clipv2}{(``m30pa.bin''}
  \opt{clipplus}{(``clppa.bin''} \opt{fuzev2}{(``fuzpa.bin''}
  \opt{clipzip}{(``clpza.bin''} \opt{c200v2}{(``c200pa.bin''} above).
\end{enumerate}

\subsection{Bootloader installation from Mac OS X and Linux}

\begin{enumerate}

\item Create a folder called ``rbinstall'' in your home directory, and then
  extract the OF, the bootloader and mkamsboot into it.

\item Open a new terminal, and navigate to the ``rbinstall'' folder.

\item Run mkamsboot, passing the name of the OF and bootloader files that
  you downloaded, in addition to a name for the output file.

% See above for the code environment arguments explanation.
  \begin{code}[firstline=\opt{fuze}{1}\opt{fuzev2}{2}\opt{e200v2}{3}%
            \opt{clipv1}{4}\opt{clipv2}{5}\opt{clipplus}{6}\opt{c200v2}{7}%
            \opt{clipzip}{8},%
               lastline=\opt{fuze}{1}\opt{fuzev2}{2}\opt{e200v2}{3}%
            \opt{clipv1}{4}\opt{clipv2}{5}\opt{clipplus}{6}\opt{c200v2}{7}%
            \opt{clipzip}{8}]
    ./mkamsboot fuzea.bin bootloader-fuze.sansa patched.bin
    ./mkamsboot fuzpa.bin bootloader-fuzev2.sansa patched.bin
    ./mkamsboot e200pe.bin bootloader-e200v2.sansa patched.bin
    ./mkamsboot m300a.bin bootloader-clip.sansa patched.bin
    ./mkamsboot m30pa.bin bootloader-clipv2.sansa patched.bin
    ./mkamsboot clppa.bin bootloader-clipplus.sansa patched.bin
    ./mkamsboot c200pa.bin bootloader-c200v2.sansa patched.bin
    ./mkamsboot clpza.bin bootloader-clipzip.sansa patched.bin
  \end{code}
 
  If mkamsboot does not report success, then retry or abort.

\item Copy the output file (``patched.bin'' in the example given above) to the
  root of the device and rename it to the same name as the OF file you downloaded earlier 
  \opt{fuze}{(``fuzea.bin''} \opt{e200v2}{(``e200pe.bin''}
  \opt{clipv1}{(``m300a.bin''} \opt{clipv2}{(``m30pa.bin''}
  \opt{clipplus}{(``clppa.bin''} \opt{fuzev2}{(``fuzpa.bin''}
  \opt{c200v2}{(``c200pa.bin''} \opt{clipzip}{(``clpza.bin''} above).
\end{enumerate}
}
        \opt{mrobe100}{\subsubsection{Installation}
\begin{enumerate}
  \item Download 
  {\download{bootloader/olympus/mrobe100/pp5020.mi4}}
  \item Connect your \playertype{} to the computer.
  \item Rename the original \fname{pp5020.mi4} file to \fname{OF.mi4} in the \fname{System} directory on your \playertype{}.
    \warn{You should keep a safe backup of this file for use if you ever wish to switch back to the \playerman{} firmware.}
    \note{If you cannot see the \fname{System} directory, you will need to make sure your operating system is configured to show hidden files and directories.}

  \item Copy the \fname{pp5020.mi4} file you downloaded to the System directory on your \dap{}.
\end{enumerate}
}
        \opt{cowond2}{\begin{enumerate}
\item Get a copy of the original \playerman{} \playertype{} firmware from
  \url{http://www.cowonglobal.com/zeroboard/zboard.php?id=C08&bmenu=support}.
  It has to be of the same version that is currently installed on your \dap{}.

\item Build the mktccboot tool or download a binary for your operating system.

\item Build the bootloader (\fname{bootloader-cowond2.bin}). Pre-built binary
  is available at \download{bootloader/cowon/d2/bootloader-cowond2.bin}.

\item Patch the bootloader into the \playerman{} firmware using mktccboot:
  \begin{code}
    mktccboot D2N.bin bootloader-cowond2.bin D2N-patched.bin
  \end{code}
  \note{The filename depends on the firmware version and hardware revision as
    shown in the table below (for the \playertype{} use the file found in folder
    ``\fname{1}'', not ``\fname{0}'')
    \begin{table}
      \begin{rbtabular}{0.6\textwidth}{XXX}%
        { \textbf{FW / HW} & \textbf{\playertype{}} & \textbf{\playertype{}+}}{}{}
        \textbf{1.xx, 4.xx} & \fname{D2.bin} & \fname{D2\_P.bin} \\
        \textbf{2.xx, 3.xx} & \fname{D2N.bin} & \fname{D2N\_P.bin} \\
      \end{rbtabular}
    \end{table}}

\item Copy the output file (``\fname{D2N-patched.bin}'' in our example) to the
  root of the \daps{} internal memory and rename it according to the above table.
\end{enumerate}
}
        \opt{vibe500}{% $Id: %
\subsubsection{Installation}
\begin{enumerate}
  \item Download 
  \download{bootloader/vibe500/jukebox.mi4}
  \item Connect your \playertype{} to the computer using UMS mode.
  \item Rename the \fname{JUKEBOX.mi4} file to \fname{OF.mi4} in the \fname{System} directory on your \playertype{}.
    \note{You should keep a safe backup of this file for use if you ever wish to switch back to the \playerman{} firmware.}
    \note{If you cannot see the \fname{System} directory, you will need to make sure your operating system is configured to show hidden files and directories.}
  \item Copy the \fname{jukebox.mi4} file you downloaded to the \fname{System} directory on your \dap{}.
\end{enumerate}
}
        \opt{ondavx777}{\input{getting_started/ondavx777_install.tex}}
        \opt{IMX233}{\input{getting_started/IMX233_install.tex}}
        \opt{samsungyh}{% $Id: %
\subsubsection{Installation}
\begin{enumerate}
  \item Download 
  \opt{samsungyh920}{\url{http://download.rockbox.org/bootloader/samsung/yh920/PP5020.mi4}}
  \opt{samsungyh925}{\url{http://download.rockbox.org/bootloader/samsung/yh925/FW_YH925.mi4}}
  \item Connect your \playertype{} to the computer using UMS mode.
  \item Rename the \originalfirmwarefilename{} file to \fname{OF.mi4} in the \fname{System} directory on your \playertype{}.
    \note{You should keep a safe backup of this file for use if you ever wish to switch back to the \playerman{} firmware.}
    \note{If you cannot see the \fname{System} directory, you will need to make sure your operating system is configured to show hidden files and directories.}
  \item Copy the \originalfirmwarefilename{} file you downloaded to the \fname{System} directory on your \dap{}.
\end{enumerate}
}
        \opt{xduoox3}{  Installing the bootloader is the trickiest part of the installation.
  We cannot distribute the bootloader directly, and thus need to patch a compatible
  version of the xDuoo firmware.

\begin{enumerate}

\item Download the Rockbox bootloader from
  \url{http://download.rockbox.org/bootloader/xduoo/X3-v11.bsdiff}

\item Get a copy of the original \playertype{} v1.1 firmware from
  \url{https://www.xduoo.net/firmware-download/} and extract the .rar file to obtain the
  update image (called \fname{update.zip}).

\item Compile bspatch in rbutil/bspatch/ (run make in the dir).

\item Run
\begin{code}
    ../rbutil/bspatch/bspatch update.zip output.zip X3-v11.bsdiff
\end{code}

\item Put \fname{output.zip} in the root of the SD card and rename it to \fname{update.zip}.

\item Disconnect the player from USB and let the original firmware boot.

\item Go into the 'Setting' menu, and select 'Upgrade'. The player will reboot into Rockbox
  upon completion.

\end{enumerate}
}
    }
}

\subsection{Finishing the install}\label{ref:finish_install}

\opt{gigabeatf}{
  After installing you \emph{need} to power-cycle the
  \dap{} by doing the following steps. Failure to do so may result in problems.
  \begin{itemize}
  \item Safely eject / unmount your \dap{} and unplug the USB cable.
  \item Unplug any power adapter.
  \item Hold the \ButtonPower{} button to turn off the \dap{}.
  \item Slide the battery switch located on the bottom of the \dap{} from
  `on' to `off'.
  \item Slide the battery switch back from `off' to `on'.
  \end{itemize}
}

\opt{iaudiom3,iaudiom5,iaudiox5}{
  After installing you \emph{need} to power-cycle the
  \dap{} by doing the following steps.
  \begin{itemize}
  \item Safely eject / unmount your \dap{} and unplug the USB cable.
  \item Hold the
     \opt{IAUDIO_X5_PAD}{\ButtonPower}
     \opt{IAUDIO_M3_PAD}{\ButtonPlay}
     button to turn off the \dap{}.
  \item Insert the charger. The Rockbox bootloader will automatically be flashed.
  \end{itemize}
}

\opt{iriverh10,iriverh10_5gb,ipod,mrobe100,sansa,cowond2,vibe500,ondavx777,samsungyh}{
  Safely eject / unmount the USB drive, unplug the cable and restart.
}

\opt{sansaAMS}{
  Safely eject the device, unplug USB and wait for the firmware update to finish.
  Don't try to power off the device, it will shutdown by itself after a minute.
}

\opt{gigabeats}{
  Safely eject / unmount your \dap{}.
}

\opt{iriverh100,iriverh300}{
  \begin{itemize}
  \item Safely eject / unmount your \dap{}.

  \item \warn{Before proceeding further, make sure that your player has a full charge
  or that it is connected to the power adapter. Interrupting the next step
  due to a power failure most likely will brick your \dap{}.}
  Update your \daps{} firmware with the patched bootloader. To do this, turn
  the jukebox on. Press and hold the \ButtonSelect{} button to enter the main menu,
  and navigate to \setting{General $\rightarrow$ Firmware Upgrade}. Select
  \setting{Yes} when asked to confirm if you want to upgrade the
  firmware. The \playerman{} will display a message indicating that the
  firmware update is in progress. Do \emph{not} interrupt this process. When the
  firmware update is complete the player will turn itself off. (The update
  firmware process usually takes a minute or so.). You are now ready to go.
\end{itemize}
}
\opt{fuzeplus}{
  \warn{Before proceeding further, make sure that your player's battery is enough charged.
    Interrupting the next step due to a power failure most likely
    will brick your \dap{}.
  }
  \note{If you are updating/reinstalling the bootloader on a previously
    rockbox installed bootloader you will need to boot into the original
    firmware in order to perform the following step. See \reference{ref:Dualboot}
    for details on how to do so
  }
  Update your \daps{} firmware with the patched bootloader. To do this,
  safely eject /unmount your \dap{}. The update process should start
  immediatly. The \playerman{} will display an animation indicating that the
  firmware update is in progress with words: ``Updating Fuze+''.
  Do \emph{not} interrupt this process. When the firmware update is complete the
  player will restart (The update firmware process usually takes one to several
  minutes.). You are now ready to go.
}
\opt{mpiohd200}{
  \begin{itemize}
  \item Safely eject /unmount your \dap{}.

  \item \warn{Before proceeding further, make sure that your player has a full charge
  or that it is connected to the power adapter. Interrupting the next step
  due to a power failure most likely will brick your \dap{}.}
  Update your \daps{} firmware with the patched bootloader. To do this, turn
  the jukebox on. The update process should start automatically. The \playerman{} will
  display animation indicating that the firmware update is in progress. Do \emph{not}
  interrupt this process. When the firmware update is complete the player will restart.
  (The update firmware process usually takes a minute or so.). You are now ready to go.
\end{itemize}
}
\opt{e200}{Your e200 will automatically reboot and Rockbox should load.}



\subsection{Enabling Speech Support (optional)}\label{sec:enabling_speech_support}
\index{Speech}\index{Installation!Optional Steps}
If you wish to use speech support you will also need a voice file. Voice files
allow Rockbox to speak the user interface to you. Rockbox Utility can install
an English voice file, or you can download it from \url{http://www.rockbox.org/daily.shtml}
and unzip it to the root of your \dap{}.
Rockbox Utility can also aid you in the creation of voice files with different voices
or in other languages if you have a suitable speech engine installed on your computer.
Voice menus are enabled by default and will come
into effect after a reboot. See \reference{ref:Voiceconfiguration} for details
on voice settings.
Rockbox Utility can also aid in the production of talk files, which allow Rockbox
to speak file and folder names.

\section{Running Rockbox}
\nopt{ipod,gigabeats,cowond2}{When
you turn the unit on, Rockbox should load.}
\opt{ipod}{Hard reset the Ipod by holding
  \opt{IPOD_4G_PAD}{\ButtonMenu{} and \ButtonSelect{} simultaneously}%
  \opt{IPOD_3G_PAD}{\ButtonMenu{} and \ButtonPlay{} simultaneously}
  for a couple of seconds until the \dap{} resets. Now Rockbox should load.
}

\opt{gigabeats}{Rockbox should automatically load when you turn on your player.\\

  \note{
    If you have loaded music onto your \dap{} using the \playerman{}
    firmware, you will not be able to see your music properly in the
    \setting{File Browser} as MTP mode changes the location and file names.
    Files placed on your \dap{} using the \playerman{} firmware can be
    viewed by initialising and using Rockbox's database.
    See \reference{ref:database} for more information.}
}

\opt{cowond2}{
  To boot the Rockbox firmware set the \ButtonHold{} switch immediately after
  power on.\\
  \note{If you have chosen to install to an SD card and it is inserted at power
    on, Rockbox will boot from that card and use it as the primary drive for
    storing settings, etc. If there is no SD card inserted, Rockbox will boot
    from the internal memory, and it will not be possible to save settings.}
}

\opt{ipod}{
  \note{
    If you have loaded music onto your \dap{} using Itunes,
    you will not be able to see your music properly in the \setting{File Browser}.
    This is because Itunes changes your files' names and hides them in
    directories in the \fname{Ipod\_Control} directory. Files placed on your
    \dap{} using Itunes can be viewed by initialising and using Rockbox's database.
    See \reference{ref:database} for more information.
  }
}

\opt{iaudiom3}{
  \fixme{Add a note about the charging trick and place it here?}
}

\section{Updating Rockbox}
Rockbox can be easily updated with Rockbox Utility.
You can also update Rockbox manually -- download a Rockbox build
as detailed above, and unzip the build to the root directory
of your \dap{} as in the manual installation stage. If your unzip
program asks you whether to overwrite files, choose the ``Yes to all'' option.
The new build will be installed over your current build.\\

\opt{gigabeats}{
  \note{When your \dap{} is in the Rockbox USB or bootloader
  USB mode, you will see two visible partitions, the 150~MB firmware
  partition (containing at least a file called \fname{nk.bin}) and
  the main data partition. Rockbox \emph{must} be installed onto the main
  data partition.\\}
}

The bootloader only changes rarely, and should not normally
need to be updated.\\

\note{If you use Rockbox Utility be aware that it cannot detect manually
        installed components.}

\section{Uninstalling Rockbox}\index{Installation!uninstall}

\nopt{gigabeatf,iaudiom3,iaudiom5,iaudiox5,mrobe100,gigabeats,fuzeplus}{
  \note{The Rockbox bootloader allows you to choose between Rockbox and
  the original firmware. (See \reference{ref:Dualboot} for more information.)}
}

\subsection{Automatic Uninstallation}
\opt{gigabeats}{\note{Rockbox can only be uninstalled manually for now.}}

You can uninstall Rockbox automatically by using Rockbox Utility. If you
installed Rockbox manually you can still use Rockbox Utility for uninstallation
but will not be able to do this selectively.

\opt{iriverh100,iriverh300,fuzeplus}{\note{Rockbox Utility cannot uninstall the bootloader due to
the fact that it requires a flashing procedure. To uninstall the bootloader
completely follow the manual uninstallation instructions below.}}

\subsection{Manual Uninstallation}

\opt{iriverh10,iriverh10_5gb,mrobe100,vibe500,samsungyh}{
  If you would like to go back to using the original \playerman{} software,
  connect the \dap{} to your computer, and delete the
  \originalfirmwarefilename{} file and rename
  \fname{OF.mi4} to \originalfirmwarefilename{}
  in the \fname{System} directory on your \playertype{}.
  \nopt{mrobe100}{As in the installation,
  it may be necessary to first put your device into UMS mode.
  }
}

\opt{e200}{
  If you would like to go back to using the original \playerman{} software,
  connect the \dap{} to your computer, and follow the instructions to install
  the bootloader, but when prompted by sansapatcher, enter \texttt{u} for uninstall,
  instead of \texttt{i} for install. As in the installation, it may be necessary to
  first put your \dap{} into MSC mode.
}

\optv{ipod}{
  To uninstall Rockbox and go back to using just the original Ipod software, connect
  the \dap{} to your computer and follow the instructions to install
  the bootloader but, when prompted by ipodpatcher, enter \texttt{u} for uninstall
  instead of \texttt{i} for install.
}

\opt{iaudiom3,iaudiom5,iaudiox5}{
  If you would like to go back to using the original \playerman{} software,
  connect the \dap{} to your computer, download the original \playername{}
  firmware from the \playerman{} website, and copy it to the \fname{FIRMWARE}
  directory on your \playername{}. Turn off the \dap{}, remove the USB cable
  and insert the charger. The original firmware will automatically be flashed.
}

\opt{iriverh100,iriverh300}{
    If you want to remove the Rockbox bootloader, simply flash an unpatched
    \playerman{} firmware. Be aware that doing so will also remove the bootloader
    USB mode. As that mode can come in quite handy (especially if you experience
   disk errors) it is recommended to keep the bootloader. It also
    gives you the possibility of trying Rockbox anytime later by simply
    installing the distribution files.
    \opt{iriverh100}{
      The Rockbox bootloader will automatically start the original firmware if
      the \fname{.rockbox} directory has been deleted.
    }
    \opt{iriverh300}{
      Although if you retain the Rockbox bootloader, you will need to hold the
      \ButtonRec{} button each time you want to start the original firmware.
    }
}

    \opt{sansaAMS,fuzeplus}{
      Copy an unmodified original firmware to your player, and then reboot into
      the Sandisk firmware. See \reference{ref:Dualboot} for more information.
}

\opt{mpiohd200}{
    If you want to remove the Rockbox bootloader, simply flash an unpatched
    \playerman{} firmware.
}

\nopt{gigabeats}{
  If you wish to clean up your disk, you may also wish to delete the
  \fname{.rockbox} directory and its contents.
  \nopt{iaudiom3,iaudiom5,iaudiox5}{Turn the \playerman{} off.
    Turn the \dap{} back on and the original \playerman{} software will load.}
}

\opt{gigabeats}{
  If you wish to clean up your disk by deleting the
  \fname{.rockbox} directory and its contents, this must be done
  before uninstalling the bootloader in the next step.

  Before installation you should have downloaded a copy of the \playerman{}
  firmware from
  \url{http://www.tacp.toshiba.com/tacpassets-images/firmware/MESV12US.zip}.
  \begin{itemize}
  \item Extract \fname{MES12US.iso} from the \fname{.zip} downloaded above.
  \item There are two files within \fname{MES12US.iso} called
  \fname{Autorun.inf} and \fname{gbs\_update\_1\_2\_us.exe}.  Extract them with
  your favourite unzipping utility e.g. 7zip.
  \item Connect your \dap{} to your computer.
  \item Extract \fname{nk.bin} from within
  \fname{gbs\_update\_1\_2\_us.exe} using e.g. 7zip and copy it to the 150~MB
  firmware partition of your \dap{}.
  \item Safely eject / unmount the USB drive, unplug the cable and restart.
  \end{itemize}

  \note{From Windows, you can also run \fname{gbs\_update\_1\_2\_us.exe}
  directly to restore your \dap{}. This will format your \dap{},
  removing all files.}
}

\section{Troubleshooting}
\begin{description}
\opt{sansa,ipod}{
  \item[Bootloader install problems]
  If you have trouble installing the bootloader,
  please ensure that you are either logged in as an administrator (Windows), or
  you have root rights (Linux)}

\opt{fuzeplus}{
  \item[Immediately loading original firmware.]
  If the original firmware is immediately loaded without going into updating
  the firmware, then the Rockbox bootloader has not been correctly installed.
  The original firmware will only perform the update if the filename is
  correct, including case. Make sure that the patched Sansa firmware is called
  \fname{firmware.sb} and present in the root directory of your player.
}

\opt{iriverh100,iriverh300}{
  \item[Immediately loading original firmware.]
  If the original firmware is immediately
  loaded without the Rockbox bootloader appearing first, then the Rockbox bootloader
  has not been correctly installed. The original firmware update will only perform
  the update if the filename is correct, including case. Make sure that the patched
  Iriver firmware is called \fname{.hex}.}

\nopt{iriverh100,iriverh300,mpiohd200}{\item[``File Not Found'']}
\opt{iriverh100,iriverh300,mpiohd200}{\item[``-1 error'']}
  If you receive a
  \nopt{iriverh100,iriverh300,mpiohd200}{``File Not Found''}
  \opt{iriverh100,iriverh300,mpiohd200}{``-1 error''}
  from the bootloader, then the bootloader cannot find the Rockbox firmware.
  This is usually a result of not extracting the contents of the \fname{.zip}
  file to the proper location, and should not happen when Rockbox has been
  installed with Rockbox Utility.

  To fix this, either install Rockbox with the Rockbox Utility which will take care
  of this for you, or recheck the Manual Install section to see where the files
  need to be located.
\end{description}

\optv{gigabeats}{
If this does not fix the problem, there are two additional procedures that you
can try to solve this:

\begin{itemize}
\item Formatting the storage partition. It is possible that using the
mkdosfs utility from Linux to format the data partition from your PC
before installing will resolve this problem. The appropriate format command is:
\begin{code}
    mkdosfs -f 2 -F 32 -S 512 -s 64 -v -n TFAT /path/to/partition/device
\end{code}
\warn{This will remove all your files.}

\item Copying a \fname{tar}. If you have a Rockbox build environment
then you can try generating \fname{rockbox.tar} instead of
\fname{rockbox.zip} as follows:
\begin{code}
    make tar
\end{code}
and copying it to the data partition. During the next boot, the bootloader
will extract it.
\end{itemize}
}
