\subsection{PictureFlow}
\screenshot{plugins/images/ss-pictureflow}{PictureFlow}{img:pictureflow}
PictureFlow is a visual browser for your albums.  After you've selected something to play,
PictureFlow will continue running by default, or can show the WPS. The optional WPS
integration enables access to a context menu for adding albums or songs
to the dynamic playlist directly from PictureFlow (see \reference{ref:currentplaylist_submenu}).

\subsubsection{Sort Options}
Albums can be sorted by:

\begin{itemize}
    \item Name
    \item Year
    \item Album Artist + Name
    \item Album Artist + Year
\end{itemize}

\subsubsection{Requirements}
PictureFlow uses both the album art (see \reference{ref:album_art}) and
database (see \reference{ref:database}) features of Rockbox.
These need to be working correctly before PictureFlow can be used.
\textbf{PictureFlow does not support embedded album art.}

\opt{tc_ramcache}{\note{It is strongly recommended you enable the database's "Load to RAM" feature
(and restart your player afterwards)! See \reference{ref:databasemenu}. This will \textbf{vastly} speed up
building the album index as well as loading of track titles when browsing albums.}}

\opt{dircache}{\note{Enabling the disk's directory cache – see \reference{ref:dircache} – noticeably
reduces the time it takes for cover art to be displayed.}}


  \begin{itemize}
    \item PictureFlow needs to build an album index the first time it's launched
    (Cancel/Quit by pressing any button) and then writes an album art cache to disk.
    During the latter, press any button to skip past the progress bar screen and
    to start browsing immediately. Processing of album art will continue in the
    background. An indicator in the screen corner shows you the completed percentage.
    When the cache is being built, the UI may be slow to react to your input.
    For optimal performance, wait for processing to complete.
    \item Each time you update your player's database, you should manually rebuild or
    update the cache in PictureFlow.
    \item PictureFlow will accept album art larger than the dimensions of the
    screen, but the larger the dimensions, the longer they will take to scale.
  \end{itemize}

\subsubsection{Keys}
    \begin{btnmap}
        \opt{scrollwheel,IRIVER_H10_PAD,PBELL_VIBE500_PAD,MPIO_HD300_PAD}{%
            \ActionStdPrev{} / \ActionStdNext
                &
            \opt{HAVEREMOTEKEYMAP}{
                &}
            Scroll through albums / track list
                \\
        }

        % only scroll wheel and `strip' targets use the same action in album and track list
        \nopt{scrollwheel,IRIVER_H10_PAD,PBELL_VIBE500_PAD,MPIO_HD300_PAD}{%
            % currently the M3 does not use buttons of the main unit which has no display
            \nopt{IAUDIO_M3_PAD,MPIO_HD200_PAD,touchscreen}{\ButtonLeft{} / \ButtonRight{}}
            \opt{MPIO_HD200_PAD}{FIXME}
            \opt{touchscreen}{\TouchMidLeft{} / \TouchMidRight}
                &
            \opt{HAVEREMOTEKEYMAP}{
                \opt{IAUDIO_M3_PAD,GIGABEAT_RC_PAD}{\ActionRCStdPrev{} / \ActionRCStdNext}
                &}
            Scroll through albums
                \\

            \nopt{IAUDIO_M3_PAD}{\ActionStdPrev{} / \ActionStdNext}
                &
            \opt{HAVEREMOTEKEYMAP}{
                % even though the M3 uses an Iaudio remote, mapping differs when used with M/X5
                \opt{IAUDIO_M3_PAD}{\ButtonRCLeft{} / \ButtonRCRight}
                \opt{GIGABEAT_RC_PAD}{\ButtonRCVolUp{} / \ButtonRCVolDown}
                &}
            Scroll through track list
                \\
        }

        \opt{FIIO_M3K_PAD}{
                \ButtonLeft{} / \ButtonRight{}
                &
                \opt{HAVEREMOTEKEYMAP}{&}
                Quickly jump to another album (alphabetically or by year) /
                browse through albums from track list
                \\
        }
        \opt{IPOD_4G_PAD,IPOD_3G_PAD,IPOD_1G2G_PAD}{
                \ButtonLeft{} / \ButtonRight{}
                &
                \opt{HAVEREMOTEKEYMAP}{&}
                Quickly jump to another album (alphabetically or by year)
                \\
        }
        \opt{FIIO_M3K_PAD}{
                \ButtonVolUp{} / \ButtonVolDown
                &
                \opt{HAVEREMOTEKEYMAP}{&}
                Adjust current album sorting
                \\
        }

        \nopt{IAUDIO_M3_PAD}{%
            \nopt{IRIVER_H10_PAD,touchscreen,PBELL_VIBE500_PAD%
                 ,SANSA_FUZE_PAD,MPIO_HD200_PAD,MPIO_HD300_PAD,SAMSUNG_YH92X_PAD%
                 ,SAMSUNG_YH820_PAD,XDUOO_X3_PAD}
                 {\ButtonSelect}
            \opt{IRIVER_H10_PAD,PBELL_VIBE500_PAD}{\ButtonRight}
            \opt{SAMSUNG_YH92X_PAD,SAMSUNG_YH820_PAD}{\ButtonFF}
            \opt{touchscreen}{\TouchCenter}
            \opt{SANSA_FUZE_PAD}{\ButtonRight}
            \opt{MPIO_HD200_PAD}{\ButtonFunc}
            \opt{MPIO_HD300_PAD}{\ButtonEnter}
            \opt{XDUOO_X3_PAD}{\ButtonPlay}
        }
            &
        \opt{HAVEREMOTEKEYMAP}{
            \opt{IAUDIO_M3_PAD}{\ButtonRCPlay}
            \opt{GIGABEAT_RC_PAD}{\ButtonRCFF}
            &}
        Enter track list
            \nopt{SAMSUNG_YH92X_PAD,SAMSUNG_YH820_PAD}{%
            / Play album from selected track}
            \\
        % Ondio & Samsung YH use a different button in album list and track list
        \opt{SAMSUNG_YH92X_PAD,SAMSUNG_YH820_PAD}{%
            \opt{SAMSUNG_YH92X_PAD,SAMSUNG_YH820_PAD}{\ButtonPlay}
                &
            Play album from selected track in track list
                \\
        }

        \nopt{IAUDIO_M3_PAD,MPIO_HD200_PAD,MPIO_HD300_PAD,touchscreen,SANSA_FUZEPLUS_PAD,%
        FIIO_M3K_PAD}{\ButtonLeft}
        \opt{MPIO_HD200_PAD}{\ButtonRec}
        \opt{MPIO_HD300_PAD}{\ButtonMenu}
        \opt{SANSA_FUZEPLUS_PAD}{\ButtonLeft{} or \ButtonBack}
        \opt{FIIO_M3K_PAD}{\ButtonBack}
        \opt{touchscreen}{
            \opt{COWON_D2_PAD}{\ButtonPower{} or}
            \TouchBottomRight}
            &
        \opt{HAVEREMOTEKEYMAP}{
            \opt{IAUDIO_M3_PAD,GIGABEAT_RC_PAD}{\ActionRCStdCancel}
            &}
        Exit track list
            \\

        \ActionStdContext
            &
        \opt{HAVEREMOTEKEYMAP}{&}
        Enter context menu (if WPS integration is enabled)
            \\

        \nopt{IAUDIO_M3_PAD,SANSA_FUZEPLUS_PAD,FIIO_M3K_PAD}{\ActionStdMenu}
        \opt{SAMSUNG_YH92X_PAD,SAMSUNG_YH820_PAD}{or \ButtonRew}
        \opt{SANSA_FUZEPLUS_PAD}{Long \ButtonSelect}
        \opt{FIIO_M3K_PAD}{\ButtonPower}
            &
        \opt{HAVEREMOTEKEYMAP}{
            \opt{IAUDIO_M3_PAD,GIGABEAT_RC_PAD}{\ActionRCStdMenu}
            &}
        Enter menu
            \\

        \nopt{IAUDIO_M3_PAD}{%
            \opt{IRIVER_H100_PAD,IRIVER_H300_PAD}{\ButtonOff}
            \opt{IAUDIO_X5_PAD,GIGABEAT_PAD,GIGABEAT_S_PAD,SANSA_E200_PAD,SANSA_CLIP_PAD%
                ,MROBE100_PAD,SANSA_FUZEPLUS_PAD}{\ButtonPower}
            \opt{SANSA_C200_PAD,IRIVER_H10_PAD}{Long \ButtonPower}
            \opt{IPOD_4G_PAD,IPOD_3G_PAD}{Long \ButtonMenu}
            \opt{SANSA_FUZE_PAD}{Long \ButtonHome}
            \opt{PBELL_VIBE500_PAD}{\ButtonRec}
            \opt{SAMSUNG_YH92X_PAD,SAMSUNG_YH820_PAD}{long \ButtonRew}
            \opt{MPIO_HD200_PAD}{FIXME}
            \opt{MPIO_HD300_PAD}{Long \ButtonMenu}
            \opt{FIIO_M3K_PAD}{\ButtonBack}
            \opt{touchscreen}{
                \opt{COWON_D2_PAD}{Long \ButtonPower{} or}
                \TouchBottomRight{} (in album view)}
        }
            &
        \opt{HAVEREMOTEKEYMAP}{
            \opt{IAUDIO_M3_PAD}{\ButtonRCRec}
            \opt{GIGABEAT_RC_PAD}{\ButtonRCRew}
            &}
        Exit PictureFlow
            \\

    \end{btnmap}

\subsubsection{Main Menu}
\begin{description}
  \item[Go to WPS.] Leave PictureFlow and enter the While Playing Screen.
  \item[Clear playlist] Removes all entries from the current playlist.
  \item[Playback Control.] Control music playback from within the plugin.
  \item[Settings.] Enter the settings menu.
  \item[Return.] Exit menu.
  \item[Quit.] Exit PictureFlow plugin.
\end{description}

\subsubsection{Settings Menu}

\begin{description}
  \item[Show album title.] Allows setting the album title / artist to be
  shown above or below the cover art, or not at all.
  \item[Show year in album title.] Set whether a year will be displayed next to each album
  title (determined by the release year of the latest track on an album).
  \item[Sort albums by.] Change the sorting of albums in PictureFlow.
  \item[Year sort order.] Determines whether albums are sorted in ascending or descending
  order, when sorting by year.
  \item[Display FPS.] Displays frames per second on screen.
  \item[Spacing.] The distance between the front edges of the side slides, i.e. changes
  the degree of overlap of the side slides. A larger number means less overlap. Scales with zoom.
  \item[Centre margin.] The distance, in screen pixels, with zoom at 100, between
  the centre and side slides. Scales with zoom.
  \item[Number of slides.] Sets the number of slides at each side, including the
  centre slide. Therefore if set to 4, there will be 3 slides on the left,
  the centre slide, and then 3 slides on the right.
  \item[Zoom.] Changes the distance at which slides are rendered from the ``camera''.
  \item[Resize Covers.] Set whether to automatically resize the covers or to leave
  them at their original size.
  \item[Rebuild cache.] Rebuild the PictureFlow cache. This is needed in order
  for PictureFlow to pick up new albums, and may occasionally be needed if albums
  are removed.
  \item[Update cache.] Update the PictureFlow cache. New albumart will be added.
  \item[WPS integration.] Integrate PictureFlow with the While Playing Screen and enable
  the Current Playlist context menu.
  \item[Backlight.] Force backlight on while running pictureflow.
\end{description}
