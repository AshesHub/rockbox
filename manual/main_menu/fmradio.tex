% $Id$ %
\section{\label{ref:FMradio}FM Radio}  
\opt{RECORDER_PAD}{
  \note{The early V2 models were in fact FM Recorders in disguise,
  so they had the FM radio still mounted. Rockbox enables it if present -
  in case this menu does not show on your unit you can skip this chapter.\\}
}
\opt{sansa}{
  \note{Not all Sansas have a radio receiver. Generally all American models do,
  but European models might not. Rockbox will display the radio menu only if it
  can find a radio receiver in your Sansa.}
}

\screenshot{main_menu/images/ss-fm-radio-screen}{The FM radio screen}{}
  This menu option switches to the radio screen.
  The FM radio has the ability to remember station frequency settings
  (presets). Since stations and their frequencies vary depending on location,
  it is possible to load these settings from a file. Such files should have
  the filename extension \fname{.fmr} and reside in the directory
  \fname{/.rockbox/fmpresets} (note that this directory does not exist after
  the initial Rockbox installation; you should create it manually). To load
  the settings, i.e. a set of FM stations, from a preset file, just ``play''
  it from the file browser. Rockbox will ``remember'' and use it in
  \setting{PRESET} mode until another file has been selected. Some preset
  files are available here: \wikilink{FmPresets}.
  
  \opt{recording}{
      \opt{swcodec}{
         It is also possible to record the FM radio while listening.
         To start recording, enter the FM radio settings menu with
         \ActionFMMenu{} and then select \setting{Recording}.
         At this point, you will be switched to the \setting{Recording Screen}.
         Further information on \setting{Recording} can be found in
         \reference{ref:Recording}.
      }
  }

  \opt{masf}{\note{The radio will shorten battery life, because the
      MAS-chip is set to record mode for instant recordings.}
  }

      \begin{btnmap}
          \ActionFMPrev, \ActionFMNext
          \opt{HAVEREMOTEKEYMAP}{& \ActionRCFMPrev, \ActionRCFMNext}
          & Change frequency in \setting{SCAN} mode or jump to next/previous
          station in \setting{PRESET} mode.\\
          %
          Long \ActionFMPrev, Long \ActionFMNext
          \opt{HAVEREMOTEKEYMAP}{& Long \ActionRCFMPrev, Long \ActionRCFMNext}
          & Seek to next station in \setting{SCAN} mode.\\
          %
          \opt{SANSA_FUZEPLUS_PAD}{
              \ActionFMPrevPreset, \ActionFMNextPreset
              & Jump to next/previous preset station.\\
          }
          %
          \ActionFMSettingsInc, \ActionFMSettingsDec
          \opt{HAVEREMOTEKEYMAP}{
              &
              \opt{IRIVER_RC_H100_PAD}{\ActionRCFMVolUp, \ActionRCFMVolDown}%
              \nopt{IRIVER_RC_H100_PAD}{\ActionRCFMSettingsInc, \ActionRCFMSettingsDec}%
          }
          & Change volume.\\
          \opt{RECORDER_PAD}{
            \ButtonPlay
            & Freeze all screen updates. May enhance radio reception in some
              cases.\\
          }

          %
          \ActionFMExit
          \opt{HAVEREMOTEKEYMAP}{& \ActionRCFMExit}
          & Leave the radio screen with the radio playing.\\
          %
          \ActionFMStop
          \opt{HAVEREMOTEKEYMAP}{& \ActionRCFMStop}
          & Stop the radio and return to \setting{Main Menu}.\\%
          %
          \nopt{ONDIO_PAD}{%
            \nopt{RECORDER_PAD}{\ActionFMPlay
              \opt{HAVEREMOTEKEYMAP}{& \ActionRCFMPlay}
              & Mute radio playback.\\}%
            %
            \ActionFMMode
            \opt{HAVEREMOTEKEYMAP}{& \ActionRCFMMode}
            & Switch between \setting{SCAN} and \setting{PRESET} mode.\\
            %
            \ActionFMPreset
            \opt{HAVEREMOTEKEYMAP}{& \ActionRCFMPreset}
            & Open a list of radio presets. You can view all the presets that 
              you have, and switch to the station.\\
          }%
          %
          \ActionFMMenu
          \opt{HAVEREMOTEKEYMAP}{& \ActionRCFMMenu}
          & Display the FM radio settings menu.\\
          %
          % software hold targets
          \nopt{hold_button}{%
            \opt{SANSA_CLIP_PAD}{\ButtonHome+\ButtonSelect}
            \opt{SANSA_FUZEPLUS_PAD}{\ButtonPower}
            \opt{ONDIO_PAD}{\ButtonMenu+\ButtonDown}
            & Key lock (software hold switch) on/off.\\
          }%
       \end{btnmap}

  \begin{description}

  \item[Saving a preset:]
    Up to 64 of your favourite stations can be saved as presets.
    \opt{RECORDER_PAD}{Press \ButtonFTwo{} to go to the presets list, press
    \ButtonFOne{} to add a preset.}%
    \nopt{RECORDER_PAD}{%
      \ActionFMMenu{} to go to the menu, then select \setting{Add preset}.%
    }
    Enter the name (maximum number of characters is 32).
    Press \ActionKbdDone{} to save.

  \item[Selecting a preset:]
        \opt{ONDIO_PAD}{\ActionFMMenu{} to open the menu, select
          \setting{Preset}}%
        \nopt{ONDIO_PAD}{\ActionFMPreset} to go to the presets list.
        Use \ActionFMSettingsInc{} and \ActionFMSettingsDec{}
        to move the cursor and then press \ActionStdOk{}
        to select. Use \ActionStdCancel{} to leave the preset list without selecting
        anything.

  \item[Removing a preset:]
        \opt{ONDIO_PAD}{\ActionFMMenu{} to open the menu, select
          \setting{Preset}}%
        \nopt{ONDIO_PAD}{\ActionFMPreset} to go to the presets list.
        Use \ActionFMSettingsInc{} and \ActionFMSettingsDec{}
        to move the cursor and then press \ActionStdContext{}
        on the preset that you wish to remove, then select \setting{Remove Preset}.

      \opt{RECORDER_PAD,ONDIO_PAD}{
          \item[Recording:]
            \opt{RECORDER_PAD}{Press \ButtonFThree}%
            \opt{ONDIO_PAD}{Double press \ButtonMenu}
            to start recording the currently playing station. Press \ButtonOff{} to
              stop recording.%
            \opt{RECORDER_PAD}{ Press \ButtonPlay{} again to seamlessly start recording
              to a new file.}
            The settings for the recording can be changed in the
            \opt{RECORDER_PAD}{\ButtonFOne{} menu}%
            \opt{ONDIO_PAD}{respective menu reached through the FM radio settings menu
              (Long \ButtonMenu)}
            before starting the recording. See \reference{ref:Recordingsettings}
            for details of recording settings.
          }
  \end{description}
  \note{The radio will turn off when starting playback of an audio file.}
